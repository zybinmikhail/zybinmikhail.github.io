\documentclass{article}
\usepackage[utf8]{inputenc}
\usepackage[russian]{babel}
\usepackage{amsmath}
\usepackage{amsfonts}
\usepackage{mathdots}

\usepackage[X2,T2A]{fontenc}

\newcommand{\Yat}{{\fontencoding{X2}\selectfont\CYRYAT}} % Буква Ять ЗАГЛАВНАЯ на рус. клавише «ея»
\newcommand{\yat}{{\fontencoding{X2}\selectfont\cyryat}} % Буква Ять строчная

\begin{document}
\title{}
\author{}
\date{}
\maketitle

--- Thursday, June 1, 2017 ---

[3:52:10 AM] Маргулис:
Миш, а что в аттестат идёт по английскому?
 Нынешний экзамен никуда не идёт?

[6:58:15 AM] Михаил:
Пока не знаю.

[9:34:44 AM] Маргулис:
У тебя есть всё по физике
 ?

--- Friday, June 2, 2017 ---

[10:30:22 AM] Маргулис:
Ты все сдал?

[2:25:41 PM] Михаил:
Я не сдавалъ вчера. Сид\yatлъ до полдевятого, потомъ ушелъ.

[4:36:46 PM] Маргулис:
Кошмар
 Не успел?
 Ты где?

[4:57:20 PM] Маргулис:
А многие не успели?

[4:58:20 PM] Михаил:
Не знаю, можетъ быть, еще челов\yatка три.

[6:12:33 PM] Маргулис:
Так можно и не успеть
 Я заранее буду его ждать

[6:54:19 PM] Маргулис:
Чего случилось-то? Почему ты мне не отвечаешь?
 Ты где?

[7:02:47 PM] Михаил:
Ничего. Сейчасъ домой прі\yatхалъ.

[7:04:40 PM] Маргулис:
Ты отписался от экзамена?

[7:08:43 PM] Михаил:
Н\yatтъ, подумалъ, что смогу сдать выше семерки.

[7:09:15 PM] Маргулис:
А я написала этой женщине, что не хочу сдавать
 Я слишком волнуюсь на устных экзаменах
 В понедельник напишу заявление

[7:10:31 PM] Михаил:
Это бываетъ, я понимаю.

[7:13:59 PM] Маргулис:
Хотя зря, конечно
edited 
Нужно вырабатывать стрессоустойчивость

[7:15:57 PM] Михаил:
Ты умная, и теб\yat д\yatйствительно незач\yatмъ сильно волноваться.

[7:44:12 PM] Михаил:
Занимайся курсовой, пожалуйста.

[7:44:44 PM] Маргулис:
Хорошо

--- Saturday, June 3, 2017 ---

[7:53:37 PM] Михаил:
Какъ д\yatла?

[8:04:43 PM] Маргулис:
Не хочу об этом говорить.
 Все будет хорошо
 У меня скоро будет паническая атака. Просто скажи: всё хорошо. Не надо уточнять, с чем все хорошо
 Я обиделась, что ты написал и ушёл
 Сильно обиделась
 Когда я отвечаю, я начинаю постоянно проверять, не ответил ли человек
 Я оскорблена до глубины души

[8:17:28 PM] Михаил:
Прости, пожалуйста. Все хорошо.

[8:21:50 PM] Маргулис:
Больно
 Не из-за тебя

[8:22:39 PM] Михаил:
Будь сильной. Ты можешь не страдать.

[8:24:38 PM] Маргулис:
Я стараюсь
 Я не страдаю.
 Я не понимаю своего отношения к этому.
edited 
Мне кажется, что он сильно меняет моё отношение к любви
 И я становлюсь менее романтичной
edited 
[8:27:18 PM] Михаил:
У меня возникли н\yatкоторыя соображенія насчетъ тебя, я ихъ расскажу теб\yat въ институт\yat.

[8:27:32 PM] Маргулис:
Может, скажешь сейчас?
 Соображение, что я ленивая и ради того, чтобы ничего не делать, ищу повод пострадать?

[8:29:04 PM] Михаил:
Н\yatтъ, конечно. Ты, кстати, почти совс\yatмъ не ленивая.

[8:29:11 PM] Маргулис:
Ага
 Ты меня просто плохо знаешь
 Я максимально пропитала ленью своё существование
 Я чай пока выпью

[8:52:10 PM] Маргулис:
Когда ты делаешь что-то, что нравится, вместо того, что должен делать,—  это лень
 Причём такая, которая даёт возможность оправдаться

[8:57:51 PM] Михаил:
Я не д\yatлаю зд\yatсь границы. И математика, и искусство мн\yat нравятся, и искусством, и математикой я заниматься долженъ.

[9:02:17 PM] Маргулис:
Нет, есть дедлайн, который делает одно первостепеннее на несколько дней
 А если ты не занимаешься этим, то это не деятельность
 Потому что результат твоей деятельности должен заключаться в выгоде для тебя
 Или по крайней мере не ущербе
edited 
Не отвлекай уже
 Я уже злюсь на тебя
> Михаил Зыбин
> Я не д\yatлаю зд\yatсь границы. И математика, и искусство мн\yat нравятся
На эту фразу
 Не извиняйся
 Уходи

[9:04:36 PM] Михаил:
Ладно-ладно.

[9:05:07 PM] Маргулис:
Я как бы ещё ничего не делала сегодня

[11:14:51 PM] Маргулис:
Расскажи о соображениях

[11:15:29 PM] Михаил:
Ну потом, потом.

[11:15:42 PM] Маргулис:
я переживаю

[11:15:53 PM] Михаил:
Не надо. Почему?
 Я сегодня былъ въ Locus Solus. Это такая врод\yat барахолка.
 Чудесное м\yatсто. Тамъ дешевыя книги.
 [Photo]
 Этотъ шедевръ сов\yatтскаго дизайна я взялъ безплатно. Правда, онъ не работаетъ, тамъ внутри батарейки 1979 года.

[11:23:48 PM] Маргулис:
Смешно
 Эти соображения делают меня плохим человеком?
 Ну скажи же

[11:24:23 PM] Михаил:
Н\yatтъ
 Не д\yatлаютъ

[11:24:45 PM] Маргулис:
Хочешь я позвоню и ты расскажешь?

[11:26:16 PM] Михаил:
Я смогу сказать только съ-глазу-на-глазъ.
 [Photo]
 Мн\yat кажется, я похожъ на Надсона.

[11:27:40 PM] Маргулис:
> Михаил Зыбин
> Я смогу сказать только съ-глазу-на-глазъ.
Да почему?

[11:29:08 PM] Михаил:
Ну пожалуйста
 Вотъ зач\yatмъ я тебя заран\yatе заинтриговалъ?

[11:29:53 PM] Маргулис:
Я не смогу так ничего доделать
> Михаил Зыбин
> Я смогу сказать только съ-глазу-на-глазъ.
Это правда так смешно писалось?

[11:31:31 PM] Михаил:
Не знаю. Самъ придумалъ. Я вид\yatлъ "мало-по-малу".

[11:32:27 PM] Маргулис:
Реформатор

--- Sunday, June 4, 2017 ---
Incoming Call 1:01:41
Incoming Call 2:23

[12:43:57 AM] Михаил:
Я похожъ на Надсона?

[1:13:05 AM] Михаил:
Зач\yatмъ мн\yat нужны пробирки - вотъ вопросъ? Хорошо еще, что я не купиль отлинявшую зм\yatиную кожу.
 И зач\yatмъ мн\yat ржавый штангенциркуль? Можетъ быть, онъ будетъ не ржавый, поскольку я положилъ его въ растворъ лимоннаго сока.

[1:53:15 AM] Михаил:
Фонаремъ я искренне восхищаюсь.

[2:21:33 AM] Маргулис:
> Михаил Зыбин
> Я похожъ на Надсона?
Немного

[2:13:23 PM] Михаил:
$https://vk.com/wall-126365649_798$
VK
МатФак НИУ ВШЭ 1 курс 2016
БМТ 161, задание по философии на 07.06: Читать «Экономическо-философские рукописи 1844 года» Маркса. Вопросы: • Что привносит Маркс в концепцию Гег

[3:16:00 PM] Маргулис:
Мы же это записали
 Эх
 Как мне не бояться людей?

[5:17:52 PM] Михаил:
Я подумаю.

[6:12:27 PM] Маргулис:
Почему я не начала это делать 1 мая

[6:36:19 PM] Маргулис:
Я не смогу это сделать

[6:36:48 PM] Михаил:
Сможешь, давай-давай.

[6:43:34 PM] Маргулис:
Я так к ЕГЭ готовилась

[6:44:01 PM] Михаил:
Ты его отлично сдала.

[6:44:15 PM] Маргулис:
Нет, я хреново написала математику
 Если я сегодня не буду спать
 То я не смогу сдать физику

[6:45:18 PM] Михаил:
Не отвл\yatкайся на меня.

[7:01:04 PM] Маргулис:
Отвратительный момент, мне даже не поможет справка)
 Я боюсь, что из-за работы моего научника будет высокий уровень плагиата
 В куске, по крайней мере

[7:02:42 PM] Михаил:
Ну, пиши другими словами.
 Банальная, конечно, мысль.

[7:04:07 PM] Маргулис:
Может, сразу начинать печатать?

[7:04:34 PM] Михаил:
Что?

[7:05:14 PM] Маргулис:
Н
 
 
 
е поняла вопрос

[7:05:39 PM] Михаил:
Я твой тоже.

[7:06:01 PM] Маргулис:
Что ты не понял?

[7:06:51 PM] Михаил:
Что сразу начинать печатать?

[7:07:10 PM] Маргулис:
Ну текст
 Курсовой

[7:08:10 PM] Михаил:
Ладно. Работай давай.

[7:08:19 PM] Маргулис:
Ты не ответил
 Я могу сейчас отвечать тебе только да и нет
 Задавай вопрос так
Incoming Call 9:23

[7:33:24 PM] Маргулис:
Ты бы успел за день?

[7:34:05 PM] Михаил:
Ч\yatстно говоря, н\yatтъ.
Incoming Call 10:50

[7:52:12 PM] Маргулис:
Зачем там какая-то длинная строчка из =?
Incoming Call 2:20

[8:43:42 PM] Маргулис:
Error : could not start the command : pdflatex -synctex=1 -interaction=nonstopmode "курсовая".tex
Incoming Call 55:27

[9:38:29 PM] Маргулис:
Как ставить тире?
 Могу сама найти, ок

[9:39:18 PM] Михаил:
Просто символом -, разве нет?

[9:41:44 PM] Маргулис:
Это короткий же?

[9:42:16 PM] Михаил:
Тогда --

[9:43:10 PM] Маргулис:
И ещё, сразу, там нужно какие-то расстояния прописывать вручную, чтобы пошире между абзацами сделать отступ?

[9:44:21 PM] Михаил:
Бла-бла-бла

$ $

Бла-бла-бла

[9:45:06 PM] Маргулис:
Ок
 А ещё я у львовского не нашла обозначения для полей

[9:46:50 PM] Михаил:
Ты хочешь шрифтъ mathbb{}, наверное, или что?

[9:47:07 PM] Маргулис:
Наверное
 Да, правда
 Извини
 Что отвлекаю

[9:48:52 PM] Михаил:
Ничего. Может быть, получится здесь. http://www.texstudio.org/

[9:59:48 PM] Маргулис:
Уже 22)

[10:00:19 PM] Михаил:
Не волнуйся

[10:00:30 PM] Маргулис:
Ну я не докажу ничего
 У меня плохо работает интернет

[10:03:23 PM] Михаил:
https://ru.sharelatex.com/
https://www.overleaf.com/
Sharelatex
ShareLaTeX, Онлайн редактор LaTeX
Простой в использовании онлайн редактор LaTeX. Не требует установки, поддерживает совместную работу в реальном времени, контроль версий, сотни шаблонов LaTeX и многое другое.

[10:06:40 PM] Маргулис:
Я уже сама нашла
 У меня какая-то херня с компом
 Не могу сделать что-либо
 Перезагружают
 Возможно, придётся заниматься с айпэда

[10:08:09 PM] Михаил:
Это неудобн\yatе.

[10:08:39 PM] Маргулис:
Ну у меня пропал интернет на компе
 Как будто я сама не понимаю, что неудобно
 Придётся не спать
 Вообще

[10:10:44 PM] Михаил:
Хоть немного поспать нужно.

[10:11:18 PM] Маргулис:
Зачем?

[10:11:50 PM] Михаил:
Чтобы лучше соображать на физике.

[10:11:56 PM] Маргулис:
Давай посмотрим правде в глаза
 Я не успею
 У нас нет причин предполагать, что она очень простая
 Никаких
 Не знаю, почему ты по её названию решил, что она простая

[10:13:30 PM] Михаил:
Курсовая?

[10:13:38 PM] Маргулис:
Да

[10:13:42 PM] Михаил:
Я не знаю названия

[10:13:56 PM] Маргулис:
Я тебе говорила
 У меня просто нельзя сделать её меньше или больше, мне нужно доказать несколько вещей, часть выкинуть нельзя, так что все плохо

[10:16:03 PM] Михаил:
Просто сделай сейчас как можно больше.

[10:16:11 PM] Маргулис:
Да я понимаю

[10:16:12 PM] Михаил:
Ну, непросто, конечно.

[10:16:28 PM] Маргулис:
Пожрать ещё хорошо бы
 Ну я всё успею

[10:52:01 PM] Маргулис:
А менее гигантский отступ как делать?

[10:54:57 PM] Михаил:
Не знаю. Это можно искать здесь. $https://en.wikibooks.org/wiki/LaTeX/Paragraph_Formatting$

[10:59:47 PM] Маргулис:
Напиши за полчаса до предполагаемой отправки ко сну

[11:14:33 PM] Михаил:
Ты какъ? Предполагаемая отправка черезъ полчаса.
 Пойду въ душъ, минутъ двадцать меня не будетъ.

[11:49:45 PM] Маргулис:
Я просто кушала

--- Monday, June 5, 2017 ---
Incoming Call 13:14

[7:52:15 AM] Маргулис:
А нумерация страниц нужна?

[7:58:07 AM] Михаил:
Она сама происходит.

[8:01:34 AM] Маргулис:
А я не видела
 Ты встал?
 Я поспала и встала

[8:29:06 AM] Маргулис:
Слушай, а как впихнуть пробел в выносную формулу?
Missed Call

[8:36:46 AM] Михаил:
/quad. Если слишкомъ большой, можно /, или /:
 У меня н\yatтъ бэкслеша на клавіатур\yat телефона.

[8:38:44 AM] Маргулис:
Зачем ты сказал, что там y?
 Ты сказал уай
 Я уже нашла про : и ;
 У меня будет курсовая занимать очень мало места

[2:07:06 PM] Маргулис:
Я еду
 Толко к 15 буду
 Ты меня не ждёшь
 Редко читаешь
 Миша
 Алло
 Расскажи, как там
 Что происходит

[2:17:37 PM] Михаил:
Ничего

[2:17:57 PM] Маргулис:
Ты стоишь в очереди?

[2:18:17 PM] Михаил:
Начало пріема въ три. Я записалъ меня и тебя.

[2:19:17 PM] Маргулис:
А насколько мы высоко?
 Успеем все обсудить?

[2:20:41 PM] Михаил:
7 и 8

[2:20:54 PM] Маргулис:
То есть последние?
 Я же говорила раньше приходить
 Что ты за человек
 Повторяй физику

[2:22:30 PM] Михаил:
Да

[2:22:37 PM] Маргулис:
И посмотри просто так другие задачки

[2:22:46 PM] Михаил:
Да

[3:40:56 PM] Михаил:
Forwarded message: Михаил Плахов [6/5/17] 
[Photo]
[Photo]
[Photo]

[3:49:18 PM] Михаил:
[Photo]
 [Photo]

--- Tuesday, June 6, 2017 ---

[11:44:00 AM] Михаил:
Д\yatлаешь курсовую, р\yatшаешь задачи?

[9:58:12 PM] Маргулис:
Я встретила Серёжу с другой девочкой в метро. Он махнул на меня рукой, чтобы я не подходила к ним

[10:16:07 PM] Маргулис:
Чего молчишь?
 Пожалей
 Я злюсь очень

[10:36:54 PM] Михаил:
Это похоже на предательство съ его стороны.

[10:37:08 PM] Маргулис:
Он ничего не понимает в общении
 Как можно не поздороваться со знакомым человеком, когда ты с другим человеком?

[10:39:41 PM] Михаил:
Видимо, онъ противопоставляетъ ту д\yatвушку теб\yat. Ну, конечно, это невежливо.
 Я сейчасъ буду ужинать.

[10:40:42 PM] Маргулис:
Ок
> Михаил Зыбин
> Видимо, онъ противопоставляетъ ту д\yatвушку теб\yat. Ну, конечно, это
Он просто ничего не понимает в общении.

[11:32:06 PM] Маргулис:
Чего ты ушёл
 Ты же возвращался вроде

[11:35:52 PM] Михаил:
Ну, не понимаетъ и не понимаетъ. Почему это тебя злитъ?

[11:36:03 PM] Маргулис:
Я с ним больше не буду общаться
 Да потому что это обидно
 Я даже встала с места и прервала чтение ради него, а он не стал со мной здороваться
 Ради девчонки, которая курит травку и пьёт, сбегает из дома и живет несколько дней с незнакомыми парнями

[11:37:33 PM] Михаил:
Что? Такое бывало?
 Забавно

[11:37:48 PM] Маргулис:
ты о ком?
edited 
Я о ней

[11:38:15 PM] Михаил:
Сережа сб\yatгаетъ изъ дома.

[11:38:16 PM] Маргулис:
Не стал здороваться со мной ради неё
 Нет, я не о том
 Это продолжение придаточной части

[11:38:57 PM] Михаил:
А-а-а, понялъ.

[11:38:58 PM] Маргулис:
Которая сбегает из дома
 И встречалась несколько раз она с какими-то агрессивными парнями, которые чуть ли не били её

[11:41:59 PM] Михаил:
Перестать съ нимъ общаться - это хорошая мысль.

[11:42:08 PM] Маргулис:
Почему?

[11:43:17 PM] Михаил:
Онъ пишетъ теб\yat гадкія или унылыя вещи.

[11:43:35 PM] Маргулис:
Да
 А откуда ты знаешь?
 Ты о чем конкретно?

[11:44:00 PM] Михаил:
Ты говорила.

[11:44:36 PM] Маргулис:
> Маргулис
> Ты о чем конкретно?
?
 Я не помню, что говорила

[11:45:04 PM] Михаил:
> Михаил Зыбин
> Онъ пишетъ теб\yat гадкія или унылыя вещи.
Ты говорила вотъ это.

[11:45:18 PM] Маргулис:
Какие вещи-то?
 Я-то слово унылый не употребляю

[11:45:43 PM] Михаил:
Я - да.

[11:45:57 PM] Маргулис:
Поэтому ты сам так назвал конкретную вещь
 Какую?

[11:46:47 PM] Михаил:
Описание симптомовъ депрессіи.

[11:47:24 PM] Маргулис:
Слово унылый слишком унылое
 А, нет, я употребляю его о писателях

[11:50:12 PM] Михаил:
Ну, это мое воспріятіе словъ такое. Унылый - значитъ связанный съ уныніемъ.
 Сегодня при мн\yat кто-то сказалъ, что у него по какому-то предмету вышла девятка. Я спросилъ, куда вышла. Меня не поняли.
 Еще я не пользуюсь словомъ "срочный".

--- Wednesday, June 7, 2017 ---
 Почти по этой же причин\yat я не ум\yatю говорить "мы".

[12:01:25 AM] Маргулис:
Не поняла связи
 Связь с вышла?

[12:01:53 AM] Михаил:
Девятка вышла.
 Ну, вышла. Это въ дух\yat анекдотовъ про Штирлица.

[12:04:42 AM] Маргулис:
Я не об этом спросила
 Это я поняла
> Михаил Зыбин
> Почти по этой же причин\yat я не ум\yatю говорить "мы".
По какой этой?

[12:06:12 AM] Михаил:
Буквальное воспріятіе словъ.
 Еще меня коробитъ эта конструкція: "Я никогда не былъ въ Африк\yat".

[12:08:36 AM] Маргулис:
Из-за никогда?

[12:08:44 AM] Михаил:
Да

[12:08:48 AM] Маргулис:
Эх
 Напоминай мне качать пресс, мне надо килограммов на 7 похудеть
 Живот есть

[12:10:19 AM] Михаил:
Да, буду напоминать. Ты сможешь похуд\yatть, я въ тебя в\yatрю.

[12:10:38 AM] Маргулис:
То есть я по-твоему толстая

[12:11:10 AM] Михаил:
Я принялъ, что ты считаешь себя толстой.

[12:11:22 AM] Маргулис:
Хорошо

[12:11:47 AM] Михаил:
Мн\yat-то твоя фигура нравится.

[12:11:49 AM] Маргулис:
На самом деле заниматься надо, потому что от отсутствия занятий страдает скелет

[12:12:07 AM] Михаил:
Въ самомъ д\yatл\yat
 Нав\yatрное, можно качать шею
 Она будетъ лучше держаться

[12:13:13 AM] Маргулис:
Я не умею

[12:13:38 AM] Михаил:
Я не догадываюсь, какъ это д\yatлается

[12:13:44 AM] Маргулис:
И я

[12:14:13 AM] Михаил:
Маркса читала?

[12:15:14 AM] Маргулис:
Сейчас буду
 Извини, мне надоело это всё

[12:16:00 AM] Михаил:
Я? Философія?

[12:16:07 AM] Маргулис:
Философия

[12:16:37 AM] Михаил:
Да, есть такое.
 Я пойду въ душъ и спать.

[12:18:02 AM] Маргулис:
Хорошо

[12:54:44 AM] Маргулис:
Ты же что-то писал

[12:55:29 AM] Михаил:
Мн\yat не нравится "блин" какъ ругательство. Не только потому, что это \yatда, но потому, что я ясно осознаю это слово какъ эвфемизмъ другого.
Еще мн\yat не нравится оборотъ "моя д\yatвушка" и "мой парень", потому что я воспринимаю "мой" какъ "принадлежащій мн\yat", а съ моей склонностью абсолютизировать т\yatхъ, въ кого я влюбленъ, я не понимаю, какъ можно им\yatть претензію на власть надъ т\yatмъ, кого любишь.
 Люди вокругъ меня говорятъ "блин" класса со второго.
 Это все просто мои чудачества.

[12:57:32 AM] Маргулис:
Тебе не кажется, что ты относишься к смешному классу людей, буквально воспринимающих понятие принадлежности?
 Устной речи без междометий не бывает, кстати
 Они есть в любом языке
 Только люди с удивительным ораторским даром могут говорить без них

[12:58:40 AM] Михаил:
Н\yatтъ, я не противъ ихъ. Совс\yatмъ не противъ.

[12:58:46 AM] Маргулис:
Мой — это относящийся ко мне
 Вот слово реально ужасное
 Это правда
 Я тоже его часто употребляю

[1:00:41 AM] Михаил:
> Маргулис
> Мой — это относящийся ко мне
Это хорошая мысль.

[1:02:27 AM] Маргулис:
Ты же говоришь моя школа и мой препод

[1:03:05 AM] Михаил:
Еще я плохо могу сокращать или коверкать имена и фамиліи. Мн\yat не нравится "Мишаня" и особенно "Миш". Маяк и Пушка - этим тоже я не пользуюсь.
> Маргулис
> Ты же говоришь моя школа и мой препод
Я стараюсь так не говорить.
 Я серіозно обдумаю этотъ вопросъ. Ты нав\yatрняка права.

[1:04:49 AM] Маргулис:
Я не говорю Мишаня

[1:05:04 AM] Михаил:
Плаховъ такъ говоритъ.

[1:05:06 AM] Маргулис:
А ты пока не имеешь права говорить маяк)
 Мы с ним лет 5 вместе)
 Пушка — это о музее?

[1:05:36 AM] Михаил:
Да

[1:05:41 AM] Маргулис:
С ним ещё дольше
 Ну а как же вышка?
 Я, кстати, тоже не люблю сокращения своего имени
 Как и Георгий

[1:06:27 AM] Михаил:
Я не говорю "вышка".
 Къ "матфаку" не сразу привыкъ.

[1:07:47 AM] Маргулис:
Кошмар
 Ты просто формалист
 Не отвлекай уже, ок?
 Бесполезный разговор же

[1:08:24 AM] Михаил:
Ладно. Пока.

[1:09:32 AM] Маргулис:
Пока

[1:09:32 AM] Михаил:
> Маргулис
> Бесполезный разговор же
Я под\yatлился съ тобой важными для меня вещами.

[1:09:54 AM] Маргулис:
Ну это можно делать не тогда, когда мне нужно прочитать книгу и помыться, ок?
 Он даже не бесполезный, а вредный
 А я злая, потому что я в очередной
 Ой
 В очках
 Извини
 Просто очки не люблю

[3:53:38 PM] Маргулис:
Две девочки в моей началке говорили, что они эмо, а я не знала, кто это такие, знала только про страусов эму. И вот я не врубалась, почему же они эму
 И что в этих страусах такого

[4:09:58 PM] Маргулис:
Не обижай меня
 Я злая, конечно, но сама на всякие мелочи обижаюсь и думаю о них сутками. Меня это все расстраивает по-настоящему
 Ты грубо со мной разговариваешь
 У меня и так стресс
 Скажи что-нибудь хорошее
 Я скоро иду на др своего дяди

[4:29:36 PM] Маргулис:
Ну Миша
 Ну ты чего прочитал и ушёл
 Ты сам не вступаешь в разговор
 И если со мной нет конструктивного диалога, то с кем есть?
 Я не видела, как ты что-то с кем-либо обсуждаешь

[4:33:05 PM] Михаил:
Ну, есть одна знакомая, съ которой я р\yatдко вижусь.
 Ты задачи р\yatшай.

[4:33:40 PM] Маргулис:
Ну, у меня нет оснований в это верить
 Я ухожу через 1,5 часа
 Не буду уже решать
 Мне через час надо начать одеваться
 Ты сам сейчас этим занимаешься?
 Задачами
> Михаил Зыбин
> Ну, есть одна знакомая, съ которой я р\yatдко вижусь.
Значит, и говоришь редко и не сильно к этому тянешься
 Ну и кстати
 Если ты мне в чем-либо отказываешь
 Если ведёшь себя по-хамски
 И если считаешь уместным забирать у меня что-то из рук
 Хотя ты сделал все, кроме как вежливо попросить о нем и сказать, что тебе нужно на лекцию
 Если не извиняешься за то, за что стоит
 И так далее
 Это только приближает меня к действиям, которые я бы не предприняла
 О том, что ты меня перебил, я забыла бы через минуту
 А о том, что не дал мне чернила, нет
 Я их вроде никогда на тебя не выливала
 У меня нет причин что-то там предполагать
 Так что ты меня, во-первых, не знаешь, во-вторых, хамишь
 Если ты что-то там сам с собой придумал — это твои проблемы
 В конце концов, все твои чувства ничего не стоят, если ты можешь отказаться от них за день
 Ну, это что-то абсолютно безусловное и не направленное на получение выгоды
 Глупый ты, Миш.
 Не понимаешь, когда люди говорят что-то абсолютно несерьёзно
 Причём постоянно не понимаешь
 И отвечаешь на это абсолютно неадекватно
 Когда можно просто отшутиться
> Маргулис
> Причём постоянно не понимаешь
Вот эта моя формулировка означает, что вообще я говорю о сегодняшнем дне, но вспомнила вдруг, что такое часто и раньше было
 Прямо ничего не можешь спросить
 Если ты сам ничего не можешь сказать прямо, то чего ты ждёшь от других людей? Какого понимания?
 Уходишь иногда абсолютно неожиданно
 Не говоря никому
 Ощущение такое, как будто у тебя есть месячные и ты периодически по гормональным причинам сходишь с ума
 По крайней мере ты ни разу не объяснил, в чем дело

[4:56:42 PM] Михаил:
Неплохое сравненіе.

[4:57:02 PM] Маргулис:
Вот например, когда мы были с Георгием и он решал физику
 Можно же прямо сказать — Маша, я ревную, или меня раздражает, что ты не уделяешь мне внимания
 Ну что-нибудь
 Я ведь правда волнуюсь, когда на меня кто-то обижается, когда я этого не хочу

[4:58:24 PM] Михаил:
Я не былъ на тебя обиженъ.

[4:58:43 PM] Маргулис:
А ты такой милый, что ты можешь просто со мной быть помягче, и я сама тебя за все прощу
 Ты очень хороший бываешь
 Почти всегда

[5:01:34 PM] Михаил:
Спасибо. 

Тогда, въ ночь на восьмое апреля, ты здорово убедила меня въ моемъ ничтожеств\yat.

[5:01:52 PM] Маргулис:
Это было на тему искусства?

[5:02:27 PM] Михаил:
Почти вообще

[5:02:50 PM] Маргулис:
Да какое ничтожество? По-моему, у тебя меньше причин так думать о себе, чем у меня

[5:03:22 PM] Михаил:
Не знаю

[5:03:55 PM] Маргулис:
По крайней мере ты не потратил сотни часов своей жизни на слёзы и мысли об одном и том же
 Мне стыдно из-за этого

[5:05:02 PM] Михаил:
Ты больше такъ не будешь, я думаю.

[5:05:44 PM] Маргулис:
Не буду
 Это правда

[8:34:39 PM] Михаил:
Эти отрывки изъ Маркса меня рассм\yatшили.
 Человек есть непосредственный предмет естествознания; ибо непосредственнойчувственной природой для человека непосредственно является человеческая чувственность (это — тождественное выражение), непосредственно как другойчувственно воспринимаемый им человек; ибо его собственная чувственность существует для него самого, как человеческая чувственность, только через другого человека.
 Какое-нибудь существо является в своих глазах самостоятельным лишь тогда, когда оно стоит на своих собственных ногах, а на своих собственных ногах оно стоит лишь тогда, когда оно стоит лишь тогда, когда оно обязано своимсуществованием самому себе.

[11:01:31 PM] Маргулис:
Это очень смешно
 Я дома
 Дяде 50 лет
 Маргулис
 Дочь на 10 лет старше меня, а сыну 12
 Дочь вегетарианка

[11:05:40 PM] Михаил:
Они отъ одной женщины?

[11:06:03 PM] Маргулис:
Нет
 Но разница в возрасте у этих женщин небольшая вроде
 И у той женщины тоже сын от другого мужчины
 И тоже примерно такого возраста

[11:07:13 PM] Михаил:
Дядя хорошо сохранился?

[11:07:16 PM] Маргулис:
Жалко сестру, честно говоря
 Да, хорошо
 На 50 не выглядит

[11:08:01 PM] Михаил:
Какіе люди были? Теб\yat было скучно?

[11:08:25 PM] Маргулис:
Только наша семья, новая семья дяди и дочь от первого брака
 7 человек
 Ну, довольно неловко, потому что мы не общаемся

[11:09:40 PM] Михаил:
Кто они по проффессіи?

[11:11:00 PM] Маргулис:
Дядя работает в IBM. Раньше он был каким-то программистом, сейчас он управляющий и много зарабатывает
 Вторая жена — какая-то женщина из Сочи
 Он её нашёл, когда уже расстался с первой

[11:15:38 PM] Михаил:
Хорошо, значитъ, онъ не совс\yatмъ негодяй.

[11:15:59 PM] Маргулис:
Не совсем?) а почему все-таки негодяй?
 Может, у первой жены характер как у меня
 Мне кажется, меня несложно бросить
 Ну, ты по себе знаешь
 Я же тебе это буду вспоминать

[11:19:14 PM] Михаил:
Не знаю, со мной было что-то не то сегодня.
 Иногда мн\yat сложно в\yatсти себя какъ челов\yatкъ.

[11:19:56 PM] Маргулис:
А я спала три часа
 Это раз
 И со мной то, о чем я упоминала, на меня вообще нельзя обижаться

[11:21:19 PM] Михаил:
Я понимаю, да.

[11:21:57 PM] Маргулис:
Физика закончилась)
 Я передумала переходить

[11:23:17 PM] Михаил:
Я очень радъ об\yatимъ вещамъ.

--- Thursday, June 8, 2017 ---

[12:36:22 AM] Маргулис:
Ну на самом деле не очень передумала
 Я надеюсь, что будет 6 по философии, если появится тройка в аттестате, то я уйду
 Только я не знаю, идти ли мне на завтра и к какой паре
 Утренник будет?

[12:39:47 AM] Михаил:
Да

[12:40:08 AM] Маргулис:
Ты мне на следующей неделе расскажешь содержание двух утренников?
 Я повторю лекции, разумеется
 Чтобы знать, что там происходит

[12:40:43 AM] Михаил:
Расскажу.

[12:40:49 AM] Маргулис:
Хорошо
 Прости меня
 Ну правда, просто гормональные перепады

[12:41:47 AM] Михаил:
У тебя или у меня?

[12:42:07 AM] Маргулис:
У меня
 Слушай, а он ведь будет выставлять не 0, если меньше 3 оценок, а те оценки, что есть, плюс нули и делить на три

[12:43:40 AM] Михаил:
Такъ говорилось, да.

[12:44:33 AM] Маргулис:
Ну если хорошая оценка за работу, то будет все хорошо

[12:44:43 AM] Михаил:
Ой, я забылъ, пишется "разскажу".

[12:44:45 AM] Маргулис:
Будет 6 и я не буду париться
> Михаил Зыбин
> Ой, я забылъ, пишется "разскажу".
Я знаю, тоже не обратила внимания

[12:45:45 AM] Михаил:
Я пойду спать.

[12:45:53 AM] Маргулис:
Хорошо
 До завтра

[12:46:13 AM] Михаил:
До завтра

[12:47:53 AM] Маргулис:
Если там будет 9, то я уже могу не париться

[1:28:52 PM] Михаил:
Ты знаешь, что завтра контрольная работа по алгебре?

[2:36:56 PM] Маргулис:
Да
 А по дискре не сегодня же была?

[2:38:16 PM] Михаил:
Не сегодня. А листокъ съ задачами для подготовки къ алгебр\yat у тебя есть?

[2:38:35 PM] Маргулис:
Да я не пишу контрольные
 Если ты ещё не заметил

[2:39:20 PM] Михаил:
Каждый разъ забываю.

[2:43:30 PM] Маргулис:
Отправь сегодня то, что у тебя есть, из 8 листочка
 По геометрии
 Завтра сдам его и дискру, надеюсь

[2:44:31 PM] Михаил:
Да

[2:52:38 PM] Маргулис:
В пушке на той старой выставке нельзя было трогать то, что трогать изначально было нужно. Шарики, которые надо крутить

[5:31:37 PM] Михаил:
[Photo]
 [Photo]
 [Photo]
 [Photo]

[5:49:03 PM] Маргулис:
Спасибо

[11:02:20 PM] Маргулис:
Я забыла поздравить одного человека.

[11:02:59 PM] Михаил:
Сегодня надо было?

[11:03:07 PM] Маргулис:
Да.
 Несколько часов назад.
 Сейчас уже никак.

[11:04:14 PM] Михаил:
Можетъ, лучше поздравить сейчасъ, ч\yatмъ совс\yatмъ не поздравить?

[11:04:29 PM] Маргулис:
Нет

[11:04:35 PM] Михаил:
Съ днемъ рожденія?

[11:04:51 PM] Маргулис:
Ну всё, уже поздно об этом говорить.
 Человек ложится спать часов в 9
 Ненавижу себя.
 А завтра ещё один раз надо поздравлять ещё одного человека.

[11:06:55 PM] Михаил:
Съ ч\yatмъ?

[11:07:04 PM] Маргулис:
Какая разница-то?

[11:07:13 PM] Михаил:
Ладно
 Не вини себя.

--- Friday, June 9, 2017 ---

[11:06:54 PM] Маргулис:
Какого чёрта ты недавно так сказал о моём дяде?
> Михаил Зыбин
> Хорошо, значитъ, онъ не совс\yatмъ негодяй.
Я об этом
 Что значит не совсем?
 И вообще
 Если это вдруг нужно уточнить
 Ну просто не бывает таких действий, которые априори делают человека негодяем
 И уж тем более развод — это же всего лишь развод
 И почему из-за ребёнка человек должен мучиться?

[11:11:26 PM] Михаил:
Я бы сказалъ, это была шутка.

[11:11:27 PM] Маргулис:
И почему он не должен этого делать, если встретил женщину, которую полюбил?
 Нет, не была

[11:11:41 PM] Михаил:
Я не серьезно.
 Извини, пожалуйста.

[11:12:03 PM] Маргулис:
Ок
 Я люблю маргулисов
 Я купила-таки то приложение арзамаса
 Оно дурацкое, всего лишь 135 русских стихотворений
 И столько же английских
 А скоро будут французские, китайские и японские, но без перевода
 Почему без перевода-то?

[11:13:42 PM] Михаил:
Для знатоковъ.

[11:13:57 PM] Маргулис:
Они же не расширят аудиторию на эти страны никогда
 А людей, знающих эти языки, слишком мало в России
 Я о Китае и Японии
 Ну, относительно их аудитории

[11:14:53 PM] Михаил:
Согласенъ.

[11:15:06 PM] Маргулис:
135 — из них 15 Пушкина и 15 Лермонтова
 Нет Пастернака и Ахматовой вообще
 2 Цветаевой, 2 Есенина, 2 Маяка, 2 Мандельштама
 Поразительный набор людей
 Ну, более-менее ясно же, кто остальные?

[11:17:30 PM] Михаил:
У меня есть это приложеніе.

[11:17:57 PM] Маргулис:
Оно не классное

[11:18:18 PM] Михаил:
Я согласенъ.

[11:18:22 PM] Маргулис:
И ещё там есть Багрицкий. Нет Ахматовой, а он есть
 А сколько ты решил?

[11:20:40 PM] Михаил:
Не скажу

[11:20:50 PM] Маргулис:
И почему именно эти стихи? Почему нет Во глубине сибирских руд, например?

[11:21:10 PM] Михаил:
Я не знаю

[11:21:14 PM] Маргулис:
И про ворона
 Забыла его)
 Меня бесит этот набор, потому что я из них многие не знаю наизусть
 Я ведь себе противоречу. Я не рассказываю что-то людям, чтобы они не стали умнее меня
 Это неправильно

[11:26:54 PM] Михаил:
Есть такое.

[11:54:06 PM] Маргулис:
Ты же не боишься показаться глупым, зачем скрываешь от меня, сколько стихов разгадал?

--- Saturday, June 10, 2017 ---

[12:12:18 AM] Маргулис:
Ты не спросил, почему меня не было
 И не извинился за то, что ведёшь себя как последняя тварь

[12:14:26 AM] Михаил:
> Маргулис
> Ты не спросил, почему меня не было
Ты не захот\yatла итти на одну пару, какъ я подумалъ.

[12:14:35 AM] Маргулис:
Наверное
 Мне было плохо и грустно
 Грустно, потому что плохо физически
> Маргулис
> И не извинился за то, что ведёшь себя как последняя тварь
Как-то не получается и уже не получится тебя простить.

[12:15:48 AM] Михаил:
> Маргулис
> Грустно, потому что плохо физически
А что съ тобой?

[12:17:52 AM] Маргулис:
Ну так, всё пройдёт
 Голова, все дела
 Давление

[12:18:48 AM] Михаил:
Повышенное?

[12:20:16 AM] Маргулис:
Наоборот
 Не надо, тебе же плевать

[12:20:54 AM] Михаил:
Н\yatтъ

[12:21:05 AM] Маргулис:
Ну всем вам не совсем плевать
 Я же типа тоже человек
 А вам на всех людей не совсем плевать

[12:21:47 AM] Михаил:
Кому - вамъ?

[12:22:20 AM] Маргулис:
Тебе, Некрасову и прочим людям, которых во мне ничего больше не интересует

[12:26:50 AM] Михаил:
Я не сталъ къ теб\yat какъ къ челов\yatку по-другому относиться. 
Въ томъ, что я любилъ тебя, я вид\yatлъ неуваженіе къ тому, что ты любишь Диму.

[12:27:14 AM] Маргулис:
Стал как к человеку относиться по-другому
 Да не бывает так. Ты хоть уверен, что любил когда-нибудь?

[12:28:43 AM] Михаил:
Тебя или кого-то?

[12:29:15 AM] Маргулис:
Вообще
 Ну и меня

[12:29:45 AM] Михаил:
Это правильный вопросъ.
 Надо подумать.

[12:30:07 AM] Маргулис:
Просто если нет, то, скорее всего, тебе просто не дано.
 Или не нужно.
 В любом случае, ты не умеешь.

[12:30:51 AM] Михаил:
> Маргулис
> В любом случае, ты не умеешь.
Вотъ это правда, да

[12:30:57 AM] Маргулис:
Это большой навык и, возможно, женский.

[12:31:29 AM] Михаил:
Я хочу научиться

[12:31:44 AM] Маргулис:
> Маргулис
> Это большой навык и, возможно, женский.
И бесполезный, даже вредный
 Кстати, помнится, классе в 5 информация делилась по подобным критериям

[12:34:18 AM] Михаил:
У меня не было, не помню

[12:35:03 AM] Маргулис:
Ладно
 Спокойной ночи
edited 
[12:37:02 AM] Михаил:
Я сожал\yatю, что я такой. 
Я хочу, чтобы ты была въ моей жизни.

[12:39:37 AM] Маргулис:
Зачем?

[6:00:42 AM] Маргулис:
Я поняла, что могу ещё многое тебе рассказать и этого никогда не сделаю.
 И не сделала бы.
 Потому что ты этого не заслуживал и не заслуживаешь
 И я жалею о том, что уже тебе рассказала
 И если ты во что-то верил раньше, то для меня это поразительно. Есть такой вариант — спросить. Люди действительно знают, кого они не полюбят. И я бы тебя не полюбила. Не потому, что ты чего-то не знаешь, а потому, что ты тоже мазохист, и ты тоже не социальный человек, и я не чувствую никакой особой внутренней силы. На самом деле ты похоронил эту возможность, когда в первый день рассказал о порезанных венах и о том, из-за чего ты страдал в школе и так далее. Не сложилось ощущение полноценного человека. Некрасиво звучит, но это так
edited 
И я, конечно, не хочу общаться таким образом, потому что ты мне ничего не даёшь в культурном плане, а только пользуешься моими знаниями
 Все-таки мне сложно это пережить
edited 
Я чувствую себя общающейся с ребёнком. Ну или вроде того. Чувствую, как будто кроме моего вклада ничего нет, хотя есть, но ты же это не показываешь

[8:52:00 PM] Маргулис:
Миш, а в понедельник выходной?
edited 
Типа день России
 ?

[8:52:35 PM] Михаил:
Я слышалъ, да.

[8:52:36 PM] Маргулис:
Да?
 Выходной?

[8:52:50 PM] Михаил:
Да

[8:52:54 PM] Маргулис:
Хорошо
 Однако у меня каникулы

[8:53:35 PM] Михаил:
Не забывай про курсовую.

[8:54:27 PM] Маргулис:
Да
 4 дня на сдачу всех предметов — это весело
 На тебе тоже много матана
 Вроде бы

[8:57:41 PM] Михаил:
Полтора листка

[8:57:44 PM] Маргулис:
Вот
 Хотела написать о чем-то
 Но это же всё равно скучно
 И нечего тебе открывать свои мысли

[9:00:13 PM] Михаил:
Мн\yat нескучно слушать тебя.

[9:11:30 PM] Маргулис:
Мне скучно читать твои ответы
> Михаил Зыбин
> Мн\yat нескучно слушать тебя.
Ты правда подумал, что я об этом?)
 Смешно)
 Я что-то забыла, какие у тебя отношения с Гарри Поттером?

[9:21:02 PM] Михаил:
Равнодушенъ къ нему. Смотр\yatлъ вс\yat фильмы, книгъ не читалъ.

[9:24:29 PM] Маргулис:
В этом всё дело.
 Все люди вокруг меня читали Гарри Поттера
 Многие несколько раз и на разных языках

[9:27:46 PM] Михаил:
И какъ ты къ этому относишься?

[9:28:13 PM] Маргулис:
Ты серьёзно? Я вообще не понимаю, как можно его не любить

[9:32:22 PM] Михаил:
Для меня это слишкомъ популярно, чтобы я это любилъ.

[9:32:37 PM] Маргулис:
Нет, для тебя не слишком
 Это детская книга, это раз, и он не более популярен в мире, чем Булгаков в России
 Хорошие детские книги одинаково хорошо раскручены
edited 
И оливер твист, и питер пэн, и хроники Нарнии, и любая другая настолько же известная детская вещь в своих условиях были гиперуспешны
edited 
[9:35:30 PM] Михаил:
Я не люблю фэнтези.

[9:35:41 PM] Маргулис:
Ты любил властелина колец
 И Булгакова, уж извини
 Грубо говоря, Булгаков тоже фэнтези
edited 
Гарри Поттера отличает по факту только то, что новые средства распространения информации и новое количество детей помогли ему стать настолько популярным в очень сжатые сроки
 От хроников Нарнии, например
 Тебе же нравятся мифологические персонажи

[9:39:51 PM] Михаил:
Я не разобрался въ своемъ отношеніи къ мистик\yat.

[9:39:57 PM] Маргулис:
Кстати, фэнтези повсеместно используется как средство ухода от реальности
edited 
[9:40:23 PM] Михаил:
Это правда. Но оно поч\yatму-то мн\yat не нравится

[9:40:23 PM] Маргулис:
К мистическим персонажам?
 Я не понимаю слова мистика в данном контексте
 Ты, кстати, любил же фильмы ужасов, а там есть мистика
 Ты разве не рассказывал на английском о чтении Пратчетта?

[9:43:09 PM] Михаил:
Было такое

[9:43:09 PM] Маргулис:
Я лично не могу чётко объяснить, почему не люблю властелина колец
 Что-то вроде того, что этот мир слишком однородный для меня

[9:44:14 PM] Михаил:
Наоборотъ, врод\yat, слишкомъ биполярный.

[9:44:28 PM] Маргулис:
Я о том же
 Неразнообразный
 Я об эстетике
 Не о сути
 В Поттере много очень милых мелочей
 Сама история, конечно, могла бы быть сильнее, но мир слишком обаятельный
 Могла бы быть сильнее, но она соблюдена в рамках детской литературы
 Нет, самого Толкиена я уважаю
 Он же типа клёвый лингвист

[9:49:51 PM] Михаил:
Д\yatйствительно
 Самому см\yatшно, какъ я мало отв\yatчаю.

[9:50:32 PM] Маргулис:
Да
 Мне смешно от того, как много я пишу

--- Sunday, June 11, 2017 ---

[1:16:17 AM] Михаил:
У меня въ голов\yat, переплетаясь и см\yatшиваясь, играетъ музыка Баха изъ Нимфоманки, сунная соната, арія "Воздухъ", Ave Maria и музыка изъ Гранатоваго браслета. Это отвлекаетъ.

[1:51:15 AM] Маргулис:
А у меня так не бывает

[3:37:26 AM] Маргулис:
С другими людьми таких проблем, как с тобой, не было. Проблема в тебе. Задумайся об этом. Я рассказываю всем подряд то, что говорила тебе, постоянно предлагаю всем фильмы сразу после просмотра. Более того, рассказываю людям о том, что я придумываю. Я не доверяю тебе лично.

[8:24:01 PM] Маргулис:
Я завтра встречусь со своим человеком

[8:24:29 PM] Михаил:
Хорошо
 Что у тебя насчетъ митинга?

[8:35:32 PM] Маргулис:
Я не хожу на митинги
 Тем более у меня полно дел
 И мои родители не любят такие мероприятия
 Это слишком неинтеллектуальное мероприятие
 Ой, тавтология, извини
 Ну и сам понимаешь, любимый человек перевешивает митинг по всем статьям
 И Навальный хорош не сам по себе, а как единственная альтернатива
 Если бы мне по-настоящему нравилась какая-то партия, я бы была к ней близка
 Хватит молчать

[8:40:45 PM] Михаил:
Я тебя воспринялъ.

[8:40:57 PM] Маргулис:
Ну а ты?
 Идёшь?
 Его санкционировали вообще?

[8:42:08 PM] Михаил:
Санкціонировали, да. Не знаю, видимо, не иду, мать противъ.

[8:42:25 PM] Маргулис:
А ты спрашиваешь разрешения?
 Слушай, ты же поможешь завтра с матаном, если я позвоню?
 Или ты обиделся?)
 И не хочешь мне больше помогать?)

[8:43:42 PM] Михаил:
Я слабо въ немъ разбираюсь сейчасъ.

[8:43:47 PM] Маргулис:
Ладно
 Ну мне же и совсем старые нужны, так что не бросай меня
 А вообще, ты обижаешься?

[8:45:00 PM] Михаил:
Н\yatтъ, не обижаюсь, конечно. Ч\yatмъ смогу, помогу.

[8:45:22 PM] Маргулис:
Слушай, а почему ты принципиально фотографируешь не под тем углом?

[8:46:20 PM] Михаил:
Я фотографирую сверху.

[8:46:41 PM] Маргулис:
Ну можно же правильно повернуть фотоаппарат
 То есть телефон
 Ладно, неважно

[8:48:43 PM] Михаил:
[Photo]
 [Photo]
 [Photo]
 [Photo]
 [Photo]
 Пустота.

[8:51:46 PM] Маргулис:
Почему так пусто? Это не твои фотки?

[8:51:59 PM] Михаил:
Мои
 Гостиный дворъ и какой-то торговый центръ въ Ветошномъ переулк\yat.

[8:53:13 PM] Маргулис:
Странно как-то

[9:59:57 PM] Маргулис:
Я не поняла
 Поняла
 Как ты? Чего делаешь?

[10:10:53 PM] Михаил:
Повторяю алгебру.

[10:16:24 PM] Маргулис:
Для чего?
 Алло
 Вот я так и знала
 Что он идёт на митинг

[10:30:01 PM] Михаил:
Дима идетъ?

[10:30:04 PM] Маргулис:
Да
> Маргулис
> Для чего?
?
> Михаил Зыбин
> Повторяю алгебру.
?

[10:30:34 PM] Михаил:
Экзаменъ будетъ.

[10:33:15 PM] Маргулис:
Когда?
 О чем ты?

[10:34:02 PM] Михаил:
Сессія.

[10:34:23 PM] Маргулис:
Ты уже начал просто?
 А что с городенцевым?
 Будет зачёт?

[10:34:53 PM] Михаил:
Будетъ, да.

[10:35:27 PM] Маргулис:
Когда и что на нем будет?

[10:36:54 PM] Михаил:
Вм\yatсто семинара на сл\yatдующей нед\yatл\yat. Будутъ задачи. Какъ въ прошлый разъ. Насчетъ темъ не знаю.
edited 
[10:37:22 PM] Маргулис:
Ну если как в прошлый раз, значит, по этому модулю
 Ой
 Семестру
 Из семинаров
 Что там можно насчёт тем не знать?

[10:39:10 PM] Михаил:
Я плохо выражаю мысли.
 Понятно, конечно, какія темы.

[10:40:04 PM] Маргулис:
Ладно
 Ну короче мы с крылом не встретимся
 А когда митинг начинается?
 Нет настроения что-то делать

[10:50:32 PM] Михаил:
Въ 14:00

[10:50:49 PM] Маргулис:
Не могу ничего делать

[10:50:58 PM] Михаил:
Можешь
 Не ленись.

[10:52:49 PM] Маргулис:
Это не лень

[10:53:29 PM] Михаил:
Катокъ?

[10:55:50 PM] Маргулис:
Я забыла, какой смысл в это вкладывала
Incoming Call 51:38

--- Monday, June 12, 2017 ---

[12:07:37 AM] Маргулис:
Миш, ты только кинь мне, пожалуйста, лекции по дискре
 Мне многие нужны
 Их он не выложил

[12:36:21 AM] Михаил:
Да, что найду.

[3:10:50 PM] Маргулис:
Миша
 Ты можешь посмотреть один номер, который должен быть простым, потому что такой же со звёздочкой легко получился, но который у меня не получается?

[4:58:41 PM] Маргулис:
Ты решил мою задачу?

[5:35:04 PM] Маргулис:
Это ответ нет?

[5:35:19 PM] Михаил:
Да, я не р\yatшилъ.

[5:37:20 PM] Маргулис:
Ты думаешь, это возможно?

[5:43:40 PM] Михаил:
Если функція обращается въ нуль при равенств\yat х1 и х2, то она обращается въ нуль при равенств\yat любыхъ двухъ иксовъ.
 [Photo]
 Гипотеза.

[6:09:21 PM] Маргулис:
Она симметрическая?
 Нет же

[6:10:53 PM] Михаил:
Да, кососимметрическая

[6:11:09 PM] Маргулис:
Мне не нужна косо

[6:11:24 PM] Михаил:
Ага
Incoming Call 14:20
Incoming Call 1:09

[6:53:23 PM] Маргулис:
Кстати
 А любая невырожденная матрица может быть линейным оператором?
 Гм
 Вообще любая матрица?
 Ну то есть не оператором, а матрицей оператора

[7:34:02 PM] Михаил:
Безусловно, да.

[8:07:34 PM] Маргулис:
Миша, найди 11 лист по матану)
 Ну или кинь мне 12-13-14
 То есть не или
 А и
 А
 Не нужен 11
 Он в прошлом модуле был

[8:10:25 PM] Михаил:
Я кину.

[8:11:48 PM] Маргулис:
И попробуй сфоткать не так криво) но это не к спеху, мне есть чем заняться

[8:31:20 PM] Маргулис:
Размышляя о логике той задачи, я поняла, почему они поставили там звездочку
 В пункте б
 Наверное, то, что ты предложил, очень логично по их мнению
edited 
А пункт б случайно так быстро я подобрала
edited 
Все, я просто ленюсь сейчас
 Блин
 Занимайся
edited 
Тороплюсь и опечатываюсь
 Смешно

[8:33:08 PM] Михаил:
Не ленись, пожалуйста

[8:33:13 PM] Маргулис:
Ок

[9:29:09 PM] Маргулис:
Слушай
 Я задумалась
 Насчёт твоих предпочтений
 Если тебе нравится кровь, то как ты относишься к менструации?
 Да, это всё геометрия наталкивает меня на эти мысли
 Потому что математика совокупляет мой мозг

[9:31:39 PM] Михаил:
Не думалъ объ этомъ въ такомъ контекст\yat. Никакъ особенно мн\yat не нравится менструація. Да я и не вид\yatлъ видео объ этомъ.

[9:44:17 PM] Маргулис:
Как действует g в листке 8 номер 4? Почему там А' остаётся на месте?
 А
 Блин
 Извини
 Я не увидела кое-что

[9:45:49 PM] Михаил:
Ладно
Incoming Call 19:53

--- Tuesday, June 13, 2017 ---

[12:05:07 AM] Михаил:
[ Егор Летов – всё идёт по плану  ] 2.8 MB
 
 Въ отд\yatлъ общаго развитія.

[12:10:57 AM] Маргулис:
Ок

[12:14:48 AM] Михаил:
[Photo]
 [Photo]
 [Photo]
 [Photo]
 [Photo]

[12:30:03 AM] Маргулис:
Завтра возьми с собой, я там попереписываю
 Сейчас все-таки другим займусь
 Не очень понятно, что за чем идёт, кстати
 [Photo]
 Это номер 7?
 Или 1?

[12:31:04 AM] Михаил:
Семь

[12:31:15 AM] Маргулис:
Я правда не понимаю твоей семерки
 И 12 найди)

[12:31:41 AM] Михаил:
Воистину

[12:32:32 AM] Маргулис:
А маленький листочек — это продолжение какого номера?

[12:32:56 AM] Михаил:
3 а

[1:18:49 AM] Михаил:
Я легъ.

[1:59:02 AM] Маргулис:
Хорошо

[10:00:15 PM] Маргулис:
Ну Миша
 Ты можешь хотя бы лекции мне отправить?
 Почему тебя надо много раз просить об этом?
Missed Call

[10:01:47 PM] Маргулис:
Я сказала об этих лекциях дня три назад
 По дискре

[10:02:23 PM] Михаил:
Начиная съ какой даты?

[10:02:26 PM] Маргулис:
Просто возьми их, сфотографируй и отправь, их по-твоему завтра легче будет найти?
edited 
Да иди ты к черту, послезавтра уже контрольная, а я просила о них несколько дней назад

[10:05:59 PM] Михаил:
[Photo]
 [Photo]
 [Photo]
 [Photo]
 [Photo]

[10:06:15 PM] Маргулис:
Когда ты собирался это сделать, мне интересно?

[10:06:33 PM] Михаил:
Когда ты напомнишь.

[10:06:34 PM] Маргулис:
Ну ладно
 Спасибо

[10:07:02 PM] Михаил:
Я слышалъ, тамъ задачи только на графы будутъ.
edited 
[10:07:14 PM] Маргулис:
Ну меня ни на одной лекции по ним не было
 Это всё, что было о них?

[10:07:46 PM] Михаил:
Ты не путаешь, что контрольная послезавтра?

[10:07:57 PM] Маргулис:
А когда,
 ?

[10:08:49 PM] Михаил:
22

[10:08:54 PM] Маргулис:
Нет
 Послезавтра
 И 22
> Маргулис
> Это всё, что было о них?
?

[10:09:25 PM] Михаил:
Хороший вопросъ
 Я не знаю

[10:10:41 PM] Маргулис:
Ты безответственный

[10:11:02 PM] Михаил:
Не безъ этого
 Я узнаю

[10:12:03 PM] Маргулис:
Нечего узнавать, 15 контрольная
 Все, не отвлекай
 Я спала днём
 Слушай, ты не спросишь у Плахова, есть ли у него лекции по графам?

[10:15:59 PM] Михаил:
Я спросилъ. Жду отв\yatтъ.

[10:57:23 PM] Маргулис:
Forwarded message: 
Полина Барон [6/13/17] 
На диск же закинули все на свете
Позавчера или вчера
А тебя нет в общей беседе? Там сказали

[10:57:29 PM] Маргулис:
А ты там есть
 И мне не сказал

[10:57:40 PM] Михаил:
Да

[10:58:05 PM] Михаил:
Forwarded message: 
Vlad

 
[10:58:36 PM] Маргулис:
Зачем? Если это там есть
 Я случайно загрузила и теперь это много места занимает наверняка
 Миша, 12 лист матан, 9 лист геом, дискра)

[11:00:44 PM] Михаил:
Да

[11:00:54 PM] Маргулис:
Хотя оказалось, что я даже делала часть 12 листа
 Но ты кинь

[11:04:47 PM] Михаил:
Листок 14, если ты не противъ.
 [Photo]
 [Photo]
 [Photo]
 [Photo]
 [Photo]
 [Photo]
 [Photo]

[11:06:19 PM] Маргулис:
Спасибо
 Скажи
 Ты считаешь, что рано с кем-то спать в 18 лет?

[11:12:43 PM] Михаил:
Я считаю, самое время.

--- Wednesday, June 14, 2017 ---

[12:32:34 AM] Маргулис:
Слушай, а ты смотрел у огарка другие задания?
 Я завтра сдаю, кстати, так что после философии надо это обсудить

[12:33:18 AM] Михаил:
Огарокъ мной преданъ забвенью.

[12:39:28 AM] Маргулис:
Жаль
 У вас не сходится 13 5 а
Missed Call

[12:58:27 AM] Маргулис:
Миша
Missed Call

[12:59:00 AM] Маргулис:
Срочно
Missed Call
Missed Call
Missed Call

[1:07:55 AM] Михаил:
Я зд\yatсь.

[1:12:58 AM] Маргулис:
Я уже успокоилась
 Наверное, лучше там все обсудить
 На матфаке
 Просто истерика
 Из-за алгебры
 И курсовой
 И матана

[1:13:40 AM] Михаил:
Они того не стоятъ.
 Все будетъ хорошо.

[1:13:57 AM] Маргулис:
Нет
 Я не прочитала Сартра
 Может, в метро
 И я не читала лекции по матану
 Дискру ты сделал?

[1:15:22 AM] Михаил:
Половину. На вторую я сегодня пару часовъ смотр\yatлъ. Съ ней ничего не случилось.
Incoming Call 10:34

[1:27:37 AM] Маргулис:
Стой, Миш
Incoming Call 0:51
Cancelled Call

[2:35:04 AM] Маргулис:
МИША
 Ты не дал мне 12 лист же
 Обязательно прочитай это утром
 Я тебя прошу, возьми его

[9:45:23 AM] Михаил:
[Photo]

[9:02:01 PM] Маргулис:
Дискру
 9 лист по геоме

[9:08:57 PM] Михаил:
Геометріи н\yatтъ, дискотеку потомъ, когда р\yatшу что-нибудь новое. У меня плохо записаны задачи, потому что он\yat почти устныя.

[9:12:37 PM] Маргулис:
Ещё раз объясни про две висячие вершины

[9:14:27 PM] Михаил:
Подумай пятнадцать минутъ. Если не получится, скажи.

[9:19:50 PM] Маргулис:
У меня нет пятнадцати минут
 Ну ок

[9:36:47 PM] Маргулис:
Ладно, я должна заниматься более полезными делами
edited 
Ссылки в Википедии — сплошные наши преподы
 Зачем после того, как я задала тебе вопрос, ты тут же задал его дальше людям, сидящим справа?

[9:53:06 PM] Михаил:
Напомни, когда?

[9:54:57 PM] Маргулис:
На Сартре

[9:55:57 PM] Михаил:
Не помню.

[9:58:21 PM] Маргулис:
Неважно
 120 рёбер
 Это смешно

[9:58:56 PM] Михаил:
Да

[9:59:00 PM] Маргулис:
Слушай
 Я тут подумала
 И решила задать тебе вопрос
 Ты вот осознаёшь, что уговоры — это форма принуждения?

[10:00:27 PM] Михаил:
Да

[10:00:56 PM] Маргулис:
И принуждение — это плохо, правильно?

[10:01:36 PM] Михаил:
Н\yatтъ.

[10:01:50 PM] Маргулис:
Я не о математике, Миш
 Не о том, что что-то лень делать, а ты меня уговариваешь

[10:03:25 PM] Михаил:
Давай въ институт\yat обсудимъ.

[10:07:39 PM] Маргулис:
> Михаил Зыбин
> Н\yatтъ.
Что значит нет?

[10:10:12 PM] Михаил:
Я не буду это обсуждать въ телеграм\yat, я знаю, ч\yatмъ это заканчивается.

--- Thursday, June 15, 2017 ---
Incoming Call 22:11

[12:22:22 AM] Маргулис:
Миш, я ещё не знаю, к какой паре приду
 Возможно, к 12

[1:09:58 AM] Михаил:
Я ложусь.

[1:10:05 AM] Маргулис:
Спокойной

[12:04:28 PM] Маргулис:
Ты где?
 Эй
 Зыбин
 Ты издеваешься что ли?

[6:17:23 PM] Маргулис:
Мур

[8:28:24 PM] Маргулис:
https://meduza.io/shapito/2017/06/15/programmisty-kotorye-ispolzuyut-dlya-otstupov-probely-okazalis-bogache-teh-kto-zhmet-klavishu-tab

Meduza
Программисты, которые используют для отступов пробелы, оказались богаче тех, кто...
Одна из главных тем споров среди программистов — как отбивать отступы строк при написании кода. Во многих языках программирования отступы нужны просто...

[8:30:22 PM] Михаил:
Я пользовался табуляціей, но чувствовалъ, что это некошерно.

[8:49:42 PM] Маргулис:
Я нашла то, что мы искали
 Но я не понимаю, как ты это сделал

[8:50:34 PM] Михаил:
Не понялъ тебя, поясни.

[8:50:52 PM] Маргулис:
> Маргулис
> Я нашла то, что мы искали
Про чосинусы и котангенсы

[8:51:13 PM] Михаил:
Хорошо.

[8:51:50 PM] Маргулис:
Не могу с тобой согласиться
 А
 Всё
 Я не прочитала, что угол прямой
 Сегодня
 Никогда бы не подумала, что вывод формулы кто-то делает снизу вверх

[9:38:11 PM] Маргулис:
Убей меня завтра

[9:38:38 PM] Михаил:
Что случилось?

[10:00:01 PM] Маргулис:
Все плохо

[10:00:47 PM] Михаил:
Дискру больше не надо сдавать.
 Алгебра не р\yatшается?
 Что Георгій?
Incoming Call 14:43
Incoming Call 4:42
Incoming Call 3:49

--- Friday, June 16, 2017 ---

[12:05:29 AM] Маргулис:
Я сначала отправила, потом поняла
 Смешно
 Какой у тебя процент заимствований?
 Это число точно процент?
Missed Call

[12:19:54 AM] Михаил:
Два.

[12:20:30 AM] Маргулис:
Это точно процент, да? К нему не надо дописывать какой-нибудь 0?
 У меня 3
 Это не отношение? Не 300% ?)

[12:21:05 AM] Михаил:
Все въ порядк\yat.

[12:21:10 AM] Маргулис:
Ладно
 Я собой почти довольна
 Ну я даже поняла, что права
 Сначала написала, а потом поняла
 Ну там, наверное, есть все-таки неточность
 В противном пункте
 Но всё хорошо
 Тебе хорошо?

[12:22:24 AM] Михаил:
Я радъ за тебя.
Incoming Call 12:41

[12:56:42 AM] Маргулис:
У меня гениальная идея
 Ты можешь сделать этот лист завтра

[1:35:13 AM] Маргулис:
Я дура
 Моя задача по дискре решалась не перебором

[5:50:45 PM] Маргулис:
У тебя есть оценка по курсовой?

[8:24:31 PM] Михаил:
Пока н\yatтъ. Въ томъ, что я загрузилъ, было, что исправить, и итоговый варіантъ я отправлю научнику сегодня.
 Ты гд\yat, какъ квартира?

[9:01:15 PM] Маргулис:
Я ушла
 Научнику?
 Ты же не перегрузишь ничего в лмс
 Я гуляю
 Просто чтобы он посмотрел?

[9:05:52 PM] Михаил:
Онъ сказалъ, что оц\yatнитъ то, что я пришлю ему.

[10:42:34 PM] Маргулис:
Ну как, прислал?
 Ты не хуже меня дедлайнишь

[10:45:02 PM] Михаил:
Я думалъ, что онъ одобрилъ то, что я отправилъ въ LMS, а онъ, оказывается, забылъ мн\yat отв\yatтить.

[10:48:25 PM] Маргулис:
Смешно
 Можно так Лизу напугать
 Она бы расстроилась и стала думать, что ей плохую оценку поставят
 И что тиморин просто забыл ответить

--- Saturday, June 17, 2017 ---

[12:17:55 AM] Маргулис:
За время, которое я была под дверью, я могла дождаться принимающего
 У меня нет оценки

[12:19:45 AM] Михаил:
Сколько ты ждала? Не пошла къ деду?

[12:20:47 AM] Маргулис:
1,5 часа
 Обидно
 Могла бы дождаться и спокойно сдать ему задачу
 Ну он все равно ещё ничего не написал и не поставил
 Поэтому я думаю, что ничего страшного, что я завтра напишу подробнее про задачу

[12:23:15 AM] Михаил:
Ну да. Почему дверь не открывали?

[12:24:15 AM] Маргулис:
А я не сказала ещё?
 А почему во множественном?
 Я же сказала, что отец один
 А мать на работе
 К нему друг пришел, они сидели на балконе и курили, дверь на балкон закрыта, окна открыты
 И ничего из квартиры не слышно

[12:25:50 AM] Михаил:
Понятно

[12:26:00 AM] Маргулис:
Это ужасно
 Причём к нему никогда неожиданно никто не приходил

[12:27:51 AM] Михаил:
Я помоюсь и лягу.

[12:28:36 AM] Маргулис:
Ок
 У тебя есть оценка?
 По курсовой?

[12:40:12 AM] Михаил:
Н\yatтъ оц\yatнки.

[9:01:26 PM] Маргулис:
Миш
 Ты тут?

[9:03:22 PM] Михаил:
Оц\yatнки н\yatтъ.

[9:03:57 PM] Маргулис:
Слава богу, что не только у меня
 Успокой меня
 Скажи, что все нормально

[9:06:43 PM] Михаил:
Все нормально, въ самомъ д\yatл\yat. Ты усп\yatешь сдать, что нужно. И экзамены хорошо напишешь.
 По курсовой нав\yatрняка хорошая оц\yatнка будетъ.

[9:10:30 PM] Маргулис:
Спасибо

[10:09:37 PM] Маргулис:
Все, алгебра готова

[10:10:06 PM] Михаил:
Прекрасно.

[10:17:00 PM] Маргулис:
[Photo]
 Он прекрасен

[10:19:11 PM] Михаил:
Да, очень красивая птица.

[10:44:27 PM] Маргулис:
Я что-то ничего не поняла
 Короче, у меня стоит довольно высокий накоп, и за листки ничего нет. Я точно помню, что когда оценка за листки уже стояла у других, у меня ещё Коршунов не проставил её. Но я не понимаю, он мне исправил накоп до или после того, как Коршунов проставил последний лист
 В этом модуле было 2 контры и 1 идз?
 Видимо, он проставил все 1 за 1 модуль
 За 3 модуль

[11:14:23 PM] Михаил:
> Маргулис
> В этом модуле было 2 контры и 1 идз?
Я не помню.

--- Sunday, June 18, 2017 ---

[12:50:18 AM] Маргулис:
Слушай
 А почему ты решил, что Кустурица русский? Что русского в его фамилии? 
 
 
 
 
 
 
 
Он же не Иванов
 Хотя печорин говорил, что знает многих Ивановых — чистокровных немцев

[1:06:52 AM] Михаил:
Мн\yat казалось, что я слышалъ, что онъ - россійскій режиссеръ. Конечно, фамилія не очень русская.
edited 
Лучше не вспоминай, что я говорилъ теб\yat раньше.

[1:07:30 AM] Маргулис:
В общем, он румын вроде

[1:07:44 AM] Михаил:
Я согласенъ.
 Я легъ.

[1:09:23 AM] Маргулис:
Хорошо
 Спокойной

[5:45:55 PM] Маргулис:
Нет оценки
 А у тебя?

[10:20:05 PM] Маргулис:
Миш
 Ты не отписывался от англа, да?
 Я завтра подъеду к вам перед англом
 Подойдёшь со мной к англичанам?

[10:30:53 PM] Михаил:
Не отписывался, подойду.
 [Photo]
 [Photo]
 Это сашино. Я только началъ это читать.

[10:58:56 PM] Маргулис:
А оценка появилась?

[11:00:10 PM] Михаил:
Н\yatтъ.

--- Monday, June 19, 2017 ---

[6:57:19 AM] Маргулис:
http://nabokov.niv.ru/nabokov/stihi/218.htm
Лилит
[Photo]

[4:19:09 PM] Маргулис:
Чего там с сашиным текстом? Тебе всё нравится? Я ещё не смотрела
 Отправь мне утренники

[4:33:30 PM] Михаил:
[Photo]
 [Photo]
 [Photo]
 [Photo]
 [Photo]
 [Photo]
 [Photo]
 [Photo]
 [Photo]
 [Photo]
 У Саши номеръ пять сд\yatланъ только для случая невырожденныхъ матрицъ, остальное хорошо.
 Хотя н\yatтъ, номеръ три я не понялъ.

[5:23:02 PM] Маргулис:
Хорошо
 Ты на консультации был?

[5:28:09 PM] Михаил:
Н\yatтъ.

[5:44:07 PM] Маргулис:
Миш
 Можно вопрос?

[5:48:42 PM] Михаил:
Давай.

[5:52:03 PM] Маргулис:
Тебе было бы приятно поцеловать меня?

[5:52:34 PM] Михаил:
Да
edited 
[5:53:43 PM] Маргулис:
А Диме не нравится

[5:54:50 PM] Михаил:
Нав\yatрное, нуженъ навыкъ, который приходитъ съ практикой, но я не знаю.
 Можешь видео на ютуб\yat посмотр\yatть объ этомъ.

[5:55:51 PM] Маргулис:
Да я не о том, он мне не дал себя поцеловать, сказал, что и с первой любовью ему не нравилось

[5:56:11 PM] Михаил:
Странно
 Непонятно. Онъ какъ-то объяснилъ?

[5:57:01 PM] Маргулис:
Нет

[5:58:29 PM] Михаил:
Хот\yatть секса и не хот\yatть цаловаться. Феноменъ какой-то.

[5:58:46 PM] Маргулис:
Как экзамен?

[5:59:04 PM] Михаил:
Простой

[5:59:46 PM] Маргулис:
Я зря отписалась, да?
 Не ожидала, что будут автоматы по устной части.

[6:02:40 PM] Михаил:
Я стараюсь поменьше думать о систем\yat, по которой организованъ Англійскій языкъ. Вс\yat мало понимаютъ. Ты отписалась совс\yatмъ?

[6:03:04 PM] Маргулис:
Нет
 Только от экзамена

[6:03:52 PM] Михаил:
Ничего страшного. Такъ себ\yat удовольствіе.

[6:17:02 PM] Маргулис:
Он однажды в Чехии был с проституткой.
 Прошлым летом

[6:18:10 PM] Михаил:
Это не былъ его первый сексъ?

[6:18:18 PM] Маргулис:
Что это?
 Мы вчера были в лесу
 Ты предлагаешь заняться сексом в лесу?

[6:19:27 PM] Михаил:
Я спросилъ, былъ ли у него сексъ до проститутки?

[6:19:32 PM] Маргулис:
А
 Вроде нет

[6:20:42 PM] Михаил:
У него къ этому д\yatлу не такое отношеніе, какъ у тебя.

[6:20:54 PM] Маргулис:
Знаю

[8:40:25 PM] Маргулис:
Я забыла площадь гиперболического n-угольника через углы
 Кажется, вспомнила
 Но ты напиши

[9:39:53 PM] Маргулис:
И вообще, напиши, когда можешь позвонить
 И приходи не впритык к 10
 Я тоже приду пораньше

[10:09:50 PM] Михаил:
Черезъ полчаса буду дома.

[10:10:11 PM] Маргулис:
Ок

[10:10:13 PM] Михаил:
> Маргулис
> И приходи не впритык к 10
Конечно, я люблю приходить заран\yatе.
> Маргулис
> Но ты напиши
Пи на эн минусъ два минусъ сумма угловъ, думаю.

[10:13:14 PM] Маргулис:
Ну в тот раз ты не пришёл заранее)

[10:48:39 PM] Маргулис:
И возьми свои конспекты

[10:49:02 PM] Михаил:
Да, возьму.

[10:53:44 PM] Маргулис:
У меня нет оценки за курсовую

[10:55:12 PM] Михаил:
Пров\yatрилъ только что, и у меня н\yatтъ.
 Оказывается, тамъ ять.
 [Photo]

[11:38:49 PM] Маргулис:
Мне кажется, уже бесполезно готовиться
 Надо отдаться течению
 Ну я готовилась сегодня, я имею в виду
 И надо остановится, потому что это одни сплошные нервы

[11:40:33 PM] Михаил:
Я радуюсь и см\yatюсь, когда мн\yat что-то непонятно.
 Особенно см\yatшная формула Стирлинга.

--- Tuesday, June 20, 2017 ---
Incoming Call 43:34

[12:31:51 AM] Маргулис:
Я не хочу больше заниматься математикой
 Но я не знаю, куда свалить

[12:34:31 AM] Михаил:
У Короленко въ очерк\yat "Парадоксъ" есть фраза "Челов\yatкъ созданъ для счастья, какъ птица для полета". Я прочиталъ это въ седьмомъ класс\yat и если не пов\yatрилъ, то принялъ это.
Incoming Call 7:42

[8:48:33 AM] Маргулис:
Миш
 Я приеду где-то к 9:30 — 9:45, не могу рассчитать время
 Если есть возможность, займи место
 Лучше на среднем ряду, он там людей не дёргает
 Ближе к концу
 Только не тормози, займи место по-человечески
 Мне просто тяжело быстро идти из-за тех следов от комаров

[3:08:20 PM] Маргулис:
Forwarded message: 
Полина Барон [6/11/17] 
https://pp.userapi.com/c840125/v840125532/5021/C8yZDvhIXOI.jpg 
Вариант значит вот
 [Photo]

[3:08:24 PM] Маргулис:
Типа вот
 Типа что-то такое должно быть на экзамене

[4:10:01 PM] Маргулис:
Посмотри на обхват лодыжки
 [Photo]
edited 
Это не пятка, это так распухла нога

[5:24:15 PM] Маргулис:
Заценил?

[5:24:29 PM] Михаил:
Да, жутко.

[5:24:53 PM] Маргулис:
Видишь на левой ноге пятно?
 В левой части
 Это шрам, который обычно в самом глубоком месте
 А теперь он на горке

[5:25:22 PM] Михаил:
Могутъ таблетки отъ аллергіи помочь?

[5:25:58 PM] Маргулис:
Ты называешь реакцию на крапиву и комаров аллергией?
 Ну это же неправильно
 Это же эффект от яда
 А не аутоиммунное

[5:26:48 PM] Михаил:
На ядъ можетъ быть и аллергія.

[5:27:07 PM] Маргулис:
Ну можно об этом порассуждать
 Но на яд не может не быть реакции

[5:27:31 PM] Михаил:
У тебя очень сильно. Это ненормально.

[5:27:46 PM] Маргулис:
Думаешь?
 Могу выпить пару таблеток от аллергии

[5:28:47 PM] Михаил:
Это в\yatдь отекъ.

[5:28:51 PM] Маргулис:
Да
 Это отёк
 Ну ё-то ставь, за что ты её игноришь?

[5:29:48 PM] Михаил:
Ея у меня н\yatтъ на клавіатур\yat.

[5:29:55 PM] Маргулис:
Гм
 Значит, это клавиатура времён Карамзина и Дашковой
 И надо писать йо
edited 
[5:31:06 PM] Михаил:
От'окъ.

[5:32:03 PM] Маргулис:
Так хорошо

[10:43:18 PM] Михаил:
Какъ ноги?

[10:55:31 PM] Маргулис:
Хреново
 Очень

[11:09:15 PM] Михаил:
Прими таблетку отъ аллергіи, если не приняла.
 Я легъ.

[11:34:42 PM] Маргулис:
Мне надо мне уже об этом говорить
 Я вроде бы тебе не пишу сейчас

--- Wednesday, June 21, 2017 ---

[2:40:44 AM] Маргулис:
Ты сам виноват, что мы мало очень говорим
 Ты испортил отношения, а не я
 Я больше не хочу с тобой ничем делиться из своей внутренней жизни, а тебе делиться, по-видимому, нечем, плюс ты почему-то не хочешь этого делать, и ты, между прочим, опять стал слишком явно мерзким маленьким эгоцентричным мальчиком.
 И если мне периодически надо с кем-то поделиться чем-то про свои отношения с крылом, то можно было бы потерпеть, если я хоть что-то для тебя значу
 Не отвечай утром, не порть настроение. Лично обсудим.

[3:38:27 AM] Маргулис:
Просто я трачу на тебя уйму нервов
 Непонятно только, зачем
 Я объясню, в чём проблема, не унижая тебя и не говоря, что я супер
 Я не вижу в твоём мышлении свободы
 Ты мыслишь настолько в рамках данной системы, что это поражает, твоё мышление находится на уровне твоего я, которое в свою очередь окружено чем-то данным, и это данное неизменно
 В твоём мышлении не происходит за несколько секунд построение хотя бы куска системы, потому что ты берёшь готовый кусок
 Как можно не мочь абстрагироваться от данности?
 Я чувствую, что что-то очевидно, когда ты меня совсем не понимаешь и считаешь, наверное, что я говорю глупости
 Мне кажется, ты не готов за секунду пересмотреть возможную трактовку понятия человечность, например
 Ты, кажется, не готов меня сразу понять, когда я говорю, что христианство как учение крайне не человечно
 И тебе именно для этого нужна психология, потому что она поддерживает систему
 Хотя люди слишком легко расширяют понятие человеческой природы
 Чтобы об этой природе судить и судить по этой данности о возможности или невозможности изменения системы
 А насчёт геноцида, кстати,— ну в чем вопрос-то, можно ждать, когда люди изменятся, и пытаться на это влиять, а можно убрать всех неподходящих и оставить способных на это развитие. Если ты устроил геноцид, ты, конечно, тоже неподходящий, но при любых попытках влиять на развитие общества ты выйдешь за рамки своего представления об идеале и станешь неподходящим, себя убирать по-любому, и ещё не факт, что твои действия будут в твоей картине мира расценены чем-то лучшим, чем убийство. Например, поступаешь ты по закону, а по закону какой-то негодяй получает пожизненный, а заключать человека в такие условия и лишать свободы гораздо хуже убийства. Но я против законодательного введения смертной казни, да, потому что есть много других причин быть против,
> Маргулис
> А насчёт геноцида, кстати,— ну в чем вопрос-то, можно ждать, ког
Ну конечно, я не сделала бы это
> Маргулис
> Я не вижу в твоём мышлении свободы
Кажется, это легко меняется

[10:22:10 PM] Маргулис:
Почему ты не ответил?
 Я не знаю, во сколько завтра экзамен
 Нашла
 Ну и?
 Ты теперь молчишь?

[10:42:01 PM] Михаил:
Про себя?

[10:42:08 PM] Маргулис:
Да
 Ты обиделся?

[10:42:29 PM] Михаил:
Не хочу.
 Я не хочу говорить о себе, я и так это слишком много делал.

[10:43:23 PM] Маргулис:
> Маргулис
> Ты обиделся?
Это ответ да?

[10:43:46 PM] Михаил:
Нет, я не обиделся.

[10:44:15 PM] Маргулис:
Ты не хочешь разговаривать больше?

[10:44:42 PM] Михаил:
Хочу, но не о себе.

[10:45:39 PM] Маргулис:
Ладно
Incoming Call 25:36

--- Thursday, June 22, 2017 ---

[9:38:55 AM] Маргулис:
Миш, у тебя получилась задача про монеты? Есть идеи?

[2:43:18 PM] Маргулис:
Почему ты не поднялся хотя бы чтобы позвонить?

[5:20:42 PM] Маргулис:
Миш, ты не посмотришь, почему первые два интеграла в 12 листе сходятся, если их не считать?

[5:40:10 PM] Маргулис:
Пожалуйста
 Просто надо написать 14 лист
 И алгебру выучить

[5:55:13 PM] Михаил:
На $[0, \pi/2] x/2 < sin x < x$. Логарифмируем, интергируем. Из сходимости интеграла от 0 до $\pi/2 ln(x)$ (в которой можно убедиться) следует наша сходимость.

[9:09:50 PM] Маргулис:
Я спала днём
 Кое-что плохое произошло
 Короче, мне очень плохо
 Я не знаю, как завтра буду сдавать
 Я просто ложусь спать, если мне грустно
 Ты извини, что я завтра буду сдавать как обычно
 Я правда хотела, чтобы было по-другому

[9:52:30 PM] Маргулис:
Я ненавижу, когда ты не отвечаешь
 Можно твой 14 лист?

[10:02:07 PM] Михаил:
[Photo]
 [Photo]
 [Photo]
 [Photo]
 [Photo]
 [Photo]
 [Photo]
 [Photo]

[10:06:38 PM] Маргулис:
Я не очень поняла, какие где номера
 И ты опять фотографируешь криво
 И опять не отвечаешь словами
 Мог бы спросить, что случилось

[10:07:15 PM] Михаил:
Я пронумеровал некоторые страницы.

[10:07:25 PM] Маргулис:
Потому что случилось и я теперь не хочу выходить из дома
 Ну ладно, не хочешь — как хочешь

[10:10:01 PM] Михаил:
Ты готова это мне рассказать? Я не очень уверен, что могу спрашивать.

[10:10:08 PM] Маргулис:
Нет, не готова
 Просто это вежливо
 Спросить
 Нужно прийти, да?
 Завтра
 На экзамен

[10:10:51 PM] Михаил:
Конечно
 Иначе хуже

[10:11:10 PM] Маргулис:
Ну да
 Слишком сильно опустится средний балл иначе
 Чертова дискра
edited 
Как же плохо я её написала, это ужасно. У меня несколько дней был на руках лист с одним из вариантов, а я не порешала ни графы оттуда, ни рекуррентное соотношение, и не вспомнила, что делать, когда есть кратные корни
 Ты готов к алгебре?
 А 1 задачи из 14 нет?

[10:14:02 PM] Михаил:
Да, около того. Задачи нет.

[10:14:36 PM] Маргулис:
И шея безумно болит, потому что я ударилась головой
 Надо было постараться, чтобы так вышло, но у меня вышло и абсолютно случайно

[10:15:25 PM] Михаил:
Мне тебя жалко.
 Завтра все будет нормально. Ты определения запомни, а там как-нибудь.
 Есть простые билеты.
 Я буду рядом.

[10:17:51 PM] Маргулис:
Слушай
 А ты смог пройти по ссылке на 4 книгу?

[10:18:19 PM] Михаил:
Я не пробовал.
> Маргулис
> Слушай
Я Черного человека вспоминаю всякий раз.

[10:28:52 PM] Маргулис:
Да

[10:56:26 PM] Маргулис:
Что такое баковые курсы и какие ты хотел бы взять?
 Базовые
 Не знаю, почему корректор исправляет на баковые

[10:57:36 PM] Михаил:
Интуитивно я понимаю, что это.
 Чего выбрать, не знаю.

[10:58:24 PM] Маргулис:
А там если тыкнешь, убрать совсем уже нельзя? Только заменять?
edited 
[11:00:02 PM] Михаил:
Нажать клавишу backspace или delete.

[11:00:48 PM] Маргулис:
А, ну я просто с планшета
 Ну ладно
 Шехтман будет вести логику

--- Friday, June 23, 2017 ---

[12:23:38 AM] Маргулис:
А знаешь, тот один балл в прошлой контре влияет на мою оценку
edited 
Так мне для 8 нужно 5 баллов
 А было бы нужно 4

[5:26:06 PM] Маргулис:
Меня проконсультировали по вопросам нисов и базовых курсов
edited 
[6:58:35 PM] Маргулис:
Я боюсь спросить, ты искренне считаешь, что пишется "занулина"?

[8:32:31 PM] Маргулис:
Я расстроилась из-за Жени
 Не люблю, когда другие люди так делают
 Ты сказал Саше про физфак?

[9:37:36 PM] Михаил:
> Маргулис
> Я боюсь спросить, ты искренне считаешь, что пишется "занулина"?
Не понял, о чем ты.
> Маргулис
> Ты сказал Саше про физфак?
Не помню. Может быть, но не уверен.

[9:41:31 PM] Маргулис:
Ну у тебя так написано в тексте
 Производная занулина

[10:01:19 PM] Маргулис:
И не говори про меня, что я тоже делаю порезы
 Это было три раза в жизни и совсем чуть-чуть, я только пробовала
 А вообще, это омерзительно

[10:11:25 PM] Михаил:
> Маргулис
> Производная занулина
В том, что я прислал? Где именно?

[10:12:07 PM] Маргулис:
Аааа
 Это не занулится
 Ну тогда ладно
 Кстати, зря ты пишешь чернилами, слишком крупный почерк и нет разницы в нажиме
 Бумагу вот тратишь
 Я, кстати, теперь из принципа не беру листовки
 Потому что нельзя тратить бумагу на такое

[10:14:19 PM] Михаил:
Неплохая мысль.

[10:14:29 PM] Маргулис:
> Маргулис
> Бумагу вот тратишь
В смысле, очень много места занимают пять строчек
 Хотя жалко раздающих листовки
 Но я держусь

[10:15:08 PM] Михаил:
Иначе неразборчиво.

[10:15:20 PM] Маргулис:
По-разному бывает
 У разных людей
 У меня неразборчиво, да
 Потому что спазм от мелких букв
 А ещё я сегодня сидела в метро с людьми, которые говорили, что хорошо сдали ЕГЭ, и тут же сказали, что это 50-60 баллов
 А ещё они мерзкие расисты, ну да ладно

[10:18:05 PM] Михаил:
Это всего лишь люди.

[10:18:06 PM] Маргулис:
Нынешние выпускники
 Ну да
> Маргулис
> Нынешние выпускники
Я не к тому, что они все такие
 Просто объяснила возраст

[10:18:52 PM] Михаил:
Это понятно

--- Saturday, June 24, 2017 ---

[3:15:14 AM] Маргулис:
Это смешно, но я звонила не тебе ночью, промахнулась один раз мимо Димы в тебя и стала перезванивать. Извини
 В последних вызовах, не в списке контактов.
 Миш
 Слушай

[3:29:02 AM] Маргулис:
Forwarded message: 
Каган Александр [Fri] 
Я вообще говоря не был точно уверен. Я судил по Мишиному вопросу вк про физфак. Поскольку он сказал, что точно не уходит, логическая цепочка привела меня к тебе)

[3:29:06 AM] Маргулис:
О чем он?

[5:54:27 AM] Михаил:
Я сделал такой ход, что задал во Вконтакте на стене запись с вопросом, кто идет на физфак и просьбой написать мне об этом. Цель была, чтобы люди рассказали мне, почему они туда идут, и я рассказал бы это тебе. Через полдня я подумал, что это выглядит странно и удалил запись.

[4:57:11 PM] Маргулис:
Ясно
 Вопрос для знакомых 11 классов?
 Как у них дела, кстати?

[5:34:16 PM] Михаил:
Не знаю. Они и существуют не совсем. Я был с Аленой, одноклассником и человеком из параллельного класса. Мы сидели в школе до трех, потом два часа шли к Алене домой. Пришли, поспали, разъехались.

[5:43:11 PM] Маргулис:
Ты о чем и когда это было? Я ничего не поняла
 Ты о какой Алёне?
 Я всех забыла
 Если бы я работала в Арзамасе, то приложение было бы великолепно
 Кстати
 Я вот очень люблю южную Америку
edited 
И тут вспомнила, что там один учёный придумал метод поиска преступников по отпечаткам пальцев
 https://m.rupoem.ru/poets/zabolockij/zakinuv-na-spinu

m.rupoem.ru
Бродячие музыканты. Николай Заболоцкий
Читать стихотворение Николая Заболоцкого «Бродячие музыканты», написанное в 1928 году. Закинув на спину трубу, Как бремя золотое, Он шел, в обиде на судьбу. За ним бежали двое.
 [Photo]
 [Photo]
 [Photo]
 [Photo]

[7:37:48 PM] Маргулис:
[Photo]
edited 
Это к фотографии моих волос

[7:44:47 PM] Михаил:
> Маргулис
> Ты о чем и когда это было? Я ничего не поняла
Выпускной в 179 ночью с пятницы на субботу.
 Алена Потапенко.
 У нее кот.

[7:53:32 PM] Маргулис:
> Михаил Зыбин
> Выпускной в 179 ночью с пятницы на субботу.
Какой ночью? Вчера?
 Я запуталась
 Ты не говорил про это
edited 
Вчера был экзамен по алгебре?
 Я ничего не помню

[7:54:24 PM] Михаил:
В самом деле, не говорил.
 Европа?

[7:55:12 PM] Маргулис:
Вчера
 Какая-то опечатка

[7:55:40 PM] Михаил:
Вчера

[7:55:48 PM] Маргулис:
Я запуталась
 С датами
 Сегодня суббота, да?

[7:56:03 PM] Михаил:
Да

[7:56:05 PM] Маргулис:
Я не шучу
 Я просто потерялась
 У меня стресс после вчерашнего
 С базовыми курсами будет конфликт расписания, но мне все равно советуют коммутативную алгебру

[7:57:36 PM] Михаил:
О, да, что насчет ИУПа?

[7:58:13 PM] Маргулис:
Я добавила пока её, потому что пока в учебной части нет человека, который сказал бы мне, что он не попадёт на день военки второго курса
 Такое может быть
 Мне там же сказали
 Потому что расчёт на третий курс идёт
 Рыбников со Смирновым сказали мне всё попробовать из нисов
 Пока Смирнов сказал, что я могу что-то понять на его нисе
 Поэтому я выбрала его
 Про алгебраическую и выпуклую геометрию, я ненормальная, зачем мне это
 Ну я думаю, я смогу полюбить это
 Геометрию
 Нет причин её не любить

[8:00:46 PM] Михаил:
А какие базовые курсы?

[8:00:50 PM] Маргулис:
Ну там всё можно будет изменить
 Ну коммутативную алгебру
 И алгебру ли советовали, но они каждый весят 5 кредитов на полгода
 Ещё там есть топология 1 весной
 Но та же проблема
 Лучше взять по два ниса, я думаю
 Мне сказали, что нис теория представлений нельзя брать на втором курсе, слишком тяжёлый
 Геометрию и динамику, кстати, сказали попробовать
 Ну в общем я в сентябре похожу и решу

[8:03:47 PM] Михаил:
Я понял, это разумно

[8:03:48 PM] Маргулис:
А базовые курсы все конфликтуют с расписанием, на самом деле
 Если не в день военки

[8:04:17 PM] Михаил:
Расписанием обязательных курсов?

[8:04:21 PM] Маргулис:
Да
 Они для третьего курса

[8:04:55 PM] Михаил:
Ясно

[8:05:54 PM] Маргулис:
Я думала просто, не взять ли программирование, но он как раз базовый
 И ещё думала, не сходить ли на что-то про логику

[8:06:46 PM] Михаил:
Я не думал

[8:07:08 PM] Маргулис:
Потому что единственная проблема в моем отношении к ней — это данияр
 Мне он не нравится и он ходит в церковь на службу
 Второе уточнение для тебя
 В церковь
 На службу
 Представь
 И его зовут данияр салкарбекович
edited 
И он ходит в православную церковь

[8:09:31 PM] Михаил:
Ему правдоподобнее быть буддистом?

[8:09:35 PM] Маргулис:
Да
 Мне так кажется

[8:10:56 PM] Михаил:
Он казах. На самом деле, там больше ислам распространен.

[8:14:04 PM] Маргулис:
Ну ладно
 А откуда ты знаешь, что он казах?

[8:14:34 PM] Михаил:
Похож

[8:14:38 PM] Маргулис:
А ты что выбрал?

[8:14:48 PM] Михаил:
Ничего еще

[8:17:31 PM] Маргулис:
Я изменила сейчас, уберу пока коммутативную алгебру
 Хотя не знаю
 Ничего не знаю
 Нет, мне нельзя делать конфликт расписания, наверняка это все на день майнора придётся

[8:18:50 PM] Михаил:
Базовые курсы тоже менять можно, как и НИСы?

[8:19:43 PM] Маргулис:
Думаю, да
 Все можно менять
 Из курсов по выбору
edited 
Ну отказаться точно дожно быть можно
 Оставила Смирнова пока
edited 
И запишу геометрию с динамикой, все равно там не будет геометрии и динамики как таковых
 Как я делаю опечатки, интересно
 Потом всё изменю
 В учебной части тоже сказали, что все можно менять
> Маргулис
> Ну отказаться точно дожно быть можно
Что я хотела сказать

[8:22:53 PM] Михаил:
Должно быть можно

[8:22:59 PM] Маргулис:
Я уже поняла

[9:09:59 PM] Маргулис:
Ты будешь ходить в независимый?

[9:11:13 PM] Михаил:
Не думал об этом

[9:11:30 PM] Маргулис:
Рыбников расстроился, что я там не была, сказал, что зря

[9:11:55 PM] Михаил:
Мне нравится там запах

[9:39:03 PM] Маргулис:
Интересно, зря ли я зачеркнула в дискре пункт б совсем в задаче про постдоков
 Если тебе что-то за него поставят, что очень зря
 А скорее всего поставят
 Ой, а ведь у меня есть ещё ошибка
 В дискре
 Там все плохо
 А может и нет
 Не помню

[9:53:33 PM] Михаил:
Не стоит волноваться, пока нет результатов. Я слышал, они немного начали появляться.

[9:59:46 PM] Маргулис:
Начали
 Почти у всех есть
 Кроме нас
 Там за каждый подпункт по 4 балла
 У меня будет баллов 9
 Я поняла, что у меня глупый ответ в 3
 Не тот знаменатель

[10:02:24 PM] Михаил:
Сейчас ничего не изменить. Или, может, что-то отапеллируешь.

[10:02:32 PM] Маргулис:
Ну вряд ли
 Я плохо написала
 И спросить со скопенковым невозможно

[10:03:30 PM] Михаил:
Не волнуйся, так или иначе.

[10:04:09 PM] Маргулис:
Ну 0 мне не поставят, остальное как-нибудь переживу
edited 
Я думаю, 0 — это маловато

--- Sunday, June 25, 2017 ---
edited 
[4:41:18 PM] Маргулис:
Почему ты удивился, что я никогда в жизни ничего не пила?
 По какой причине у тебя были об этом другие представления?
 Какой именно стереотип ты примерил на меня?
 Это обидно

[5:03:17 PM] Михаил:
Что всякий к восемнадцати годам пробовал спиртное.
edited 
[5:22:46 PM] Михаил:
Я написал одному проверяющему, он сказал, что Скопенков заметил, что у нас списана третья или четвертая задача и хочет обнулить работы. Баллы у тебя 4 1 0 0 1 0 1, мои 4 2 4 0 0 0 1.

[5:26:57 PM] Маргулис:
Как у нас может быть списана третья и четвёртая задача, если они не об одном и том же?
 И я сначала написала свою первую часть, потом только твою дисперсию посмотрела
 А кому ты написал?
> Маргулис
> Как у нас может быть списана третья и четвёртая задача, если они
Я о третьей
 Я не видела твою третью в принципе

[5:36:42 PM] Михаил:
Евгению Гончарову.
 Я уже не помню условий.

[5:37:37 PM] Маргулис:
У тебя разве была сумма 45 в третьей?

[5:37:58 PM] Михаил:
Сумма чего?

[5:38:09 PM] Маргулис:
Сумма трёх чисел
 20 31 и 42
 Ну, это их максимумы

[5:38:38 PM] Михаил:
Значит, речь о четвертой задаче.
 Она про что?

[5:38:54 PM] Маргулис:
Про дисперсию
 Какого числа показ работ?

[5:41:48 PM] Михаил:
Завтра в два.

[5:42:17 PM] Маргулис:
Я только дисперсию переписала, а остальное у меня к тому моменту было записано уже
 Прости

[5:43:36 PM] Михаил:
Ничего, это со мной бывало.

[5:43:47 PM] Маргулис:
Когда?
 Я приеду завтра

[5:44:18 PM] Михаил:
В первой моей школе. Я тоже приеду, видимо.

[5:47:27 PM] Маргулис:
А он уже обнулил или как?
 Или с ним можно поговорить?

[5:48:08 PM] Михаил:
Евгений Гончаров просит Скопенкова этого не делать.

[5:48:35 PM] Маргулис:
А почему ты именно ему написал?

[5:49:00 PM] Михаил:
Миша Плахов посоветовал.

[5:58:56 PM] Маргулис:
Прости
 Надо было одну строчку не полениться досчитать
 Не могу отвлечься теперь
 Прости
 Так со всеми, кому ещё не выставили оценки, что ли?
 Может, написать скопенкову?
 Миша
Missed Call

[6:30:43 PM] Маргулис:
Миша, не уходи
 Я уйду с матфака, если будет тройка в аттестате
 Я прошу тебя, ответь
 Умоляю тебя, не уходи
 А ты изначально Гончарова спросил про нас двоих или он сам тебе сказал?

[7:36:02 PM] Михаил:
Про двоих.

[7:44:32 PM] Маргулис:
Больше не буду сидеть с тобой никогда
 Прости меня
 Если что, попробую хотя бы попросить, чтобы он только мне поставил 0, и уйду с матфака

[8:38:44 PM] Маргулис:
Это всё, что тебе сказал Гончаров?

[8:39:06 PM] Михаил:
Да, все.

[8:39:40 PM] Маргулис:
А я сижу в очках
 В них мои глаза выглядят ещё меньше

[8:40:19 PM] Михаил:
Они же большие.

[8:41:00 PM] Маргулис:
Мне классе в 7 одноклассница сказала, что маленькие
 А они не растут
 Мне страшно

[8:43:00 PM] Михаил:
Это загадочно, потому что я действительно считаю твои глаза больше, чем средние.

[8:43:18 PM] Маргулис:
Ну офтальмологи тоже
 Но они такими не выглядят из-за узости
 Запах Лизы — это духи Лизы?
 Ты обращаешь внимание всего лишь на шампунь и духи?

[8:47:13 PM] Михаил:
Я не знаю, что это.

[8:49:20 PM] Маргулис:
В смысле?
 Не понимаю, почему у меня такая оценка по геоме

[8:50:07 PM] Михаил:
Не знаю, запах чего это.

[8:51:28 PM] Маргулис:
Опять ехать туда...

[8:52:04 PM] Михаил:
Ой, да, ты нехорошо написала. Я посмотрел результаты сейчас.

[8:52:31 PM] Маргулис:
Ну просто там плюс, кажется, только за кватернионы
 А задача про решётку правильная вроде

[8:53:49 PM] Михаил:
Может быть, отапеллируешь чего-нибудь. Не грусти.

[8:54:10 PM] Маргулис:
Скажи, о чем были задачи

[8:54:31 PM] Михаил:
Я не помню

[8:54:34 PM] Маргулис:
Я не помню одну
 А
 Вспомнила
 То есть это неправильно про окружность в верхней полуплоскости?

[8:55:58 PM] Михаил:
Правильно, но, может, ты криво оформила.

[8:56:37 PM] Маргулис:
Не знаю
 Может, повыше поставит шварцман

[8:58:48 PM] Михаил:
Это правда, он такой.

[8:59:38 PM] Маргулис:
Ну у меня сейчас 7 получается

[9:00:16 PM] Михаил:
Какой накоп?

[9:01:22 PM] Маргулис:
Я не знаю точно
 Вроде не меньше 8

[9:18:33 PM] Маргулис:
Очень хочется пошутить, что меня проверял скопенков

[9:20:12 PM] Михаил:
Ты предполагала что-то около десяти?

[9:26:01 PM] Маргулис:
Нет
 Я предполагала, что мне первые три задачи поставят
 И последнюю, может быть
 Ну что-то типа 7
 Или 8, но это неважно, у меня все равно тогда была бы 8
 По крайней мере я не вижу причин снижать мне за решётки
 Ну в декабре мне подняли сильно достаточно
 Так что про решётки я его, может быть, смогу убедить
 Ну я давно поняла, что этот семестр неудачный, черт с геометрией, меня волнует дискра
 Миш, с сколько у тебя была дина одного из векторов решётки?
 Примерно

[9:34:59 PM] Михаил:
Корень из трех и 1.48

[9:35:51 PM] Маргулис:
А у меня был 1,46 с чем-то, это тоже в квадрате больше 2
 Я просто без твоего числа не была уверена, как мое выглядело
 Странно
 Ну ладно
 Боже мой, неужели я забыла исправить косинус на синус
 Нет, не забыла

[9:37:45 PM] Михаил:
Ты мне это сказала во время работы.

[9:38:05 PM] Маргулис:
Ну а вдруг я тебе сказала, а у себя не исправила
 Нет, я написала, что синус меньше корня из трёх на два, значит, угол меньше 60 градусов
 Не забыла
 Ладно, иди готовься

[9:39:44 PM] Михаил:
Готовиться?
 Неееет

[9:39:50 PM] Маргулис:
К матану

[9:39:51 PM] Михаил:
Рано

[9:39:56 PM] Маргулис:
Хи-хи
 А если угол меньше 60 градусов, то проекция больше половины

[9:40:27 PM] Михаил:
Да

[9:40:35 PM] Маргулис:
И решетка плохая
 Гм, может, они не нашли просто? Там вакханалия с оформлением

[9:41:35 PM] Михаил:
Да, скорее всего так.

[9:41:58 PM] Маргулис:
Проекция же должна быть меньше или равна половине, да?

[9:42:14 PM] Михаил:
Да

[9:42:47 PM] Маргулис:
Подожди, а это обязано дважды выполняться что ли?
 Для проекции на каждого на каждый?
 Иди просто большего на меньший?
 А, поняла

[9:43:40 PM] Михаил:
Нет, для проекции на самый короткий

[10:25:20 PM] Маргулис:
Как думаешь, скопенков поставит 0?
 Гончаров не сказал, не даст ли он переписать?

[10:34:07 PM] Михаил:
Не знаю, не сказал.

--- Monday, June 26, 2017 ---

[12:07:38 AM] Маргулис:
Объясни, чем для тебя похожа математика на физику
 Ну она есть в физике, но это не называется похожестью
edited 
Корова на говядину или дерево на бумагу по такому принципу тоже похожи
 Миш
 А что обычно значит фраза "с учётом кратности"? Я никогда не могу её понять не думая. N корней с учётом кратности — это когда все кратные считаются отдельно или как один?
 То есть все одинаковые

[12:45:35 AM] Маргулис:
Мишка
 Мишунь
 Миш
 Кстати, ты не спросил, что у меня с алгеброй

[11:16:01 AM] Михаил:
> Маргулис
> А что обычно значит фраза "с учётом кратности"? Я никогда не мог
Все кратные считаются отдельно.
> Маргулис
> Кстати, ты не спросил, что у меня с алгеброй
Посмотрел оценку в кондуите.

[5:09:07 PM] Михаил:
Не переживай.
 Вообще говоря, банально грустить из-за неприятностей.
 Я, кстати, поговорил с Скопенковым еще, сказал про красный диплом, он ответил, что обсудит с Артамкиным.
 Я только себя виню, это моя слабость, я не умею отказывать.
 Хорошо
 Нет

[5:36:20 PM] Михаил:
Нет
 Я опечалюсь, если и в другом месте тебе не понравится и ты будешь жалеть, что ушла. Если тебе понравится там, куда ты уйдешь, я порадуюсь.

[5:59:50 PM] Михаил:
Полагаю, что да, потому что, может быть, ты и не уйдешь, и от пропущеннаго экзамена будет хуже.
 Порой тебе нравится матфак.
 Какой у тебя накоп?
 И это не соматические проявления невроза?
 Что с тобой?
 О тебе я не говорил.
 Тогда да
 Действительно
 Это непонятно
 Про проблемы со здоровьем?
 Спасибо тебе за все, Маша.
 Я дейсвительно благодарен тебе за след, который ты оставила в моей жизни.
 А меня прости, что я так без спроса вошел в твою и неаккуратно себя вел.

[7:37:46 PM] Михаил:
Не будь такой слабой.
 Я бы сказал, нет.
 Я опустил "что".
 Успокойся, пожалуйста.
 Надо быть достаточно уверенной, что в другом месте не будет хуже.

[8:55:41 PM] Михаил:
Вот это мне сильно не понравилось. 


\end{document}