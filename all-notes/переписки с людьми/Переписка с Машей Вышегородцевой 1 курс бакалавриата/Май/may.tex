\documentclass{article}
\usepackage[utf8]{inputenc}
\usepackage[russian]{babel}
\usepackage{amsmath}
\usepackage{amsfonts}
\usepackage{mathdots}


\usepackage[X2,T2A]{fontenc}



\newcommand{\Yat}{{\fontencoding{X2}\selectfont\CYRYAT}} % Буква Ять ЗАГЛАВНАЯ на рус. клавише «ея»
\newcommand{\yat}{{\fontencoding{X2}\selectfont\cyryat}} % Буква Ять строчная


\begin{document}
\title{}
\author{}
\date{}
\maketitle


--- Monday, May 1, 2017 ---

[12:38:04 AM] Михаил:
> Маргулис
> Я бы не пошла с тобой внутрь
Маша, ты можешь доверять мне больше. Я тебя очень уважаю. Мне нужна твоя душа, она у тебя самая прекрасная на свете. Тебя мне раньше очень не хватало. Благодаря тебе разрушились границы моего маленького мира, и теперь я могу видеть всякие замечательные вещи, которые были недоступны.
 Спокойной тебе ночи.

[1:50:17 AM] Маргулис:
Ты дурачок)
 Я не об этом)
 Я имела в виду, что там опасно и вообще
 Можно встретить маргинальные слои населения
 И что-то может на тебя упасть

[5:59:04 AM] Маргулис:
Тебе никто не разрешал находить мою страницу в Кинопоиске
 И добавлять меня в друзья
 Скорее всего, я удалю страницу сегодня
 Из-за тебя
 Спасибо, удалила
 Я обиделась и тебя блокирую
 У меня к моим фильмам личное отношение, я не хочу, чтобы кто-то видел, что я смотрела и какие оценки считаю нужным ставить, если я сама ему об этом не сказала. Иди нахер. Я больше не хочу с тобой общаться.

[2:05:33 PM] Маргулис:
Из-за тебя я хочу себя убить.
 Я — злобная тварь
 Не лезь ко мне. Я сама предлагаю всё, чего хочу. Никаких самостоятельных действий, это моё личное пространство. Не смей к нему приближаться.
 Не пиши ничего, всё равно не прочитаю.
 Ладно, я так не могу.
 Пиши, если хочешь
 Прости меня.

[2:53:13 PM] Михаил:
Жаль тебя, Маша, что ты так переживаешь. Только не теряй любовь к себе и веру, что можешь стать лучше. Может быть, здесь уместны слова "Полнота человеческой жизни определяется тем, насколько человек сегодня лучше, чем был вчера". Извини, если это звучит глупо. 
Страдание из-за своего несовершенства - это здоровое чувство, но надо и действовать. Извини, если это звучит глупо.

[3:07:26 PM] Маргулис:
Я ничего не поняла, откуда три одинаковых сообщения?
 Да, это полный идиотизм, потому что ты ничего не понимаешь. Ты не понимаешь, что проблема в тебе и что ты должен перестать вести себя как придурок.
 Это моя жизнь, мои интересы и моё личное пространство. Я никого не пускаю в своё личное пространство.
 Зачем ты постоянно меня злишь?
 У меня есть гениальный вид действия—прекратить это дурацкое общение с одной тварью
 Меня так давно никто не злил, пока я тебя не встретила
 Всё было идеально
 Меня, кстати, бесит, что все думают, что мы встречаемся
 Вот объясни, что интересного мне в общении с тобой?
 Ты рассказываешь что-то интересное?
 У тебя полно интересных мыслей?
 Если это так, то я этого не вижу
 Убери, пожалуйста, весь свой пафос и всю свою назидательную интонацию
 Твои чувства ко мне—твоя проблема, то, что ты меня бесишь—твоя проблема, сделай так, чтобы этой проблемы больше не было
 Меня бесишь ты, не ситуация, а лично ты. Ты мне неприятен как человек. С тобой скучно и я постоянно попадаю в неловкие ситуации. Как можно постоянно о меня тереться разными частями тела, когда я просто хочу спросить что-то на ухо? По-твоему, это незаметно?
 Мне это всё противно.

[3:31:11 PM] Михаил:
Я отрекаюсь от тебя.

[10:33:20 PM] Маргулис:
Я просила добавить меня уже кучу времени назад
 А ты сделал это только сейчас
 Все равно ты хороший
 Не отрекайся от меня)


[10:35:45 PM] Михаил:
Эти слова имеют для меня большую силу. Это выстрел в голову моей любви к тебе. Надеюсь, у нее нет головы, и все нормально.

[10:36:10 PM] Маргулис:
Больше не любишь?

[10:37:04 PM] Михаил:
Она выжила, похоже. Я тебя люблю.

[10:45:51 PM] Маргулис:
Ты не знаешь, кагану 18 или 19?

[10:47:06 PM] Михаил:
Не знаю.
 "Эти слова" - это те мои четыре слова.

[10:48:38 PM] Маргулис:
> Михаил Зыбин
> "Эти слова" - это те мои четыре слова.
Я поняла

--- Tuesday, May 2, 2017 ---

[12:07:28 AM] Маргулис:
Я перестала читать стихи

[6:38:16 PM] Маргулис:
Представила, что оторвавшийся у нас над домом кусок провода с порывом ветра попадёт ко мне в окно. Представила, что я толкаю его из окна чем-то металлическим. Представила заголовки газет: человек со 100 баллами по физике не знал, что металл проводит ток, и умер в результате этого. Да, пожалуй, тогда бы в системе ЕГЭ серьёзно усомнились.
 А ещё я таки согласилась на самый неинтеллектуальный и самый действенный способ избавления от депрессии

[7:17:08 PM] Михаил:
Маша, что с тобой? Давай вместе успокоимся, перестанем грустить и страдать, пожалуйста.

[7:54:44 PM] Михаил:
Я испугался оттого, что ты пишешь о самоубийстве.

[10:23:10 PM] Маргулис:
Я не знаю, почему
 Я не убью себя по-любому

[10:39:51 PM] Михаил:
> Маргулис
> Представила, что оторвавшийся у нас над домом кусок провода с по
Ты представляешь, что убьешь себя. К тому же, самоубийство - самый действенный способ избавления от депрессии.

[10:40:36 PM] Маргулис:
Нет
 Я не представляла, что случайно это сделаю
 Я представляла, что так случайно получится
 Я представила себе гипотетическую ситуацию и поняла, что в ней у меня в руке кочерга

[10:42:23 PM] Михаил:
Не пугай меня так.

[10:43:37 PM] Маргулис:
Я стала толстая

[10:44:23 PM] Михаил:
Ты нравишься мне любой.
 Думаю, ты слишком самокритична.
 У тебя очень красивая фигура.

[10:45:24 PM] Маргулис:
Да, я нравлюсь любой, и это удручает

[10:45:48 PM] Михаил:
Нет

[11:15:56 PM] Маргулис:
Худею

--- Wednesday, May 3, 2017 ---

[8:18:17 PM] Маргулис:
Как ты?

[8:50:17 PM] Михаил:
Вчера были Мечтатели, сегодня Рассекая волны.
 Что-то заставляет меня перечитывать историю Ирландии.

[9:04:56 PM] Маргулис:
Не надо говорить мне про все фильмы, я просила уже
 А них я прямо спрашиваю
 Кстати, ты не потерял и не испортил мою красную скрепку?

[9:16:24 PM] Михаил:
Какая скрепка?..
 О, нашел.

[9:34:07 PM] Маргулис:
Кстати, ты отправишь то, что было ей скреплено?
 Прочитай 20 сонетов Марии Стюарт

[10:20:53 PM] Михаил:
[Photo]
 [Photo]
 [Photo]
 [Photo]
 [Photo]
 [Photo]

[10:52:19 PM] Маргулис:
Спасибо

[11:21:21 PM] Маргулис:
Почему ты не беседуешь со мной здесь?
 Кстати, о живописи до леонардо—а откуда ты знаешь, что тогда рисовали плохо, если ты ничего об этих художниках не знаешь?

--- Thursday, May 4, 2017 ---

[9:53:24 AM] Михаил:
Приходилось видеть иллюстрации в средневековых книгах.

[10:30:08 AM] Михаил:
Предполагал, что тебе не нужны беседы со мной.

[1:48:00 PM] Маргулис:
> Михаил Зыбин
> Приходилось видеть иллюстрации в средневековых книгах.
Глупо предполагать, что это и есть то, о чем я говорю
 И едва ли они во времена леонардо были бы лучше
 Ты разочаровал меня
 Немножко
edited 
[3:58:28 PM] Маргулис:
На самом деле сильно, потому что ты так ничего и не понял.

[10:29:39 PM] Маргулис:
Я порой хочу чем-то с тобой поделиться, но потом вспоминаю, что ты меня поймёшь неправильно
 Я толстая, тупая и некрасивая

[10:57:46 PM] Маргулис:
Я не стою твоей любви, если она ещё осталась.

[10:58:14 PM] Михаил:
Что с тобой?

[10:58:44 PM] Маргулис:
Мои родители согласны, что я толстая
 Тупая, потому что не пишу курсовую
 Типа отдыхаю
 Конечно, сяду сегодня

[11:00:27 PM] Михаил:
> Маргулис
> Мои родители согласны, что я толстая
Это для тебя имеет большое значение?
 Курсовой я тоже пока не занимался.

[11:03:07 PM] Маргулис:
> Михаил Зыбин
> Это для тебя имеет большое значение?
Огромное
 Я похудею
 Ты можешь говорить что угодно про то, что внешний вид неважен,  но это не так. Ты бы не обратил на меня внимание в 11 классе, если бы я не была симпатичной

[11:08:04 PM] Михаил:
Ну, допустим. Хотя я не знаю, что меня цепляет в девушках.

[11:08:21 PM] Маргулис:
Допустим, что внешность важна?
 Ты, кстати, сейчас согласился, что я толстая и страшная

[11:09:49 PM] Михаил:
Нет
 Маша, перестань так думать.
 Я считаю тебя очень красивой.

[11:11:59 PM] Маргулис:
И я далеко не толстая относильно наших однокурсниц

[11:12:17 PM] Михаил:
Ну да
 Вот
 Ну и разумеется, ты не тупая.
 Не слишком концентрируйся на этой самокритике.
 Ты пиши мне, я хочу тебя понимать. Сейчас я усну, утром прочитаю.

[11:40:27 PM] Маргулис:
Хорошо
 Просто день сегодня был плохой. Не смогла заниматься. А первые три дня было лень.

--- Friday, May 5, 2017 ---

[1:09:01 AM] Маргулис:
Посмотри историю двух сестёр
 Корейский

[10:45:08 AM] Михаил:
Я вчера ходил в Третьяковскую галерею на Крымском Валу. Посмотрел де Кирико и оттепель.

[11:05:32 AM] Михаил:
Не грусти, если тебе что-то в себе не нравится.
 Это, наоборот, хорошо.

[12:43:24 PM] Маргулис:
Ну я же просила
 Всё, задолбал.
 Я говорила не говорить мне здесь о фильмах и музеях.
 Не пищи мне больше.
 Какой же ты придурок, я тебя ненавижу.
 Миша, в кино ты не понимаешь очевидных вещей, литературу не чувствуешь, с искусством ещё хуже, слава богу, после пушки ты не говорил об этом со мной. Это просто ужасно. Хватит портить мне настроение.

--- Saturday, May 6, 2017 ---

[11:33:16 PM] Маргулис:
Миша
 Ты звонил?
 МИША
 Всё хорошо?

--- Sunday, May 7, 2017 ---

[2:41:36 AM] Маргулис:
Миш, я не могу простить человеку режим сна. И вечное отсутсвие в важные моменты. Не пиши и не звони мне больше, на матфаке увидимся. Ушами, носами не смей больше ко мне прикасаться. При Саше не ведёшь себя так, чтобы он думал, что можно предложить тебе взять меня на выставку.

[3:10:04 AM] Маргулис:
И одна фраза из нимфоманки
 Совершенно искренняя
 Я ничего не чувствую.
 Не могу читать и смотреть картины
 Я ничего не чувствую
 Я хочу выкричать это
 В глаза одному человеку
 Я ничего не чувствую
 http://www.vavilon.ru/texts/rubinstein/1-14.html
www.vavilon.ru
ВАВИЛОН: Тексты и авторы: Лев РУБИНШТЕЙН: Регулярное письмо: "ЭТО Я" (1995)
Книга современного израильского русского поэта Льва Рубинштейна. Collected poems by present-day Russian poet Lev Rubinstein.

[3:35:36 PM] Михаил:
Я могу ответить?

[4:04:06 PM] Маргулис:
На что ответить?
 Я не в гневе
 Я просила не писать мне глупости
 Я хотела удалить, не читая
 Я не поверю в тебя, ты этого не заслуживаешь. Люди, которые достойны того, чтобы я с ними общалась, родились другими. Я никогда не стану относиться к тебе лучше, потому что ты с самого начала не был достаточно хорош для общения со мной.
 Почему я такая идиотка и забыла тебя заблокировать вчера?
 Теперь заблокировала.
edited 
И ты не дозвонишься до меня больше.
 Я не буду в тебя верить
 Ты не боишься показаться глупым — ок, но я вторых шансов не даю. Людей много, я всегда найду другого.
 Вот эта вот фраза про "тебе не станет лучше"
 Да мне плевать на моих бывших одноклассников, если ты такой идиот и этого не понимаешь
 Мне не нужно никого прощать
 Я о них не думаю
edited 
Я сама знаю, как мне жить, я, в отличие от тебя, не устраиваю этих безумных сцен на матфаке, перестань учить людей, которые взрослее тебя и больше понимают

[4:17:58 PM] Михаил:
Ладно, мне надоело уже.

[4:18:05 PM] Маргулис:
Ты ведёшь себя так, как будто тебе лет 10
 Слава богу

[5:10:25 PM] Маргулис:
Думаю, после лета точно все будет хорошо.
 На матфаке поговорим.

--- Tuesday, May 9, 2017 ---

[4:53:43 PM] Маргулис:
Миш, правда, что завтра нет философии?
 Это очень плохо

[6:42:40 PM] Михаил:

VK
МатФак НИУ ВШЭ 1 курс 2016
ФИЛОСОФИЯ Уважаемые студенты, наше занятие завтра, 10 мая, отменяется (приболел). Но вы не расслабляйтесь – 17 мая у нас контрольная работа. Напомина...

[8:08:53 PM] Маргулис:
Как у тебя дела?
 Ну ладно
 Я хочу тебе сказать, что и искусство никогда не влияет на жизнь человека в обществе. Это второстепенное действие, возможно, это может косвенно повлиять, но только если человек, поменявшись внутренне, принял такое решение.
 И на поведение гораздо сильнее влияют более статичные и фундаментальные черты человека.
 Искусство не меняет характер
 Вживую я добрая, потому что не могу при всех быть такой мразью.
 Я не знаю, меня почему-то раздражает твой стиль письма, в жизни ты приятный. Это со мной бывает, а бывает ровно наоборот. Лучше так, чем наоборот.
 Я звонила из музея
 Поэтому связь пропала
 И потом я не смогла перезвонить
 Ну ты напиши мне
 Простишь ли меня

[9:41:30 PM] Маргулис:
Если простишь, я кину хорошую статью, которую ты должен расценить как примирительный жест
 А если не простишь, то я пожадничаю и не кину
 Вредный мальчик
 Очень противный
 Не отвечает мне
 

 
 Противный Зыбин
 Я клянусь, ты ещё пожалеешь, что хотел говорить со мной об искусстве
 Я буду задалбывать тебя Мандельштамом каждый день
 Тебе будет снится он
 Он, кстати, прекрасен даже в своём мировоззрении

[10:32:50 PM] Маргулис:
Ясно
 Пока

[10:33:18 PM] Михаил:
Нет, стой, нормально все.
 В Мандельштама я как раз позавчера влюбился.

[10:34:19 PM] Маргулис:
Сейчас найду кое-что
 Правда, ты начнёшь взапой читать этого человека и станешь умнее меня
 Или мне придётся читать его больше, чем ты
 Но тогда я перестану делать хоть что-то по математике
 http://www.easyschool.ru/books/literatura/literaturnie-leitmotivi/lamark-shelling-marr
www.easyschool.ru
Ламарк, Шеллинг, Марр - Easyschool
Заказ всех видов письменных работ: рефератов, курсовых, дипломных, кандидатских и магистерских диссертаций. Все работы написаны самостоятельно от перв...
 Про одно стихотворение
 Статья Гаспарова
 Про него спрашивал Ракитин
 Он лучший человек в мире
 Слушай, меня бросили из-за того, что я каждый день говорила об искусстве.
 Ну это так
 К слову

[11:27:36 PM] Маргулис:
Слушай
 Ты бороду не сбрил?

[11:28:00 PM] Михаил:
Не сбрил.

[11:28:07 PM] Маргулис:
Хорошо
 Мне нравится твоя борода
 Я завтра не приду
 А ты?

[11:28:35 PM] Михаил:
Приду.

[11:28:41 PM] Маргулис:
На 1 пару?

[11:29:00 PM] Михаил:
Мне не очень нравится дома.

[11:29:06 PM] Маргулис:
Ладно
 Ты был в отделе личных коллекций на сокровищах нукуса?

[11:29:45 PM] Михаил:
Нет.

[11:29:56 PM] Маргулис:
Там не очень, на самом деле
 Я вчера туда пошла
 В прекрасную погоду
 [Photo]

[11:31:22 PM] Михаил:
Кстати, завтра начало действия Мастера и Маргариты. Опять холодно, как год назад. Должно же быть жарко.
edited 
[11:31:41 PM] Маргулис:
А я не запомнила дату

--- Wednesday, May 10, 2017 ---
edited 
[8:53:05 PM] Михаил:
Ты помнишь про идз по алгебре? Я вот забыл. У него дедлайн сегодня, но не страшно.

[9:04:42 PM] Маргулис:
Я помню, что я не хожу на контрольные и не делаю идз
 Это я хорошо помню
 Мне сказал друг, что я заслуживаю того, как складывается моя жизнь
 Наверное, это бесспорное утверждение
 По отношению к любому человеку
 Если брать в целом

[9:08:46 PM] Михаил:
Оно бессмысленное.

[9:09:03 PM] Маргулис:
Но он мне сказал, что я заслуживаю того, как ко мне относится мой человек
 Фраза "мой человек" не моя
 Её Варя часто употребляла

[9:10:07 PM] Михаил:
Я не понял.

[9:12:41 PM] Маргулис:
Человек, которого я люблю
 Мне грустно
 Ещё он сказал, что было бы слишком несправедливо, если бы я была счастлива

[9:15:41 PM] Михаил:
Тоже бессмыслица, по-моему.

[9:15:51 PM] Маргулис:
Почему?

[9:16:40 PM] Михаил:
Нет же никакого высшего суда и абсолютной справедливости.

[9:16:48 PM] Маргулис:
Кстати, что за танец ты танцевал?
 Наберу в интернете

[9:17:11 PM] Михаил:
Хастл.

[9:17:25 PM] Маргулис:
Как ты? Хорошо день прошёл?
 Прости ещё раз за все эти истерики на каникулах
 Я неадекватная, тупая и ленивая
 По крайней мере неадекватная и ленивая

[9:18:34 PM] Михаил:
Так или иначе, ты лучший на свете человек.

[9:18:56 PM] Маргулис:
Я посмотрела 2,5 русских фильма
 Меня до сих пор от них тошнит
 Груз 200, "да и да" и "все умрут, а я останусь"
 Да и да я не смогла досмотреть
 Груз 200 я тоже смотрела частями, не все 1,5 часа, но до конца

[9:21:44 PM] Михаил:
Сегодня была ещё лекция по анализу, там говорилось про линейные отображения, дифференциал и нормы. Я делал алгебру. Потом доехал до Охотного Ряда, пошёл к Патриаршим. Был снег с дождём. По пути зашёл в церковь, мне она понравилась.
 [Photo]
 [Photo]
 [Photo]
 [Photo]
 [Photo]
 [Photo]
 Зашёл в музей Булгакова, долго читал, что на стенах написано.

[9:24:13 PM] Маргулис:
Однажды я загадала там желание
 Сбылось
 Больше не загадывала
 Боюсь, что больше не сбудется

[9:25:16 PM] Михаил:
Много раз там есть какой-то кусок из Дома, в котором. Про то, что в раю есть шишки.

[9:25:40 PM] Маргулис:
Разве это не какая-то слащавая гадость?

[9:25:45 PM] Михаил:
Ты читала Дом, в котором?

[9:25:51 PM] Маргулис:
Нет
 Ты спрашивал
 Давай так: если ты посоветуешь, я потрачу на это следующий час
 Все равно мне нехорошо

[9:26:42 PM] Михаил:
Нет, я тоже не читал.

[9:26:54 PM] Маргулис:
Не поняла
 А откуда ты тогда знаешь?
edited 
А вообще, мне стыдно, что я медленно читаю

[9:28:10 PM] Михаил:
Я уже видел этот кусок где-то. Позволю себе его привести.
Мальчики, не верьте, что в раю нет деревьев и шишек. Не верьте, что там одни облака. Верьте мне, ведь я Старая Птица, и молочные зубы сменила давно, так давно, что уже и не помню их запах. 
Мысленно с вами всегда, Ваш Папа Стервятник.

[9:29:08 PM] Маргулис:
Ты наизусть?
 Медленно читаю в плане скорости
 А не в плане частоты

[9:29:57 PM] Михаил:
Мне иногда кажется, что я медленно читаю.
 Может, это нестрашно.

[9:30:50 PM] Маргулис:
Ты прочитал статью?
 О Мандельштаме

[9:31:04 PM] Михаил:
Да.

[9:32:48 PM] Маргулис:
Нет, честно говоря, не хочу читать дом

[9:33:15 PM] Михаил:
Я и не советую.

[9:33:21 PM] Маргулис:
Я поняла уже
 В смысле, ты тоже не веришь в эту книжку?
 Я читаю сейчас кое-что, что давно планировала

[9:35:06 PM] Михаил:
Я не знаю. Я её брал в конце февраля у знакомой, собирался прочитать, но появилась куча других планов, и я не стал.
edited 
[9:38:49 PM] Маргулис:
"Морфологию волшебной сказки" проппа
 И ещё метр и смысл Гаспарова
 Только у меня метр и смысл неудачный скачан
edited 
В формате дежавю. Я не помню, как е там буквы латинские, поэтому смешно пишу
 Я опечаталась
 И корректор исправил смешно
 Я могу отвезти тебя в один хороший книжный, кстати
 Там полно анархистской литературы
 Очень хорошая коллекция гуманитарной литературы
 Полка с поэзией
 И он дешевле большинства книжных
 И цены там на всё неровные, это очень смешно
 Я пойду туда скупать Мандельштама
 Когда вспомню пароль от своей стипендии
 Которая копится с сентября
 Такого набора книг, как там, больше нигде нет

[10:01:24 PM] Михаил:
Я не знаю, как это нормально сказать. Дело в деньгах.
 Я, конечно, хочу сходить в этот магазин.
 Завтра обсудим.

[10:23:01 PM] Маргулис:
Да ладно, можно же просто сходить туда
 Я понимаю, все в норме
 Просто показать тебе, что он есть
 Давай я завтра принесу тебе две книжки маяка, кстати?

[10:56:41 PM] Маргулис:
Только ты аккуратный читатель?

[11:21:55 PM] Маргулис:
Займи мне завтра место на втором ряду
 Жалко, что все ещё холодно
 Ты курил когда-нибудь?

[11:37:54 PM] Михаил:
Нет
> Маргулис
> Давай я завтра принесу тебе две книжки маяка, кстати?
Неси, я буду рад. Я думаю, я очень аккуратный читатель. Когда я вижу, что люди разворачивают книгу полностью и проглаживают в середине, я испытываю глубокое страдание. Иногда я читаю книгу в перчатках, чтобы не залапать.

[11:48:14 PM] Маргулис:
Нет, в перчатках не обязательно
 И эта книга довольно старая
 И я не уверена, что она сама по себе в идеальном состоянии
 Посмотрела на хастл, почувствовала себя толстой

--- Thursday, May 11, 2017 ---

[2:37:27 AM] Михаил:
Лечь позже, если нужно вставать раньше. Особая логика.

[3:24:57 AM] Маргулис:
Ты о себе?
 Это такая шутка
 О человеке, который пишет через 45 минут
> Маргулис
> Ты о себе?
Я об этом в 2 сообщениях вообще
 Я качала пресс в час ночи, теперь не могу уснуть
edited 
Ещё я тянула ноги и качала спину
 А ещё я не умею отжиматься
 [Photo]
 У кого-то внутри вселенная

[7:06:25 AM] Маргулис:
Я не спала сегодня

[8:24:46 AM] Маргулис:
Только всего сильней электрический свет в глазах

[8:05:50 PM] Маргулис:
Я ложусь скоро
 Может, сейчас, может, в 9
 Я несла какой-то бред, завтра объясню нормально
 Просто давно не говорила

[8:15:57 PM] Михаил:

VK
МатФак НИУ ВШЭ 1 курс 2016
Уважаемые студенты 1-го курса! 1. Согласно порядку, действующему на факультете, выбор студентами научного руководителя на 2-й курс происходит в 4-м мо...

[8:19:57 PM] Маргулис:
Блин
 Кого возьмёшь?
 Сон отбил
 Одни сплошные нервы
 Опять только математика и математическая физика. Говорили, что на первый год. Нет, я просто так, я так и собиралась
 Дурдом

[8:24:06 PM] Михаил:
Не волнуйся. Я не знаю.

[8:24:42 PM] Маргулис:
Я думала о скопенкове, но я сдохну и он меня не возьмёт

[8:40:36 PM] Маргулис:
С рвбниковым можно поговорить

--- Friday, May 12, 2017 ---

[11:24:44 AM] Маргулис:
Я опаздываю
 Довольно сильно
 Я проснулась в 6:30, но не встала
 А потом не хотела вставать
 И сейчас спать хочу
 Не могу не хотеть спать

[7:15:59 PM] Михаил:
http://www.world-art.ru/lyric/lyric.php?id=7682

[7:18:26 PM] Михаил:
Forwarded message: Михаил Зыбин [4/13/17] 
[Photo]
[Photo]

[7:27:05 PM] Маргулис:
А ты до этого ничего мне не отправлял?

[8:20:28 PM] Михаил:
Есть еще эта, но она мне не очень нравится.

[8:20:28 PM] Михаил:
Forwarded message: Михаил Зыбин [4/13/17] 
[Photo]

[8:22:19 PM] Михаил:
subsmovies.com Там не всегда есть, но часто.

[8:30:53 PM] Маргулис:
Спасибо
 Я — это когда клянёшься, что перед 1 курсом посмотрела последний свой фильм ужасов, и сейчас смотрю, что бы еле посмотреть
 Спасибо, там можно будет посмотреть американских богов, когда все серии выйдут
 Но Твин пикс только на большом экране, конечно
 Блин, надо что-то делать, а не лазать по этому сайту
 Мне нужно пересмотреть старый Твин пикс, посмотреть новый, когда выйдет
 Не начинай ни в коем случае смотреть его с фильма

[10:06:12 PM] Маргулис:
В фильме спойлер всего сериала
 Давай-ка ты займёшься делом
 Физикой

[10:33:21 PM] Маргулис:
Ты летом дома сидишь?

[11:27:19 PM] Михаил:
Прошлым летом я две недели преподавал в лагере. В Москве каждые дня два по утрам бегал. В неделю раза два-три ходил на съёмки телешоу зрителем. Ну, в магазин за едой ходил. А так дома был.

--- Saturday, May 13, 2017 ---

[12:06:47 AM] Михаил:
Тебя добавить в беседу матфака?

[12:21:12 AM] Маргулис:
Нет
 Ни в коем случае
 Я не о том, я о слове Москва
 В Москве ли ты
 Я в Москве
 И мы будем ходить
 По много часов
 Это не вопрос

[1:17:01 AM] Маргулис:
Я случайно отправила один вопрос тебе Георгию
 Ты фильмы Годара на каком языке смотришь?
 Если ты станешь клятвопреступником, я клянусь, что ты больше не будешь со мной общаться

[1:55:52 AM] Маргулис:
Саша ничего не надо говорить хотя бы потому, что я ещё не разобралась со своей старой любовью
 Я приняла важное решение
 Я не расскажу тебе больше ни об одном фильме и режиссёре
 Ты смотришь их др меня, и меня это бесит
 Если бы я не рассказала, ты бы о них не узнал.

[4:37:17 AM] Маргулис:
Я посмотрела жюля и Джима
 Я не видела твоих ответов и не увижу
 Я зря это все написала

[11:41:53 AM] Михаил:
> Маргулис
> И мы будем ходить
Я с радостью.
> Маргулис
> Ты фильмы Годара на каком языке смотришь?
На русском.
> Маргулис
> Я посмотрела жюля и Джима
Я не собирался смотреть этот фильм, я же тебе обещал.

[2:47:32 PM] Маргулис:
Можешь посмотреть
Он оказался все-таки на 9, а не на 10

[3:52:41 PM] Маргулис:
Но он очень хорош

[4:27:46 PM] Маргулис:
Оказалось, я еду в Америку летом

[5:14:23 PM] Маргулис:
Напиши мне критерии нравственности, которые он в результате перечислил

[5:30:30 PM] Маргулис:
А вообще, напиши-ка сегодня-завтра, что ты помнишь в ответ на тот вопрос
 По философии

[6:39:47 PM] Михаил:
Я почему-то спал последние два часа.

[6:40:18 PM] Маргулис:
Молодец
 Это бывает правильно

[10:31:36 PM] Михаил:
У меня очень плохая память, нужно ее развивать.

[10:39:31 PM] Маргулис:
Я надеюсь, ты это не насчёт философии
 Когда-то я отправила своему бывшему очень смешную вещь
 [Photo]
 Я ничего не имела в виду
 Но получилось многозначительно
 
 Выбрать файлы Выбрать файлы
 
 
 
 Но получилось многозначительно

--- Sunday, May 14, 2017 ---

[12:49:39 AM] Михаил:
Да

[12:49:50 AM] Маргулис:
Можешь написать
 Я сейчас миролюбива
 А вообще, я демонстрирую поведение раненого зверя

[12:52:22 AM] Михаил:
Я увидел мораль, что ты не можешь заранее знать, каким окажется случайный человек. Варя оказалась девушкой, с которой ты больше всего общалась. То есть нужно много пробовать общаться с людьми и узнавать их ближе, потому что можно найти родственную душу.

[12:53:02 AM] Маргулис:
Мне не близка Алёна, например, это видно сразу

[12:53:31 AM] Михаил:
Да, вы разные.

[12:53:43 AM] Маргулис:
И Настя

[12:53:57 AM] Михаил:
И она тоже.

[12:54:38 AM] Маргулис:
Крыл сказал, что Лиза сильнее меня. Наверное, он прав.
 Наверное, я слабый человек.

[12:55:16 AM] Михаил:
Я хочу тебе помогать.
edited 
Ты ценный человек.

[12:56:25 AM] Маргулис:
По-моему, я выгорела дотла.

[12:57:35 AM] Михаил:
У тебя же есть силы и желание что-то мне говорить, чтобы я увидел то, что не вижу. Ты кажешься мне достаточно живой.

[12:57:56 AM] Маргулис:
Я мало говорю осмысленного.
 Я не высказываю ярких мыслей.
 Раньше могла.

[12:59:15 AM] Михаил:
Ну а я и раньше не мог. Или мог, но совсем забыл, когда это было. Мне некому, в общем, было.

[1:00:22 AM] Маргулис:
Я помню ощущение, когда знаешь, что внутри есть силы. Что ты готов думать, чувствовать, спорить, жить.
 Я не чувствую эмоций, читая стихи
 Они не вызывают ураган чувств

[1:03:41 AM] Михаил:
Совсем ничего?

[1:04:27 AM] Маргулис:
Мало

[1:08:48 AM] Михаил:
У меня, как ты понимаешь, совсем недавно было ничего.  Ты, видимо, духовно истощена. Твои духовные ресурсы расходуются на поддержание жизни любви. 

Нужно уметь убивать любовь, я считаю. Это как застреливать раненую лошадь.

[1:09:13 AM] Маргулис:
Я хочу его увидеть.
 Я хочу с ним поговорить.
 Я хочу добиться от него хотя бы извинения.
 Как максимум—понимания. Я хочу, чтобы он меня понял.

[1:53:17 AM] Маргулис:
Мне было одиноко каждое лето

[1:53:57 AM] Михаил:
У него там родные, что ли?

[1:54:03 AM] Маргулис:
Нет.
 В том и дело, что нет.
 Просто его родители и он любят этот город
 И вообще мелкие города
edited 
Меньше калининграда

[1:55:10 AM] Михаил:
Тебе было тяжело, я вижу.
edited 
[2:00:06 AM] Маргулис:
Ты спать хочешь?

[2:00:21 AM] Михаил:
Это правда

[2:00:27 AM] Маргулис:
Жалко

[2:01:04 AM] Михаил:
Ты хочешь еще рассказывать?

[2:01:05 AM] Маргулис:
Кстати, какого размера повышенная стипендия?
 Хочу

[2:01:19 AM] Михаил:
Не знаю.

[2:01:27 AM] Маргулис:
Ну, на 2 курсе буду стараться
 В целом, сейчас у меня все нормально
 И брльшого завала нет

[2:02:24 AM] Михаил:
Куда он делся?

[2:02:25 AM] Маргулис:
Нужно только курсач отправить и досдать пару долгов
 Ну, это не большой завал

[2:02:50 AM] Михаил:
Отправить? Ты сделала что-то?

[2:02:59 AM] Маргулис:
Не всё
 Завтра сделаю много
 Я не пойду в пушку
 Отложу на неделю

[2:04:05 AM] Михаил:
Надеюсь, ты в самом деле будешь продуктивной.

[2:04:29 AM] Маргулис:
Только проблема в том, что теперь нужно ещё заниматься американским посольством

[2:05:05 AM] Михаил:
Это долгий процесс?

[2:05:17 AM] Маргулис:
Непредсказуемый

[2:05:55 AM] Михаил:
Какие возможны варианты?
 Дело в том, что я опасаюсь, что начну писать что-то неадекватное и невпопад

[2:06:47 AM] Маргулис:
Ну, типа до свидания друг мой
> Михаил Зыбин
> Какие возможны варианты?
Я об этом
 Могут вижу не дать
 Я же студент
 Они не любят студентов
 И старших школьников
 Возможно, я даже похожа на проститутку и они решат, что это цель моей поездки

[2:08:37 AM] Михаил:
Не похожа. Ты бывала уже в штатах?

[2:08:41 AM] Маргулис:
Нет
 Мне вышлют приглашение родственники, должны дать визу
 Ладно, или спать, я пока ещё позанимаюсь
 А то я поняла, что все-таки завал есть

[2:10:08 AM] Михаил:
Пойду. Спокойной мне ночи.

[2:10:11 AM] Маргулис:
Да
 Спокойной тебе ночи
 Как в тему—на моем календаре страшный суд

[2:28:30 AM] Маргулис:
Меня немножко напрягает то, что я только что узнала о лагранжиане

[2:42:32 AM] Михаил:
Восток сейчас великолепен.

[2:45:13 AM] Маргулис:
Ты не спишь?
 Я делаю физику
 По-моему, я начала что-то понимать
 Где-то глубоко в душе я поняла, что такое лагранжиан и с чем его едят
 У меня чёрным-черно

[2:46:20 AM] Михаил:
Я хотел отправить тебе песню, но, наверное, не стоит.

[2:46:20 AM] Маргулис:
А у меня окна на восток выходят

[2:46:38 AM] Михаил:
Да ладно, на востоке светло.

[2:46:47 AM] Маргулис:
Ты согласен, что секунду

[2:47:06 AM] Михаил:
Что?

[2:47:21 AM] Маргулис:
Извини
 Не стёрла начало
 На моём востоке тучи
 Ты согласен, что в задаче раздел 1 задача 2 нужно сложить две кинетические и вычесть две потенциальные энергии?
 Продифференцировать
 И приравнять к 0

[2:49:05 AM] Михаил:
Я не имею понятия, я не решал ничего

[2:49:33 AM] Маргулис:
Там светает, но там тучи

[2:49:46 AM] Михаил:
[ Мельница – Дорога сна (zaycev.net)  ] 7.4 MB
 

[2:49:58 AM] Маргулис:
Эх

[2:50:03 AM] Михаил:
Мда

[2:50:10 AM] Маргулис:
Что?

[2:50:25 AM] Михаил:
То же, что эх.
 Ладно, я пойду.

[2:50:55 AM] Маргулис:
Ок
 А почему ещё не пошёл?

[2:51:36 AM] Михаил:
Стыдно признаться, я был в душе.

[2:52:12 AM] Маргулис:
Почему стыдно?
 Ну ладно
 Было бы стыдно, если бы ты там не бывал

[2:20:30 PM] Маргулис:
Мне снился сон
 По мне ползал питон
edited 
[4:13:52 PM] Маргулис:
Хочу уговорить кого-нибудь написать моему бывшему, что я покончила с собой

[4:52:10 PM] Маргулис:
И что в этом виноват мой бывший)

[5:10:27 PM] Маргулис:
Напиши мне про философию

[5:10:51 PM] Михаил:
Я не знаю

[5:11:04 PM] Маргулис:
Что не знаешь?
 Напиши, что помнишь из его объяснений

[5:11:32 PM] Михаил:
Ничего

[5:58:37 PM] Маргулис:
Я должна признаться, что не понимаю смысла задачи 2.2

[6:08:36 PM] Михаил:
Бродский говорил, что смысл поэзии - впасть от нее в зависимость. Я впал. Мандельштам мне, однако, не снится.

[6:10:33 PM] Маргулис:
> Маргулис
> Я должна признаться, что не понимаю смысла задачи 2.2
Т.е. Задачи 4
 Чего там с движением может быть-то? Там ничего это поле не меняет
 Жёсткий стержень, направлен вдоль линий
 Если я правильно поняла условие
 С обеих сторон либо на этот стержень давит одинаковая сила, либо растягивает

[6:12:34 PM] Михаил:
Я бездельник

[6:12:55 PM] Маргулис:
И никакого вклада в скорости нет

[6:13:27 PM] Михаил:
Какой учебник ты читаешь?

[6:13:45 PM] Маргулис:
Ландау, википедию, Фейнмана
 Фейнмана  по электричеству
 И магнитам
 Ландау по механике

[7:57:59 PM] Маргулис:
У тебя же есть наша старая переписка?
 Переотправка мне физику
 Которую уже кидал
 Последнее занятие третьего модуля

[8:21:49 PM] Михаил:
Forwarded message: Михаил Зыбин [4/5/17] 
[Photo]
[Photo]

[9:53:15 PM] Маргулис:
Какие успехи?

[9:53:47 PM] Михаил:
Эмм... ну...
 Хмм...

[9:55:26 PM] Маргулис:
Я знаю, как решать половину физики вроде
 Потому что я молодец

[9:56:25 PM] Михаил:
Я очень рад.

[9:56:47 PM] Маргулис:
Миша
 Где правильная 3 задача?

[9:57:35 PM] Маргулис:
Forwarded message: Михаил Зыбин [5/3/17] 
[Photo]
[Photo]
[Photo]
[Photo]
[Photo]
[Photo]

[9:59:02 PM] Михаил:
Здесь

[9:59:02 PM] Михаил:
Forwarded message: Михаил Зыбин [5/3/17] 
[Photo]

[9:59:45 PM] Маргулис:
Сердечка нет

[9:59:59 PM] Михаил:
О, я вспомнил, успех же есть. Я узнал, что Бродского можно звать Бутербродским.

[10:00:45 PM] Маргулис:
Я это много лет знаю
> Маргулис
> Сердечка нет
Сердечка нет

[10:01:29 PM] Михаил:
Да, его нет.

[10:02:07 PM] Маргулис:
Я ставила их там, где тебе зачли задачу

[10:03:13 PM] Михаил:
Я сдал третью, мы ее проглядели.

[10:03:24 PM] Маргулис:
Хорошо
 А свои успехи можно где-то отслеживать?

[10:03:59 PM] Михаил:
Нет

[10:13:47 PM] Маргулис:
Когда ты занимаешься делом, ты помогаешь и мне
 Так что займись

--- Monday, May 15, 2017 ---

[12:04:37 AM] Маргулис:
Я боюсь самолётов
edited 
Я боюсь, что визу не дадут

[12:18:51 AM] Михаил:
Раньше летала уже?
 Ты говоришь, приглашение будет, значит, должны дать.

[12:19:51 AM] Маргулис:
Много раз
 Не факт
 Даже маме той женщины не всегда давали
 Один раз отказали, причём не в первый
 Они студентов не любят

[12:21:18 AM] Михаил:
Ты ведь не можешь на это повлиять, так что бояться незачем.
 Почему самолётов боишься? Хотя это глупый вопрос, наверное.
 Я понимаю твое волнение, впрочем.

[12:25:04 AM] Маргулис:
Ну, я боюсь процесса смерти
 Мне неприятно было бы тонуть или гореть

[12:26:03 AM] Михаил:
Самолёт же безопаснее многих других видов транспорта, как говорят.

[12:26:03 AM] Маргулис:
Или чтобы меня съели
 Уж очень небыстрая смерть
 В самолёте

[12:27:12 AM] Михаил:
Она маловероятна.
 Сгореть - это очень неприятно, я согласен.

[12:28:21 AM] Маргулис:
Я понимаю, что глупо бояться
 Но я ещё и одна
 И надо одной пересаживаться

[12:28:50 AM] Михаил:
Почему одна?
 Где пересаживаться? В какой город ты, собственно, летишь?

[12:30:34 AM] Маргулис:
Ну я одна
 В Сиэтл
 В Амстердаме
 Ну там родственники мои
 А я одна
 Ладно, пойду пока делом займусь
 Я очень психую
 Из-за всего
 Хотя не из-за чего

[12:31:57 AM] Михаил:
Не волнуйся, у тебя все получится.

[12:32:06 AM] Маргулис:
Я выпила успокоительное
 А, философия

[12:32:34 AM] Михаил:
Я как раз думал, посоветовать тебе валерьянку или нет.

[12:32:37 AM] Маргулис:
Она меня волнует
 Больше всего
edited 
Родители моего бывшего бы мне сейчас пригодились

[12:49:08 AM] Михаил:
Зачем?

[12:50:26 AM] Маргулис:
Они закончили философский
 Мама преподаёт
 [Photo]
 [Photo]
 [Photo]

[1:00:33 AM] Михаил:
Мур-мур-мур

[1:05:18 AM] Маргулис:
Да
 Приходиться читать википедию, чтобы сдавать философию

[1:08:18 AM] Михаил:
Я читаю паблик про этимологию.

[1:08:52 AM] Маргулис:
Два непонятных слова
 Одно
 Третье

[1:09:34 AM] Михаил:
Это означает сообщество вконтакте.

[1:09:41 AM] Маргулис:
Этимология—это в смысле происхождение слов в лингвистике?

[1:10:05 AM] Михаил:
Да. Случайно нашел его.

[1:10:09 AM] Маргулис:
Чортъ, какая же я стала тупая
 Чортъ прекрасно выглядит, не могу писать это через е

[1:10:55 AM] Михаил:
Мне тоже нравится.

[1:11:56 AM] Маргулис:
Хотя он не должен же так писаться
 Есть слово черти

[1:12:14 AM] Михаил:
Раньше так писался.

[1:12:36 AM] Маргулис:
Но так пишется же только когда нет замены...
 Гм
 А
 Нет
 Я путаю с ятем
 Мне нравится, что ять непонятно какого рода

[1:15:14 AM] Михаил:
Звезда через ять пишется, кстати. Это исключение.

[1:16:20 AM] Маргулис:
Хорошо
 Я люблю исключения
 Я слышала о людях, которым плохо, когда на них не висит миллион дел

[1:18:50 AM] Михаил:
Бывает
 Еще писалось "шопот".

[1:26:07 AM] Маргулис:
Мне нравится старая орфография
 Она красивая
 Она очень романтична
 У меня есть Евгений Онегин в старом варианте

[1:27:14 AM] Михаил:
И ведь первая эф в слове орфография - это фита.

[1:27:20 AM] Маргулис:
Я знаю
 Потому что греческий корень
 Объясни мне понятие предикат, я забыла уже

[1:28:14 AM] Михаил:
Это было у Данияра.

[1:28:49 AM] Маргулис:
Я знаю
 Я забыла уже

[1:30:04 AM] Михаил:
Это не тот предикат, который может в философии появляться.

[1:30:12 AM] Маргулис:
Я знаю
 Я хочу тот, что в философии

[1:32:27 AM] Михаил:
Википедия говорит, что это то, что утверждается о субъекте.
 Тебе это не нравится?
 Я не помню, кстати, говорили ли мы об этом на семинарах.

[1:34:48 AM] Маргулис:
Нормально
 Нравится
 Но я, видимо, не всегда это понимаю по ходу текста
 Я поняла про трансцендентность и трансцендентальность
 (Да не может суффикс менять значение слова на противоположное)
 Да не может быть три взаимных антонима

[1:36:22 AM] Михаил:
А какой третий?

[1:37:17 AM] Маргулис:
Я завтра расскажу
 Надо учится сейчас
 Я немного забыла, как прекрасен процесс познания
 Звучит ужасно, но это правда.
 
IMG1392.PNG 196 KB
 
edited 
[1:40:25 AM] Михаил:
А я и не знал, как здорово читать стихи. Так себя чувствую, как будто давно сидел под одеялом с головой, продышал там весь воздух, а теперь вылез.

[1:41:02 AM] Маргулис:
> Маргулис
> Я немного забыла, как прекрасен процесс познания
Я не не знала все-таки, если бы я не знала, я была бы глупее, благо есть куда

[1:42:00 AM] Михаил:
Ну, я тоже догадывался про стихи.

[1:43:35 AM] Маргулис:
Как-то унизительно сейчас звучит

[1:43:50 AM] Михаил:
Извини

[1:43:59 AM] Маргулис:
Учитывая то, я читаю сейчас философию
 И говорю о ней
 И уже читала философию в своей жизни

[1:45:11 AM] Михаил:
Вот и написал.

[1:45:11 AM] Михаил:
Forwarded message: Михаил Зыбин [5/14/17] 
Дело в том, что я опасаюсь, что начну писать что-то неадекватное и невпопад

[1:45:24 AM] Михаил:
Мне пора

[1:45:36 AM] Маргулис:
Хорошо
 Спокойной ночи
 Извини, что задержала
 Я больше не буду

[1:46:21 AM] Михаил:
Ладно, нормально все. Ложись не слишком поздно сегодня.

[1:50:43 AM] Маргулис:
Хорошо

[10:35:06 AM] Маргулис:
Я приеду к 12, наверное
 Я умираю
 Идиотское успокоительное
 Я не могу подняться с постели

[4:07:01 PM] Михаил:
Я ушел с дискры. Поеду домой.
 Наверняка я глупо сделал.
 Что ушел.
 Прости меня.
 Тебе обязательно станет легче. Это всегда проходит.
 Ты можешь справиться.
 Я буду сегодня заниматься учебой.
 Я умоляю, прости меня, что я ушел.

[4:45:41 PM] Маргулис:
Ничего страшного
 Все в норме

[5:09:08 PM] Маргулис:
А ещё я люблю змей
 Я псих

[6:18:13 PM] Маргулис:
Мне нравятся только очень носатые мальчики, и сама я носатая. Мне жалко свою дочь, если она будет

[6:19:23 PM] Михаил:
Я тебя отлично понимаю.

[6:19:54 PM] Маргулис:
Даже страшно подумать, о чем ты

[6:20:44 PM] Михаил:
Ой
 То есть я ценю в девушках большие носы.

[9:16:43 PM] Михаил:
Forwarded message: Алексей [5/15/17] 
У Хорошкина др 20 августа
У Рыбникова 19 августа
в один год
81

[9:48:55 PM] Маргулис:
Классно

[10:31:25 PM] Маргулис:
Что ты сегодня делал?
 Мог бы интересоваться иногда, как я тут
 Что значит 81?
 У хорошкина ребёнку 18 лет
 Или 17

[11:35:29 PM] Михаил:
1981 год
 Ну, я слегка понял, что такое лагранжиан, немного решил матан, что-то добавил в курсовую, почитал Канта.
 Тебе менее плохо, чем днем?

--- Tuesday, May 16, 2017 ---

[12:03:24 AM] Маргулис:
Нет
 Хуже
 Я никуда не лечу
 Потратила день на визу
 И никуда не лечу
 Потому что дата ближайшего собеседования только в июле

[12:04:28 AM] Михаил:
Это грустно

[12:04:32 AM] Маргулис:
Ага

[12:04:54 AM] Михаил:
Родственники тоже расстроились?

[12:05:55 AM] Маргулис:
Только что узнала
 Они ещё нет
 Пиши про канта

[12:06:43 AM] Михаил:
Тебе, сейчас? Я не готов пока.

[12:06:48 AM] Маргулис:
Ок
 Мне очень плохо, но ныть не буду
 Очень жалко этих суток
 Несколько часов заполняла визу
 Отец ещё фотографию обрабатывал

[12:25:34 AM] Михаил:
Да, я сочувствую тебе.

[10:27:20 AM] Михаил:
Буду к 12:00

[10:30:34 AM] Маргулис:
Так и нужно же?
 Займи мне место

[5:02:07 PM] Маргулис:
Чего ты ушёл?
 Не нравится мне Саша, забей, если ты из-за этого

[5:03:00 PM] Михаил:
Я семинаре я все равно ничего бы не делал.

[5:03:21 PM] Маргулис:
Мог бы позвать меня
 Мы с Георгием порешали физику
> Маргулис
> Не нравится мне Саша, забей, если ты из-за этого
Ты из-за этого?

[5:04:09 PM] Михаил:
Нет
 Лагранжиан нужно из головы выдумать, или он определяется по-человечески в задачах?

[5:05:20 PM] Маргулис:
Давай вечером напишу? Я до 6 сплю/отдыхаю
 Его нужно определить

[5:05:42 PM] Михаил:
Да, хорошо

[5:06:03 PM] Маргулис:
Он равен сумме кинетических минус сумма потенциальных
edited 
По каждому элементу системы

[5:06:22 PM] Михаил:
Действительно

[6:34:19 PM] Михаил:
Отправь Георгию записи Лизы по философии.

[7:45:32 PM] Маргулис:
Он стесняется меня попросить?

[7:46:02 PM] Михаил:
Наверное
 Не берусь сказать

[7:47:25 PM] Маргулис:
Простите
 Я уснула надолго

[7:47:52 PM] Михаил:
Молодец

[7:48:45 PM] Маргулис:
Прости

[7:49:29 PM] Маргулис:
Forwarded message: Маргулис [5/16/17] 
[Photo]
[Photo]
[Photo]
[Photo]
[Photo]
[Photo]
[Photo]
[Photo]
[Photo]
[Photo]
[Photo]
[Photo]
[Photo]
[Photo]
[Photo]
[Photo]
[Photo]
[Photo]
[Photo]
Forwarded message: Anastasia Sukacheva [5/16/17] 
[Photo]
[Photo]
[Photo]

[7:50:39 PM] Маргулис:
Там не очень записи, по-моему

[7:51:01 PM] Михаил:
Спасибо

[7:51:29 PM] Маргулис:
Прости меня
 Мне просто одиноко

[7:53:49 PM] Михаил:
За что простить?

[7:56:54 PM] Маргулис:
За то, что ставлю тебя в такие условия, что ты уходишь
 То есть все-таки я виновата?

[8:00:12 PM] Михаил:
У меня в голове была нерабочая атмосфера, поэтому я ушел.

[8:03:26 PM] Маргулис:
Почему молча?
 Почему?
 Это так мерзко

[8:05:45 PM] Михаил:
Да, я так иногда делаю. Старая привычка, что мое присутствие не имеет значения. Извини, я не буду больше.

[9:05:28 PM] Маргулис:
Мне очень плохо

[9:05:50 PM] Михаил:
Почему?

[9:07:01 PM] Маргулис:
Не знаю
 Жить не хочется

[9:07:58 PM] Михаил:
Маша, перестань.
 Я в таких случаях просто жду, когда это чувство пройдёт.
 Пойми, что это не ты хочешь, а что-то инородное, от чего надо избавиться.
 Вернее, хочешь-то ты, но понять надо.

[9:15:04 PM] Маргулис:
Мур

[9:15:23 PM] Михаил:
Мур
 В Полутора комнатах, кстати, написано про мур немного.
 Нужно не поддаваться. Понимай, что приступы временные, а ты, когда они случаются - это ненастоящая ты. Ты больше и богаче, чем печаль.

[9:20:46 PM] Маргулис:
Кстати, я так и не дочитала
 Трижды первую половину
 Мне поездки в метро не хватает
 А здесь грустно его читать

[9:21:53 PM] Михаил:
Почему?

[9:22:36 PM] Маргулис:
Не знаю
 Я же говорю — каток. Я не могу любить больше.

[9:24:14 PM] Михаил:
Каток?

[9:24:41 PM] Маргулис:
Я уже говорила

[9:25:22 PM] Михаил:
Не помню.

[9:25:30 PM] Маргулис:
http://www.world-art.ru/lyric/lyric.php?id=7594

[9:27:54 PM] Михаил:
Прелесть.
 Какой каток?

[9:28:39 PM] Маргулис:
Которым асфальт раскатывают

[9:29:14 PM] Михаил:
А, этот. Да, ты говорила.

[9:35:00 PM] Маргулис:
Кажинный раз на этом самом месте
edited 
Мне нравится момент про на той неделе

[9:36:57 PM] Михаил:
Я прочувствовал этот стих в контексте того, что тебе мог бы нравится Саня.

[9:37:27 PM] Маргулис:
Нравиться
 )
 Ь
 Корректор исправил на о

[9:38:08 PM] Михаил:
Извини, я опечатался

[9:38:17 PM] Маргулис:
Ничего
 Я не знаю, что я о нем думаю

[9:38:54 PM] Михаил:
Ну, я тоже

[9:40:02 PM] Маргулис:
Ты-то почему?

[9:40:27 PM] Михаил:
Я не знаю, что ты о нем думаешь.

[9:40:38 PM] Маргулис:
Мне он понравился тем, что в нем я увидела какую-то возможность нормальной счастливой жизни.
 Но кого я обманываю — я так не могу
 Очень не хочется портить отношения с ним из-за этого

[9:42:22 PM] Михаил:
Да, хорошо иметь такого человека рядом.

[9:43:06 PM] Маргулис:
Я очень не люблю быть не в тех отношениях с людьми, когда их можно обнимать
 Я люблю обнимать людей

[9:44:04 PM] Михаил:
Ты это хорошо скрываешь.
 Я тоже люблю и тоже хорошо скрываю.

[9:44:42 PM] Маргулис:
Мы настоящие спецагенты
 Гении конспирации

[9:45:15 PM] Михаил:
Смешно

[9:46:28 PM] Маргулис:
Я хочу чего-то светлого в своей жизни, да. Но это всё слишком правильно для меня. Саша может начать хуже ко мне относиться, потому что я гораздо более раскрепощённый человек

[9:51:54 PM] Михаил:
Давай позже поговорим, я хочу математикой заняться. Ты ей тоже занимайся. 
Ты в трех предложениях попыталась выразить что-то, для чего три предложения - маловато.

[10:23:45 PM] Маргулис:
Ты занимаешься математикой всегда, когда не нужно ей заниматься
 Спасибо
Missed Call

[10:32:42 PM] Маргулис:
Алло
 Да ну тебя
Outgoing Call 1:22:03

--- Wednesday, May 17, 2017 ---

[12:07:35 AM] Маргулис:
Таким образом, моральный закон может состоять лишь в законодательной форме принципа:
«поступай так, чтобы максима твоей воли могла бы быть всеобщим законом».
 Я же говорю, по сути это и есть моральный закон

[12:36:37 AM] Маргулис:
Не могу я это писать
 Я не знаю, как мне начать
 Я не воспринимаю это как однородные понятия, которые можно перечислить
 "Не только, но и" обозначает, что он и априорен, и ноуменален
 Кстати, когда Алёна пишет нАуменален, она видит за этим какую-то этимологическую основу, наверное

[1:01:12 AM] Маргулис:
Ты что-то написал?

[1:01:58 AM] Михаил:
Да. Но мне еще есть, что дописать. Я не закончил.

[1:02:14 AM] Маргулис:
Мне не нравится то, я пишу

[1:02:47 AM] Михаил:
Напиши так, чтобы нравилось)

[1:04:50 AM] Маргулис:
Ты скинешь мне своё?
 Я могу бесконечно переписывать начало
 А конец меня устраивает

[1:06:00 AM] Михаил:
Ну, потом. И у меня неразборчиво сейчас написано, я перепишу.

[1:06:11 AM] Маргулис:
А, ты не печатаешь
> Маргулис
> "Не только, но и" обозначает, что он и априорен, и ноуменален
Это правда,
 ?

[1:12:36 AM] Михаил:
Да
 То есть бывают априорные неноуменальные вещи.

[1:15:14 AM] Маргулис:
А
 То есть да
 Я неправильно прочитала
 Ты же не видишь разницу
 Я наоборот не вижу сходства
edited 
Потому что ноуменальность — св-во реального мира
 А априорность — это св-во мышления
> Михаил Зыбин
> То есть бывают априорные неноуменальные вещи.
По-моему, не бывает
 Я сейчас подумала

[1:18:01 AM] Михаил:
Априорность - свойство суждений. Ноуменальность связана с нашим восприятием вещей.

[1:18:11 AM] Маргулис:
Нет, конечно
 Она существует вне восприятия
 Ноумен — это просто вещь, как она есть
 В реальности
edited 
Но ты о ней знаешь только то, что воспринимаешь
 Возможность априорных суждений заложена в мышлении
 Давай так, ты не можешь постись, откуда у тебя есть способность априорных суждений
 Так что это по-любому вещь в себе

[1:20:45 AM] Михаил:
Да
 Которая вещь?

[1:21:15 AM] Маргулис:
Априорность
 Способность к ней
 Все суждения, из которых берётся априорность
edited 
Хотя нет, пространство-время все-таки не вещь в себе?
Incoming Call 41:20
edited 
[3:09:37 AM] Маргулис:
По канту, нравственность человека должна подчиняться принципу автономии воли, который формулируется следующим образом: максима, в соответствии с которой мы действуем, не должна выбираться нами ради какой-то внешней цели, цели, несущей выгоду. Конечной целью является сам человек. Максима исходит только из чувства внутреннего долга, закона, которым человек добровольно себя ограничивает, и в этом выражается истинная свобода личности, свобода как независимость от внешних условий. Нравственный закон — это категорический императив, то есть, облагая себя им, мы не можем не исполнять его, потому что это означало бы, что мы поступаем безнравственно. Вообще с точки зрения человека цель исполнения морального закона — быть достойным быть счастливым, но напрямую такая максима к счастью (по крайней мере в бытовом, корыстном понимании) не ведёт. Чтобы проверить соответствие максимы нравственности, человек обращается к принципу: "поступай так, чтобы максима твоей воли могла быть всеобщим законом.". То есть, моральный закон есть нечто всеобщее, а это обеспечивается тем, что он, должен исходит из самой структуры сознания, общей у всех людей, потому что от опыта он, как сказано выше, зависеть не может. В этом выражается априорность нравственности. Но человек является существом ноуменального мира (то есть мы познаём себя с посредством самоощущения, самовосприятия, а это значит, что являемся по отношению к себе явлениями, но вещь не исчерпывается лишь своим явлением,  и в нас есть непознаваемая для нас часть, ноумен). К этой ноуменальной части относятся и базовые заложенные в сознание принципы, рассудочные понятия, ведь мы не можем их вывести посредством разума. Как сказано выше, нравственность мы имеем заложенной в наше мышление, значит, она так же относится к ноуменальность миру.
 Сейчас ещё допишу
 По канту, суждения опыта — это те суждения, в которых наши восприятия связаны посредством рассудочных понятий, заложенных в наше сознание, и потому являющиеся объективными. Таким образом, законы естествознания есть суждения опыта. Суждения же восприятия субъективны и выражают лишь то, что мы непосредственно воспринимаем.

Это надо расширить

[3:45:16 AM] Маргулис:
Ты извини, там кривой текст, быстро писала и не перечитывала

[2:06:24 PM] Маргулис:
Мишка
 Ты хороший
 Я забыла объяснить, что такое предмнение, и в связи с этим стоило написать, что понимание возможно, когда есть изначальное представление об объекте, но это, я надеюсь, не очень важно
 В смысле, надеюсь, что на это закроют глазе
edited 
Я загрустила из-за этого
 Я пошла лучше ответить на второй вопрос

[3:28:10 PM] Михаил:
Я понял герменевнический круг в контексте понимания текстов, а в контексте гуманитарной науки нет. Не знаю, есть ли суть дела в том, что я написал.

[3:28:30 PM] Маргулис:
Есть

[3:28:36 PM] Михаил:
Что у тебя, ты говоришь, с ногами?

[3:29:00 PM] Маргулис:
> Маргулис
> Я забыла объяснить, что такое предмнение, и в связи с этим стоил
Ну забыла. Думаешь, стоит ему позвонить и сказать об этом?
> Михаил Зыбин
> Что у тебя, ты говоришь, с ногами?
Натерла до крови
 До мяса
> Маргулис
> Ну забыла. Думаешь, стоит ему позвонить и сказать об этом?
Может, это не очень страшно?
 Я сегодня потрачу день на Гегеля. Я полюбила нашу философию

[3:30:51 PM] Михаил:
Звонить не надо, конечно. Что сделала, то сделала, теперь незачем волноваться.

[3:31:04 PM] Маргулис:
Ну, спросить, страшно ли это

[3:31:12 PM] Михаил:
У тебя такая неудачная обувь?

[3:31:19 PM] Маргулис:
Я поняла, что забыла, когда объясняла Алёне, что это
 Новая обувь почти всегда хоть немного натирает
 А если ходить с моей скоростью, когда я одна иду, то натирает по-любому сильно, если у неё есть такая возможность
 Она разносится
 Через неделю перестанут натирать

[3:33:09 PM] Михаил:
Я тоже так думаю

[3:33:25 PM] Маргулис:
Обидно будет, если именно за это снимут
 На следующем занятии ему скажу об этом, если он не принесёт проверенные работы
 Если принесёт, объясню, что решила, что проверяющий и так это знает
 И что суть вопроса не в этом
 И мне в момент описания было это совершенно очевидно.

[3:34:38 PM] Михаил:
У него нет оснований тебе верить, что ты это именно забыла на работе, а не не знала.

[3:35:07 PM] Маргулис:
Странное утверждение
 А ты объяснил это?

[3:35:24 PM] Михаил:
Я в метро, мне сейчас выходить.

[3:35:33 PM] Маргулис:
Я не умею оперировать понятиями, которые не понимаю
 Я тоже написала его в контексте понимания текста, но потом уточнила, в чем заключается научный метод при его использовании
 Ну, ладно, разумеется, мой ответ я сейчас могу сильно расширить

[3:42:51 PM] Михаил:
Я думаю, что если человек допустил в работе недочет и осознал его потом, то это не является для проверяющего поводом как-то иначе относиться к работе.
 Может быть, это осознала не ты, а кто-то тебе сказал.

[3:43:44 PM] Маргулис:
{приходится надеяться на то, что многие написали плохо и оценки он будет выставлять относительно нас} 


Я поняла это как толко вышла 

Он возникает в гуманитарной науке, потому что она отличается от нормальной науки (кхм)) отсутствием прямой проверяемости. Нормальный научный текст ты можешь проверить опытом, а гуманитарный — нет
> Маргулис
> А ты объяснил это?
Про предмнение

[3:44:30 PM] Михаил:
Вроде да

[3:45:21 PM] Маргулис:
Я объяснила, что оно корректируется и стремится к максимальной близости со смыслом текста, но не сказала, что это наше представление об объекте в данный момент
 И что поэтому для понимания у нас должно быть некое начальное представление

[3:46:03 PM] Михаил:
Я написал, что гуманитарные науки занимаются оценкой культурных явлений с точки зрения ценностей, существовавших в соответствующую эпоху.

[3:46:38 PM] Маргулис:
> Маргулис
> Про предмнение
Ты ответишь мне на вопрос?

[3:46:38 PM] Михаил:
Не знаю, стоит ли сейчас это обсуждать.

[3:46:49 PM] Маргулис:
Ты определил его?
 Нет, лучше обсудить сейчас. Потому что позже он сам все объяснит и я буду чувствовать, что не могу доказать тебе, что я не дура
> Михаил Зыбин
> Я написал, что гуманитарные науки занимаются оценкой культурных
Я не писала это, потому что не сочла и не считаю нужным
 В данном вопросе
 Вопрос не о об отличие от остальных наук
> Маргулис
> {приходится надеяться на то, что многие написали плохо и оценки
Насчёт последнего абзаца, проверяемость гуманитарной науки заключается в этом круге
 Ладно, это слишком неважно для меня. Мне неприятно только, что от этого зависит чьё-то мнение обо мне. Я просто знаю, что я понимаю это и что это меня радует

[3:58:05 PM] Михаил:
Чьё? Моё, что ли? Или Кирсберга?
 Мне тоже эти дела не очень важны, как ты должна понимать.

[4:01:54 PM] Маргулис:
Кирсберга и всех людей, которые услышат оценку
 Все равно я не тупая
 И я хорошо отвечу по Гегелю
 И он мне поверит
> Маргулис
> И он мне поверит
Что я не тупая

[4:03:02 PM] Михаил:
Да, я согласен

[4:04:46 PM] Маргулис:
Ну вообще, если бы он мне поставил оценку выше 7, было бы очень хорошо

[4:04:47 PM] Михаил:
http://www.world-art.ru/lyric/lyric.php?id=7809

[4:06:37 PM] Маргулис:
Очень здорово
 Не видела раньше

[5:23:18 PM] Маргулис:
Ты хороший

[5:48:50 PM] Маргулис:
[ Sticker]
 [ Sticker]

[6:09:30 PM] Маргулис:
http://paperpaper.ru/tour-legends/

Интернет-газета "Бумага"
20 сказок о России: что нелегальные гиды рассказывают китайским туристам про ист...
Пушкин-негр, подделки в Эрмитаже, война из-за Янтарной комнаты
 "Все экспонаты Эрмитажа и других музеев украдены Россией: крайне глупо покупать произведения искусства, имея такие армию и флот."
 Честно скажу, ни разу не видела, чтобы китайские туристы трогали музейные экспонаты
 Они, конечно, часто ведут себя крайне неприлично и выглядят очень слово образоваными
 Статуи они могут трогать, это правда, но не картины
 И кстати, они безумно меркантильны

[7:29:31 PM] Маргулис:
Посмотри жизнь Адель

[7:48:13 PM] Маргулис:
Миша, с тобой не происходило в жизни совсем плохих событий, о которых ты мне не решился рассказать?

[7:56:25 PM] Михаил:
Я посмеялся.
edited 
[7:56:56 PM] Маргулис:
Кстати, я нашла на бумаге ссылку на сайт для родителей гомосексуальных подростков. Я редко вижу что-то настолько однобокое
 Он ужасно написан

[7:57:21 PM] Михаил:
> Маргулис
> Миша, с тобой не происходило в жизни совсем плохих событий, о ко
Кажется, нет.

[7:57:47 PM] Маргулис:
То есть они считают, что вся психика человека предопределена природой и ничто внешнее на неё не влияет

[7:58:22 PM] Михаил:
Это неправда

[7:58:31 PM] Маргулис:
Я знаю
 Ну это такой бред, я даже расстроилась
 Вместо образовательного ресурса они просто хотят сказать, что никто в гомосексуальности не виноват

[8:01:16 PM] Михаил:
Однобоко, в самом деле.

[8:01:57 PM] Маргулис:
И что никакие травмирующие события на сексуальность не могут повлиять, хотя могут
 И куча маньяков в детстве подверглась насилию

[8:03:32 PM] Михаил:
Да-да-да.

[8:03:44 PM] Маргулис:
Что-то не так?

[8:04:00 PM] Михаил:
Нет, я согласен

[8:04:13 PM] Маргулис:
Я прекрасно отношусь к людям любой ориентации
 И менее прекрасно к людям некоторых гендерных идентичностей, потому что я не верю в то, что над этим не стоит работать
edited 
Почему мужчины менее терпимы к геям, чем женщины?
 Опечаталась
edited 
Интересно, там исправилось?
 Чем женщины, а не чем к женщинам, я отвлеклась

[8:13:40 PM] Михаил:
Маша, я планирую заниматься учебой сейчас.

--- Thursday, May 18, 2017 ---

[8:02:34 PM] Маргулис:
Я поняла, в чем твоя проблема
 Я чувствую, что к себе ты относишься лучше, чем ко мне
 Меня это принципиально не устраивает
 Ты все ещё заблокирован
 Не отвечай

[9:49:08 PM] Маргулис:
На питру идёшь?

[9:52:37 PM] Михаил:
Нет

[9:52:49 PM] Маргулис:
Почему?
 Тебе не кажется, что она поставит тебе 0?

[9:53:53 PM] Михаил:
У меня в голове сидит идея отписаться английского.

[9:58:49 PM] Маргулис:
Не отписывайся

[10:00:01 PM] Михаил:
Ну я не знаю, у меня ненависть к изучению английского давно, нужно с этим бороться.

--- Friday, May 19, 2017 ---
Incoming Call 2:49:29

[12:50:22 AM] Маргулис:
169 минут

[3:45:56 PM] Маргулис:
Встреть меня после проективки

[4:39:38 PM] Маргулис:
Я расстроюсь, если ты ушёл

[6:40:08 PM] Маргулис:
Сколько вас было в шляпе?

[7:22:28 PM] Михаил:
Я пошёл на семинар по логике. Не понимаю, почему не ходил на него раньше.

[8:56:25 PM] Михаил:
http://www.world-art.ru/lyric/lyric.php?id=7692
 http://www.world-art.ru/lyric/lyric.php?id=7913

[10:54:03 PM] Маргулис:
Самое время сказать, что Бродский весьма дурновкусен и я его не люблю и никогда не любила

[10:59:35 PM] Михаил:
Это самый частый поэт, стихи которого ты мне присылала и давала читать.

[10:59:53 PM] Маргулис:
И?
 Кстати, ты часто меня поправляешь, так что я скажу, что поэты не бывают "частыми"

[11:01:36 PM] Михаил:
Ладно, ничего. Бог с ним.

[11:01:42 PM] Маргулис:
Мне нравится несколько стихов
edited 
Штук 20, может быть, лень считать
 (Я с телефона печатаю, не попадаю по кнопкам толстыми большими пальцами)
 Но он очень много писал
 И большинство является явным перебором этого жуткого стиля
 Я не люблю поздние его стихи и их только ты мне кидаешь периодически

[11:04:06 PM] Михаил:
Понятно

[11:04:16 PM] Маргулис:
Ты обиделся,
 ?

[11:04:20 PM] Михаил:
Да

[11:04:27 PM] Маргулис:
Почему?

[11:05:37 PM] Михаил:
На самом деле, я уверен, что я прочитал меньше 20 его стихов и не могу вполне о нем судить. Просто пока что он мне очень нравится.

[11:06:06 PM] Маргулис:
Просто он очень попсовый
 Это не аргумент за или против, но в некотором роде индикатор

[11:07:06 PM] Михаил:
Попсовый - это как?

[11:08:14 PM] Маргулис:
Это когда его любят все четырнадцатилетние девочки
 Да брось, не обижайся

[11:09:32 PM] Михаил:
Слушай, я ведь недавно разговаривал с двумя семиклассницами, и они его любят.

[11:10:15 PM] Маргулис:
Чтд

[11:11:01 PM] Михаил:
Я должен быть более критичным, чем я есть.

[11:12:01 PM] Маргулис:
Он, кстати, пошловат в прямом смысле слова
 Прочитай дебют

[11:16:50 PM] Михаил:
Этот стих не вызвал у меня протеста.

[11:17:17 PM] Маргулис:
Но он все равно пошловат

[11:21:18 PM] Михаил:
Этот стих бесчувственный.

[11:22:31 PM] Маргулис:
Нет, я думаю, он весьма точный, но для этих персонажей в жизни нет той пошлости, которую он вытягивает из ситуации

[11:24:03 PM] Михаил:
Я смутно понимаю значение слова пошлость, вот.

[11:26:56 PM] Маргулис:
Я пытаюсь сформулировать

[11:27:12 PM] Михаил:
Еще мне нужно научиться любить поэтов за что-то, а не безусловно.
 Это несложно наверняка

[11:28:19 PM] Маргулис:
Ну ты же понимаешь, что в нимфоманке почти нет пошлости. Так что сформулировать сложно
 Мне не нравится момент в больнице
 И мне не нравится её параллели с вавилонской блудницей

[11:31:55 PM] Михаил:
Я опасаюсь написать глупость
 Эти моменты мешают тебе воспринимать ее как положительного героя, наверное? Кажется, что в течение всех фильмов она и режиссёр пытаются убедить зрителя, что она сволочь. Я до последнего ей сочувствовал, но она убила того мужчину. Этого я ей не простил. И от этого драма героини получается очень глубокая.

[11:36:28 PM] Маргулис:
> Михаил Зыбин
> Эти моменты мешают тебе воспринимать ее как положительного героя
Разумеется, не мешают
> Михаил Зыбин
> Эти моменты мешают тебе воспринимать ее как положительного героя
Ты хоть что-то видишь?
 Они не пытаются никого в этом убедить
 Ни в одной детали
 Она правильно сделала, что убила его
 Я все это тебе уже говорила
 Я очень её люблю

[11:38:20 PM] Михаил:
Давай не будем обсуждать фильмы в телеграме.

[11:39:14 PM] Маргулис:
Мне просто кажется, что до тебя не доходит посыл, заложенный в фильме
 Причём очень часто
 Ты проецируешь себя на персонажей, кстати
 Это очевидно по посторонним

[11:39:51 PM] Михаил:
О, это чистейшая правда.

[11:39:51 PM] Маргулис:
А так нельзя воспринимать искусство

[11:40:43 PM] Михаил:
Я не буду. Я был себе слишком интересен. Теперь меньше.
> Маргулис
> Ты проецируешь себя на персонажей, кстати
И на людей тоже.
 Я тебя обожаю. Ты так хорошо меня видишь.

[11:42:30 PM] Маргулис:
Ты дополняешь ситуацию своими смыслами, которые не заложены автором, и искажаешь посыл
 Я так делаю с ламарком
 Мандельштама

[11:45:29 PM] Михаил:
Искажаю, ух, как искажаю.

[11:46:00 PM] Маргулис:
Я уверена, что триер любит героиню

[11:47:49 PM] Михаил:
Я хочу сказать, что он хочет показать, что очень мало людей могут ее полюбить.
 Ты точно хочешь обсуждать фильмы в телеграме?

[11:48:29 PM] Маргулис:
Что мало могут её понять
 Нет, не очень

[11:49:02 PM] Михаил:
Ну и я.
 Ты на выходных ложишься раньше или позже обычного?

[11:50:26 PM] Маргулис:
Позже

[11:51:56 PM] Михаил:
А я раньше.

[11:52:06 PM] Маргулис:
Спокойной ночи

[11:52:30 PM] Михаил:
Спокойной ночи.

[11:55:17 PM] Маргулис:
Понимаешь, я не тебя хорошо понимаю, это просто общее место, в этом нет ничего необычного

--- Saturday, May 20, 2017 ---

[5:53:55 AM] Маргулис:
То, как он полез к Джо после всего её рассказа по крайней смешно и нелепо, но на самом деле, гораздо больше. Это предательство, это смешно до недопустимой пошлости и нелепо до омерзительной приземленности. Она никогда не опускалась до того, кем он был, кем он себя показывает. Это преступление и он был достоин этого убийства
 И она не стала от этого хуже
 А ещё режиссёр этим показал, что человека, который смог бы её понять, не было и нет в её жизни
 В ней нет ничего, за что ты  имел бы право её судить

[1:29:31 PM] Михаил:
Ты права.
 По предыдущей теме: Мастер и Маргарита - очень попсовая вещь. Ты это тоже понимаешь. 
Я в этот раз не смог дочитать.
 Я сам, не зная твоего мнения, хотел найти в Бродском что-то, что бы мне не понравилось.

[1:41:17 PM] Маргулис:
Хорошо

[7:03:53 PM] Маргулис:
Я бы сказала, что ему надо было избавиться от хэппи эндом и подчеркнуть её байроническое одиночество
 Я посмотрела фильм манхэттен и фильм прочь
 Скажем, одиночество — её душа и её судьба

[8:45:02 PM] Маргулис:
Посмотри манхэттен. Мне есть, что о нем сказать
 Он так себе

[9:38:02 PM] Михаил:
Хорошо. Ты читала Софью Парнок?

[9:38:31 PM] Маргулис:
Да

[9:39:07 PM] Михаил:
Я сегодня читал и еще почитаю.

[9:39:44 PM] Маргулис:
Посмотри манхэттен, мне есть, что о нем сказать
> Михаил Зыбин
> Хорошо. Ты читала Софью Парнок?
Она никакая

[9:42:41 PM] Михаил:
Не пиши о ней.

[9:43:04 PM] Маргулис:
Одни сплошные никакие фильмы. У меня завышенные ожидания, видимо. Хорошо.
 Просто это выглядит так, словно после моего замечания ты решил найти кого-то не очень известного
 Почитай Ходасевича, Мережковского, и человека, которому посвящён сорокоуст Есенина, не помню фамилию
 А ты делал 8 лист по геометрии, кстати?

[11:08:41 PM] Маргулис:
Мне грустно
 Почему ты не проверяешь телеграм?
 Вчера ты похвалил, что я понимаю тебя
edited 
Но я просто увидела вполне типичную черту

[11:26:47 PM] Маргулис:
Я обиделась, кстати
 Ты должен был предупредить, что уходишь, и дождаться моего ответа
 Вдруг я хочу сказать что-то важное
 Завтра не пиши.
 В общем, до личной встречи. Принеси книгу.
 В понедельник.
 Я не прочитаю тебя сегодня-завтра.

--- Sunday, May 21, 2017 ---

[10:26:13 PM] Маргулис:
Миш
 Книгу можешь не нести
 Почитай пока
 Я была в пушке

[10:27:11 PM] Михаил:
Я не уверен, что успею посмотреть Манхэттен.

[10:27:35 PM] Маргулис:
До 19:50, нас даже выгонять начали

Да, это не обязательно, разумеется

[10:28:19 PM] Михаил:
А пришла во сколько?

[10:29:10 PM] Маргулис:
Ну часа в 2
 Может, в 2:30
 Скорее в 2:15

[10:30:26 PM] Михаил:
Я там был часа три.

[10:31:35 PM] Маргулис:
Я не знаю, почему так долго вышло
 Я сначала предприняла обход выставки, и, видимо, это затянулось
 Иногда меня даже удивляет моя память. Я очень хорошо помню, что и где висело на почти что всех выставках, где я была
 Под скульптурой был значок, что её можно трогать. Я потрогала. Ко мне подошла смотрительница и сказала, что я не слепая
 Это очень смешно
 Там три экспоната для слепых, какое-то издевательство

[10:35:11 PM] Михаил:
Да. Я тоже потрогал, мне сказали, что экспонаты нельзя трогать.

[10:39:40 PM] Маргулис:
Я наконец-то полностью прочувствовала такое легендарное направление, как венский акционизм

[10:40:45 PM] Михаил:
А я нет.
edited 
[10:40:52 PM] Маргулис:
А видел фильм?
 Я успела посмотреть толькл один, потому что мне лень было ждать начала
 У остальных
 А когда я была в последнем зале, их уже не крутили

[10:42:04 PM] Михаил:
Нет

[10:42:33 PM] Маргулис:
Там было в зале с распятым обрубком
 Очень крутой фильм. Очень противный, болезненный, крутой фильм. Мне это было нужно.

[10:43:29 PM] Михаил:
Помню, было. Я не досмотрел.

[10:44:20 PM] Маргулис:
> Маргулис
> Там было в зале с распятым обрубком
Я весьма тактична
 Потрясающая выставка
 Я расширила свои познания
 И в скульптуре тоже

[10:46:51 PM] Михаил:
Я рад. Я тоже много нового узнал. Ты лучше ее восприняла, чем я, наверное.

[10:48:21 PM] Маргулис:
Жалко, что у меня мало времени на конец осталось. Я опять сошла с ума и слишком долго проторчала в одном зале, который был так себе

[10:48:42 PM] Михаил:
В каком?

[10:51:45 PM] Маргулис:
[Photo]
 [Photo]
 В этом

[10:54:10 PM] Михаил:
Там еще афиши вроде.

[10:54:37 PM] Маргулис:
Деколлажи?
 Да
 Ты все читаешь на выставках?
 Я, видимо, долго читала кучу текста. Лестницу я ещё как-то помиловала, читала часть, а там, где был глобус, я читала все

[10:57:10 PM] Михаил:
Я читал все, но примерно четверть забывал сразу.

[10:58:10 PM] Маргулис:
Я тормознутая и иногда меня клинит
 Даже когда я прохожу зал быстро, к меня включается что-то и я начинаю запоминать его очень хорошо
 Я забыла сфотографировать вторую часть длинной подписи

[11:02:29 PM] Михаил:
Какая длинная подпись?

[11:03:51 PM] Маргулис:
К одной фотке
 Ну, там ясна суть
 Просто я как раз это прочитать не успела
 Ну, что я могу сказать — придётся ехать в Будапешт и смотреть на эти вещи снова

--- Monday, May 22, 2017 ---

[12:50:00 AM] Маргулис:
Кстати, я сегодня по-настоящему поняла весь потенциал цвета
 По той картине Матисса, которая была на выставке
 После этой выставки мне захотелось уйти с матфака
 Ну это же настоящая жизнь
 Я тоже хочу делать выставки)
 И писать о картинах

[12:54:33 AM] Михаил:
Мне вчера захотелось уйти, и я вообще пожалел, что перешел в 179. Из-за статей про сравнение символизма с акмеизмом. Это намного интереснее тензорной алгебры.

[12:55:25 AM] Маргулис:
Ага.
 Чёрт.
 Наверное, я завтра скажу родителям, что жалею, что пошла на матфак.
 Нет, надо было сегодня
 Они подумают, что что-то случилось
 Потом. Напишу курсовую и скажу

[12:57:03 AM] Михаил:
Они спят уже?

[12:57:46 AM] Маргулис:
Да, кажется. Даже если нет, они в кровати и читают. Сейчас неудобно. Нужен подходящий момент, за столом,
 С другой стороны, я могу закончить матфак. Многие личности, связанные с анархией или искусством, имели хорошее техническое образование

[12:59:49 AM] Михаил:
Да
edited 
[1:00:07 AM] Маргулис:
Всего три года осталось..
Но курсовая отделяет меня от этих трёх лет
 Ладно, все, я возьму себя в руки завтра
 Отсидела же я в этой проклятой школе, там было хуже, чем на матфаке

[4:14:04 AM] Михаил:
Я подозреваю, что не пойду к Ракитину.

[9:51:04 AM] Маргулис:
Почему? Ты не спал?
 А в тот раз ты был? Он давал список академ слов?

[10:48:24 AM] Маргулис:
А к алгебре ты придёшь?
 Ты всё проспал?

[11:03:04 AM] Михаил:
Я в 4:20 лег, хотел подольше поспать. В тот раз я был, список он не давал. К алгебре я приду.

[11:03:18 AM] Маргулис:
Я не пошла на ракитина
 Тебе не кажется, что в 11:05 уже бесполезно об этом писать?
 Ты в метро?

[11:03:56 AM] Михаил:
Нет, в троллейбусе

[11:04:02 AM] Маргулис:
О боже
 Тебе ещё и на нем ехать
 Кошмар
 Я читаю Гегеля

[11:04:43 AM] Михаил:
Меня не печалит троллейбус
 Я вчера прочитал с 41 до 63.

[11:05:47 AM] Маргулис:
> Михаил Зыбин
> Меня не печалит троллейбус
А меня трамвай
 Это шутка про гумилева и она не получилась
 Я пешком хожу
 Кстати, а чем тебе и моему бывшему нравится Гумилёв?
 Я кроме трамвая не видела ни одного хорошего его стиха

[11:07:19 AM] Михаил:
Я не знаю, мне он уже не нравится

[11:07:26 AM] Маргулис:
Так, конечно, не бывает, но я не видела
> Михаил Зыбин
> Я не знаю, мне он уже не нравится
Как тебя жизнь помотала
 Цветаева
 Какой-нибудь предок мой был — скрипач,
Наездник и вор при этом.
Не потому ли мой нрав бродяч
И волосы пахнут ветром!

Не он ли, смуглый, крадет с арбы
Рукой моей — абрикосы,
Виновник страстной моей судьбы,
Курчавый и горбоносый.

Дивясь на пахаря за сохой,
Вертел между губ — шиповник.
Плохой товарищ он был, — лихой
И ласковый был любовник!

Любитель трубки, луны и бус,
И всех молодых соседок…
Еще мне думается, что — трус
Был мой желтоглазый предок.

Что, душу черту продав за грош,
Он в полночь не шел кладбищем!
Еще мне думается, что нож
Носил он за голенищем.

Что не однажды из-за угла
Он прыгал — как кошка — гибкий…
И почему-то я поняла,
Что он — не играл на скрипке!

И было все ему нипочем, —
Как снег прошлогодний — летом!
Таким мой предок был скрипачом.
Я стала — таким поэтом.
 Мне это нравится
 Ну вот, а мой не бродяч
 И ветром волосы не пахнут
 Я слишком домашняя

[11:12:13 AM] Михаил:
Я сейчас зайду в метро.

[11:12:18 AM] Маргулис:
Ок

[2:34:49 PM] Маргулис:
Ты же чттал это на Арзамасе?
 А может лучшая победа
Над временем и тяготеньем —
Пройти, чтоб не оставить следа,
Пройти, чтоб не оставить тени

На стенах…
                     Может быть — отказом
Взять? Вычеркнуться из зеркал?
Так: Лермонтовым по Кавказу
Прокрасться, не встревожив скал.

А может — лучшая потеха
Перстом Севастиана Баха
Органного не тронуть эха?
Распасться, не оставив праха

На урну…
                 Может быть — обманом
Взять? Выписаться из широт?
Так: Временем как океаном
Прокрасться, не встревожив вод…

[6:40:27 PM] Михаил:
Не читал раньше. Хороший стих.

[9:14:49 PM] Маргулис:
Читаешь Гегеля?
 Читай

[10:03:45 PM] Михаил:
В целом, читаю, да.

[10:27:39 PM] Маргулис:
Мне понравилось, что ты оценил Бартона Финка
 Это комплимент моему вкусу четырёхлетней давности, так что я знаю, что я и тогда была ничего
 Слушай
 Я не люблю обсуждать фильмы, потому что после этого я помню только то, что говорил ты. Ты заражаешь меня своей теорией

[10:35:50 PM] Михаил:
У меня есть мысли о Бартоне Финке, но я и сам хочу оставить их при себе.

[10:36:02 PM] Маргулис:
Как я объясняла тебе поведение Патриции из последнего дыхания?
 Я говорила, что она действительно хотела проверить, любит ли его?

[10:36:44 PM] Михаил:
Нет

[10:36:56 PM] Маргулис:
А
 Я вспомнила
 Она поняла, что у неё в жизни есть смысл, любимая работа, будущее, а с ним она всё это потеряет.

[10:37:47 PM] Михаил:
Да

[10:37:55 PM] Маргулис:
Ну да, это так

[10:38:18 PM] Михаил:
Я согласен

[10:38:42 PM] Маргулис:
Но все равно она его не любила по-настоящему
 Я думаю, что она надеялась, что он сбежит от неё и полиции после этого
 Она же просила его уходить?

[10:39:45 PM] Михаил:
Просила
 Или нет
 Я не уверен

[10:40:19 PM] Маргулис:
Мне тоже кажется, что она говорила, что поздно бежать
 Ну, неважно. Она просто хотела, чтобы он каким-либо образом исчез
 А ты проходил тест iq когда-нибудь?

[10:41:27 PM] Михаил:
Нет

[10:41:29 PM] Маргулис:
И я нет
 Боюсь расстроиться

[10:42:17 PM] Михаил:
Я мало об этом знаю и не доверяю.
 Мне понравилось начало Утра акмеизма.

"При огромном эмоциональном волнении, связанном с произведениями искусства, желательно, чтобы разговоры об искусстве отличались величайшей сдержанностью."

[10:51:59 PM] Маргулис:
Миша
 Г
 Е
 Г
 Е
 Л
 Ь
 
 
 Ты, кстати, вовремя, потому что я чуть было не начал читать Мцырей.

[10:59:46 PM] Маргулис:
Слушай
 Мне не очень нравятся в раннем Мандельштаме вечные отсылки к древней Греции
 А тебе?

[11:00:49 PM] Михаил:
Мне нравятся пока.
 Я чувствую, что не зря читал Илиаду и так далее.

[11:03:35 PM] Маргулис:
Они невкусные
 В них нет яркости
 В древнегреческой тематике
edited 
Не знаю, почему в моем сознании это всё принимает такие формы
 Но я не вижу в этом чувства
 Для меня древняя Греция — это что-то очень статичное и спокойное
 Даже какие-то знания её истории не спасают
 Я из тех людей, у которых всякие варвары вызывают больший восторг
 А Рим — статичное, спокойное и очень жестокое.

[11:08:33 PM] Михаил:
Я тебя понял. Я еще не имею мнения на этот счет.

[11:08:53 PM] Маргулис:
Моё довольно необоснованно
 Что-то тут не так
 Ладно, мне лень думать, как правильно
 С одной же н, да?

[11:10:21 PM] Михаил:
Например, книга "Метаморфозы" - это точно не спокойное и не статичное.
 Буколики и Георгики - наверное, но я не читал.

[11:12:05 PM] Маргулис:
Я почитаю
 Но сейчас гегель

[11:35:15 PM] Маргулис:
Кстати, а что было на философии?
 Неужели Гегель?

[11:38:09 PM] Михаил:
Меня очень глубоко затронуло то, что говорил лектор. Основной вопрос был "Что значит быть?"
 В конце он читал нам из Сартра.

[11:38:34 PM] Маргулис:
Гм

[11:38:44 PM] Михаил:
Мне надо читать Сартра.
 Летом

[11:38:58 PM] Маргулис:
О нет. Сартра не нужно читать
edited 
Сартр — это общепризнанно худший французский писатель

[11:39:56 PM] Михаил:
Это же философ-экзистенциалист.

[11:40:11 PM] Маргулис:
И общепризнанно худший французский писатель
 Ты что, никогда о нём не слышал раньше?...
 Ну ок

[11:40:40 PM] Михаил:
Слышал

[11:41:30 PM] Маргулис:
Да нет, читай конечно, но я не могу хорошо к такому относиться
> Маргулис
> Сартр — это общепризнанно худший французский писатель
Если хочешь, худший французский философ
 Надо признать за французами наличие хороших

[11:42:57 PM] Михаил:
Ему давали нобелевку в 1964
 По литературе
 Может, ты и знаешь

[11:47:00 PM] Маргулис:
Мало ли кому давали нобелевку по литературе
 Бунину тоже давали нобелевку по литературе. То, что он умеет писать, не делает его хорошим писателем. Это я о Бунине.
 Нобелевка слишком не аполитична
 Ну это к слову

[11:49:03 PM] Михаил:
Ну да, я тут читаю, Сартр из-за этого от нее и отказался.

[11:50:08 PM] Маргулис:
Каннский кинофестиваль не даёт гарантию хорошести фильма, Нобелевка не даёт гарантии хорошести писателя, это всё субъективно и во многих случаях они бы сами в новом составе её бы не дали снова тому человеку

[11:51:11 PM] Михаил:
Я не спорю.

[11:53:25 PM] Маргулис:
Но Гамсун клёвый
 Кстати, у Бродского же тоже Нобелевка

[11:53:57 PM] Михаил:
Вообще дают кому попало.

[11:54:16 PM] Маргулис:
Но эссе у него бывают хорошими

--- Tuesday, May 23, 2017 ---

[12:08:57 AM] Михаил:
Мне стало нравиться изучать Гегеля. Сам Гегель не стал.

[12:17:38 AM] Маргулис:
Хорошо

[12:19:53 AM] Михаил:
Я в 9-10 классах думал о себе как о философе, хотя книг об этом не читал.

[12:20:18 AM] Маргулис:
Все так думают в подростковом возрасте
 Я пораньше так думала
 В 7-8
 Нет, в 8-9

[12:21:24 AM] Михаил:
Я мало знаю людей

[12:21:35 AM] Маргулис:
В 7 я считала себя глупой
> Михаил Зыбин
> Я мало знаю людей
По количеству или качеству?

[12:22:00 AM] Михаил:
По качеству

[12:22:09 AM] Маргулис:
Наверное
 Я тоже не очень хорошо знаю
 В некоторых сферах
 Мне кажется, я не понимаю людей, которые делают какие-то мелкие гадости. Я не вижу смысла в мелких обманах. У меня, когда я была маленькая, одна девочка украла кружку, потому что она ей нравилась. Я не понимаю этого
 Причём просто сказала мне, что это её кружка
 И её мама это подтвердила потом, это смешно
 А ещё я однажды оставила на парте конфеты, а их забрала другая девочка 
Смешная у меня жизнь
 Сейчас эта девочка учится в мгимо на коррупционера

[12:28:43 AM] Михаил:
Ну, просто глупость и несовершенство людей.

[12:29:07 AM] Маргулис:
Гм, она ведь мне ещё и какие-то нитки не отдала на труде
 Однако, я злопамятная

[12:29:31 AM] Михаил:
Ага

[12:29:47 AM] Маргулис:
Просто память хорошая
 А ещё мне было плохо на всех пробных ЕГЭ по математике
 Я даже не написала нормально ни один из тех, что приходили официально.
 Потому что меня то тошнило, то болела голова

[12:31:42 AM] Михаил:
Соматические симптомы невроза.

[12:32:02 AM] Маргулис:
Ага
 А русский и информатику я, по-моему, всегда писала ровно на один и тот же первичный балл
 А физику пробные я писала баллов на 75-80, а на экзамене было 100
> Маргулис
> А физику пробные я писала баллов на 75-80, а на экзамене было 10
Потому что я не повторяла ничего к пробным
 И не помнила либо механику, либо оптику, либо переменные токи

[12:34:04 AM] Михаил:
> Маргулис
> А физику пробные я писала баллов на 75-80, а на экзамене было 10
У меня так же было, но я готовился к пробным.

[12:34:34 AM] Маргулис:
Из-за пробных мои одноклассники перестали считать меня очень умной, кажется
 Потому что был один не очень глупый парень, который очень хотел быть популярным и делал им всё
 И писал физику ровно на 1 балл лучше, чем я
 А на егэ на 8 баллов хуже, кажется. У остальных было где-то по 80
> Маргулис
> И писал физику ровно на 1 балл лучше, чем я
Но он готовился к ЕГЭ

[12:37:15 AM] Михаил:
Я не очень хочу вспоминать эти дела.

[12:37:19 AM] Маргулис:
Ладно
 Я просто школу вспомнила
 Противный был год
 Я к тому, что ко мне плохо относились

[12:38:29 AM] Михаил:
Ты хочешь мне звонить?

[12:38:33 AM] Маргулис:
Нет
 Я плохо готова
 Ты завтра пойдёшь на лекцию?
 По матану
 А ты сколько прочитал Гегеля?
 Я могу позвонит в 1:00, но это поздновато?

[12:41:17 AM] Михаил:
Что-то до 141. Он же начинается с добра, и я выяснял, что это. Звони, это не поздно.

[12:41:56 AM] Маргулис:
Слушай, а ведь у меня безумный почерк
 Мне поставят 0
 За контру

[12:42:54 AM] Михаил:
[Photo]

[12:46:07 AM] Маргулис:
Ничего не видно

[12:46:29 AM] Михаил:
Я о том, что у меня плохой почерк.
 Это я пишу для себя.
 Пишу я это в себе или не в себе?

[12:57:17 AM] Маргулис:
Слушай, подожди ещё немножко, хочу дочитать до 198
 Я на 194

[1:00:58 AM] Михаил:
Оказалось, что абстрактный - это не то, что я думал.

[1:04:40 AM] Маргулис:
Я медленно читаю, прости. Я просто вникаю

[1:05:08 AM] Михаил:
Ну я-то тоже читаю Гегеля сейчас.

[1:11:27 AM] Маргулис:
Я звоню?

[1:41:38 AM] Маргулис:
[Photo]
 [Photo]
 [Photo]
Incoming Call 41:41

[2:18:13 AM] Маргулис:
Я узнала про для-себя-бытие
 Но из-за диалектики не могу это объяснить
 Потому что не могу использовать слово противоположность
 Такое выражение придумано как антоним бытия-для-других, то есть наличного бытия, для себя бытие включает это осознание себя

[2:20:01 AM] Михаил:
Мне грустно.

[2:20:49 AM] Маргулис:
Какие-то моменты
 Ерунда какая

[2:21:14 AM] Михаил:
Ерунда
 Это правда

[2:21:43 AM] Маргулис:
Я боюсь, что не хочу идти завтра вообще
 Я, наверное, приду на семинар по матану и уйду
 Наличное бытие — это не в себе бытие
 Оно конкретно
 А в себе — нет
 Что такое дурное и отрицательное?
 Дурная и отрицательная бесконечность

[2:30:50 AM] Михаил:
Я не готов говорить об этом.

[2:31:56 AM] Маргулис:
Я поняла
> Маргулис
> Дурная и отрицательная бесконечность
Это бесконечность прогресса, вечного повторения одного и того же

[3:36:49 AM] Михаил:
$https://vk.com/wall-126365649_759$
VK
МатФак НИУ ВШЭ 1 курс 2016
:boom: ЭПИЧНОЕ ВОЗВРАЩЕНИЕ ГЕРОЯ :boom: С 22 во вторникам второй парой снова будут лекции по геометрии с Шварцманом.
 Не знаю, знаешь ли ты.

[10:41:59 AM] Маргулис:
Не знала

[2:45:36 PM] Маргулис:
Отрицание — возможность противоположного понятия

[5:57:03 PM] Маргулис:
Хотела кое-что рассказать
 Но боюсь, что ты совсем бросишь математику

[6:02:21 PM] Михаил:
Мне сейчас нужно немаленькое нравственное усилие, чтобы ей заниматься.

[6:04:04 PM] Маргулис:
Понимаю тебя

[8:08:56 PM] Маргулис:
Как там гегель?

[9:21:18 PM] Михаил:
Про государство и общество понятнее, чем про мораль и собственность.

[9:49:24 PM] Маргулис:
270 параграф.....................
 Он такой огромный
 Я не увидела своего прогресса за сегодня
 Как определяется особенность?
 Скажи, я плохо на тебя влияю?
 Мне стыдно , что ты перестал много заниматься учёбой

[9:57:02 PM] Михаил:
Ты мне здорово помогла разобраться в себе. Ты помогаешь мне выкидывать кучу невротического мусора из головы. Я вспоминаю и понимаю, кто я на самом деле такой.

[9:57:26 PM] Маргулис:
А я теперь занимаюсь по часам
 Час назад придумала
 Ставлю таймер
 Сейчас нарушаю

[9:58:04 PM] Михаил:
Я оцениваю знакомство с тобой как наиболее значительное событие в моей жизни.
 Я именно тебя ждал с конца 9 класса.
 Надо матан делать. Мне Саня много рассказал.

[10:00:12 PM] Маргулис:
Скажи, когда можно созвониться

[10:01:07 PM] Михаил:
Не знаю, часа через два.

[10:03:17 PM] Маргулис:
Ок
 В мое расписание плохо умещается еда

[11:28:48 PM] Маргулис:
Если душа родилась крылатой —
Что ей хоромы и что ей хаты!
Что Чингисхан ей — и что — Орда!
Два на миру у меня врага,
Два близнеца, неразрывно-слитых:
Голод голодных — и сытость сытых!

--- Wednesday, May 24, 2017 ---

[12:09:37 AM] Маргулис:
Можно я умру
 В час созвонимся, ок?
 Или не ок?

[12:10:05 AM] Михаил:
Ок. Я решаю матан.

[12:10:12 AM] Маргулис:
Мне нужно будет стресс снять
 Разговором
 Это ужасный стресс
 Гегель

[12:10:44 AM] Михаил:
Я понимаю

[1:26:43 AM] Михаил:
Ты как?

[1:29:25 AM] Маргулис:
Я тут
 Прости
 Я умерла
 Подожди ещё 4 минуты

[1:40:38 AM] Михаил:
Маша?

[1:44:13 AM] Маргулис:
Ты торопишься?

[1:44:47 AM] Михаил:
Начинаю хотеть спать.
Incoming Call 43:07

[2:34:23 AM] Михаил:
После третьей пары я пойду в 179 на последний звонок.

[2:34:30 AM] Маргулис:
Слушай
 Можно ещё на одну минуту?
Incoming Call 5:04

[2:03:51 PM] Маргулис:
В Европе холодно, в Италии темно
 У меня все смешалось
 Куда мне деться в этом январе? 
Открытый город сумасбродно цепок. 
От замкнутых я,что ли,пьян дверей? 
И хочется мычать от всех замков и скрепок. 

И переулков лающих чулки, 
И улиц перекошенных чуланы, 
И прячутся поспешно в уголки 
И выбегают из углов угланы. 

И в яму, в бородавчатую темь 
Скольжу к обледенелой водокачке, 
И, спотыкаясь,мертвый воздух ем, 
И разлетаются грачи в горячке, 

А я за ними ахаю, крича 
В какой-то мерзлый деревянный короб: 
- Читателя! Советчика! Bрача! 
На лестнице колючей - разговора б!
edited 
В Европе холодно. В Италии темно.
Власть отвратительна, как руки брадобрея.
О, если б распахнуть, да как нельзя скорее,
На Адриатику широкое окно.

Над розой мускусной жужжание пчелы,
В степи полуденной — кузнечик мускулистый.
Крылатой лошади подковы тяжелы,
Часы песочные желты и золотисты.

На языке цикад пленительная смесь
Из грусти пушкинской и средиземной спеси,
Как плющ назойливый, цепляющийся весь,
Он мужественно врет, с Орландом куролеся.

Часы песочные желты и золотисты,
В степи полуденной кузнечик мускулистый — 
И прямо на луну влетает враль плечистый...

Любезный Ариост, посольская лиса,
Цветущий папоротник, парусник, столетник,
Ты слушал на луне овсянок голоса,
А при дворе у рыб — ученый был советник.

О, город ящериц, в котором нет души, — 
От ведьмы и судьи таких сынов рожала
Феррара черствая и на цепи держала,
И солнце рыжего ума взошло в глуши.

Мы удивляемся лавчонке мясника,
Под сеткой синих мух уснувшему дитяти,
Ягненку на дворе, монаху на осляти,

Солдатам герцога, юродивым слегка
От винопития, чумы и чеснока, — 
И свежей, как заря, удивлены утрате...
 Это два разных стихотворения, а я их смешала
 Очень ритм одинаковый

[3:27:33 PM] Маргулис:
Слушай-ка
 А как же духовность
 Ты сегодня попросил скинуть тебе порно с похожими на меня девушками
 А как же только душа?
 Кстати, у нас совершенно не совпадают вкусы
 Я люблю пожёстче
 Аааа, боже мой, ты же ещё не знаешь

[5:29:25 PM] Маргулис:
Неудачно я с одним человеком поговорила
 Сегодня утром
 Опять кто-то говорил со мной, как с маленькой
 Кто-то, для кого философия началась с приходом Чернавина
 Бесит меня
 Больше не буду с ним разговаривать

[7:16:09 PM] Михаил:
> Маргулис
> Ты сегодня попросил скинуть тебе порно с похожими на меня девушк
Я зря, я не должен был, извини.
> Маргулис
> Неудачно я с одним человеком поговорила
Поподробнее объясни, пожалуйста.

[7:19:20 PM] Маргулис:
> Михаил Зыбин
> Поподробнее объясни, пожалуйста.
Потом
> Михаил Зыбин
> Я зря, я не должен был, извини.
Да я шучу

[7:19:36 PM] Михаил:
Ладно

[7:20:08 PM] Маргулис:
Витгенштейн
 Надо читать Витгенштейна

[7:20:41 PM] Михаил:
Очень вероятно.

[11:53:00 PM] Маргулис:
Я все-таки убью кого-нибудь на матфаке

[11:54:00 PM] Михаил:
Почему?

[11:57:33 PM] Маргулис:
Из-за того человека утром
 Его зовут Данат

[11:57:54 PM] Михаил:
Есть такой.

[11:58:09 PM] Маргулис:
Я знаю, я много раз с ним говорила
 Мы утром случайно часто встречаемся

[11:59:26 PM] Михаил:
Расскажешь подробнее?

--- Thursday, May 25, 2017 ---

[12:01:30 AM] Маргулис:
Ну, если прибегать к необоснованным оскорблениям, то он попал под влияние чернавина и все его философские воззрения поверхностны и появились полгода назад
 Но говорит он так, словно он разбирается в искусстве и в моём любимом 68 году, от которого он услышал от чернавина, лучше, чем я
 И не даёт мне вставить свою мысль
 Потому что мы шли втроём с гаицгори
 А вы, мерзкие мужчины, всегда затыкаете женщин. Это шутка
 Меня это бесит
 И он неправильно выделяет его для себя
 Он совсем не о том и пафос 68 года не в том

[12:04:49 AM] Михаил:
Несовершенство и глупость совершенно не должны удивлять - это совершенно неудивительно.
 Черт, я забыл про середину предложения в конце.

[12:05:18 AM] Маргулис:
Ну он, по-моему, не глупый, меня просто, как всегда, задела какая-то фраза
 И я злюсь
 Я не уверена, что он сам действительно читал всех, о ком говорит

[12:06:37 AM] Михаил:
Так дело во фразе или в том, что тебя не устраивает его мировоззрение?
 Я думаю, он не стоит того, чтобы из-за него злиться.
 Это незачем.

[12:09:25 AM] Маргулис:
В том, что я сказала такую очевидную вещь, как "если общество в нынешнем состоянии несовершенно и не готово к идеальному мироустройству, это не говорит о том, что общество после нравственного развития не сможет достичь этого мироустройства", и сослалась на Гегеля, а он ответил, что существуют мыслители помимо Гегеля
 Это, конечно, наверняка мысль Гегеля, но я её не у него взяла
 Я хотела сказать, что если коммунизм лучше капитализма, то это не означает, что надо начинать его строить завтра
 К тому, что он сказал, что он аполитичен и не придерживается никакого мировоззрения на этот счёт, что он не анархист и его в этом плане 68 год не интересует

[12:13:00 AM] Михаил:
Я понял. Ну, человек как человек. Не общайся с ним, и все.

[12:13:16 AM] Маргулис:
Я не понимаю
 Он вроде бы очень начитанный
 Сегодня возил с собой огромную книгу по искусству, кстати
 Издательства музея гараж
 Там искусство рассматривалось в числе прочего с позиции психоанализа, а я это не признаю

[12:14:58 AM] Михаил:
Сами по себе книги мало значат.

[12:15:27 AM] Маргулис:
Да, разумеется
 Но люди редко читаю просто для того, чтобы пройтись глазами по страницам
 Люди читают то, что соответсвует их уровню
 Либо тому уровню, которому они хотят соответствовать, да
 Ты же не потому так отвечаешь, что боишься, что я буду с ним больше общаться, чем с тобой?

[12:17:29 AM] Михаил:
Нет

[12:17:56 AM] Маргулис:
Извини
 А ты какое-то мнение о нем уже составил?
 До меня

[12:18:29 AM] Михаил:
Дело в том, что я его ну совершенно не знаю.
 Иногда слышал обрывки его фраз, но по ним не могу ничего сказать.

[12:19:28 AM] Маргулис:
Понятно
 Просто ты сказал, что он есть, я подумала, что ты его знаешь
 Надо его подергать

[12:22:28 AM] Михаил:
В тебе есть стремление делать людей лучше?

[12:22:39 AM] Маргулис:
Определённо, да
 Во мне есть преподавательский инстинкт. Возвышенные представления о развитии человека и том, насколько и как можно в этом участвовать
 Кстати, извини
 Я часто нарушаю эти представления и пользуюсь запрещёнными приемами
 И оскорбляю Бродского и Сартра
 Читай, конечно, моё мнение ничего не значит

[12:25:17 AM] Михаил:
У меня такого нет. Я никогда не видел своего влияния на людей.

[12:27:55 AM] Маргулис:
Я однажды слышала от косвенной знакомой, что она рассталась с парнем, потому что поняла, что он мешает её развитию. Это же очень важная идея. Люди нередко расстаются потому, что один имеет слишком большое влияние на другого. Наверное, это одна из причин нашего с Васей расставания
 Меня было слишком много

[12:31:04 AM] Михаил:
Понятно. Я сейчас опять думаю, что я разочарован в людях, потому что так спокойно отношусь к глупости (чьей угодно, но не своей).

[12:31:33 AM] Маргулис:
Наоборот же
 Ты спокойно относишься, потому что разочарован
edited 
[12:33:02 AM] Михаил:
Там дурацкий синтаксис. Я вижу, что спокойно отношусь и поэтому понимаю, что разочарован.

[12:33:41 AM] Маргулис:
Да, это так

[12:34:00 AM] Михаил:
Но это же неправильно. Моя глупость ничем не лучше и не хуже другой.

[12:34:00 AM] Маргулис:
Ты знаешь меньше крутых людей, чем я, как ни странно

[12:34:55 AM] Михаил:
Да, конечно. Я бы только тебя назвал крутой из моих знакомых.

[12:35:24 AM] Маргулис:
Спасибо
 Кстати, у Гегеля была одна хорошая мысль
 Про то, что если человек никак себя не определяет, не совершает выбор в пользу каких-то воззрений, то он так и остаётся ничем и не развивается
 Она дословно не так звучала
 Но была именно про воззрения
 Это не то, о чем мы говорили в плане труда
 И в другом разделе

[12:37:42 AM] Михаил:
Можно самому выдумывать воззрения.

[12:37:44 AM] Маргулис:
Я это о Васе. Ты извини, я достала им
 Он кретин
 С огромными аналитическими способностями
 Но кретин

[12:38:42 AM] Михаил:
У тебя с ним было понимание с полуслова?

[12:38:54 AM] Маргулис:
Да
 С 1/32 слова

[12:39:42 AM] Михаил:
Почему он кретин?

[12:40:24 AM] Маргулис:
Он не любил определённых взглядов
 Они, конечно, были
 Но по малому числу вопросов
 И ему не нравилась моя убеждённость в их необходимости
 Я занята
 Ты уже очень долго не даёшь мне досмотреть фильм
 Минут 45
 Ты мне должен эти 45 минут, я могу на них тебя отвлечь в любое время

[12:41:59 AM] Михаил:
Ладно. Пока.

[12:44:20 AM] Маргулис:
Пока

[1:33:03 AM] Михаил:
Говорят, что завтра ко второй паре.

[1:33:03 AM] Михаил:
Forwarded message: 
Полина Барон [5/24/17] 
Написала Шварцману. Завтра утренника НЕ БУДЕТ.
Forwarded message: Misha Pavlovskiy [5/24/17] 
А Тихомиров то завтра будет? Утром?
Forwarded message: Алексей [5/24/17] 
Шварцмана лекция алло
Forwarded message: Anastasia Sukacheva [5/24/17] 
То есть всё-таки к 10:30? :expressionless::expressionless::expressionless:
Forwarded message: Алексей [5/24/17] 
Ну да

[1:55:47 AM] Маргулис:
Я уже это поняла
 И к ней хотела прийти
 Спасибо

[3:54:53 PM] Маргулис:
https://meduza.io/feature/2017/05/25/tesnota-kantemira-balagova-uchenik-sokurova-snyal-film-o-evreyskoy-semie-na-kavkaze-v-1990-e

Meduza
«Теснота» Кантемира Балагова: ученик Сокурова снял фильм о еврейской семье на Ка...
На Каннском фестивале в рамках программы «Особый взгляд» состоялась мировая премьера картины «Теснота» — дебютной работы режиссера Кантемира Балагова,...
 Прочитай до конца

[6:37:19 PM] Михаил:
Прочитал.
 Кстати, если позволишь, я буду писать здесь в старой орфографии.

[8:47:02 PM] Маргулис:
Давай
edited 
[10:47:24 PM] Маргулис:
Скажи, ты часто, когда видишь цвет, чувствуешь вкус?

[10:48:10 PM] Михаил:
Не бывало такого.

[10:51:00 PM] Маргулис:
Гм
 Опять я одна
edited 
Я вообще не могу смотреть на некоторые цвета, начинаю хотеть мороженное определённого вкуса
 А у Набокова такое было

[10:53:01 PM] Михаил:
Любопытное д\yatло. Давно такъ случается?

[10:53:14 PM] Маргулис:
Не знаю, вряд ли
 Откуда у тебя раскладка с ятью?

[10:53:43 PM] Михаил:
Я скачалъ позавчера.

[10:54:00 PM] Маргулис:
А, ты поэтому только тогда об этом спросил?

[10:54:37 PM] Михаил:
Да
 Теб\yat не мешаетъ это картины воспринимать, или помогаетъ?

[10:55:30 PM] Маргулис:
Не мешает или помогает?
 Смешно
 Я картины запоминаю не так
 Я это не контролирую, конечно, но я прилагаю усилия для запоминания картин
 А синестезия течёт естественно
 Ещё у цветов есть температура
 А запахи я не особо выделяю, так что запахов нет

[10:57:19 PM] Михаил:
Этого я тоже, признаться честно, не понимаю, про температуру.
 Это не связано в\yatдь со зв\yatздами?

[10:58:07 PM] Маргулис:
На самом деле, это не очень приятно — когда рот слюной наполняется аж до боли при виде какого-то цвета
 Какими звёздами?
 А
 Поняла
 Нет
 Нет, у одного цвета есть запах
 Он пахнет воском и ладаном

[10:59:18 PM] Михаил:
Это крайне любопытно. А какой?

[10:59:39 PM] Маргулис:
Непонятный цвет, который я не могу назвать красным
 Красно-розово-коричневый
 Я из-за названия этого цвета ссорилась с кучей людей
 Он ещё и фиолетовый
 [Photo]
 Плохо отобразился цвет
 Хуже другое
 Для меня лаванда пахнет кровью
 И это кошмарно
 В Англии надо было меня видеть в момент, когда я встретила лаванду

[11:02:57 PM] Михаил:
> Маргулис
> Для меня лаванда пахнет кровью
Я не знаю, что сказать.

[11:03:08 PM] Маргулис:
Ты знаешь, почему?
 Я знаю

[11:03:19 PM] Михаил:
Н\yatтъ

[11:03:47 PM] Маргулис:
Я не так много запахов выделяю
 В жизни
 Ладан, воск, лаванда с кровью, хвоя, что-то цитрусовое
 Наверное, больше, конечно
 Просто вспомнила чистый понедельник
 Поэтому сказала об этом
 В ещё я задыхаюсь каждый год в первый день мороза

[11:06:10 PM] Михаил:
Я не знаю запах крови.
 А опавшие листья, дождь, старые книги?

[11:07:31 PM] Маргулис:
Я не люблю старые книги
 А, ещё запах плохой бумаги тоже похож на кровь
 Но не старой
 Просто плохой
 У меня есть одна книга, которая пахнет кровью
 Дождь, может быть
 Листья — нет
 Я знаю запах старых книг

[11:08:52 PM] Михаил:
Почему ты знаешь запахъ крови?

[11:10:07 PM] Маргулис:
> Михаил Зыбин
> Я не знаю запах крови.
Хи-хи
 Ъ
 Просто ассоциация
 Это на запах крови
 Но вообще, я часто ещё встречаю
 И знаю настоящий

[11:11:47 PM] Михаил:
Думаю, можетъ, сд\yatлать пор\yatзъ и принюхаться.

[11:11:52 PM] Маргулис:
Нет
 Не стоит
 Я расстроюсь
 А как выглядит большая ять?

[11:12:26 PM] Михаил:
\yat

[11:12:45 PM] Маргулис:
Я их себе скинула
 В переписку с собой
 А ижица есть?

[11:13:19 PM] Михаил:
Я знаю на матфак\yat еще двухъ людей, которые д\yatлали пор\yatзы. Есть. 


[11:13:33 PM] Маргулис:
Кто эти люди?

[11:13:56 PM] Михаил:
Женя Иткина и Лиза Нестерова.

[11:14:02 PM] Маргулис:
Можно отдельно две ижицы? Лень столько удалять
 Откуда ты знаешь о них?

[11:14:24 PM] Михаил:


[11:14:31 PM] Маргулис:
Спасибо

[11:14:47 PM] Михаил:
Я разговаривалъ съ ними.

[11:14:59 PM] Маргулис:
Хорошая тема

[11:16:00 PM] Михаил:
Не только объ этомъ.

[11:16:16 PM] Маргулис:
Про лизу удивилась

[11:18:14 PM] Михаил:
Она православная, я ее расспрашивалъ про это сегодня. Мн\yat понравилась, что она говорила. Это очень романтично и красиво.

[11:18:29 PM] Маргулис:
Она верующая?
 Ты меня поражаешь
 Когда ты успел, кстати?
 После моего ухода?

[11:18:59 PM] Михаил:
Угу

[11:19:22 PM] Маргулис:
Ты же понимаешь, что тебе нельзя что-либо говорить обо мне?
 ?
 ?
 ?

[11:19:53 PM] Михаил:
Да, разум\yatется. Я и не говорилъ.

[11:20:26 PM] Маргулис:
Хорошо
 Спасибо

[11:20:54 PM] Михаил:
Я тебя не отвлекаю?

[11:21:13 PM] Маргулис:
Немного
 Я не знаю, захотелось поговорить о синестезии
 Я отключаю это, когда смотрю на картины, насколько это возможно. Нужно чувствовать настроение напрямую
 Я люблю цвета
 Люблю фильмы, в которых на них акцент. Но и чб люблю
 Какие ты любишь цвета?
 Знаешь, я ведь знаю вкус, который никогда не пробовала
 Из вина из одуванчиков
 Сливочно-лимонное мороженое
 Я по-моему сегодня не исправила корректор в этом слове, да?

[11:25:11 PM] Михаил:
А само вино изъ одуванчиковъ?

[11:25:30 PM] Маргулис:
Мне иногда бывает лень это делать и я не перечитываю свои фразы
 Не знаю
> Маргулис
> Какие ты любишь цвета?
Ну?

[11:26:20 PM] Михаил:
Пока никакіе.

[11:26:35 PM] Маргулис:
По мне не скажешь, но я в восторге от красного
 Я живу этим цветом
 Я испытываю счастье, когда его много

[11:27:15 PM] Михаил:
Мн\yat это совершенно непонятно.

[11:27:24 PM] Маргулис:
Раньше я любила цвет с той картины Матисса
 Бирюзовый

[11:27:39 PM] Михаил:
Помню

[11:27:48 PM] Маргулис:
Но тогда уже не любила
 Когда мы там были
edited 
А какое-то время я любила жёлтый
 Он был для меня противоположностью бирюзового

[11:29:17 PM] Михаил:
Теб\yat идетъ красная кофта.

[11:29:20 PM] Маргулис:
Мне кажется, он мне не очень идёт из-за цвета волос
 Желтый
> Михаил Зыбин
> Теб\yat идетъ красная кофта.
Спасибо, я тоже так считаю
 А ещё я без ума от одного оттенка синего

[11:30:48 PM] Михаил:
Он теб\yat тоже идетъ. Он какъ у Кляйна.

[11:31:45 PM] Маргулис:
> Михаил Зыбин
> Он теб\yat тоже идетъ. Он какъ у Кляйна.
Кто он? Синий?
 [Photo]
 Этот синий хорош
 Хотя это не совсем тот оттенок
 А в целом я очень не люблю синий. По неясной причине

[11:33:30 PM] Михаил:
Въ д\yatтств\yat тебя били синевой.

[11:33:58 PM] Маргулис:
Нет
 Меня не били вроде

[11:34:27 PM] Михаил:
Хорошо

[11:36:09 PM] Маргулис:
Красный прекрасен же

[11:37:52 PM] Михаил:
Красной ручкой в школ\yat пров\yatряютъ работы.
 Ч\yatмъ его меньше, т\yatмъ меньше ошибокъ и т\yatмъ лучше.

--- Friday, May 26, 2017 ---

[12:06:30 AM] Маргулис:
> Михаил Зыбин
> Красной ручкой в школ\yat пров\yatряютъ работы.
Ерунда какая
 Видимо, у меня было так мало ошибок, что я об этом не думаю)
 А вообще, самые классные мои учителя всё делали синей ручкой
 Историчка, физик и женщина-биолог
 По химии ничего не помню
 Возможно, тоже синим
 Сейчас понимаю, что я на самом деле не ничего не делала в школе
 Что я дофига писала по биологии и истории дома
 Даже в последние годы
 Кошмар какой
 И химию приходилось делать

[12:09:21 AM] Михаил:
Это тебя не радуетъ?

[12:09:26 AM] Маргулис:
Помню, как я шла сдавать что-то под дождем
 Потому что забыла взять с собой
> Михаил Зыбин
> Это тебя не радуетъ?
Вообще абсолютно безразлично
 Просто я поняла, что ты слишком доверчивый и всё воспринимаешь буквально

[12:11:15 AM] Михаил:
Я дов\yatрчивый, да.

[12:12:14 AM] Маргулис:
Нет, это всего лишь сформированное мнение

[12:13:52 AM] Михаил:
Не понялъ.

[12:14:17 AM] Маргулис:
Если бы ты считал, что я сейчас каждую минуту занимаюсь делом, ты бы мне не поверил
 Что я ничего не делала в школе
 Ты согласился?

[12:29:15 AM] Михаил:
Похоже на то.

[12:36:44 AM] Маргулис:
Мило
 Я с тобой не разговариваю
 Ок, молчи дальше

[12:38:16 AM] Михаил:
Прости меня.

[12:38:35 AM] Маргулис:
За что, интересно?

[12:40:00 AM] Михаил:
Я не знаю.

[12:41:17 AM] Маргулис:
Ну вот поэтому и нет
 Я тебе больше ничего не расскажу

[12:45:45 AM] Михаил:
Давай объяснимся въ институт\yat. Я не хот\yatлъ тебя обид\yatть.

[12:46:48 AM] Маргулис:
Ладно
 Это безумно атмосферно
 Дореволюционный
 И слово объяснимся — как будто мы из Чехова
 Я не люблю Чехова

[12:48:46 AM] Михаил:
Ок

[12:49:30 AM] Маргулис:
Не поняла
 Что ок
 Почему ты вечно делаешь меня больной?
 [Photo]
 Где ты видишь тут только биологию?

[12:51:19 AM] Михаил:
Похоже, я не дочиталъ оглавленіе до конца.
 Даже до середины.

[12:51:37 AM] Маргулис:
Ага
 Самое главное, говоришь со мной так, как будто я не помню ничего
edited 
Ок — не в стиле, меняй стилистику речи

[12:53:32 AM] Михаил:
Воистину.

[12:53:54 AM] Маргулис:
Да, так хорошо
 Как настроение?

[12:56:30 AM] Михаил:
У меня давно сложности съ этимъ понятіемъ.
 Ну, безпричиннаго страха и паники сейчасъ н\yatтъ.
 Планирую спать.
edited 
[12:58:21 AM] Маргулис:
Ты мерзкий
 Я пойду резать ноги

[12:59:36 AM] Михаил:
Я не намекаю на тебя, что ты!

[12:59:47 AM] Маргулис:
Что?
 А где ты мог намекать на меня?

[1:00:21 AM] Михаил:
Почему я мерзкій?
> Михаил Зыбин
> Ну, безпричиннаго страха и паники сейчасъ н\yatтъ.
Тутъ

[1:01:10 AM] Маргулис:
Потому что ты обижаешь меня

[1:01:42 AM] Михаил:
А-а-а, д\yatйствительно.

[1:02:03 AM] Маргулис:
Поэтому я пойду себя резать
 Ты же от этого расстроишься

[1:02:22 AM] Михаил:
Ноги?

[1:02:34 AM] Маргулис:
Ну да, повыше, где особенно опасно
 И где не увидят родители, они же расстроятся

[1:03:19 AM] Михаил:
Давай я позвоню.
Outgoing Call 1:07:18
Missed Call

[2:11:56 AM] Маргулис:
Что случилось?

[2:12:47 AM] Михаил:
Телефон разрядился.
Outgoing Call 1:32:44

[3:47:48 AM] Маргулис:
Боже, какой рассвет
 Блин
 Да что со мной сегодня
 Ладном неважно

[3:48:35 AM] Михаил:
Спи

[3:48:41 AM] Маргулис:
И ты

[4:23:18 AM] Маргулис:
Слушай, а почему я по-твоему не могла отказать крылу, если его хотела? Согласие и желание — абсолютно разные категории.
 Больше всего меня тогда волновали отношения с Васей и неуверенность в том, что он не бросит меня в тот же день
 Нужны какие-то гарантии
 В виде отношений
 И было обидно, что он только за этим меня привёл
 Нам было что обсудить
 И он не делал ничего похожего на изнасилование, на самом деле. Он был очень мягким, только обнимал меня крепче, чтобы я не могла уйти, и стянул как-то кофту. В остальном он очень милый
 Он знал, что я не хотела, потому что я говорила об этом

[3:52:25 PM] Маргулис:
Автомат есть
 Таки есть

[8:26:20 PM] Михаил:
Хочешь фотографіи съ кладбища?

[8:29:19 PM] Маргулис:
Да

[8:29:56 PM] Михаил:
[Photo]
 [Photo]
 [Photo]
 [Photo]
 [Photo]
 [Photo]
 [Photo]
 [Photo]
 [Photo]
 [Photo]
 Тамъ очень красиво п\yatли соловьи. Въ той части, гд\yat я былъ, могилы въ основномъ новыя и не очень интересныя.
 Людей никого.
 Оно въ семь закрылось.

[8:32:19 PM] Маргулис:
Красиво
 На новодевичьем гораздо хуже
 Зато там лежит Петр Алексеевич
 А в самом монастыре Бакунин

[8:34:28 PM] Михаил:
По пути я купилъ пленку для фотоаппарата.

[8:34:29 PM] Маргулис:
[Photo]
 [Photo]
 [Photo]
 [Photo]
 [Photo]

[8:35:09 PM] Михаил:
Хорошій вопросъ, зач\yatмъ.

[8:35:19 PM] Маргулис:
А у тебя он есть или нет?

[8:35:29 PM] Михаил:
Есть.

[8:35:57 PM] Маргулис:
Будешь фоткать на него

[8:36:21 PM] Михаил:
Тамъ 36 кадровъ.

[8:36:26 PM] Маргулис:

> Михаил Зыбин
>  Photo
Это часовенка или склеп?

[8:37:13 PM] Михаил:
Думаю, что часовенка. Она закрыта.

[8:37:37 PM] Маргулис:
Обрати внимание, на могиле Кропоткина посадили дерево. Это хорошая традиция
 Я тоже хочу дерево на могилу
 Я ещё не решила, какое
 Ясень ли

[8:38:51 PM] Михаил:
Будетъ еще время подумать, если повезетъ.

[8:38:52 PM] Маргулис:
От места смерти зависит
 Я бы в Италии умерла
 Если выбирать
 Хотя, грустно все это

[8:39:44 PM] Михаил:
Какая разница, гд\yat?

[8:39:44 PM] Маргулис:
Все равно будут катаклизмы или войны
 Все равно кладбище разровняют
 И не будет там дерева
> Михаил Зыбин
> Какая разница, гд\yat?
Красиво там

[8:41:14 PM] Михаил:
Я склоняюсь къ тому, чтобы мой пепелъ гд\yat-нибудь развеяли.

[8:41:47 PM] Маргулис:
Это вроде не очень льзя
 Не хочу спойлерить, но в одном фильме сказали, что нельзя

[8:43:09 PM] Михаил:
Слушай, мн\yat сейчасъ неудобно писать, я не дома пока.

[8:55:09 PM] Маргулис:
Ок
 Слушай
 А над чем мы ночью вчера смеялись?

[9:35:57 PM] Михаил:
Это такая реакція на тему разговора. Я бы сказалъ, мы см\yatялись оттого, что нарушали табу на такіе разговоры. Можетъ, и отъ удивленія.
 Отъ смущенія. Отъ радости, что переживания такого рода можно высказать.

[10:11:38 PM] Маргулис:
Нет, вчера было что-то смешное
 Это немного странно, но я не то чтобы не привыкла к этому

[10:22:02 PM] Михаил:
Было, что ты указала у себя на какое-то слово-паразитъ и сразу посл\yat этого употребила его.

[10:45:52 PM] Маргулис:
Да

[11:04:20 PM] Маргулис:
Слушай
 А в Бартона финке Бартон Финк постоянно молчит или нет?

[11:04:47 PM] Михаил:
Н\yatтъ

--- Saturday, May 27, 2017 ---

[12:19:36 AM] Михаил:
Я сейчасъ лягу спать. Спокойной ночи.

[12:29:27 AM] Маргулис:
Жалко

[3:26:52 AM] Маргулис:
Я вспомнила
 Что смотрела фильмы до 40 года
 Даже много
 20-х или 10-х, не помню
 Это были очень известные высокохудожественные фильмы
 Но я тебе о них не расскажу
 Потому что ты сволочь
edited 
И я не буду больше участвовать в твоим развитии
 Но ты мне должен, так что тебе придётся со мной общаться
 Я не знаю, как я могла об этом забыть.
 Просто это тоже было много лет назад
 И 10-х, и 20-х.
 Но ты о них не узнаешь, хи-хи
 Мне не нравится, когда ты смотришь что-то до меня и потом говоришь мне об этом
 Не надо мне в таких случаях об этом говорить

[5:46:30 PM] Маргулис:
Ты переборщил в своих рассказах. Извини, мне неприятно.

[7:19:14 PM] Михаил:
Согласенъ.

[7:35:26 PM] Михаил:
Я совершенно не настаиваю, что эта тема для меня важна. Къ тому-же, я выложился весь, и добавлять мн\yat больше нечего.
 Захот\yatлось побыть неправильнымъ.

[11:43:29 PM] Маргулис:
Прости
 Я уже успокоилась
 Просто это бывает тяжеловато
 Хорошо знать человека
 Я не из-за того, что ты рассказал. Просто иногда мне не хватает одиночества
 Тебе нравится актриса, игравшая Гермиону?

[11:45:37 PM] Михаил:
Что-то въ ней есть.

[11:46:12 PM] Маргулис:
Мне немножко жаль, что ей не нравится быть актрисой
 Если бы она хотела, она могла бы больше играть

[11:46:27 PM] Михаил:
> Маргулис
> Я не из-за того, что ты рассказал. Просто иногда мне не хватает
Загадочно.

[11:47:49 PM] Маргулис:
Когда кого-то очень хорошо знаешь и постоянно общаешься с этим кем-то, очень тяжело выкинуть человека из мыслей
 Каких ты любишь актрис?

[11:51:19 PM] Михаил:
Я не знаю.

[11:55:21 PM] Маргулис:
Сейчас я скину тебе описание главной героини мультика красавица и чудовище
 [Photo]
 Что значит имеет?
 Меня теперь занимает этот вопрос
 И не было там вроде никакого моряка
 Там был Гастон

[11:56:54 PM] Михаил:
Я хохочу.

[11:57:01 PM] Маргулис:
Таки что-то с этим описанием не так
 Соперника за сердце чудовища что ли?

--- Sunday, May 28, 2017 ---

[12:05:28 AM] Михаил:
Я тоже думаю, тамъ не было моряка. Это какая-то случайная фраза, нев\yatсть откуда взявшаяся.

[3:25:47 PM] Маргулис:
Миш
 Коту моему плохо
 Я еду в ветеринарную

[3:52:25 PM] Михаил:
Над\yatюсь, обойдется.
 Ты тоже над\yatйся. Я знаю, котъ для тебя очень важенъ.
 Что сейчасъ?

[4:04:39 PM] Маргулис:
Процедуры
 Там нужно промыть полость

[4:05:21 PM] Михаил:
Какую полость?
 Опасности для жизни н\yatтъ?
 Я подозр\yatваю, ты очень волновалась.

[4:13:36 PM] Маргулис:
Нет опасности
edited 
[4:37:19 PM] Маргулис:
Если бы ты слышал, как он орёт

[5:12:32 PM] Маргулис:
Если случится заражение, то будет очень плохо

[7:11:17 PM] Маргулис:
Я волнуюсь
 Все меня бросили

[7:40:36 PM] Михаил:
Я съ тобой.
 Котъ въ клиник\yat? Ты съ нимъ?
> Маргулис
> Если случится заражение, то будет очень плохо
Изъ-за плохо обработанныхъ инструментовъ, или какое-то другое?

[7:47:27 PM] Маргулис:
Ну там же гной
 Нужно промывать каждый день
 Нет, мы уже дома
 Кот с нами

[7:48:43 PM] Михаил:
Гд\yat гной-то? Что съ нимъ вообще произошло?

[7:54:34 PM] Маргулис:
Потом опишу
 Я очень устала
 Мне плохо
 Со мной такого давно не было

[9:45:27 PM] Маргулис:
Есть железы, через которые у скунсов выходит вонючая жидкость. Они есть и у котов с кошками, только менее развиты. Вот они и воспалились
 Очень сильно
 Очень давно, видимо.
 Котов с собаками
 Я опечаталась
 Выше
 Задумалась

[10:18:04 PM] Маргулис:
Ты молчишь
 Ты молодец

[10:18:53 PM] Михаил:
Теб\yat грустно?

[10:19:36 PM] Маргулис:
Да нет, что ты, офигенно себя чувствую

[10:20:00 PM] Михаил:
Теб\yat очень грустно и ты устала.

[10:20:41 PM] Маргулис:
Да.

[10:22:18 PM] Михаил:
То есть сейчасъ за нимъ требуется особый уходъ?

[10:22:24 PM] Маргулис:
Да
 Два раза в день нужно промывать эту полость
 Колоть антибиотики
 И его не надо пускать в комнаты и нельзя давать лизать рану

[10:23:30 PM] Михаил:
А гд\yat онъ долженъ находиться?

[10:23:51 PM] Маргулис:
Ну просто там, где нет ковров
 Это всё к ране липнет
 И ковёр тоже пачкается от крови
 Кухня, коридор

[10:25:26 PM] Михаил:
Жутковато. Вы справитесь. То есть дома долженъ всегда кто-то быть?

[10:26:03 PM] Маргулис:
Ну, не можем же мы всё бросить ради него
 Ну он по-любому редко бывает один

[10:26:20 PM] Михаил:
Его вялость не изъ-за этого?

[10:26:32 PM] Маргулис:
Из-за этого, видимо
 Я на английский не могу идти

[10:28:16 PM] Михаил:
Понимаю. Глупо просить тебя не грустить, но в\yatдь не такъ все плохо. Вы справитесь.

[10:28:39 PM] Маргулис:
Да я знаю
 Просто я очень испугалась за него

[10:28:57 PM] Михаил:
Да, конечно

[10:29:02 PM] Маргулис:
Мы не сразу поняли, откуда кровь
 Подумали, что почки

[10:29:33 PM] Михаил:
То есть началось съ того, что пошла кровь?
 Я не отвлекаю тебя сейчасъ?

[10:30:46 PM] Маргулис:
Нет, не отвлекаешь. Я ещё полчаса буду пытаться успокоиться
 Потом брошу это дело и займусь чем-то в таком состоянии
> Михаил Зыбин
> То есть началось съ того, что пошла кровь?
Да
 Мы когти стригли ему
 Мы для этого кладём его на спину
 Когда возвращали нормальное положение, очень сильно полилась кровь с гноем
 Там что-то разорвалась

[10:34:05 PM] Михаил:
Понятно. Ну, сейчасъ-то изв\yatстно, что съ нимъ, такъ что д\yatлай то, что въ твоихъ силахъ, над\yatйся на лучш\yatе и не унывай.
edited 
[10:34:47 PM] Маргулис:
Меня раздражают такие советы

[10:35:05 PM] Михаил:
Не буду больше.
 Я не знаю, что еще говорить.

[10:35:36 PM] Маргулис:
Ничего страшного, сил злиться у меня нет
 Ты не смотрел маленькую мисс счастье?

[10:39:48 PM] Михаил:
Не смотр\yatлъ.

[10:39:55 PM] Маргулис:
Ну и ладно
 Можно посмотреть, но не супер
 И это типа комедия
 Меня просто напугало в этом фильме, что действительно может быть конкурс красоты для девочек 6-7 лет, где они красятся и ведут себя так, как будто им 20 лет и они не отягощены особыми моральными принципами

[10:43:47 PM] Михаил:
Я вовсе не понимаю конкурсовъ красоты.
 Зд\yatсь моя школа названа милліономъ разныхъ способовъ. https://sistema.lksh.ru/2017/entrance/results/
sistema.lksh.ru
Поступившие в ЛКШ 2017 • SIStema
Летняя компьютерная школа — это летняя школа для учащихся 6-10 классов, увлечённых программированием. ЛКШ ориентирована в основном (но не только) на ш...

--- Monday, May 29, 2017 ---

[1:53:12 AM] Маргулис:
Миш
 Серьёзный вопрос
 Мне автра идти на матан или уйти после одной пары, которая с 12 до 13:30?
 Если завтра матан
 Вечером
 С 3:30 до 5:00

[1:54:50 AM] Михаил:
Дискотека

[1:55:02 AM] Маргулис:
Надо курсач делать
> Михаил Зыбин
> Дискотека
Чего
 А
 Ну да
 Правда?

[1:55:35 AM] Михаил:
Четная нед\yatля

[1:55:36 AM] Маргулис:
Я просто много уже матана пропустила
 Занятия три точно
 Я чувствую падение оценки на балл
 Ну это так
 Я многое чувствую
 Чего на самом деле нет

[1:57:12 AM] Михаил:
Въ сл\yatдующемъ учебномъ году ты будешь бол\yatе трудолюбивой и такъ дал\yatе.
 На дискотеку ты больше или меньше хочешь идти?

[1:58:52 AM] Маргулис:
> Михаил Зыбин
> На дискотеку ты больше или меньше хочешь идти?
На дискотеку можно не ходить
 Там мое присутствие никого не волнует
 Хотя, оно нигде никого не волнует

[1:59:31 AM] Михаил:
Это я такъ говорю часто.

[2:00:42 AM] Маргулис:
Это заразно
 И вот от этого я хотела спрятаться в скайпе

[2:01:06 AM] Маргулис:
Forwarded message: Дарья Ремизовская [5/28/17] 
Ладно)

[2:01:17 AM] Маргулис:
Я ненормальная
 Я умерла
 Сегодня
 Зато вчера я хорошо провела ночь
 Смотрела фильм

[2:03:18 AM] Михаил:
Загадочно. Въ скайп\yat ты написала, что чувствовала себя живой.

[2:03:32 AM] Маргулис:
Ну, на контрасте
 Сначала умерла

[2:03:47 AM] Михаил:
Ожила уже?

[2:03:59 AM] Маргулис:
А ещё я сегодня видела Ивана Грозного
 Фильм
 Эйзенштейна
 Если тут нежное количество й
 Случайно

[2:04:40 AM] Михаил:
Н\yatжное
 И шелковистое
 Случается.
 Я пойду въ душъ.

[2:05:54 AM] Маргулис:
Нужное
 Хорошо

[12:15:34 PM] Михаил:
Заходи съ правой стороны.

[12:43:42 PM] Михаил:
Ты не придешь, подозр\yatваю.

[12:50:07 PM] Маргулис:
Ага
 Прости
 Но я даже доехала до фрунзенской
 А потом поняла, что все равно не хочу ничего писать на алгебре, лучше почитаю пока. Все равно я на одну пару собиралась

[1:32:11 PM] Маргулис:
Миша, знаешь, что я тебе скажу
 Как бы я не относилась к человеку, мне бы не пришло в голову читать все книги, о которых он говорит
 Это неприлично, в конце концов
 Читай уже то, что знаешь сам, ты выглядишь так, как будто только ради меня это делаешь
 Я тебе больше ничего не расскажу о книгах
 И о фильмах
 Займись своим делом а не лезь к моим писателям
 Это ужасно глупо выглядит
 Мне противно. Я больше не заговорю с тобой ни о чем.
 Если ты продолжишь, мы больше не будем общаться.

[1:37:38 PM] Михаил:
Я понимаю.

[1:38:51 PM] Маргулис:
Ну хватит уже копаться во мне, ты начинаешь быть похож на маньяка
 Если бы ты был девочкой, можно было бы подумать, что ты хочешь меня убить и занять моё место
 В смысле, пластическую операцию там сделать
 Жить в моей семье
 И так далее
 Кстати, ты прочитал уже Петра Алексеевича?

[1:41:20 PM] Михаил:
Въ этомъ есть значительная доля правды. Но мн\yat это уже самому не нравится.
 Дв\yat главы. Я небыстро читаю.
 Во мн\yat есть мотивъ самоотреченія, это правда.

[1:47:16 PM] Маргулис:
Просто ты вчера заговорил уже и о Гамсуне
 И меня это все сбивает
 Я же тебе о них рассказала
 Значит, я должна прочитать больше их книг, чем ты
 Значит, я сажусь и делаю это
 Значит, не делаю курсач

[2:07:10 PM] Михаил:
> Маргулис
> Значит, я должна прочитать больше их книг, чем ты
Я бы сказалъ, ты не должна.

[3:10:52 PM] Маргулис:
> Михаил Зыбин
> Я бы сказалъ, ты не должна.
Обязана
 Они мои
 И всегда будут моими, убери от них себя подальше

[9:28:23 PM] Маргулис:
Ок. Объект культурного наследия. 
Но вот какая штука
Я, например, японка, и, как я могла бы делать, будь я японкой, разбираюсь в японской поэзии. И знаю ну умопомрачительного японского поэта, но достаточно неизвестного, чтобы его не знал ни один из моих знакомых, и вообще меньше 0,1% японцев. Хотя это много японцев, пусть будет 0.0001%, остановимся на этом. А ты, скажем, студент, и внезапно тебе нужно  учить японский. Ну, случайное стечение обстоятельств нас сталкивает, мы общаемся, ты говоришь ерунду, я её оспариваю, разговор доходит до японской поэзии и я вспоминаю о нём, моём родном, любимом, далее далее далее. Ты, конечно, с ним ознакомишься и почитаешь его, абсолютно естественно в такой ситуации. Не в нашей ситуации, в нашей есть один придурок, который занимается паразитированием на моём интеллекте, прошедшем определённый путь развития в плане искусства, не самый лёгкий, кстати, потому что на самом деле я не знаю лично ни одного человека, который разбирался бы в живописи так, как это делаю я. Я не претендую на систематичность знаний, и причина некоторого перекоса в определённые стороны именно в самостоятельности. Ну да ладно. Ты ознакомился с тем японцем. Разве можно это ознакомление считать хоть сколько-нибудь частью тебя, ведь оно является совершенно вырванным из любого контекста, твоего личного и общего, оно не обусловлено твоими жизненными обстоятельствами и не является частью твоей культуры. Ты должен был бы прийти к этому поэту сам, жизнь должна была подвести тебя к нему, образовать прочные, личные связи. Иначе в тебе зарождается и начинает жить своей жизнью посторонний и не имеющий к тебе отношения кусок. У тебя даже нет личного отношения к нему и может не появиться. Даже не это страшно — страшно, что ты, не зная об этом ничего заранее и не придя к этому естественно, не имея никакого внутреннего стремления к какой-то идее, знакомишься с чем-то инородным, стройным, сильным, продуманным мыслями, близких с которыми у тебя не было, и аргументов за и против которого заранее ты не имеешь, и это инородное и стройное тебя давит своей внутренней системой, влияет на тебя и меняет так, чтобы максимально с тобой слиться и получить дальнейшее развитие. Оно просто закладывает начальные данные, которые на самом деле не начальные
 Ты за две минуты согласился, что бродский — дрянь
 Ты не читая Сартра согласился со мной за те же 2 минуты 
А с мнением нобелевского комитета — за 0 секунд
 Я обратно тебя блокирую
 Ты представляешь, как на некоторых людей влияют Достоевский и Ницше?
 Вот абсолютно в этом причина
 Я, возможно, скажу удивительную вещь, но в искусстве можно начать разбираться, не прикасаясь к критике, и ко всем выводам можно прийти лично
 Это абсолютно второстепенная вещь, биографии, критика и прочая ерунда
 Я не видела твоего ответа, ты заблокирован
 И сейчас
 Мне очень весело так делать
 Если ты хочешь знать, почему меня это так сильно бесит — мне, может, и хочется какого-то нового культурного опыта, но ты мне его не даёшь. Это правда паразитирование, я от тебя ничего не получаю

--- Tuesday, May 30, 2017 ---

[12:19:05 AM] Маргулис:
Ты спишь?

[12:19:29 AM] Михаил:
Н\yatтъ.

[12:19:49 AM] Маргулис:
А ты кому-то ещё пишешь в старой орфографии?
 Я не люблю математику.
 Я не знаю, зачем я здесь.
 Меня не радуют лекции.

[12:20:46 AM] Михаил:
> Маргулис
> А ты кому-то ещё пишешь в старой орфографии?
Н\yatтъ, но я и мало кому пишу.
> Маргулис
> Меня не радуют лекции.
Это грустно.

[12:21:19 AM] Маргулис:
Я не знаю, как это может радовать.

[12:22:11 AM] Михаил:
Сложная ситуація у тебя.

[12:22:55 AM] Маргулис:
Я не могу сказать родителям.

[12:23:51 AM] Михаил:
Почему? Не хоч\yatшь ихъ расстроить?

[12:24:31 AM] Маргулис:
Я боюсь их. Начнётся чорт знает что. Они скажут, что я не довожу ни одно дело до конца
 Я бросила художку
 А в началке я бросила гимнастику, танцы и плавание. Мне отец до сих пор блин это вспоминает

[12:26:40 AM] Михаил:
Не знаю. Думаю, передъ родителями нужно быть собой. Они же нав\yatрняка тебя очень любятъ.
 Это изв\yatстное д\yatло, когда н\yatтъ пониманія. Ну, надо стремиться къ нему.

[12:28:12 AM] Маргулис:
Ну ты понимаешь, что это просто смешно — я реально не занимаюсь математикой весь этот год
 Причём с 1 сентября себя так чувствую
> Маргулис
> Причём с 1 сентября себя так чувствую
Вот как это можно объяснить?

[12:31:06 AM] Михаил:
Мн\yat это сложно понять, но я готовъ тебя поддерживать. Хотя бы на этомъ курс\yat. Что потомъ - и я не знаю. А можетъ, изъ прошлыхъ учителей теб\yat кто-то поможетъ разговоромъ, или, видимо, н\yatтъ?
 На мехмат\yat, видимо, не было бы лучше.

[12:32:11 AM] Маргулис:
Как можно объяснить, что я не захотела говорить учителям, что хочу сдавать другие егэ, и поэтому не стала их сдавать?
 Я бы могла оттуда вылететь уже
 Тут хоть дз есть

[12:35:23 AM] Михаил:
На физфак\yat было бы лучше, или ты именно гуманитарное хочешь?

[12:35:44 AM] Маргулис:
Я не знаю. Мне кажется, было бы лучше
 Я не знаю, кто туда пойдёт и какие там преподы

[12:39:41 AM] Михаил:
У меня есть тамъ однокласница.

[12:39:57 AM] Маргулис:
Я о нашем физфаке
 В вышке
 Ты об МГУ?

[12:40:14 AM] Михаил:
Ага

[12:40:28 AM] Маргулис:
Не хочу на физфак МГУ, там физра
 Ну, не только физра
 Есть ещё причины
 И, кстати, мои родители по неясной мне причине против физики ;)
 При том, что у них сильно с ней связан профиль

[12:43:12 AM] Михаил:
Это невесело. Ты говорила съ к\yatмъ-то о физфак\yat по существу?

[12:43:20 AM] Маргулис:
Против теоретической физики
 Да
 У меня там есть хорошая знакомая доцент
 Вроде доцент, я не помню
 Смешно, но я не знаю её фамилию

[12:44:22 AM] Михаил:
И что, сов\yatтуетъ?

[12:44:23 AM] Маргулис:
Ей вообще МГУ не нравится)
 У неё дочь в вышке, это не молодая знакомая
 Ровесница мамы
 Дочь на год старше
> Маргулис
> Я о нашем физфаке
Ты что-то о нем знаешь?

[12:46:00 AM] Михаил:
Нетъ

[12:46:07 AM] Маргулис:
Он только откроется
 Первый набор будет
 Ну это страшно
 Ничего не известно
 Зато такой халявный был бы год
 Мне же матан, историю и философию зачли бы
 С кем поговорить раньше
 С тимориным или с родителями?
 С родителями, я понимаю

[12:50:17 AM] Михаил:
Хорошій вопросъ.
 Тиморина ты боишься меньше родителей?

[12:50:46 AM] Маргулис:
Да.
 Но будет неловко сказать ему, что меня тошнит от математики и остаться
 А меня тошнит
 Возможно, я ей просто не занимаюсь и не даю ей шанса меня увлечь
 Но меня даже от здания матфака подташнивает
 Оно такое унылое
 Внутри
 И на лекциях мне скучно, с этим ничего не поделаешь. На топологии было ничего, если бы не данияр, на логике тоже было бы ничего.
 Но в целом скучно
 Оцени по десятибалльной шкале свою заинтересованность на лекциях

[12:55:24 AM] Михаил:
Въ школ\yat математика теб\yat нравилась больше? 
Лекціи - основное, ради чего я сейчасъ живу. 10, если угодно.

[12:57:00 AM] Маргулис:
Я не узнала ничего нового по математике от учителя за последние несколько лет. А, нет, узнала. Но то, что я узнала за год, я бы могла понять за 5 часов.
 Безумно скучно

[12:57:33 AM] Михаил:
Годъ - прошлый?

[12:57:39 AM] Маргулис:
Да
 Ну нас готовили к ЕГЭ в основном. А ещё дали в какой-то момент интегралы (я не помню, в каком классе проходят производную), но так, что я могла бы это понять за пять часов
 Это я ещё много часов даю
 Это понять и запомнить
 Ну у меня просто тупые одноклассники
 Не удивляйся

[1:00:15 AM] Михаил:
Теб\yat не повезло.

[1:00:25 AM] Маргулис:
Я не открыла за год ничего и не сделала ничего
 У меня был средний балл под 5.0
 4.8 — 4.9
 Абсолютно честный
 Причём было не 5.0 разве что по невнимательности

[1:02:20 AM] Михаил:
Напомни, поч\yatму ты не перешла въ другую школу?

[1:02:29 AM] Маргулис:
Нет, остальные в это время тоже нихрена не делали. У нашей мадам просто не было дисциплины в классе
 Я боялась новых людей
 Я об этом не думала в 8 классе, точнее думала только после февраля, а мне нужно долго решаться
 А после 9 почти везде сформированы классы и мне было страшно приходить в готовый класс
edited 
Я не была уверена, что не пойду в архитектурный
 В 8-9 классе не была уверена
 Я знала, что мои родители хотят от меня идеальную успеваемость
 А в другом месте это было бы сложно
 Ну, везде, куда стоит идти, это было бы сложно

[1:05:46 AM] Михаил:
Сейчасъ уже, конечно, н\yatтъ смысла говорить, что перейти стоило.

[1:05:52 AM] Маргулис:
Да
 Ну и
 Мои родители почему-то не хотели
 Мне этого не понять
 Они волновались, что мне нужно будет утром ездить на метро

[1:06:52 AM] Михаил:
О, моя мама тоже.

[1:06:58 AM] Маргулис:
На пару лет раньше начала бы — не такая уж проблема
 Ну я не могу без их поддержки такие решения принимать, тем более я это юридически не могу сделать
 Тьфу
 Все в прошедшем времени

[1:09:00 AM] Михаил:
Они хотя бы догадываются о твоей гуманитарной природ\yat?

[1:09:19 AM] Маргулис:
Я думаю, её нет
 Это просто высокий уровень образованности
 Я не очень хорошо говорю и пишу
 Очень долго подбираю слова
 Я для гуманитария очень мало читала
 Я не офигенно учу языки
 У нас была одна девочка, которая была восхитительна в этом
 Я боюсь гуманитарных профилей.
 Точнее, год назад боялась
 Я боялась высказывать свои мысли на письме и устно
 Я боялась огромных объёмов материала
 Но у меня была депрессия год назад, а сейчас её нет
 Если сейчас я не решусь уйти, будет поздно
 Надо решаться.

[1:14:17 AM] Михаил:
У тебя будетъ бви на физфак\yat?

[1:14:32 AM] Маргулис:
Нет, по-моему

[1:15:00 AM] Михаил:
Когда подача документовъ?

[1:15:11 AM] Маргулис:
Я не знаю
 В начале июля
 Ну, весь июль
 Ты о каком физфаке? Если об МГУ, то можно не сомневаться, что я пройду
 Если о вышке, то, кажется, можно перевестись
 У Тимоши надо спросить

[1:18:24 AM] Михаил:
Да, это надо узнать.

[1:22:53 AM] Маргулис:
А если мне только кажется, что я люблю физику
 Хотя вряд ли
 Смотрю на наши задачи и сомневаюсь в этом)
 Надо их сдать

[1:24:11 AM] Михаил:
Надо ихъ р\yatшать.

[1:25:19 AM] Маргулис:
Forwarded message: 
Georgy Gaitsgori [5/29/17] 
[Photo]
[Photo]

[1:25:28 AM] Маргулис:
У тебя есть уже это?

[1:25:40 AM] Михаил:
Н\yatтъ
 Спасибо

[1:26:47 AM] Маргулис:
Чую, что я ещё что-то у него спрошу
 Я не знаю, кто пойдёт на физфак вышки
 Кто туда пойдёт?
 Давайте рассуждать
 О, ещё и не ясно, как там будет с англом
 Философы и историки, полагаю, дублируются

[1:31:24 AM] Михаил:
Маша, если ты позволишь, давай договоримъ въ институт\yat.

[1:33:50 AM] Маргулис:
Конечно
 Прости

[4:06:34 PM] Михаил:
Ты не пришла, потому что р\yatшила, что незач\yatмъ?

[4:30:10 PM] Маргулис:
Да
 Я не говорила с родителями

[4:31:16 PM] Михаил:
Я еще на семинар\yat, потомъ отв\yatчу.

[4:31:54 PM] Маргулис:
Ок
 Поговори с кем-нибудь о физике

[4:54:30 PM] Маргулис:
Вот это ты зря не прочитал
 Кстати, какой-то из Миш же тоже должен её делать

[5:20:34 PM] Михаил:
Плахову что-то изв\yatстно. Я ему написалъ.

[5:21:12 PM] Маргулис:
Хорошо
edited 
Я месяца два уже хочу себя убить
 В моей жизни нет смысла
 Эх
 Зря я не пришла сегодня
 Глупая
 Яне спала просто ночью, потом легла

[7:06:41 PM] Маргулис:
Я убью себя
 Мои родители против перевода

[7:07:26 PM] Михаил:
Поговорила съ ними, значитъ.

[7:09:05 PM] Маргулис:
Да
 У них истерика

[7:10:03 PM] Михаил:
Почему они противъ?

[7:11:34 PM] Маргулис:
Потому что
 Второй раз на первый курс
edited 
Ну и вообще
 Физика
 Я её никуда не приложу
 Современная физика — это дорогостоящие аппараты, в России их либо нет, либо запускают их два раза в год
 Это цитата
 Я ничего не довожу до конца
 Я сплю до 16:30
 Хожу к 12
 И наверняка я просто не хочу сдавать сессию
 И на физике проще не будет
 И там тоже сплошная математика
 Это всё цитаты

[7:14:09 PM] Михаил:
Ты сейчасъ грустишь.

[7:14:19 PM] Маргулис:
Я хочу повеситься

[7:14:58 PM] Михаил:
> Маргулис
> И там тоже сплошная математика
Это же похоже на правду.

[7:15:16 PM] Маргулис:
Ну и что?
 Я не вижу смысла в том, чем я сейчас (не) занимаюсь
 Чем я буду заниматься после вуза?
 Если я у них спрошу, они скажут, что они мне об этом говорили год назад
> Маргулис
> Чем я буду заниматься после вуза?
И я уже не очень верю, что в состоянии буду доучиться и защитить диплом. Мне правда скучно здесь

[7:24:46 PM] Михаил:
На матфак\yat есть физики. Можно писать курсачъ по физик\yat.

То есть имъ твое будущее посл\yat матфака видится опред\yatленн\yatе и лучше?

[7:26:32 PM] Маргулис:
Да
 Тебя почему-то стало тяжелее читать
 Я не могу заниматься метафизикой
 Блин
 Матфизикой

[7:28:14 PM] Михаил:
Теб\yat ц\yatнна связь съ реальностью?

[7:29:47 PM] Маргулис:
Я не знаю
 Что такое не иметь этой связи

[9:55:21 PM] Маргулис:
Я царапала вену
 Я не умею доставать лезвия из станка

[9:56:20 PM] Михаил:
Б\yatдная Маша.
 Не грусти.

[9:57:38 PM] Маргулис:
> Маргулис
> Я не умею доставать лезвия из станка
Да, несчастная
> Михаил Зыбин
> Б\yatдная Маша.
Лиза

[9:58:42 PM] Михаил:
Ты очень умный челов\yatкъ, у тебя въ жизни все будетъ хорошо.

[9:58:54 PM] Маргулис:
Угу.
 Когда я не напишу курсач

[10:00:04 PM] Михаил:
Напишешь, конечно.

[10:02:12 PM] Маргулис:
Ладно
 А физику ты не взял у кого-нибудь?

[10:02:55 PM] Михаил:
Только что Миша прислалъ.

[10:02:55 PM] Михаил:
Forwarded message: Misha Pavlovskiy [5/30/17] 
[Photo]
[Photo]
[Photo]
[Photo]

[10:03:26 PM] Маргулис:
А сколько здесь задач,
 ?

[10:03:44 PM] Михаил:
Я еще не читалъ это.

[10:03:56 PM] Маргулис:
Тут две задачи
 В сумме пока есть три

[10:05:00 PM] Михаил:
[Photo]
 Я не знаю, гд\yat тутъ начало и конецъ.

[10:09:01 PM] Маргулис:
Что такое Н у Миши?
 Я не поминаю, какая это задача
 Ты нормально не мог сфоткать?
> Маргулис
> Я не поминаю, какая это задача
Первая у Миши
 А на фото понимаю
 Поле что ли

[10:10:50 PM] Михаил:
Н есть въ условіи.

[10:10:57 PM] Маргулис:
Поняла
 Это третья задача?

[10:11:07 PM] Михаил:
Да

[10:11:09 PM] Маргулис:
А не вторая
 Поняла
 А почему фото под углом?

[10:11:45 PM] Михаил:
Это не я снималъ.
 Мн\yat не нравится, что мое р\yatшеніе задачи 3 занимаетъ полстраницы.

[10:19:02 PM] Маргулис:
Задачи три,
 ?
 А

[10:19:21 PM] Михаил:
Три

[10:26:20 PM] Маргулис:
Голова болит

[10:26:55 PM] Михаил:
Это ты переневрничала.
 Ты обычно принимаешь обезболивающее?

[10:27:28 PM] Маргулис:
Да
 Но сейчас не буду
 Можно позвонить тебе?
Incoming Call 1:03:8

--- Wednesday, May 31, 2017 ---

[12:03:41 AM] Михаил:
Я сегодня вид\yatлъ челов\yatка, который шелъ босикомъ по улиц\yat.

[12:03:48 AM] Маргулис:
Кошмар
 Скажи
 Пишется ведь прийти?
 Или я псих?

[12:04:33 AM] Михаил:
Сейчасъ это пишется такъ, да.
edited 
[12:04:47 AM] Маргулис:
Крыл пишет исключительно придти
 Тьфу
 Придти

[12:05:59 AM] Михаил:
У этого есть основанія.

[12:06:09 AM] Маргулис:
Кстати
 Хотела тебе показать

 Первые две греческие
 Ты видел, что по-другому шрифт выглядит?
 Я знаю, что это другая буква

[12:07:50 AM] Михаил:
Да, они по-разному выглядятъ, я знаю.

[12:08:09 AM] Маргулис:
Мне греческие нравятся больше

[12:08:35 AM] Михаил:
Мн\yat Русскія.
> Михаил Зыбин
> Я сегодня вид\yatлъ челов\yatка, который шелъ босикомъ по улиц\yat.
Это былъ я.

[12:09:03 AM] Маргулис:
Почему?

[12:09:21 AM] Михаил:
Просто такъ
 Тепло уже

[12:09:33 AM] Маргулис:
> Михаил Зыбин
> Ты обычно принимаешь обезболивающее?
Тут разве все хорошо?

[12:10:03 AM] Михаил:
Въ смысл\yat буквъ?

[12:10:11 AM] Маргулис:
Да

[12:10:51 AM] Михаил:
По-моему, да.

[12:11:21 AM] Маргулис:
Как пишутся окончания прилагательных?

[12:12:01 AM] Михаил:
Безъ ятей.

[12:12:15 AM] Маргулис:
А в сравнительной степени с?

[12:12:20 AM] Михаил:
Да

[12:12:32 AM] Маргулис:
Хорошо

[1:04:55 AM] Маргулис:
Боже мой

[1:29:33 AM] Михаил:
Чего?
 А-а-а, фотографія?

[1:30:17 AM] Маргулис:
Да

[9:03:02 PM] Маргулис:
Мур

[9:06:23 PM] Михаил:
Проснулась?

[9:11:34 PM] Маргулис:
Да
 В 19:50
 Обидно

[9:11:54 PM] Михаил:
А хот\yatла когда?

[9:12:03 PM] Маргулис:
В 10

[9:12:52 PM] Михаил:
Физикой ты завтра займешься? Сейчасъ что д\yatлаешь?

[9:13:13 PM] Маргулис:
Физикой завтра 
Сейчас нужно заставить себя поесть
 Курсовая
 Я не вижу в еде смысла

[9:14:23 PM] Михаил:
Онъ есть
 У меня отъ голода хуже настроеніе

[9:15:23 PM] Маргулис:
Мне еда настроение не поднимет

[9:15:47 PM] Михаил:
Ея отсутствіе его опуститъ

[9:16:13 PM] Маргулис:
Нет
 Оно на дне
 Я просто не хочу есть
 Нет желания

[9:17:36 PM] Михаил:
Оттого, что придетъ Дима, ты волнуешься?

[9:17:47 PM] Маргулис:
Да, но не очень
 Надо будет ещё убрать все кошачьи волосы в квартире. У него аллергия

[9:18:38 PM] Михаил:
А насчетъ самого кота?

[9:19:37 PM] Маргулис:
Ну куда его деть
 Он в другой комнате спит
 А ты почему не спрашиваешь о его самочувствии?

[9:20:28 PM] Михаил:
Меня это интересуетъ. Какъ онъ?

[9:20:34 PM] Маргулис:
Он лучше
 Это выражается в том, что он стал рваться к нам в комнаты
 И мы его пускаем
 Он милый очень

[9:21:51 PM] Михаил:
Сталъ, стало быть, активн\yatе и жив\yatе?

[9:21:57 PM] Маргулис:
Да

[9:22:13 PM] Михаил:
Хорошо

[9:22:40 PM] Маргулис:
Он очень смешно прячется в моей комнате от родителей

[9:22:58 PM] Михаил:
Почему прячется?

[9:23:11 PM] Маргулис:
Уколы
 И промывание
 Он не любит это
 Как ты понимаешь

[9:23:59 PM] Михаил:
Да

[10:08:53 PM] Маргулис:
Люди делятся на тех, кто пишет а-а-а и м-м-м, и тех, кто пишет ммм и ааа
 В смысле звукоподражания

[10:09:20 PM] Михаил:
Да

[10:09:35 PM] Маргулис:
Мы с тобой по разные стороны баррикад

[10:10:16 PM] Михаил:
Въ книгахъ я вид\yatлъ только съ дефисами.
edited 
[10:10:53 PM] Маргулис:
Ты не прав, кстати. Правда, ещё лучше, когда такого вообще не пишут
 В книгах
 Попросишь Георгия физику скинуть?

[10:11:41 PM] Михаил:
В томъ, что въ книгахъ я вид\yatлъ только съ дефисами, я правъ.
 Попрошу

[10:12:09 PM] Маргулис:
Откуда ты знаешь?
 Вдруг ты внимания не обращал
 Обычно пишут одну букву и троеточие
 Это не про а
edited 
Про м

[10:14:04 PM] Михаил:
Щекотливый разговоръ.

[10:14:14 PM] Маргулис:
Да
 Мы поссоримся

[11:08:43 PM] Маргулис:
Оказывается, убрать всю шерсть из квартиры с котом невозможно

\end{document}