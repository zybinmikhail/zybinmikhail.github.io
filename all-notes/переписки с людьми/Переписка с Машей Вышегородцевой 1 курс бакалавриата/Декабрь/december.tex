\documentclass{article}
\usepackage[utf8]{inputenc}
\usepackage[russian]{babel}
\usepackage{amsmath}
\usepackage{amsfonts}
\usepackage{mathdots}


\begin{document}
\title{}
\author{}
\date{}
\maketitle

Маргулис:
Можешь грустному несчастному больному человеку его отправить или дать завтра? Я умру, если буду это делать

[11:09:00 PM] Михаил:
Неловко, что я не сразу увидел сообщения.
 [Photo]
 [Photo]

[11:09:54 PM] Маргулис:
Ничего страшного

[11:10:04 PM] Михаил:
На консультации будешь?
edited 
[11:10:04 PM] Маргулис:
Просто у меня куча долгов по английскому, поэтому нет времени на геометрию
 На какой консультации?

[11:10:28 PM] Михаил:
Имею ввиду геометрию

[11:10:38 PM] Маргулис:
На семинаре?
 Я не знаю просто, когда консультация

[11:10:54 PM] Михаил:
Которая в пять

[11:11:38 PM] Маргулис:
Не уверена. Это завтра? А ты не будешь на семинаре утром?

[11:11:42 PM] Михаил:
[Photo]
 Буду на семинаре. Некоторые задачи неадекватно записаны, их надо разъяснять.

[11:13:06 PM] Маргулис:
Давай тогда на семинаре встретимся

[11:13:07 PM] Михаил:
[Photo]
 Ну да

[11:13:38 PM] Маргулис:
Спасибо большое

[11:13:48 PM] Михаил:
[Photo]
 [Photo]
 Остальное принесу

[11:16:33 PM] Маргулис:
Хорошо, спасибо

--- Monday, December 19, 2016 ---

[9:58:07 PM] Маргулис:
Можно пункт б в 6?

[10:23:32 PM] Михаил:
Нет его.
 [Photo]
 [Photo]
 У меня оригинальное решение 5а
 [Photo]

[10:28:21 PM] Маргулис:
3 я вроде ему уже сделала и сдала
 И 5 вроде тоже, сейчас 10 посмотрю, спасибо

[11:28:39 PM] Маргулис:
Кстати, в тебя есть 7 из 4 листка? Я её одну из 4 не сдала ещё

--- Wednesday, December 21, 2016 ---

[12:48:26 AM] Михаил:
Я почему-то проигнорировал твое последнее сообщение.

[1:14:47 AM] Маргулис:
Неважно
 Я все сдала
 Спасибо

[9:28:16 PM] Михаил:
Почему Маргулис, расскажи, пожалуйста.

[9:28:28 PM] Маргулис:
Папина фамилия, все просто
 А Вышегородцева мамина

[9:29:33 PM] Михаил:
Почему ты в списках Вышегородцева?

[9:29:59 PM] Маргулис:
Потому что мои родители, когда я была маленькая, решили дать мне мамину фамилию
 Я не знаю, почему
 Мама оставила свою фамилию после свадьбы, а когда у ребёнка и матери разные фамилии, это не очень удобно с бюрократической точки зрения

[9:33:07 PM] Михаил:
У меня к тебе еще вопросы есть. Почему у тебя такие волосы и кто ты вообще?

[9:33:26 PM] Маргулис:
По национальности кто?
 Волосы такие всю жизнь
 И у папы такие
 И у дедушки

[9:33:55 PM] Михаил:
Здорово

[9:35:25 PM] Маргулис:
Вообще я человек. Вряд ли математик. На четверть еврейка. Атеистка. А ещё я не люблю Достоевского

[9:38:00 PM] Михаил:
Я считаю себя математиком. Я тоже на четверть еврей, и бывает грустно. Не могу отнести себя ни к тому, ни к другому народу. Я тоже атеист, а мнения о Достоевском у меня нет.

[9:39:10 PM] Маргулис:
Бывает грустно безотносительно к народу? Там несколько неочевидная синтаксическая конструкция

[9:40:46 PM] Михаил:
Грустно, что не чувствую единства ни с теми, ни с другими. Тем более оба народа религиозные.
 Я недавно понял, что математика - особая ветвь культуры. Она не является ни наукой, ни искусством в полной мере, но имеет с ними черты сходства.
 Надо к геометрии готовиться.

[9:46:20 PM] Маргулис:
Да, надо

[10:17:15 PM] Маргулис:
Я забыла, мы во сколько начинаем? В 11 или 10:30?

--- Thursday, December 22, 2016 ---

[3:26:30 PM] Маргулис:
Будет время на выходных—расскажи мне условие своей задачи на экзамене, если помнишь
edited 
Это алгебра

[8:57:16 PM] Михаил:
Доказать, что для всякой верхнетреугольной матрицы А существует многочлен f (x) степени не выше n (n+1)/2, такой что  f (A) - нулевая матрица.

[8:57:48 PM] Маргулис:
Спасибо. Просто некоторые люди говорят, что была очень сложная задача

--- Tuesday, December 27, 2016 ---

[6:59:41 PM] Маргулис:
Ты спрашивал, что такого в том доказательстве. Оно в нормальном виде в сокращении занимает страницы три мелким шрифтом, а эти расширение длиннее каждое
edited 
Можно брать не непрерывные коллинеации, комплексное поле, наверное, можно ещё другие поля туда вставить
 Это большая и прекрасная тема

[7:22:11 PM] Маргулис:
Ну и это нигде ещё не записано, так что есть где подумать

[8:49:24 PM] Маргулис:
Ну вот откуда у тебя суицидальные наклонности?

[9:24:27 PM] Михаил:
Их у меня нет, тебе показалось. Были, но я победил. Имеется ввиду, что выбор Скопенкова в качестве принимающего очень странен.
 [Photo]
 [Photo]
 Это здание конторы новосухаревского рынка, я его сегодня нашел.
 Иногда мне кажется, что я что-то хочу тебе сказать, но проблема в том, что я плохо умею общаться, поэтому ничего не получается.
 Я умею писать бумажные письма, но это вряд ли поможет.
 Я только недавно начал понимать, что у меня впереди жизнь, и это удивительно.

--- Friday, February 17, 2017 ---

[12:23:04 AM] Маргулис:
Миш, а ты мне 2 и 4 задачки по матану не скинешь?
 Ты, наверное, уже не прочитаешь сегодня. Зайди ко мне в 214 завтра, если возьмёшь решения с собой
 Я, как всегда, все делаю в последнюю ночь

[1:32:40 PM] Маргулис:
А после лекции ты сможешь прийти?

--- Monday, February 27, 2017 ---

[9:07:25 PM] Маргулис:
Миш, привет, а ты умеешь решать 6 лист по геометрии?

[9:37:34 PM] Михаил:
1 3 4 6 7 8 10

[9:43:50 PM] Маргулис:
А можешь завтра взять с собой, я тебя опять подёргаю с этим всем?

[11:24:46 PM] Маргулис:
Не в смысле совсем всем

--- Tuesday, February 28, 2017 ---

[12:25:54 AM] Михаил:
У меня привычка иногда отвечать мысленно и забывать писать. Да, я расскажу о задачах.

[12:26:17 AM] Маргулис:
Слушай, ты чего-нибудь понимаешь вот с этим:
 [Photo]
 Одна информация
 А вот вторая
 $https://electives.hse.ru/mn_students$
Как записаться на майнор
[Photo]
 В одном месте говорят о 1 марта, в другом о 14
 Кстати, ты куда будешь записываться?

[12:27:57 AM] Михаил:
Я не знаю об этом ничего.

[12:28:13 AM] Маргулис:
Ты не думал о майноре?

[12:28:33 AM] Михаил:
Я не думал о майноре.

[12:28:53 AM] Маргулис:
Говорят, что от него можно отказаться вообще
 Но надо обсуждать это с научником

[12:30:04 AM] Михаил:
Я не знаю, в общем.
 Говорят, действительно.

[12:30:20 AM] Маргулис:
Понятно
 Я вот думаю об этом
 Но я не хочу тебя отвлекать ночью, так что завтра поговорим 
А заодно о геометрии
\end{document}