\documentclass{article}
\usepackage[utf8]{inputenc}
\usepackage[russian]{babel}
\usepackage{amsmath}
\usepackage{amsfonts}
\usepackage{mathdots}


\begin{document}
\title{}
\author{}
\date{}
\maketitle



--- Thursday, March 2, 2017 ---

[7:50:32 PM] Михаил:
Что у тебя получилось?
$ https://docs.google.com/spreadsheets/d/1-gtwLhbkErF7exliSkdz45vT3PDb3ZCe35d4GoQzMrw/editgid=0$

Я не очень знаю генезис этой штуки, но она есть.

Google Docs
Отзывы по майнораам
Лист1

Майнор, Расположение, Имя, Преподаватели, Сложности, Время на подготовку к семинарам и выполнение дз, Что понравилось, Что не понравилось, Неп...
 Тебе сегодня надо рано лечь спать. А завтра и на выходных решай матан. Листок приятный, я в каждой задаче рисуночек рисовал помимо интеграла.

[8:46:58 PM] Маргулис:
У меня получилось 7 за экзамен и 8 в результате. Это очень грустно
 Я не успела вчера сдать много задач
 Если бы сдала, то было бы 9 в результате
 Спасибо за табличку
 Я буду решать листок по алгебре
 Ну и матан
 Сейчас покажу, почему 7 за экзамен
 [Photo]
 Видишь отображение в центре картинки?
 Из гамма мю отдельно в гамма и мю
 Так вот, оно непрерывно
 Но я не смогла это нормально рассказать

[8:52:22 PM] Михаил:
Ага
 Там что-то про согласованность.
 В точке 1/2
edited 
[8:53:45 PM] Маргулис:
Это делается через какие-то там сужения и потом через то, что прообраз замкнутого замкнут
 Ой
 Замкнут

[8:54:08 PM] Михаил:
8 - это хорошо в действительности.

[8:54:26 PM] Маргулис:
Согласованность —это очевидно, я сказала
 Но это только показывает, что такое отображение задано корректно

[8:55:39 PM] Михаил:
А что нужно?

[8:56:05 PM] Маргулис:
Там что-то не очень приятное

[8:56:30 PM] Михаил:
Кому ты сдавала?

[8:56:54 PM] Маргулис:
Сначала Ване, потом конкретно эту задачу добивала львовскому
 Я не знала, что там что-то кроме того, что пути в точке 1/2 имеют общую точку, нужно

[8:58:26 PM] Михаил:
У тебя в листе и билет 17, и билет 18. Какой он?

[8:58:58 PM] Маргулис:
Это не мой текст
 Это я из Гугл диска тебе решение показала
 Я просто так же сделала
 Мой про корректность был
 17
 Видимо
 А потом нужно ещё непрерывность показать в следующем отображении, таким же образом

[9:01:46 PM] Михаил:
Думаю, дело в том, что надо доказать, что если а гомотопно а', b гом b', то  аb гом а'b' и это и есть корректность.

[9:03:12 PM] Маргулис:
а и а это пути, которые я обозначила через  и ?
 Так я это ниже сделала
 А, у тебя не мой лист
 Забыла
 Ну я показала
 Нужно задать саму композицию

[9:04:36 PM] Михаил:
Загадочно

[9:04:47 PM] Маргулис:
Не хватает непрерывности

[9:05:56 PM] Михаил:
Не понимаю. Завтра объясни в институте.

[9:06:05 PM] Маргулис:
Да, хорошо

[9:06:20 PM] Михаил:
У тебя есть мнение о Фрейде?

[9:06:50 PM] Маргулис:
Ты это по какой-то причине спросил или нет?
 А ты о Зигмунде или Люсьене?

[9:07:49 PM] Михаил:
Зигмунд. Я внезапно решил его почитать.

[9:08:57 PM] Маргулис:
Я не читала, но в целом я не очень люблю психологию, потому что ощущаю в любых обобщениях о людях нотку тоталитарности, и это становится мне неприятно
 Отдельно о психоанализе я давно не думала

[9:10:22 PM] Михаил:
Люди же действительно похожи, в этом ничего страшного.

[9:10:31 PM] Маргулис:
Психология даёт основание уничижительного отношения к проявлениям человечности

[9:11:09 PM] Михаил:
Не думаю.

[9:11:50 PM] Маргулис:
Ну вот какому-то человеку очень грустно, а ты все сводишь к каким-то формальностям и смеёшься над тем, что он так близко к сердцу принимает стандартный механизм функционирования мозга, например

[9:13:55 PM] Михаил:
Смысл в том, чтобы избавлять от грусти, помогать людям. По мне, идея психологии по сути человеколюбива.

[9:14:11 PM] Маргулис:
Ну, не знаю, мне кажется, что человек должен рассуждать о своих эмоциях в проекции на другую плоскость

[9:15:12 PM] Михаил:
Непонятно

[9:15:23 PM] Маргулис:
Потом могу сказать
 Но в ряде случаев она полезна, да

--- Friday, March 3, 2017 ---

[3:21:46 PM] Маргулис:
Кстати, я имела в виду не молекулярно-кинетическую теорию, а термодинамику
 И то, и другое имела в виду

[8:21:26 PM] Михаил:
В какие музеи ты ходишь?

[8:23:12 PM] Маргулис:
На выставки в Пушкинский, в третьяковку, в какие-нибудь более современные, если мне выставка интересна. Ещё есть еврейский музей, в нем очень здорово и тоже бывают хорошие художественные выставки

[8:26:20 PM] Михаил:
В еврейском я не бывал, мимо однажды только проходил. В эти выходные куда-нибудь идешь? Я вот думаю, может вместе сходим?

[8:27:13 PM] Маргулис:
Я ещё не решила. В принципе, я завтра свободна, можно сходить
 Я немножко не уверена, что в еврейском завтра не выходной

[8:27:45 PM] Михаил:
Действительно

[8:28:49 PM] Маргулис:
А через неделю у них пурим
 Ну точно, суббота выходная
 А ты в каком районе живёшь?

[9:01:13 PM] Маргулис:
http://www.arts-museum.ru/events/archive/2017/morimura/index.php

Вот эта выставка мне нравится, но я не уверена, что завтра хочу пойти. У меня это очень много энергии отнимает

[9:07:14 PM] Михаил:
Что именно отнимает?
edited 
[9:13:52 PM] Маргулис:
Я очень подолгу хожу по выставкам и очень активно при этом думаю

[9:16:13 PM] Михаил:
Это здорово на самом деле.
 Я же буду помогать, так что должно быть легче.

[9:19:49 PM] Маргулис:
Помогать думать?

[9:20:02 PM] Михаил:
Помогать думать.

[9:20:10 PM] Маргулис:
Хорошо
 Можно сходить

[9:20:51 PM] Михаил:
Во сколько?

[9:32:42 PM] Маргулис:
Не знаю
 А где мы встретимся? Ты где живёшь?

[9:33:14 PM] Михаил:
На ВДНХ.

[9:33:40 PM] Маргулис:
А я на щёлковской, не по пути
 Я вот не знаю, что там с очередями. На моей памяти они были только в главное здание, а в крайние никогда

[9:35:24 PM] Михаил:
Японский человек в крайнем?

[9:35:30 PM] Маргулис:
Да
 В музее стран Европы и Америки
 Забавно, кстати

[9:36:15 PM] Михаил:
Там я не был даже.

[9:36:28 PM] Маргулис:
Лучшие выставки обычно там и проходят
 Там была очень крутая выставка Бердслея несколько лет назад

[9:37:45 PM] Михаил:
Не знаю, кто это.

[9:38:08 PM] Маргулис:
Я до выставки не знала

[9:40:39 PM] Михаил:
В 12?

[9:40:45 PM] Маргулис:


Как-то так
обри бёрдслей - Поиск в Google
[Photo]
 В 12 там?
 Давай в 12:30
 Вообще смешно, там по крайней мере одна работа не его

[9:42:50 PM] Михаил:
Загадочные работы.

[9:43:06 PM] Маргулис:
В основном иллюстрации
 К известным вещам и к не очень
 Ну так что, в 12:30 или как?

[9:45:51 PM] Михаил:
Да

[9:45:56 PM] Маргулис:
Хорошо
 У меня есть твой номер

[9:46:55 PM] Михаил:
Потом можем поесть и дойти до Арбатской, например.

[9:47:05 PM] Маргулис:
Да, можно

[9:49:08 PM] Михаил:
В центре Кропоткинской.

[9:49:32 PM] Маргулис:
Договорились

--- Saturday, March 4, 2017 ---

[11:47:11 AM] Маргулис:
Я немножко позже буду
edited 
Подожди меня, если уже туда едешь

[11:51:54 AM] Михаил:
Да, я понял.

[11:53:57 AM] Маргулис:
Я отпечаталась

[12:54:09 PM] Михаил:
Непонятно.

[8:40:56 PM] Маргулис:
Опечаталась дважды. Точне, один, а слово отпечаталась придумал корректор

[9:55:42 PM] Маргулис:
http://arzamas.academy

Ты знаешь этот сайт? Там можно найти много интересных вещей, особенно если перейти в журнал и покопаться по архиву

Arzamas
История культуры в видео, текстах и фотографиях

[11:18:55 PM] Михаил:
Спасибо, я буду его исследовать.

[11:19:14 PM] Маргулис:
Не за что
 А вообще, Арзамас—это литературное объединение, где Пушкин состоял. Там про это тоже есть в спецпроекте 200 лет Арзамасу
 Ну и город

[11:20:53 PM] Михаил:
Есть план пожелать тебе спокойной ночи.

[11:21:13 PM] Маргулис:
Я ещё не ложусь, но спокойной ночи

[11:21:18 PM] Михаил:
Я тоже

--- Monday, March 6, 2017 ---

[11:38:54 PM] Михаил:
Я не дал это ясно понять, но у меня восторг и эйфория от по-новому открывшегося мне мира искусства. Это необычайно любопытно и загадочно, и ты очень интересный человек, мне безумно понравилось, я хочу узнавать вас ближе, тебя и искусство.

--- Tuesday, March 7, 2017 ---

[9:34:10 PM] Маргулис:
Слушай, Миш, я хотела тебе сказать вчера и сегодня, но не смогла найти момент, чтобы нас никто не мог слышать. Я не хочу встречаться, извини. Общаться можно, но не более. Не обижайся.

[9:37:33 PM] Михаил:
Очень хорошо.

[11:54:03 PM] Михаил:
Действительно ничего страшного, Маша, ты молодец, что сказала. Думаю, я тебя понимаю. Однако дело не в нас, а в искусстве и духовном обогащении. Я все равно хочу с тобой ходить в музеи. Обещаю в тебя не влюбляться)

--- Wednesday, March 8, 2017 ---
 С Восьмым марта, Маша. Позволю все-таки отметить, что считаю тебя красивой.

[12:10:59 AM] Маргулис:
Спасибо

--- Friday, March 10, 2017 ---

[6:33:31 PM] Михаил:
Какова интерпретация этой картины с падающим человеком? И летящий бородатый слон тоже что значит?

[7:37:51 PM] Маргулис:
Я ничего не поняла
 Которая сейчас стоит?
 Satire on romantic suicide называется
 Слон не бородатый
edited 
У него много хоботов
 Какая-то индийская легенда, я не заморачивалась со смыслом

[10:29:19 PM] Маргулис:
> Маргулис
> Я ничего не поняла
Это относилось к твоему вопросу, если что
 И он дурацкий
 Поставила то, что было раньше, потому что изучала, что с живописью было у индусов. Это просто иллюстрация, что ты называешь интерпретацией? А насчёт нынешней—чего ты именно от неё хочешь? Я так понимаю, ты её только в совсем мелком варианте видишь, почему не начать с с просьбы показать её в большом варианте, или можно спросить, что на ней или о чём она, но при чем тут интерпретация?

[11:36:54 PM] Михаил:
Я же имею ввиду смысл, трактовку.

[11:38:52 PM] Маргулис:
Есть вполне буквальные вещи
 А имеют в виду с пробелом) извини, не обращай внимания)
 Какой вообще смысл в твоём вопросе, если ты не видишь картину
 А название нынешней исчерпывает смысл

[11:41:35 PM] Михаил:
Ну хорошо.

[11:42:03 PM] Маргулис:
У меня плохое настроение

[11:42:36 PM] Михаил:
Мне тоже сегодня очень не по себе было.
 Паника, пустота, потеря чувства реальности.
 А что, собственно, у тебя случилось?

[11:56:57 PM] Маргулис:
Да всё в целом нормально.
 Не хочу рассказывать пока

[11:57:51 PM] Михаил:
Все будет хорошо, ты не грусти, пожалуйста.

[11:58:05 PM] Маргулис:
Ты тоже
 Не смотрел запах женщины?

[11:59:01 PM] Михаил:
Не нашел времени, зачитываюсь перемешкой Булгакова и Фрейда.

--- Saturday, March 11, 2017 ---

[12:00:09 AM] Маргулис:
И как тебе всё?

[12:01:39 AM] Михаил:
В дьяволиаде нет дьявола. По духу она похожа на изучение математики. Там ужасающий абсурд и непонятно ничего.

[12:02:34 AM] Маргулис:
А почему Булгакова ты вдруг решил всего прочитать?

[12:02:39 AM] Михаил:
"Вытесненные из сознания желания переходят в сферу бессознательного и продолжают влиять на поведение."

[12:02:49 AM] Маргулис:
Никогда не читаю подряд много 1 автора

[12:03:04 AM] Михаил:
Там короткие повести.

[12:03:10 AM] Маргулис:
Я имею в виду, именно Булгакова
 Там без слова именно теряется смысл
 А я всё равно всегда мешаю с чем-то
 А, ну ты тоже мешаешь, с Фрейдом

[12:05:14 AM] Михаил:
Мне Ракитин дал рассказы Брэдбери, знакомая дала Дом, в котором, а я читаю свое.
 Загадочно, в общем.
 Сегодня на Патриарших был, там знак убрали.
 Что не разговаривать с неизвестными.

[12:07:25 AM] Маргулис:
Его постоянно воруют
 Настолько, что я его не могу встретить

[12:08:21 AM] Михаил:
Мне везло.

[12:08:27 AM] Маргулис:
Я ради него туда привела родственников из Америки, а его не было
 А до этого мы были в его квартире
 Они её не так давно здорово переделали

[12:09:42 AM] Михаил:
Там висит ящик писем мастеру, я раньше не замечал.
 Давно в самой квартире был.

[12:10:20 AM] Маргулис:
Ты не о той квартире же, да?
 Ты о той, что на первом этаже
 А я о той, что высоко и в соседнем доме
 Есть дом-музей и есть квартира

[12:11:12 AM] Михаил:
Любопытно.
 Он жил не в музее?
edited 
[12:11:43 AM] Маргулис:
Нет, там он не жил
 А там, где жил, на более низких этажах жили Ларионов с Гончаровой
 Но пораньше
 А я писала письмо. Сбылось

[12:13:29 AM] Михаил:
Как попасть в квартиру?

[12:14:38 AM] Маргулис:
В другой подъезд того же дома зайти. Или в соседнем доме, который во дворе прямо по курсу при входе. Я не помню точно. На двери табличка с домофоном. В подъезде много граффити
 Раньше мне казалось, что в соседнем доме, а недавно поняла, что это тот же дом.
 Когда была этим летом
 Она как раз номер 50
 А вообще, в том доме много известных людей жило
 В квартире про это есть комната

[12:17:29 AM] Михаил:
Это интересно.
edited 
А как твой курсач? Мой научник намекнул в письме, что хочет меня видеть сегодня, но я не пришел, потому что у меня ничего нового с нашей прошлой встречи месяц назад.

[12:20:32 AM] Маргулис:
Мой не намекал, но у меня то же самое. Сдвиг есть только в моей голове, а мой хочет видеть уже в латехе

[12:21:09 AM] Михаил:
Ты умеешь пользоваться техом?

[12:21:17 AM] Маргулис:
Нет, но я научусь скоро
 Мне немножко лень этим заниматься. Гораздо более лень, чем самой курсовой.

[12:22:47 AM] Михаил:
Я нахожу в этом что-то приятное.

[12:23:01 AM] Маргулис:
В записи в техе?

[12:23:08 AM] Михаил:
Да.

[12:23:47 AM] Маргулис:
Мне проективка нравится все больше, но НИС плохой. Зря я не отписалась, чтобы пойти на дискру
 Я из-за курсовой не отписалась, чтобы быть логичной.
 Но на самом деле они не имеют друг к другу отношения
 Кроме того, что и там, и там есть проективка

[12:25:39 AM] Михаил:
Много рисунков делается?

[12:26:17 AM] Маргулис:
В курсовой нет, не очень. В нисе много. Но нис все равно скучный

[12:27:52 AM] Михаил:
Ты ходишь и на физику тоже. Почему так?

[12:28:10 AM] Маргулис:
В смысле?

[12:28:55 AM] Михаил:
Ты записана на два ниса?

[12:29:08 AM] Маргулис:
Да
 Многие так
 Ты тоже?

[12:30:26 AM] Михаил:
Нет, я только на физику, и меня удивило, что можно на несколько.

[12:31:06 AM] Маргулис:
Можно набирать 66 кредитов. В 1 семестре я была на 1 нисе. Поэтому у меня 63 кредита заполнены
 А ты когда-нибудь общался с очень суицидальными людьми?

[12:33:17 AM] Михаил:
Нет, я ни в ком, кроме себя, не встречал этих мыслей.

[12:34:05 AM] Маргулис:
А я сейчас общаюсь с одним таким человеком. Со школы общаемся

[12:34:55 AM] Михаил:
Надо с психологом говорить.

[12:35:04 AM] Маргулис:
И я не помню, когда это с ним началось.
 Он учится на психолога, кстати

[12:35:34 AM] Михаил:
Смешно

[12:36:21 AM] Маргулис:
Он описывает очень странные вещи. Есть ощущение, что у него проблемы с физическим здоровьем, а не с душевным
 Но, конечно, ко врачу он не идёт

[12:37:28 AM] Михаил:
Органическое поражение мозга вроде шизофрении?

[12:38:48 AM] Маргулис:
Нет. Я даже не знаю, связано ли это всё. Но у него супер низкое давление, проблемы с сердцем и не только
 Что-то от солнца ещё с кожей. Похоже на гормональное
 Я уже так давно от этого устала.
 Но уйти не могу
 Иногда не могу уйти всю ночь из переписки

[12:40:30 AM] Михаил:
Да, суровое дело.
 Может, я как-то с ним пообщаюсь?

[12:41:39 AM] Маргулис:
Вряд ли. Не знаю сейчас, как ему это преподнести

[12:43:07 AM] Михаил:
У него бывают проблемы с потерей чувства реальности?

[12:43:36 AM] Маргулис:
Ещё какие-то ужасные проблемы с концентрацией. Начинаются головные боли при длительных нагрузках. Это страшно.

[12:43:36 AM] Михаил:
Он испытывает одиночество?

[12:43:42 AM] Маргулис:
Испытывает
 Про чувство реальности не знаю

[12:44:14 AM] Михаил:
Да, жутко.

[12:45:41 AM] Маргулис:
Тебе спать пора, наверное

[12:45:56 AM] Михаил:
Тебе тоже

[12:46:04 AM] Маргулис:
Да

[12:46:17 AM] Михаил:
Спокойной ночи, Маша.

[12:46:22 AM] Маргулис:
И тебе

[7:19:04 PM] Михаил:
Надо начать с того, чтобы он отделил себя от своей болезни, воспринял ее как нечто инородное, противоположное ему, его хорошим чертам характера. Я не вижу пока четкой картины, но, может быть, это соматические проявления невроза, и ему нужно найти внутренний конфликт в себе и разрешить его. Что у него в жизни происходило?

[7:27:02 PM] Маргулис:
Извини, я занята сейчас

[7:28:09 PM] Михаил:
Хорошо, потом ответь.
 Ты единственный человек, с кем он общается? Как его зовут, кстати?

[7:30:35 PM] Маргулис:
Сергей
 Нет, уже не единственный, у него новые знакомые в вузе
 Ну, я надеюсь, что с ними есть какое-то неофициальное общение

[7:35:26 PM] Михаил:
На тебе большая ответственность. Я думаю, мы справимся, надо не бояться. Еще он должен осознать свою причастность к жизни, неразрывную связь с миром. Черт, правда, знает, как...

[7:38:01 PM] Маргулис:
Миш, ты перебарщиваешь.

[7:38:10 PM] Михаил:
Ок
edited 
[7:38:15 PM] Маргулис:
Мне не нужно писать об ответсвенности и этом всем.
 Думаю, я сама разберусь. Не заморачиватся с этим.

[7:39:41 PM] Михаил:
Я понял.

--- Sunday, March 12, 2017 ---

[3:49:49 AM] Маргулис:
Извини, но я уже несколько раз пишу тебе и удаляю одну вещь.
edited 
Феномен, из-за которого я вела себя определенным образом, называется вежливость.
 Поэтому я не знаю, в какой момент я могла дать повод для заблуждений. Я не давала этих поводов.
 И это значит, что обнимать меня на выставке было более чем неприлично. Хотя бы потому, что при людях.
 Я пришла на выставку, а это меня от неё ужасно отвлекало и не давало прочувствовать. Можно было хотя бы спросить, можно ли это сделать.
 Осадок у меня остался ужасно неприятный. Не извиняйся и не обсуждай это со мной. Просто мне нужно это высказать.

[6:41:07 PM] Михаил:
Я понял.

--- Monday, March 13, 2017 ---

[6:19:49 PM] Михаил:
Почему ты сегодня не пришла?

[9:22:03 PM] Маргулис:
Просто так, не захотела идти

[10:34:29 PM] Михаил:
Очень теплеет сейчас, хорошо. Ты не грусти только, и выспись тоже, пожалуйста.

[10:43:53 PM] Маргулис:
Ты тоже не грусти
 Я уже выспалась
 Делаю алгебру

[10:48:15 PM] Михаил:
Меня с Сашей Каганом сегодня Аня учила, как ее делать, сам я не мог разобраться.

[10:55:15 PM] Маргулис:
Ты о чём?
 Ты делаешь листочек?
 Если ты про идз, то я уже выражала отказ его делать

[11:01:00 PM] Михаил:
Я понял.

[11:03:26 PM] Маргулис:
А я ничего не поняла
 Спрашиваешь у человека, о чём он, а он отвечает: я понял
 Это смешно

[11:04:28 PM] Михаил:
Действительно смешно.

[11:06:05 PM] Маргулис:
Чёрт возьми
 Ладно
 Насчёт того человека, о котором я тебе рассказывала: вы просто одинаковые.
 Я никогда не пойму, почему нужно устраивать из ответа на вопрос какую-то сцену, а потом не отвечать.
 Как же это надоело.

--- Tuesday, March 14, 2017 ---

[9:01:37 PM] Маргулис:
http://www.world-art.ru/lyric/lyric.php?id=1531
 Цветаева
 https://m.rupoem.ru/poets/gumilev/shel-ya-po
Гумилёв

m.rupoem.ru
Заблудившийся трамвай. Николай Гумилев
Читать стихотворение Николая Гумилева «Заблудившийся трамвай», написанное в 1919 году. Шёл я по улице незнакомой И вдруг услышал вороний грай, И звоны...
 У блока я люблю Девушка пела в церковном хоре, Из газет, и что-то про каменного гостя
 И ещё что-то про первую мировую у блока нашла недавно очень хорошее
 У Ахматовой тоже есть стихотворение времён самого начала первой мировой
 У Мандельштама Стихи о неизвестном солдате
 Нужно искать стихи самой поздней версии, с вот этой частью 
"Но закончилась та перекличка, и пропала, как весть без вестей, и по выбору совести личной, по указу великих смертей, я, дичок испугавшийся света, становлюсь рядовым той страны, у которой попросят совета все, кто жить и воскреснуть должны..." и так далее, по памяти записала
 И союза её гражданином становлюсь на призыв и учёт, и Вселенной её семьянином всяк живущий меня назовёт

[11:49:38 PM] Михаил:
Отлично, спасибо, Маша. Сейчас я пытаюсь решать дискру, стихи когда-нибудь скоро обязательно прочитаю.

--- Wednesday, March 15, 2017 ---

[4:53:45 PM] Михаил:
Что ты, Маша, делала сегодня?

[7:10:40 PM] Маргулис:
Болела
 И завтра буду болеть
 Это, конечно, кошмар, потому что сделать всё я успеть могу, но я больная и не могу сдавать
 А сейчас и решать

[9:11:55 PM] Михаил:
Грустно, да. А что с тобой, если не секрет?

[9:43:10 PM] Михаил:
К этому можно отнестись философски, в соответствии с принципом "Не для школы, а для жизни мы учимся". То есть важно именно то, как ты развиваешься, решая задачи. Но все равно жалко, и болеть вообще плохо. Выздоравливай, пожалуйста.

--- Thursday, March 16, 2017 ---
edited 
[12:33:55 AM] Маргулис:
Ну горло болит
 Кстати
 Зачем писать, что я делала, в 5 вечера?
 Я в 5 ещё ничего не начинаю обычно
 Температура была немножко
 Какие-то дурацкие у тебя вопросы, не задавай такие больше
 В институте могу сказать, что я и когда делала

[9:39:14 PM] Маргулис:
Как ты читаешь наши лекции по алгебре без учебников? Эти конспекты хуже Википедии

--- Friday, March 17, 2017 ---

[12:07:07 AM] Михаил:
Сидя на лекциях, я стараюсь понимать.
 Я считаю это важным - до конца не терять нить. Это помогает при восприятии конспектов.
 Мы в эти выходные пойдем куда-нибудь? Я обещаю себя хорошо вести. В музее Москвы сейчас выставка оттепели.

[12:24:37 AM] Маргулис:
Я была на прошлой неделе, это раз.
 Про два я и говорить не буду
 И ещё есть очевидное три
 Неужели непонятно, что у меня полно математики, которую надо делать? Я эту алгебру ещё не доделала
 И самое главное, на каком языке мне нужно сказать, что я болею, чтобы это стало понятно
 Я люблю ходить одна, не люблю, когда мне задают вопросы
edited 
Я сама спокойно все анализирую
 Я не пойду больше никуда с тобой, потому что ты предлагаешь мне пойти куда-то в то время, когда мне нужно заниматься математикой. За неделю до сессии. В тот раз ты позвал меня ещё хуже.
 Нет, сейчас хуже
 Но тогда я тоже была очень занята
 Не пиши мне больше здесь.
 И у меня ещё есть курсовая
 Всё, отстань от меня.
 Возможно, я тебя заблокирую здесь, потому что ты позволяешь себе меня отвлекать, в то время как сам никогда не отвечаешь сразу, потому что, видите ли, занят.

--- Tuesday, March 21, 2017 ---

[3:56:01 PM] Маргулис:
Все будет хорошо

[5:45:53 PM] Маргулис:
https://m.rupoem.ru/poets/pasternak/prizhimayus-schekoyu-k

m.rupoem.ru
Зима. Борис Пастернак
Читать стихотворение Бориса Пастернака «Зима», написанное в 1913 году. Прижимаюсь щекою к воронке Завитой, как улитка, зимы. "По местам, кто не хочет - к сторонке!" Шум
 http://www.stihi-xix-xx-vekov.ru/pasternak63.html
www.stihi-xix-xx-vekov.ru
Борис Пастернак - поэма «Высокая болезнь»
Борис Леонидович Пастернак - поэма «Высокая болезнь»
 https://m.rupoem.ru/poets/pasternak/melo-melo-po

m.rupoem.ru
Зимняя ночь (Мело, мело по всей земле...). Борис Пастернак
Читать стихотворение Бориса Пастернака «Зимняя ночь (Мело, мело по всей земле...)», написанное в 1946 году. Мело, мело по всей земле Во все пределы. С...
 https://m.rupoem.ru/poets/pasternak/za-oknami-davka

m.rupoem.ru
После дождя. Борис Пастернак
Читать стихотворение Бориса Пастернака «После дождя», написанное в 1915 году. За окнами давка, толпится листва, И палое небо с дорог не подобрано. Все стихло. Но что это было с
 У Пастернака много всего скоро кину
 Я в метро
 Уже некогда

[7:33:05 PM] Маргулис:
Миш, слушай
 Если ты не очень занят
 Посмотришь 18 и 19 задание в листочке по алгебре? Вдруг у тебя будут идеи

[9:55:51 PM] Михаил:
Я начал смотреть.
 18 а гуглится. math.phys.msu.ru/data/25/LinearAlgebra5.pdf
[LinearAlgebra5.pdf] 114 KB
 
[10:08:12 PM] Маргулис:
Спасибо

[11:06:41 PM] Маргулис:
Я не могу зайти на Гугл диск
 Это конец

[11:59:11 PM] Маргулис:
Все очень плохо

--- Wednesday, March 22, 2017 ---

[12:25:28 AM] Михаил:
Там конспекты лекций?
 Ну, может, потом получится зайти. Это как бы техническая неполадка? Зачем тебе сейчас гугл диск?
 Не расстраивайся сильно. Нужно больше бесконечного непоколебимого оптимизма.
 Я могу помочь?

[12:39:34 AM] Маргулис:
Прости, я не видела сообщений
 Ну, мне нужно было искать ответы к алгебре
 Она сложная во второй половине, но немножко есть в лекции
 Я зашла с компа

[7:39:02 PM] Маргулис:
Скажи мне, Миша
 Ты понимаешь номер про экспоненту?
 Он 15-й
 У меня ещё такая штука есть
 $https://en.wikipedia.org/wiki/Exponential_polynomial
Wikipedia
Exponential polynomial
In mathematics, exponential polynomials are functions on fields, rings, or abelian groups that take the form of polynomials in a variable and an exponential function.$

[11:53:26 PM] Михаил:
Экспонента - это ряд. У оператора есть аннулирующий многочлен, мы можем рассмотреть ряд по модулю этого многочлена.

--- Thursday, March 23, 2017 ---

[1:04:42 AM] Маргулис:
Какой ты хороший
 Я схожу с ума
 У меня скоро будет нервный срыв

[3:17:26 PM] Михаил:
Сейчас ты как себя ощущаешь? Отдохни, если перезанималась.

[3:18:49 PM] Маргулис:
Я сдаю

[3:19:02 PM] Михаил:
Ты на матфаке?

[3:19:23 PM] Маргулис:
Да

[10:01:01 PM] Маргулис:
Покажешь мне свою дискру? Я с ума сойду, не могу больше ничего решать
edited 
А ещё я хотела спросить про номера 3 и 4 в матане
 Ты ходишь к питре?

[10:13:01 PM] Михаил:
[Photo]
edited 
[10:14:21 PM] Маргулис:
Кстати, что за буква C перед всякими промежутками? Я забыла, давно не делала матан
 Что это за фамилия на Л? Да, я правда ничего не помню

[10:15:38 PM] Михаил:
С - это непрерывные функции, С1 - непрерывные, дифференцируемые с непрерывной производной.

[10:16:37 PM] Маргулис:
Спасибо
 К Питре ходишь?

[10:18:18 PM] Михаил:
Нет.

[10:19:09 PM] Маргулис:
Я тоже
 Была раза 2
 И завтра не хочу идти

[10:20:04 PM] Михаил:
Я не пойду.

[10:20:17 PM] Маргулис:
Я хочу лечь рано, прийти пораньше и делать там дискру
 Но мне надо разобраться с геомой

[10:21:06 PM] Михаил:
С листком 6?
 Матан без номеров 5 и 10 я нашел.
 Дискра моя точно в 219.

[10:24:32 PM] Маргулис:
Да
 Лист 6
 Он у меня есть

[10:24:56 PM] Михаил:
Я ее сдавал человеку по имени Дмитрий Олегович. Ты ему сегодня что-то сдавала.

[10:25:10 PM] Маргулис:
Что такое 219? Я не помню такого кабинета
 Дмитрий Коршунов? 319?
 Это, наверное, 3 этаж

[10:25:51 PM] Михаил:
Рядом с 219

[10:26:11 PM] Маргулис:
Где этот кабинет?

[10:26:37 PM] Михаил:
Ладно, видимо, его нет.

[10:26:43 PM] Маргулис:
Это на 2 этаже? Ты ничего не путаешь?
 Не рядом с компьютерным классом?

[10:27:50 PM] Михаил:
Под компьютерным классом.

[10:27:58 PM] Маргулис:
214?
edited 
Комната для самостоятельных занятий?

[10:28:34 PM] Михаил:
А кто-то зовет ее боталкой.

[10:29:17 PM] Маргулис:
То есть, ты хочешь сказать, что твоя дискра может быть там?

[10:29:39 PM] Михаил:
Да

[10:29:47 PM] Маргулис:
Хорошо

[10:33:51 PM] Михаил:
У тебя очень красивый голос.

[11:05:15 PM] Михаил:
Я ко второй паре приду, расскажу что-нибудь. Там несколько ошибок есть у меня в дискре.

[11:20:51 PM] Маргулис:
Хорошо
 Тогда встретимся на 2 паре

--- Friday, March 24, 2017 ---

[6:53:00 PM] Михаил:
Отличная погода.

[7:44:24 PM] Маргулис:
Ага
 Я промокла и замёрзла

--- Saturday, March 25, 2017 ---

[8:37:57 PM] Маргулис:
Ты не делал билеты?

[9:43:56 PM] Маргулис:
Миш, поделись последними 3 семинарами по геоме

[10:22:38 PM] Михаил:
[Photo]
 [Photo]
 [Photo]
 Я ужасно веду семинары.
 Билеты я не делаю. Непонятные места я пишу, но бессвязно.

--- Sunday, March 26, 2017 ---

[5:14:08 PM] Маргулис:
Так, Миша, мне нужно уточнить, всё ли у тебя хорошо
 Ты дома?

[5:16:29 PM] Михаил:
Я дома.
 Я готовлюсь к сессии, я никуда не пошел.

[5:17:12 PM] Маргулис:
Это хорошо
 Мне сказали, что одного моего знакомого задержали
 Как там наши? Никто ничего не писал?

[5:18:15 PM] Михаил:
Второкурсников каких-то задержали.
$ https://vk.com/wall232091520_652$
Roman Krutovsky
Только что на моих глазах задержали Илью Думанского и Петра Огарка за то, что они стояли у памятника Пушкина!!!! UPD: Ребята целы, едут. ОнНамНеДимон


[5:23:08 PM] Маргулис:
Двух моих знакомых, кстати. Он там не один был
 Ну, не считая второй курс

[5:42:24 PM] Маргулис:
Что значит едут?
 Домой или в отделение?

[8:47:39 PM] Маргулис:
Завтра будь рядом на экзамене, мне так будет спокойнее. Я достаточно готова, так что дело не в том, что я хочу использовать тебя

[10:17:04 PM] Михаил:
Forwarded message: Anastasia Sukacheva [3/26/17] 
Илья уже вышел
С Петей пока непонятно

[10:18:11 PM] Михаил:
Я буду рядом, хорошо.
 Очень хороший вопрос, когда начало.

[10:22:56 PM] Михаил:
Forwarded message: Sanek Kirillov [3/26/17] 
[Photo]
Forwarded message: Sonya Ivanova [3/26/17] 
[Photo]

[10:40:19 PM] Михаил:
Я приду в десять, с Каганом поготовлюсь.

[11:19:29 PM] Маргулис:
Я полагаю, что 12
 Вряд ли к 11 приеду
 Толком не готовилась, как всегда)

[11:22:46 PM] Михаил:
Очень интересно, что же ты делала) Готовиться надо, ты же понимаешь. Жизнь - это ведь во многом борьба с самим собой, со своими ошибками и недостатками.

[11:23:26 PM] Маргулис:
Ну, я думала о том, что нужно готовиться
 Переживала за знакомого
 Его не отпустили
 Не знаю, можно ли ему звонить
 И не поздно ли уже по времени
 Ну, у меня же ещё есть время
 На подготовку

[11:25:07 PM] Михаил:
Спать уже надо

[11:25:19 PM] Маргулис:
Не, спать в 3
 И я многое помню и без подготовки
 Потому что решала задачи

[11:26:46 PM] Михаил:
Звучит почти оптимистично.

[11:27:04 PM] Маргулис:
Что рассказывали наши? Сколько в билете вопросов и задач? Как обычно?

[11:29:15 PM] Михаил:
Как в программе сказано, так и есть, думаю. Два вопроса и задача.

[11:29:28 PM] Маргулис:
Там сказано не так
 Там сказано 1-2 задачи и 1-2 вопроса
edited 
[11:32:20 PM] Михаил:
Кеша так думает, и Миша Плахов тоже.

[11:32:20 PM] Михаил:
Forwarded message: Кеша Хумонен [3/26/17] 
2 вопроса
1 задача

[11:32:39 PM] Маргулис:
Хорошо

--- Monday, March 27, 2017 ---

[1:13:39 AM] Маргулис:
Я вряд ли приеду сильно раньше, потому что я отвлекаюсь на вас от подготовки. Сядь со мной, мне, наверное, будет нужна твоя помощь
 Прости, что так вышло
 Больше так не будет

[2:57:30 AM] Маргулис:
Ну то есть точно не будет.
 Я больше не буду не готовиться
 Прости, это безответственно
 На экзамене по алгебре я нормально подготовилась и получила 10

[11:07:35 AM] Маргулис:
Ещё не началось?

[2:32:05 PM] Михаил:
[Photo]
 Если не поздно.

[3:54:17 PM] Михаил:
Ты не обиделась, что я ушел? Как ты сдала?

[4:00:25 PM] Маргулис:
Не обиделась
 Я сдала на 8, он за этот вопрос два балла снял
 Но он сказал, что я на зачёте смогу оценку исправить

[5:04:12 PM] Маргулис:
Грустно что-то

[5:05:41 PM] Михаил:
Что именно? Насчет знакомого, насчет оценки?

[5:05:54 PM] Маргулис:
Знакомого
 Оценку исправлю на зачёте

[6:00:48 PM] Михаил:
Что ему вменяют? Хотят дать 15 суток или просто не отпускают пока? Его же когда-нибудь отпустят так или иначе, не волнуйся. Полицейские и сокамерники вежливые? Он где, ты знаешь?

[6:01:14 PM] Маргулис:
Он в преображенском увд провёл ночь

[6:01:21 PM] Михаил:
Вот кого-то уже выпустили.

[6:01:21 PM] Михаил:
Forwarded message: 
Petr Ogarok [3/27/17] 
Да, ещё ночью. Меня лично выпустили около полуночи, Илью - тремя часами пораньше.

[6:01:26 PM] Маргулис:
Хотят дать 15 суток

[6:01:50 PM] Михаил:
Он что-то активное делал,
 ?

[6:01:52 PM] Маргулис:
Вроде с сокамерниками все нормально
 Не знаю
 Не факт
 С кроссовками шёл
 Ну их было трое на 2 койки
 Не знаю, спал ли он
 В соседних камерах четверо на две

[6:03:14 PM] Михаил:
Нелегко им там, это верно.

[6:04:02 PM] Маргулис:
Его вот хумонен и гехт знают

[6:04:17 PM] Михаил:
Во сколько его задержали и где?

[6:04:26 PM] Маргулис:
Около 5 часов, кажется
 Может раньше
 Ну то есть точно раньше, но я не знаю, насколько
 На тверской
 Близко к бульвару
 Но ещё не на самой пушкинской
 Не спрашивай сам ничего у хумонена с гехтом, ок?

[6:06:12 PM] Михаил:
Ты волнуешься, что ему дадут 15 суток и он пропустит важные вещи в институте?

[6:06:18 PM] Маргулис:
Да
 Именно на экзамен

[6:06:27 PM] Михаил:
Не буду

[6:06:56 PM] Маргулис:
Я бы написала им, но вряд ли что-то новое услышу
edited 
Вообще, он какое-то время в 57 учился, а потом в 1543
Его многие наши знают
 1543, опечатка

[6:08:58 PM] Михаил:
Если его сегодня выпустят, он не пропустит? Экзамен когда?

[6:09:04 PM] Маргулис:
Завтра
 В 57 в матклассе, в 1543 в гуманитарном

[6:09:50 PM] Михаил:
Как его имя?

[6:09:58 PM] Маргулис:
Дмитрий Крылов
 Только не спрашивай про него у наших для меня, мне будет неудобно
 Хотя он сам меня к ним послал
 Только не говори, что ты его знаешь тоже

[6:11:55 PM] Михаил:
Не буду, не буду. Он сам расстроен? Не знаю его.
 Ты не больше него переживаешь?

[6:12:23 PM] Маргулис:
Меня просто пугает количество его знакомых
 Я постоянно их встречаю
 Я больше него переживаю
 Потому что он ничего не хочет рассказать
 Но вообще, он переживал из-за экзамена
 Сильно заранее готовился

[6:14:02 PM] Михаил:
Бывают в жизни неприятности, да.

[6:14:19 PM] Маргулис:
Я не знаю, можно ли ему звонить, боялась, что телефоны отнимают, и тогда я бы его подставила
 Ну, в любом случае, я узнаю, когда он выйдет

[6:16:40 PM] Михаил:
Риск, что заберут, был велик, это ясно. Там кормят их хоть?
 У него высокий средний балл?

[6:17:36 PM] Маргулис:
Думаю, да, но я постеснялась спросить
 Потому что у меня всего 9,22 и мне было бы неудобно
 Им вроде туда принесли печенье, соки и прочее
 Колбасу, хлеб
 Ну мне так передали
 А то, что там дают—какая-то дрянь
 Я не знаю, откуда другой мой бывший одноклассник это знает, вроде они лично не переписывались
 От кого-то ещё узнал, наверное
 Так что это все неточно
 Он был там не один, но взяли, видимо, только его из компании. Я знаю точно одного человека, не знаю, было ли больше

[6:21:48 PM] Михаил:
Сдавать на неделе пересдач - это не так уж плохо.

[6:23:24 PM] Маргулис:
А 15 суток кончатся к пересдачам?

[6:24:26 PM] Михаил:
Не знаю, когда пересдачи. По-моему, после 4 модуля. Нужно только надеяться, что не дадут 15 суток.

[6:24:44 PM] Маргулис:
Почти невероятно. Но я надеюсь

[6:25:52 PM] Михаил:
Какие-нибудь юристы могут помочь? Он пытается их искать?

[6:26:02 PM] Маргулис:
Я ничего не знаю
 Он сказал, что у него разрывается телефон, что все нормально и спросить у кого-нибудь ещё
 Но он бы вряд ли сказал, что все плохо, что он дурак, что пошёл перед сессией на митинг, и что расстроился из-за экзаменов

[6:31:38 PM] Михаил:
Я тебя понимаю. Но это политика, в ней постоянно происходят плохие вещи.
 Ты на его страницу вконтакте ходила?

[6:38:09 PM] Маргулис:
Нет
 Меня же там нет
 А у него пару лет назад была закрыта от неавторизованных

[6:38:58 PM] Михаил:
$https://vk.com/wall48038134_4918$
$ https://vk.com/wall48038134_4931$
 Работает?

[6:39:53 PM] Маргулис:
Нет
 Да ладно, все нормально будет

[6:42:43 PM] Михаил:
Я это и имею в виду.
 Конечно
 У него написано, что суд сегодня до восьми вечера.

[6:47:44 PM] Маргулис:
Не слишком долго?
 Мне говорили, что утром начинается

[6:48:08 PM] Михаил:
То есть в какое-то время с утра до восьми вечера

[6:48:15 PM] Маргулис:
Наверное, в порядке очереди?
 Готовься лучше к геометрии

[6:49:25 PM] Михаил:
[Photo]
 Это у него там

[6:50:12 PM] Маргулис:
Совершил(а), (совершило)

[6:50:48 PM] Михаил:
Кое-кому тоже надо готовиться.

[6:50:56 PM] Маргулис:
Надо
 Я не мог открыть Гугл диск
 Сейчас опять через комп полезу
 Все нормально с контрольной будет

[6:53:04 PM] Михаил:
Да, тема понятная.

[7:20:25 PM] Маргулис:
Я нашла лекцию, где есть немножко сферической геометрии
 Её не будет?
 Миш, а папка на Гугл диске с лекцией 02.09 пустая или не могу открыть?

[9:55:42 PM] Маргулис:
Я уже смогла открыть папку
 Можно я опять у тебе завтра сяду?)
 Я постараюсь приехать чуть раньше, чем вовремя

[10:05:23 PM] Михаил:
Сферической не будет. Пожалуйста, садись. Я странно готовлюсь. Прочитал повесть "Морфий", что было потом, не помню.
edited 
[10:14:18 PM] Маргулис:
Но ты же и так всё помнишь

[10:15:14 PM] Михаил:
Сейчас я выясняю владельцев российских сми.

[10:20:15 PM] Маргулис:
А ты какой-то "список Шварцмана" того, что нужно знать, видел и знаешь?

[10:21:44 PM] Михаил:
$:triangular_ruler:$Необходимо знать к КР по геометрии:

1. Модель Кэли-Клейна. Уметь считать расстояния (между точками, между расходящимися прямыми);
2. Знать 2 теоремы косинусов (если используются какие-то новые формулы, необходимо уметь выводить их через эти теоремы), теорему синусов;
3. Знать, как считать площадь гиперболического выпуклого n-угольника, если известны его углы;
4. Принцип переноса действия (см. фото);
5. Знать что-то про отражения (см. фото).

Для решения 4-х задач из будущей контрольной необходимо и достаточно перечисленных навыков. Пятая задача будет посложнее, поэтому знания, нужные для её решения нельзя объяснить в двух словах.

:exclamation:Шварцман снова обращается к очень умным ребятам и просит их выводить умные формулы, которые они используют.

[10:22:03 PM] Маргулис:
Хорошо

[10:22:26 PM] Михаил:
[Photo]

[10:22:58 PM] Маргулис:
Не поняла сути фото
 Попробую лечь пораньше
 Просто две формулы?

[10:23:43 PM] Михаил:
Это пункты 4 и 5 из текста.

[10:24:10 PM] Маргулис:
Forwarded message: 
Liza [3/27/17] 
[Photo]

[10:24:34 PM] Маргулис:
Почему тут просил скопенков сделать 2 минуса и плюс в скалярном?
 А не два плюса и минус

[10:25:53 PM] Михаил:
Я бы сказал, что он не прав.

[10:26:30 PM] Маргулис:
Он сказал, давайте так, это удобно
 А, все, я поняла

[10:27:22 PM] Михаил:
Это действительно удобно.

[10:27:54 PM] Маргулис:
Это он чтобы к 1, а не -1 приравнять
 Спать хочу

[10:28:55 PM] Михаил:
В формуле расстояния между точками везде минусы, а если писать так, то их нет.
 [Photo]

[10:29:39 PM] Маргулис:
Ты имеешь в виду, что в билинейной форме 1 минус
 ?
 А
 Ты об этом
 Я тебя поняла
 Ну минусы же все равно бы здесь сократились

[10:32:47 PM] Михаил:
Это маловажный нюанс, я не буду этим пользоваться. Сигнатура все-таки 2,1. Шварцман так не делал.

[10:33:19 PM] Маргулис:
Ну я поняла

[10:37:11 PM] Михаил:
Ложись спать тогда. Спокойной ночи.

[10:37:25 PM] Маргулис:
Надо все-таки что-то повторить
 Скоро пойду
 Часов в 12
 Надеюсь, раньше
 Если бы с 12 была контра, могла бы утром повторить
 А сейчас никак

[10:39:36 PM] Михаил:
У меня странное чувство, потому что я не понимаю, что повторять.

[10:40:00 PM] Маргулис:
Последние 2 месяца
 Я тупая и не помню геому, потому что не ходила на неё и не учила

[10:42:12 PM] Михаил:
Ну, ты ведь отлично знаешь, что делать в будущем - ходить и учить.

[10:42:23 PM] Маргулис:
Знаю
 Ну у меня был такой период в жизни

[10:43:32 PM] Михаил:
Да, периоды в жизни - дело серьезное, я знаю.

[10:44:12 PM] Маргулис:
Это сарказм?

[10:45:13 PM] Михаил:
Нет

[10:58:30 PM] Маргулис:
Хочу приехать пораньше, чтобы тебя никуда не отпустить
 С тобой спокойно и ты все помнишь

--- Tuesday, March 28, 2017 ---

[12:13:44 AM] Михаил:
Мне приятно, что я нужен.

[12:14:40 AM] Маргулис:
А мне приятно, что со мной бывает интересно
 Все, я легла

[9:08:31 AM] Маргулис:
В каком ты будешь кабинете?
 Я, к сожалению, только к 10 приеду
 Может, 9.55

[4:55:45 PM] Маргулис:
Он написал, что ему 8 суток осталось
 Могли же 7 дать, а дали 10
 Осталось бы 5 дней

[7:02:32 PM] Михаил:
Теперь неизвестности нет хотя бы. На стене он пишет, что ему много еды приносили. Опять-таки, это власть, политика, здесь плохие вещи часто происходят.
 [Photo]
 [Photo]
 [Photo]
 [Photo]
 [Photo]
 [Photo]

[7:32:02 PM] Маргулис:
Сегодня?
 Нет, подожди
 Сегодня не столько снега

[9:20:35 PM] Маргулис:
Как там у тебя?

[9:59:56 PM] Маргулис:
http://www.stihi-xix-xx-vekov.ru/severynin105.html
А Северянин как тебе?
www.stihi-xix-xx-vekov.ru
Игорь-Северянин - «Сонаты в шторм»
Игорь-Северянин - стихотворение «Сонаты в шторм»

[10:00:57 PM] Михаил:
[Photo]
 [Photo]
 [Photo]
 [Photo]
 [Photo]
 [Photo]
 [Photo]

[10:01:10 PM] Маргулис:
Он, конечно, довольно провинциален, но не самый плохой поэт

[10:01:30 PM] Михаил:
Так рядом с моим домом. Не знаю, кто такой Северянин.

[10:02:49 PM] Маргулис:
Я вспомнила, что должна тебе отправить

[10:03:15 PM] Михаил:
Яркий образ в стихе, интересный.

[10:03:32 PM] Маргулис:
$http://slova.org.ru/guro/gorod_pahnet_kroviu/$
slova.org.ru
Елена Гуро — Город
Елена Гуро - Город
 Северянин, Маяк и Гуро—очень разные футуристы
 Надо дать тебе книжку Северянина

[10:06:29 PM] Михаил:
Сильный стих. Я его почувствовал.
 Чересчур безнадежный, пожалуй. Я верю в людей.

[10:07:25 PM] Маргулис:
Да, сильный
 Такое, пожалуй, встречается редко
 http://www.world-art.ru/lyric/lyric.php?id=7697

[10:11:20 PM] Михаил:
Разве в самом деле так обстоят дела?

[10:11:43 PM] Маргулис:
Предупреждаю, в тексте есть очевидная опечатка

[10:11:51 PM] Михаил:
Люди в основном пустые и невосприимчивые?

[10:12:28 PM] Маргулис:
Думаю, я сейчас не так определяю основную массу людей
 Наверное, это всё синонимично
 Очевидная опечатка — Я знал, что я существую, пока ты была со мной 

Без не, разумеется 

Там написано— Я не знал
 Там лишнее НЕ и нету Я
 Ну как?
 http://www.world-art.ru/lyric/lyric.php?id=7425

[10:20:00 PM] Михаил:
Очень романтическая атмосфера

[10:20:09 PM] Маргулис:
Быкову очень нравится это стихотворение
 И мне
 Ну вот этот кусок с райскими кущами с адом голосов за спиною—это великолепно

[10:21:42 PM] Михаил:
Да, я согласен.
 Быков - это кто?

[10:22:22 PM] Маргулис:
Эх)
 Журналист, литературовед
 https://m.rupoem.ru/poets/mayakovskij/poslushajte-ved-esli

m.rupoem.ru
Послушайте!. Владимир Маяковский
Читать стихотворение Владимира Маяковского «Послушайте!», написанное в 1914 году. Послушайте! Ведь, если звезды зажигают - значит - это кому-нибудь нужно? Значит - кто-то хочет, ч

[10:26:10 PM] Михаил:
Коридор я не понял, про звезды тоже.
 Звезды - это метафора чего-то? Чего?

[10:27:02 PM] Маргулис:
Звезды—это звезды

[10:28:20 PM] Михаил:
Звезды же не зажигают. И бога будто бы нет.

[10:28:48 PM] Маргулис:
И почему из этого следует, что это не звезды?

[10:30:24 PM] Михаил:
Я не понимаю, объясни, что выражает этот стих

Маргулис:
Я не хочу объяснять
 Меня убедили в том, что надо чувствовать такие вещи
 Ну вот как можно объяснить такие вещи? На этом же основано человеческое общение. На умении понимать чувства
 Я думаю, ты начнёшь со временем понимать

[10:34:36 PM] Михаил:
Да, я же раньше очень мало стихов читал. И вообще последние полгода у меня была диктатура разума в голове.
 То есть неправда, что если что-то нельзя выразить словами, то этого вовсе нет?

[10:36:53 PM] Маргулис:
Что значит нельзя?
 Я могу выразить
 Но не хочу
 Мне это кажется бессмысленным—объяснять такое стихотворение
 В нем один такой яркий юношеский надлом

[10:38:49 PM] Михаил:
Сейчас будет максимально глупый вопрос: надлом чего?

[10:39:15 PM] Маргулис:
Душевный надлом

[10:39:54 PM] Михаил:
А зачем этот надлом?

[10:40:54 PM] Маргулис:
Ну ок 
Это всё о той же бесчувственности мира 
А молодой Маяковский удивляется, кому в этом мире нужны звезды. На них никто не обращает внимания. Но ведь кому-то это нужно? Кто-то что-то чувствует, глядя на них?
 Надлом не зачем, надлом всегда почему
 http://www.stihi-xix-xx-vekov.ru/m-stih37.html
www.stihi-xix-xx-vekov.ru
Владимир Маяковский - «Мама и убитый немцами вечер»
Владимир Маяковский «Мама и убитый немцами вечер»
 http://www.stihi-xix-xx-vekov.ru/m-stih36.html
www.stihi-xix-xx-vekov.ru
Владимир Маяковский - «Война объявлена»
Владимир Маяковский «Война объявлена»
 Ну Послушайте максимально романтично
 http://www.stihi-xix-xx-vekov.ru/m-stih38.html
www.stihi-xix-xx-vekov.ru
Владимир Маяковский - «Скрипка и немножко нервно»
Владимир Маяковский «Скрипка и немножко нервно»
 http://www.stihi-xix-xx-vekov.ru/m-stih68.html
www.stihi-xix-xx-vekov.ru
Владимир Маяковский - «Себе, любимому, посвящает эти строки автор»
Владимир Маяковский «Себе, любимому, посвящает эти строки автор»

[10:49:12 PM] Михаил:
Я чувствую, как от меня откалываются глыбы и мягко уплывают куда-то.

[10:50:09 PM] Маргулис:
Прочитай ещё поэмы Маяка 
Я могу принести книжку

[10:52:50 PM] Михаил:
Этот жалкий голос продолжает лепетать: "Почему? Зачем это?" Мой ли это голос, не знаю.
 Я склонен все алгоритмизировать, а эти стихи успешно посылают всякие алгоритмы к черту.

[10:54:11 PM] Маргулис:
Это хорошо
 А ты в школьные годы влюблялся?

[10:59:06 PM] Михаил:
Да, конечно, это всё были жалкие маленькие трагедийки, жившие только в моей голове.

[10:59:39 PM] Маргулис:
Во множественном числе, значит)
 Интересно)
 А почему так прохладно искусство воспринимаешь тогда?
 Я думала, это влияет

[11:04:25 PM] Михаил:
Так или иначе, я хочу его воспринимать, но не вполне умею пока. Я привык не верить чувствам. Дурная привычка.

[11:04:33 PM] Маргулис:
Это на тебя вообще повлияло?
edited 
Влюблённости
 Не хочешь об этом говорить?
 Если неприятно, то я, конечно, не буду лезть

[11:05:34 PM] Михаил:
Я думаю
 Это сложный вопрос
 Не в них дело. Полагаю, бояться чувств и эмоций я начал после первого периода мыслей о самоубийстве. Но это очень сложно, должно пройти больше времени, чтобы я все осознал.
 Еще пара дней, может

[11:13:33 PM] Маргулис:
Забавное сочетание фраз

[11:17:39 PM] Михаил:
Еще мама в детстве иногда, то есть нечасто, кричала на меня. Это было ужасно, и моя восприимчивость притупилась. Мама сама потом извинялась всякий раз. Я поэтому давно понял абсурдность гнева, и никогда не испытываю его.

[11:18:12 PM] Маргулис:
Теперь чувствую себя виноватой
 Извини, что сорвалась на тебя когда-то

[11:18:29 PM] Михаил:
Нормально всё
 Я и прощать поэтому отлично умею

[11:21:19 PM] Маргулис:
Я не знаю, стоит ли об этом говорить. Однажды мы с человеком, который очень мне нравился, были вдвоём в его квартире, я просто хотела с ним пообщаться, это была первая встреча за много месяцев, и я искренне поверила, что он хочет поговорить. Он стал ко мне приставать. Он отпустил меня в конце концов, и всё было нормально, ничего страшного не произошло. Но я с тех пор немножко изменилась.
 Поэтому я так сорвалась тогда ночью.

[11:23:04 PM] Михаил:
Я понимаю

[11:23:19 PM] Маргулис:
Я убеждаю себя, что всё нормально, и могу не бояться и не злиться. Но часто это заканчивается срывом.
 Я еле справляюсь с панической атакой, когда остаюсь одна в кабинете с проверяющим.
 Это смешно

[11:24:28 PM] Михаил:
Жутковато
 Важно отделять себя от своих патологических сил. Хорошо, что ты понимаешь себя в этом вопросе.
 Это пройдет. Все проходит.

[11:26:51 PM] Маргулис:
Я знаю
 Сейчас мне вроде бы легче

[11:29:42 PM] Михаил:
Ну и здорово. Это ведь как бы рана. Ты говорила, кажется, что рана помогает воспринимать искусство, как я понял. А совершенно здоровый человек может это делать, или таких людей нет?

[11:30:07 PM] Маргулис:
Мне кажется, что их нет
 Но кто-то из моих знакомых верит, что есть

[11:32:14 PM] Михаил:
Парадокс выходит. Люди избегают ран, но без ран люди слепы. Хотя тут кругом парадоксы.
edited 
Это на ту же тему: люди избегают трудностей, но именно в трудное время проявляются лучшие стороны их душ.
 Спать?

[11:35:34 PM] Маргулис:
Спать.
 Спокойной тебе ночи

[11:35:56 PM] Михаил:
Спокойной ночи, Маша.

--- Wednesday, March 29, 2017 ---

[10:02:56 PM] Михаил:
Не понял я все-таки про коридор и про Мехико-сити.

[10:03:11 PM] Маргулис:
Извини, повторяю всё
 Сколько лекций было по дискре?

[10:03:40 PM] Михаил:
Три

[10:04:02 PM] Маргулис:
Можешь посмотреть, сколько лежат на Гугл диске?

[10:05:38 PM] Михаил:
Две, первая и вторая.

[10:05:48 PM] Маргулис:
Кинешь третью?


[10:07:00 PM] Михаил:
[Photo]
 [Photo]
 [Photo]

[10:07:44 PM] Маргулис:
Спасибо
 Мне тот парень отправил смс из своего спецприемника, что условия нормальные
 Но я не знаю, можно ли ему верить
 Может, он не хочет меня расстраивать

[10:10:27 PM] Михаил:
Наверняка он не хочет расстраивать, но условия действительно могут быть нормальными.

[10:11:27 PM] Маргулис:
Вообще, мы общаемся с ним редко, потому что заняты по очереди, но сейчас приходится немножко подвинуть дела

[10:13:46 PM] Михаил:
Приходится. Подвинуть, чтобы общаться?

[10:14:01 PM] Маргулис:
Чтобы переживать)
 Телефон-то у него забрали, раз в день дают, планшет тем более

[10:16:27 PM] Михаил:
Да, я понимаю. Но, может, поверить ему - что еще остается делать?

[10:16:32 PM] Маргулис:
Ну да

[10:20:50 PM] Михаил:
Так вот, ты мне коридор и Мехико объясни. Можем завтра обсудить, если объяснение большое. А то толку нет, если стихи будут проходить через меня, как вода сквозь сито.

[10:21:12 PM] Маргулис:
Давай послезавтра лучше
edited 
А что в мехико объяснять?
edited 
Это была опечатка

[10:24:23 PM] Михаил:
Этот стих не затронул меня. Я спрашиваю себя: "И чего?" И ничего.

[10:25:19 PM] Маргулис:
Кстати, я полюбила математику к концу модуля
 Она классная
 Оказывается

[10:27:29 PM] Михаил:
Молодец. Хорошо меняться. Мне тоже нравится.
 К Навальному в результате ты как относишься?

[10:35:24 PM] Маргулис:
Я не уверена. Ну то есть лучше, чем к нашим властям, но не знаю, насколько
 По мне разве можно определить мои политические взгляды, не разговаривая о них?

[10:37:31 PM] Михаил:
Нельзя, думаю.

[10:37:54 PM] Маргулис:
Но ты же откуда-то знаешь, что я не единорос
 Про Навального я просто мало знаю
 Из того, что я знаю, я отношусь к нему хорошо
 Возможно, есть какие-то но

[10:39:28 PM] Михаил:
Надо узнавать. Посмотри его канал на ютюбе.

[10:39:39 PM] Маргулис:
Потом
 Мне это скучно
 Лучше я буду узнавать режиссёров хорошего кина
 Вчера это делала

[10:41:24 PM] Михаил:
У меня перестало получаться игнорировать политику.

[11:30:31 PM] Маргулис:
Попробуем зайти вместе, но если по отдельности и будет возможность, займём друг другу место?

[11:45:45 PM] Михаил:
Это-то да. Душа воспламенилась, и я ясно и отчетливо почувствовал стих про звезды.

--- Thursday, March 30, 2017 ---

[8:40:28 PM] Маргулис:
Мы будем в 427?

[8:42:19 PM] Михаил:
306, 427, с одиннадцати

[8:42:32 PM] Маргулис:
Хорошо
 Я хочу в 427, если есть выбор
 Там хорошо
 А 306 проклята зачётом по алгебре

[8:43:22 PM] Михаил:
Плохо его сдала?

[8:43:32 PM] Маргулис:
Очень хорошо
 Ну, если у меня была 10 в семестре, очевидно, городенцев поставил мне положительную оценку

[8:44:36 PM] Михаил:
Тогда я не понял, почему проклята.

[8:44:45 PM] Маргулис:
Я не люблю городенцева
 Я его боюсь
 Он злой
 Если я выйду к доске, он будет смеяться, когда я перепутаю право и лево или неправильно умножу 7 на 8
 А у доски я нервничаю и так делаю

[8:47:40 PM] Михаил:
Неправильно он делает

[8:48:07 PM] Маргулис:
Я ещё не была у него у доски, кстати

[8:50:52 PM] Михаил:
Мне у доски весело

[8:51:10 PM] Маргулис:
У меня это со школы осталось
 Меня не любили одноклассники

[8:51:56 PM] Михаил:
Как тебя можно не любить?

[8:52:20 PM] Маргулис:
Они считали меня высокомерной
 На самом деле было не так и они мне долго нравились
 А потом я слишком часто видела, что со мной не хотят общаться и совсем от них закрылась

[8:54:38 PM] Михаил:
Это бывает так, что человек хороший, но коллектив не подходящий.
 Я бы сейчас мог разводить философию про обычных и необычных людей, но там сложно, и я не понимаю, как оно устроено.
 Ждать и надеяться - вот что нужно делать в такой ситуации.
 Как можно не хотеть с тобой общаться? Глупые люди.
 Любовь или нелюбовь людей - штука зыбкая и мало о чем говорящая.
 Любовь к людям у тебя есть?

[9:10:54 PM] Маргулис:
К людям в целом?
 Я не знаю
 Наверное, нет

[9:13:02 PM] Михаил:
Ладно, надо готовиться, потом договорим.

[9:13:11 PM] Маргулис:
Верно

--- Friday, March 31, 2017 ---

[12:23:33 AM] Маргулис:
Я не готова

 Завтра всё будет очень плохо
 Хочу самые простые вопросы
 Надо ввести новое понятие
 "Готова я как всегда"

[12:35:44 AM] Михаил:
Конечно, все будет отлично.
 Я приду к десяти в институт, готовиться с людьми.

[12:36:57 AM] Маргулис:
Что "конечно"?
 Готова как всегда = готовилась 1,5 часа
 Да, я ненормальная
 Но почти всегда можно что-то придумать
 Многие из рядов можно доказать на месте
 Я про интегралы можно залезть в конспект)
 Интересно, настанет ли экзамен, к которому я буду готова?
 Хотя нет, он уже был
 Зимой

[12:43:31 AM] Михаил:
Что ты вчера делала?

[12:44:07 AM] Маргулис:
Дискру читала 
Причём не ту дискру 
Читала разные лекции с сайта ФКН
 Не люблю, когда меня заставляют заниматься конкретной вещью
 Не грусти из-за меня
 Про то, что я не готовлюсь
 Я исправлюсь
 Тебе спать пора

[12:46:30 AM] Михаил:
Несомненно, ты исправишься.

[12:46:40 AM] Маргулис:
Звучит саркастично
 Я приеду как сегодня
 Раньше не встану
 Лучше выспаться, чем готовиться всю ночь и умереть на экзамене
 У меня была просто идея не спать

[3:24:19 PM] Маргулис:
Я забыла дома кошелёк, а какой-то дедушка на улице попросил у меня денег
 Теперь я расстроилась

[6:23:29 PM] Маргулис:
Что ты мне отправляешь?

[6:23:31 PM] Михаил:
Значит, ты добрая
 Отправляю я тебе песни, которые я понимаю.

[6:24:16 PM] Маргулис:
А я не слушаю музыку и не буду

[6:24:19 PM] Михаил:
Они почти стихи.
[ Виктор Берковский – Снега выпадают  ] 1.5 MB
 
 Ты зря так.
 Попробуй

[6:29:28 PM] Маргулис:
Знаешь, что? Есть полное ощущение, что ты уже это знал 
Но я этого не говорила
Странно, да?

[8:34:09 PM] Михаил:
В самом деле. А что я знал?
$ http://lib.ru/INOFANT/BRADBURY/r_sarscent.txt$

[8:47:37 PM] Маргулис:
Позже прочитаю
 Знал, что я не слушаю музыку

[8:55:48 PM] Михаил:
Могу тексты присылать. 
$http://www.bards.ru/archives/part.php?id=24003$
$ http://bard.ru.com/php/print_list.php?id=20723$

[8:57:14 PM] Маргулис:
Знаю автора стихов
 Первого
 И мне не нравится)
 Сейчас, чуть позже остальное прочитаю

[8:59:35 PM] Михаил:
$http://www.bard.ru/cgi-bin/listprint.cgi?id=115.037$
 $http://bard.ru.com/php/print_list.php?id=26116$
 $http://www.bards.ru/archives/part.php?id=8369$
 $http://bard.ru.com/php/search_song.php?name=26550$
 Это ты знаешь, наверное.$ http://www.stihi-rus.ru/1/Esenin/6.htm$

[9:07:03 PM] Маргулис:
Знаю
 Остальные ещё не посмотрела

[9:07:59 PM] Михаил:
И вот это я понимаю. $http://www.museum-esenin.ru/tvorchestvo/465$

[9:09:43 PM] Маргулис:
Мне у него другие нравятся
 Сукин сын, собака качалова (ок, это слишком известное, впрочем, как и почти весь Есенин), Сорокоуст (кусок про милый милый смешной дуралей), ещё то, которое синий туман снеговое раздолье тонкий лимонный лунный свет
 Чёрного человека, ты, конечно, знаешь? Не знаю, мне оно надоело уже

[10:33:36 PM] Михаил:
Я умру от избытка чувств.
 Это не от Есенина, а в общем.

[10:34:22 PM] Маргулис:
От этого не умирают, не бойся
 Качаю сейчас себе кинцо

[10:35:51 PM] Михаил:
Я ужасаюсь от того, какой я глупый, как много не знаю из того, что очень хотел бы знать.

[10:36:30 PM] Маргулис:
Я тоже это чувствую

[10:36:56 PM] Михаил:
Сейчас?

[10:37:05 PM] Маргулис:
Чувствую?
 Сейчас
 Из-за кино

[10:37:46 PM] Михаил:
Какое кино?

[10:37:54 PM] Маргулис:
Не поняла
 Я качаю кино сейчас
> Михаил Зыбин
> Я ужасаюсь от того, какой я глупый, как много не знаю из того, ч
А чувствую это насчёт того, что мало его смотрела
 Ну то есть много
 Но не достаточно
 А качаю 10000 фильмов

[10:40:06 PM] Михаил:
Почему 10000?

[10:40:14 PM] Маргулис:
Просто много
 Большое число
 Старые фильмы
 То есть в этот раз очень старые

[10:42:06 PM] Михаил:
Я сейчас умру еще больше, потому что фильмов до 2000 года смотрел вряд ли больше десяти.

[10:42:28 PM] Маргулис:
А я смотрела вряд ли больше 10 фильмов до 60 года

[10:50:19 PM] Михаил:
http://raybradbury.ru/library/story/51/12/1/
raybradbury.ru
Рэй Брэдбери.Ру Ревун - Рэй Брэдбери (Бредбери) - скачать рассказ - перевод Лев Жданов
 http://lib.ru/INOFANT/BRADBURY/goldeyed.txt
 http://raybradbury.ru/library/story/51/5/1/
raybradbury.ru
Рэй Брэдбери.Ру То ли ночь, то ли утро - Рэй Брэдбери (Бредбери) - скачать расск...