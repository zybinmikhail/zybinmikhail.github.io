\documentclass{article}
\usepackage[utf8]{inputenc}
\usepackage[russian]{babel}
\usepackage{amsmath}
\usepackage{amsfonts}
\usepackage{mathdots}


\begin{document}
\title{}
\author{}
\date{}
\maketitle


\renewcommand{\d}{\Diamond}
\newcommand{\s}{\square}

Вышел в лес погулять. Недавно было холодно, снег шел, сейчас потеплело. Грязи много. Но что, в сущности, такое грязь? Почему влажную землю мы называем грязью? Ничего в ней страшного нет, если идти не спеша. Вот, штаны не запачкал. Кругом эта грязь. И опавшие листья. Как кладбище. То ли умерших надежд, то ли умершей боли. Вижу женщин с собаками. Женщин три, а собак почему-то две. После - компания мужчин. Вижу среди них ребенка и седого. Подхожу ближе. Ребенок - не ребенок, а взрослый маленького роста; седой - не седой, а лысый. 

Серый, коричневый, черный. Что я испытываю? Восторг. Восторг и больше ничего. Этот запах, запах влажной земли. Похоже на осень. Люблю такую погоду. Красиво кругом. Грязь, серость, старые опавшие листья. На тропинках снег утоптанный, он дольше тает. Кругом земля, а тропинки белые. Людей мало; еще бы - грязно же. Вот человек с собакой. Собака лает. Я иду, а она лает. Ну и ладно. Что я испытываю? Тоска. Ужасная тоска. Грязь, листья, умиращий снег. Мне нравится. Дорога стала шире и похожа на речку. Вода. Пришлось свернуть, идти вдоль. О, запах!

Есть такое состояние - самадхи. Термин из дзен-буддизма. Что-то вроде потери "Я". У меня это так: мысли начинают идти по кругу, круг сужается-сужается, превращается в точку. Потеря "Я". Потом - опять круг. Начинается пульсация - круг-точка-круг-точка. Мысли начинают идти по кругу. Круг-точка-круг-точка. 

Есть такое состояние - самадхи. Точка. Умирать не хочется; ничего не хочется. Вспышка - я на планете Земля, таких, как я, еще семь миллиардов. Потухло. Иду вдоль дороги, которая похожа на речку. Есть такое состояние - самадхи. Потеря "Я". Такое со мной бывает. Что я чувствую? Восторг. Смеяться захотелось, хорошо-то как! Захотелось - засмеялся. Услышал смех. Точка. Серый, коричневый, черный. Ну и цвета. Небо, деревья, снег. Дорога больше не похожа на речку, пошел по ней. Не надышаться этим запахом, не насмотреться на это небо! Мысли идут по кругу, круг пульсирует. Что я чувствую? Ужас. Паника. Слышу стук. Это я стучу рукой по мосту. Ручей черный. Как кровь, но кровь, кажется, не черная. Кто-то кричит. Душа, по-моему. Ручей исчез, мост тоже. Значит, я иду обратно. Потеря "Я". Мысли движутся по кругу, потом точка. И каждый раз то ужас, то восторг. Людей мало - грязно же сейчас в лесу. Танго ужаса и восторга. Не разобрать, кто из них кто. Смеяться захотелось. Серое небо, о, серое небо! Проникает в меня, восхищает меня! Услышал смех. Танго. Тоскливо-то как, и паника, паника. Иногда такое бывает. Вспышка - я на планете Земля, таких, как я, еще семь миллиардов. Потухло. Что, в сущности, такое грязь? Во влажной земле нет ничего страшного. Есть такое состояние - самадхи. Потеря "Я". Такое со мной бывает.
Ужас и восторг танцуют. Не поймешь, кто есть кто. Надо глубже, глубже! О, как же тоскливо кругом! Все это - я. Я разлился. Ничего нет и не будет.

Шлагбаум. Выхожу из леса, значит. Смешное слово - шлагбаум. Сколько раз я уже повторил эту фразу? Здание здание здание здание. Каша. О, запах. На осень похоже. Нет, не очень. Асфальт течет назад. Я иду вперед. Танцуют еще, восторг с ужасом-то. Черная земля, черная грязь. В сущности, грязь - это что? Услышал смех. Тоскливо-то как! Серое небо. Ключ, дверь, лифт, ключ, дверь, кончилось.


\end{document}