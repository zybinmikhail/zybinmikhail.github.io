\documentclass{article}
\usepackage[utf8]{inputenc}
\usepackage[russian]{babel}
\usepackage{amsmath}
\usepackage{amsfonts}
\usepackage{mathdots}

\begin{document}
\title{Бессонница}
\author{Михаил Зыбин}
\date{3 января 2016}
\maketitle

\begin{center}
I
\end{center}

В ту ночь Джону Смиту опять не спалось – как и всегда, когда он утром забывал принимать успокоительное. Джон был очень добр и миролюбив, но, когда он оставался один, какие-то части мозга иногда напрочь забывали об этом. У Джона была важная, полезная, интересная работа, но она забирала у него всё. Вернее, так: она забирала у него всего его. Джон любил людей, но не мог говорить с ними, потому что в его душе, как он думал, кроме работы, ничего не осталось. Он не хотел путешествовать, не отмечал праздников, потому что его душа была слишком утомлена.

Джон Смит лежал и смотрел в никуда. Мысли его были безумны и беспредметны, они роились, копошились, неслись куда-то в безумном вальсе, но истины не было в тех мыслях, а была темнота. Тогда-то впервые и пришла она. Худая девушка среднего роста в белом платье и сама совершенно белая. Она виновато и смущённо улыбалась, а в её глазах застыло выражение ужаса. 

Джон сел на кровати. Он давно догадывался, что сходит с ума, поэтому не удивился. Девушка села рядом. Они молча смотрели друг другу в глаза, и постепенно на лице Джона тоже появилась улыбка. В глазах Бессонницы (а это была она) он видел то же, что и в своих, но разбавленное смирением и надеждой. Бессонница, глядя в глаза Джону, видела в них$ \ldots $ да, то же, что и в своих, но разбавленное смирением и надеждой. 

Оба они в ту ночь впервые за очень долгое время ощущали понимание и сочувствие. Они просто молчали: о счастье и боли, о смерти и жизни, о свете и тьме, и молча понимали друг друга.

Под утро Джон всё же уснул. И так было каждую ночь в течение недели. Джон очень мало спал и уставал всё более и более. Он сказал Бессоннице об этом. Тогда она впервые заговорила. <<Я не могу смотреть, как ты мучишься из-за меня>>. Девушка положила ладони ему на голову, и Джон почувствовал, что полон сил. <<Джон, -  сказала она, - я не кажусь тебе, я реальна и я люблю тебя>>.

\newpage

\begin{center}
II
\end{center}

Теперь каждую ночь она приходила, а под утро уходила. Они разговаривали, и мудрость была в их разговорах. Они любили друг друга и были друг для друга всем. А работа$ \ldots $ всё стало только лучше, ведь белая дева давала Джону сил. А когда она однажды сказала Джону: <<Полетели>>, он без колебаний открыл окно и первым шагнул. Да, они летали, не чувствуя ни холода, ни нехватки кислорода, ведь их питала их любовь. 

Они могли плыть над морем затихшего леса в бледном свете луны, могли слушать музыку над шумными большими городами. А иногда ложились на воду и позволяли нести себя течению неведомо куда, глядя на небо. 

Небо$ \ldots $ однажды, когда Джон и Белая Дева стояли на берегу океана, она закрыла ему глаза руками, а когда открыла, Джон увидел то, чего прекраснее до него никто не видел. <<Я добавила в твой зрительный диапазон частоту 300,148 Герц. На ней видна тёмная материя. Представляешь, почему-то ещё никому не пришло в голову настроить на неё радиотелескоп>>. Всем известно, звёздное небо красиво. Но только если оно не переливается, как мыльный пузырь - слова для описания такого появятся в языке людей через много-много лет. <<А что это за огромный светлый плывущий граф?>> <<О, об этом людям ещё предстоит узнать>>. Они поцеловались. 

Взошла луна и осветила их, два усталых, одиноких создания, вдруг нашедших друг друга и ставших счастливыми.

$ $

\begin{center}
III
\end{center}

Но в одну ночь Джон уснул - Бессонница не пришла. Она была в другом месте. Любой человек, заглянувший бы туда хоть на минуту, заснул бы только через пару недель, навсегда. Это был суд над Белой Девой. Дело в том, что она - часть Бессонницы, один из её обликов. Бессоннице запрещено давать силы, это может делать лишь Сон, и он был недоволен. Он требовал вернуть силы, которые Дева давала Джону, и смерти Девы. 

Но Судья был милостив. Деву решили отсечь, то есть сделать самостоятельной сущностью, с тем чтобы она сама собой затихла, как эхо, а у Джона силы всё же забрать. Получилось так, что Джон больше месяца не спал, и тело его больше не проснулось. Но душа его осталась с Белой Девой, и они не затихли, поскольку любовь питала их. Они стали Утешителями, приходя к людям по ночам и унося их скорбь.

$ $

\begin{center}
IV
\end{center}

Через много-много лет они всё ещё жили на Земле. Она превратилась в дальний закоулок обитаемой части Галактики, и многие уже не верили, что жизнь зародилась здесь. Рас людей стало очень много, и почти никакие из них не могли иметь детей от людей Земли. 

В конечном счёте люди на Земле вымерли. Встретив рассвет на Египетской Пирамиде, (говорят, их было несколько, но остальные занесло песком, а эту вообще построили на десять тысяч лет позже тех, под песком) Джон Смит и Дева отправились в далёкое путешествие до ближайшей обитаемой планеты. Но и на ней жизнь исчезла, как неизбежно исчезает всё в этой Вселенной. 

И было так ещё много раз, и везде Утешители любили друг друга и старались помогать жителям планет. Где-то их не замечали, где-то о них знали, но не верили в них, где-то они были богами или даже королями. 

Но пришло время, и Джон Смит с Белой Девой оказались на последней планете, летящей вокруг последнего солнца, рядом с последним живым существом. Оно было разумным, и оно умирало. Джон и Дева, а также стоящие рядом многочисленные боги, в которых верило существо, знали, что умрут вместе с ним. <<О, эта Вселенная была замечательна. Надеюсь, где-нибудь когда-нибудь это случится ещё разок>>. Таковы были последние слова во Вселенной. Рассыпалась планета, потухло солнце, и наступило царство покоя, которому не будет конца.

\end{document}