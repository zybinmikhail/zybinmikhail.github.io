\documentclass{article}

\usepackage[T2A]{fontenc}
\usepackage[utf8]{inputenc}
\usepackage[russian]{babel}
\usepackage{amsmath, amssymb}
\usepackage{chngpage}
\usepackage{enumitem}

\usepackage[X2,T2A]{fontenc} 
\usepackage[utf8]{inputenc} 
\newcommand{\И}{{\fontencoding{X2}\selectfont\CYRII}} % Буква И-десятеричное ЗАГЛАВНАЯ на рус. клавише «Ии» 
\newcommand{\и}{{\fontencoding{X2}\selectfont\cyrii}} % Буква И-десятеричное строчная 
 
\newcommand{\Е}{{\fontencoding{X2}\selectfont\CYRYAT}} % Буква Ять ЗАГЛАВНАЯ на рус. клавише «Ее» 
\newcommand{\е}{{\fontencoding{X2}\selectfont\cyryat}} % Буква Ять строчная 
 
\newcommand{\Ф}{{\fontencoding{X2}\selectfont\CYROTLD}} % Буква Фита ЗАГЛАВНАЯ на рус. клавише «Фф» 
\newcommand{\ф}{{\fontencoding{X2}\selectfont\cyrotld}} % Буква Фита строчная 
 
\newcommand{\Ы}{{\fontencoding{X2}\selectfont\CYRIZH}} % Буква Ижица ЗАГЛАВНАЯ на рус. клавише «Ыы» 
\newcommand{\ы}{{\fontencoding{X2}\selectfont\cyrizh}} % Буква Ижица строчная

\begin{document}
\title{Безсонница}
\author{Михаил Зыбин}
\date{3 января 2016}
\maketitle

\begin{center}
I
\end{center}

Въ ту ночь Джону Смиту опять не спалось – какъ и всегда, когда онъ утромъ забывалъ принимать успокоительное. Джонъ былъ очень добръ и миролюбивъ, но, когда он оставался одинъ, как\ия-то части мозга иногда напрочь забывали объ этомъ. У Джона была важная, полезная, интересная работа, но она забирала у него все. В\ерн\ее, такъ: она забирала у него всего его. Джонъ любилъ людей, но не могъ говорить съ ними, потому что въ его душе, какъ онъ думалъ, кром\е работы, ничего не осталось. Онъ не хот\елъ путешествовать, не отм\ечалъ праздниковъ, потому что его душа была слишкомъ утомлена.

Джонъ Смитъ лежалъ и смотр\елъ въ никуда. Мысли его были безумны и безпредметны, они роились, копошились, неслись куда-то въ безумномъ вальс\е, но истины не было въ т\ехъ мысляхъ, а была темнота. Тогда-то впервые и пришла она. Худая д\евушка средняго роста въ б\елом плать\е и сама совершенно б\елая. Она виновато и смущенно улыбалась, а въ ея глазахъ застыло выражен\ие ужаса. 

Джонъ селъ на кровати. Онъ давно догадывался, что сходитъ съ ума, поэтому не удивился. Д\евушка села рядомъ. Они молча смотр\ели другъ другуъ въ глаза, и постепенно на лиц\е Джона тоже появилась улыбка. Въ глазахъ Безсонницы (а это была она) онъ вид\елъ то же, что и въ своихъ, но разбавленное смирен\ием и надеждой. Безсонница, глядя въ глаза Джону, вид\ела въ нихъ$ \ldots $ да, то же, что и въ своихъ, но разбавленное смирен\ием и надеждой. 

Оба они въ ту ночь впервые за очень долгое время ощущали пониман\ие и сочувств\ие. Они просто молчали: о счасть\е и боли, о смерти и жизни, о св\ет\е и тьм\е, и молча понимали другъ друга.

Подъ утро Джонъ все же уснулъ. И такъ было каждую ночь въ течен\ие нед\ели. Джонъ очень мало спалъ и уставалъ все бол\ее и бол\ее. Онъ сказалъ Безсонниц\е объ этомъ. Тогда она впервые заговорила. <<Я не могу смотр\еть, какъ ты мучишься изъ-за меня>>. Д\евушка положила ладони ему на голову, и Джонъ почувствовал, что полонъ силъ. <<Джонъ, -  сказала она, - я не кажусь теб\е, я реальна и я люблю тебя>>.

\newpage

\begin{center}
II
\end{center}

Теперь каждую ночь она приходила, а подъ утро уходила. Они разговаривали, и мудрость была въ ихъ разговорахъ. Они любили друг друга и были другъ для другаъ вс\емъ. А работа$ \ldots $ все стало только лучше, в\едь б\елая Д\ева давала Джону силъ. А когда она однажды сказала Джону: <<Полет\ели>>, онъ безъ колебаний открыл окно и первым шагнул. Да, они летали, не чувствуя ни холода, ни нехватки кислорода, ведь их питала их любовь. 

Они могли плыть над морем затихшего леса в бледном свете луны, могли слушать музыку над шумными большими городами. А иногда ложились на воду и позволяли нести себя течен\ию неведомо куда, глядя на небо. 

Небо$ \ldots $ однажды, когда Джонъ и Белая Д\ева стояли на берегу океана, она закрыла ему глаза руками, а когда открыла, Джонъ увидел то, чего прекраснее до него никто не видел. <<Я добавила в твой зрительный диапазон частоту 300,148 Герц. На ней видна тёмная матер\ия. Представляешь, почему-то ещё никому не пришло в голову настроить на неё радиотелескоп>>. Всем известно, звёздное небо красиво. Но только если оно не переливается, как мыльный пузырь - слова для описан\ия такого появятся в языке людей через много-много лет. <<А что это за огромный светлый плывущий граф?>> <<О, об этом людям ещё предстоит узнать>>. Они поцеловались. 

Взошла луна и осветила их, два усталых, одиноких создан\ия, вдруг нашедших друг друга и ставших счастливыми.

$ $

\begin{center}
III
\end{center}

Но в одну ночь Джонъ уснул - Бессонница не пришла. Она была в другом месте. Любой человек, заглянувший бы туда хоть на минуту, заснул бы только через пару недель, навсегда. Это был суд над Белой Девой. Дело в том, что она - часть Бессонницы, один из её обликов. Бессоннице запрещено давать силы, это может делать лишь Сон, и он был недоволен. Он требовал вернуть силы, которые Д\ева давала Джону, и смерти Девы. 

Но Судья был милостив. Деву решили отсечь, то есть сделать самостоятельной сущностью, с тем чтобы она сама собой затихла, как эхо, а у Джона силы всё же забрать. Получилось так, что Джонъ больше месяца не спал, и тело его больше не проснулось. Но душа его осталась с Белой Девой, и они не затихли, поскольку любовь питала их. Они стали Утешителями, приходя к людям по ночам и унося их скорбь.

$ $

\begin{center}
IV
\end{center}

Через много-много лет они всё ещё жили на Земле. Она превратилась в дальний закоулок обитаемой части Галактики, и мног\ие уже не верили, что жизнь зародилась здесь. Рас людей стало очень много, и почти никак\ие из них не могли иметь детей от людей Земли. 

В конечном счёте люди на Земле вымерли. Встретив рассвет на Египетской Пирамиде, (говорят, их было несколько, но остальные занесло песком, а эту вообще построили на десять тысяч лет позже тех, под песком) Джонъ Смитъ и Д\ева отправились в далёкое путешеств\ие до ближайшей обитаемой планеты. Но и на ней жизнь исчезла, как неизбежно исчезает всё в этой Вселенной. 

И было так ещё много раз, и везде Утешители любили друг друга и старались помогать жителям планет. Где-то их не замечали, где-то о них знали, но не верили в них, где-то они были богами или даже королями. 

Но пришло время, и Джонъ Смитъ с Белой Девой оказались на последней планете, летящей вокруг последнего солнца, рядом с последним живым существом. Оно было разумным, и оно умирало. Джонъ и Д\ева, а также стоящ\ие рядом многочисленные боги, в которых верило существо, знали, что умрут вместе с ним. <<О, эта Вселенная была замечательна. Надеюсь, где-нибудь когда-нибудь это случится ещё разок>>. Таковы были последн\ие слова во Вселенной. Рассыпалась планета, потухло солнце, и наступило царство покоя, которому не будет конца.

\end{document}