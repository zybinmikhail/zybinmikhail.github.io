\documentclass{article}
\usepackage[utf8]{inputenc}
\usepackage[russian]{babel}
\usepackage{amsmath}
\usepackage{amsfonts}
\usepackage{mathdots}


\begin{document}
\title{}
\author{}
\date{}
\maketitle


--- Saturday, April 1, 2017 ---

[3:05:26 PM] Маргулис:
А почему ты смотрел мало фильмов старше 2000 года?

[6:28:56 PM] Михаил:
Глупый был. Но сегодня я посмотрел "Нежность" 1966 года, очень понравилось мне.

[10:18:55 PM] Михаил:
Благодаря тебе я становлюсь собой; будто просыпаюсь от дурного сна.

[10:22:53 PM] Маргулис:
Это хорошо

--- Sunday, April 2, 2017 ---

[9:50:16 PM] Маргулис:
Миш
 Что у нас сейчас по философии?
 На самом деле я не отвечала, потому что мне было лень это читать
 Что к следущей паре должно быть прочитано?

[9:52:07 PM] Михаил:
Я не знаю.
 Фихте, наверное.

[9:52:38 PM] Маргулис:
Ну а что именно?
 Я знаю, что Фихте
 Я уже нашла

[9:54:27 PM] Михаил:
И что нужно читать?

[9:54:52 PM] Маргулис:
Основа общего наукоучения

[9:55:50 PM] Михаил:
Спасибо

[10:06:16 PM] Маргулис:
Знаешь, чем писатели отличаются от философов? Только первые иногда, не все, умеют писать. Неужели нельзя научиться говорить связными предложениями? Не может быть в хорошем предложении 10 основ

[10:10:29 PM] Михаил:
Мне дали прочитать интервью с Михаилом Громовым, и теперь кажется, что философия, которой мы занимаемся - чушь. Может, это пройдет.

[10:11:10 PM] Маргулис:
А ещё в переводе Фихте гений-переводчик делает грамматические ошибки
 Забывает предлоги из-за перенагроможденности предложения

[10:12:31 PM] Михаил:
Я бы отнесся к этому с юмором.

[10:14:02 PM] Маргулис:
Мне просто тяжело так читать

[10:15:12 PM] Михаил:
Да, я понимаю.

--- Monday, April 3, 2017 ---

[5:08:12 PM] Михаил:
Ты не заболела снова? Как твои дела?

[5:08:25 PM] Маргулис:
Нормально дела
 Немножко, но я выйду завтра точно
 Просто не очень хорошо себя чувствовала

[5:11:28 PM] Михаил:
Понятно. Бывает.

[5:15:42 PM] Маргулис:
Слышал про питер?

[8:38:36 PM] Маргулис:
Ты прочитал про наукоучение?

[9:24:36 PM] Маргулис:
Завтра до 2 поговоришь со мной о старом листочке по геоме? У меня есть пара вопросов

[11:36:44 PM] Маргулис:
Сколько у тебя за дискру?

[11:43:39 PM] Михаил:
Слышал, читаю, поговорю, одиннадцать.
 По геометрии забавно.
 Результаты контрольной.
edited 
[11:47:23 PM] Маргулис:
11 из 10?
 Да
 Забавно

[11:47:38 PM] Михаил:
Из двадцати

[11:47:38 PM] Маргулис:
По геометрии
edited 
А сколько это?
 Из 10

[11:48:11 PM] Михаил:
Не знаю

[11:48:18 PM] Маргулис:
Там же написано
 Где ты взял результат?
 На том сайте?
 Если есть список общий?

[11:49:51 PM] Михаил:
Здесь$ https://docs.google.com/spreadsheets/d/1P3QBR0Z7qNJtavEzg1mz6vEVAjjiEY-zS47jclYNNJw/edit?ts=58bdc296gid=1255350898$

[11:51:08 PM] Маргулис:
У тебя 7

[11:52:08 PM] Михаил:
В самом деле семь
 Принеси мне книгу стихов каких-нибудь, пожалуйста.

[11:53:47 PM] Маргулис:
Да, забыла
 Принесу

[11:55:25 PM] Михаил:
Я тебе Брэдбери дам, хочешь? Там 451 градус, Марсианские хроники и рассказы.

[11:55:39 PM] Маргулис:
Я читала
 Пока не хочу
 Я люблю электронные

[11:57:55 PM] Михаил:
Надо обсудить тогда

[11:58:23 PM] Маргулис:
Кого принести?

--- Tuesday, April 4, 2017 ---

[12:00:24 AM] Михаил:
Обсудить Брэдбери. Кого принести, не знаю.

[12:00:29 AM] Маргулис:
Маяка?

[12:00:40 AM] Михаил:
Приноси его

[12:00:58 AM] Маргулис:
Есть блок, Северянин и Чёрный ещё
 Есть больше, но надо искать их

[12:02:18 AM] Михаил:
Правильно было бы мне всё прочитать. Начнем с Маяка.
 Хотя погоди
 Блок.

[12:02:59 AM] Маргулис:
Ок
 Просто сразу всех тяжело нести

[12:03:34 AM] Михаил:
Да, это понятно
 Когда пойдешь спать?

[12:06:48 AM] Маргулис:
Хочу сейчас кое-что посмотреть
 Потом пойду

[12:07:34 AM] Михаил:
Ну смотри. А что?

[12:07:43 AM] Маргулис:
На последнем дыхании
 Годар

[12:07:57 AM] Михаил:
Любопытно

[12:03:53 PM] Маргулис:
Forwarded message: 
Ekaterina Akilbaeva [4/4/17] 
$http://iphras.ru/uplfile/root/biblio/2008/Problema_demarkacii_1.pdf
[Problema_demarkacii_1.pdf] 737 KB$
 
[9:04:33 PM] Маргулис:
Я скучаю по своему физику
 Учителю в школе

[11:37:00 PM] Маргулис:
Он был ужасно милым
 Только тоже хотел спать на первом уроке и давал на нем контрольные
 А потом рассказывал о чем-нибудь, было здорово
 Рисовал какие-то огромные схемы электрических приборов
edited 
[11:39:28 PM] Михаил:
Это хорошо, если есть по кому скучать.
 Чёрт, я ни по кому не скучаю.
 К тебе физик хорошо относился?

--- Wednesday, April 5, 2017 ---

[12:14:47 AM] Маргулис:
Хорошо

[12:48:57 AM] Михаил:
Я перестал быть уверен, что мне нужно сейчас читать стихи, потому что через неделю апрельское полнолуние. Есть красивый индийский рассказ, который я перечитаю в этот день, и следующие дни я буду перечитывать другие индийские рассказы, чтобы настроиться.

[12:55:43 AM] Маргулис:
И что должно ознаменовывать апрельское полнолуние?
 Ты вроде далёкий от астрологии человек

[12:57:23 AM] Михаил:
Оно имеет значение именно из-за этого рассказа.

[12:57:39 AM] Маргулис:
Понятно
 Индийский?
 Что за рассказ?

[1:00:02 AM] Михаил:
"Ночь в полнолуние", Кришан Чандр

[1:01:04 AM] Маргулис:
Какие тебе нравятся фильмы?
 Я могу что-то посоветовать

[1:03:35 AM] Михаил:
Я думаю, мои вкусы не сформировались.

[1:03:58 AM] Маргулис:
Ты должен посмотреть Аризонскую мечту. Всем, кому я советовала, она нравилась

[1:04:42 AM] Михаил:
Посмотрю, значит
 Когда планируешь спать лечь?

[1:05:28 AM] Маргулис:
Не знаю

[1:05:47 AM] Михаил:
Ложись сейчас

[1:05:54 AM] Маргулис:
Нет
 Не могу
 До завтра

[1:07:09 AM] Михаил:
До завтра - это как пока и до свидания?

[1:07:17 AM] Маргулис:
Да
 Я занята
 Извини

[10:21:46 PM] Маргулис:
Кинь последнюю лекцию по физике

[11:13:08 PM] Михаил:
[Photo]
 [Photo]

[11:30:23 PM] Маргулис:
Спасибо

--- Thursday, April 6, 2017 ---

[12:11:34 AM] Михаил:
Оказывается, существует тотальный диктант. В эту субботу он будет. 
https://totaldict.ru/
 Мне очень грустно, что ты мало спишь ночью. У меня к этому очень трепетное отношение. Мама с давних пор говорит, что ее бабушка говорила, что сон - самое лучшее лекарство. От чего, почему, не знаю, но на эмоциональном уровне отношение мое ко сну определяется этой фразой. И еще причудливой фантазией, взятой мной с потолка. Сон - это дань смерти, которую если мало и неисправно платить, то смерть придет раньше.
 Ну и, разумеется, если высыпаться, то на лекциях легче. Можно и привыкнуть мало спать и прекрасно себя чувствовать, но ты, как я понимаю, не прекрасно себя чувствуешь и иногда хочешь спать днем в неподходящее время.

[12:39:12 AM] Михаил:
Конечно, это твое дело, но ты усилила во мне способность испытывать чувства, в том числе воспринимать близко к сердцу проблемы людей, не являющихся мной.
 Люди, не являющиеся мной, вообще очень интересные.

[9:11:18 AM] Маргулис:
Что такое неподходящее время для сна?
 Бредовая фраза
 Сама решаю, когда мне спать

[4:24:26 PM] Маргулис:
Ты где?

[10:34:56 PM] Маргулис:
[Photo]
 [Photo]
 Я тебя не поняла
 У тебя бессвязная запись
 И я не вижу, где у тебя ответ
 У тебя записаны 4 разных выражения
 И зачем тебе именно ортогональный базис? Что у тебя за запись вообще, это невозможно понять
 Если ... то: и дальше какая-то сумма без равенства
 Ничего непонятно
 И в квадрат ты странно матрицу возводишь
 Питру завтра отменили
 Придёшь за полчаса до городенцева, чтобы этот и 7 номер объяснить?

[11:51:03 PM] Маргулис:
Кстати, у тебя здесь есть Каган?
 Ты обещал ответить мне здесь на вопросы и не пришёл
 Это не по совести

--- Friday, April 7, 2017 ---

[12:34:52 AM] Михаил:
Я буду более ответственно относиться к телеграму, извини.

[12:35:25 AM] Маргулис:
Все, я окончательно легла
 Спокойной ночи

[12:35:45 AM] Михаил:
Спокойной ночи, Маргулис.

[5:29:53 PM] Маргулис:
Пока мы были на проективке, был теракт в Стокгольме
 В моем любимом Стокгольме

[7:19:10 PM] Михаил:
Это, конечно, печально. Мир несовершенен, человеческая глупость бесконечна. То, что мы можем делать - это локально уменьшать злобу вокруг себя, быть миролюбивыми и любить людей. Вести себя противоположно террористам.
Я думаю, когда-нибудь тех ужасных людей обязательно победят.
Не нужно бояться, ведь они этого и хотят.
 Больно, что люди способны такими быть. Но человечество еще юно, пройдет много времени, пока оно помудреет, если не исчезнет перед этим.
 В мире, однако, есть добро, от этого тоже никуда не деться.

[8:11:39 PM] Маргулис:
Ты о ФКН не думал при поступлении?
edited 
Я имею в виду, ты вообще выбирал между чем-то и чем-то, или сразу знал, куда пойдёшь?

[9:16:10 PM] Михаил:
Программировать я однозначно не хотел. Хотел именно изучать математику. Выбирал между мехматом и матфаком. Матфак мне советовали мои учителя математики в школе. Выпускники матфака тоже его советовали. О мехмате сложилось впечатление, что там меньше контроля, можно что-то делать только на сессии, а остальное время ничего. Мне так не нравится. К матфаку я склонялся еще потому, что он молодой, из-за чего не успел еще сгнить, и программа поэтому современная.
 Сейчас с кем-то с первого курса мехмата общаюсь, у них программа однозначно слабее.
 Программирование - это очень неромантично.
 Сейчас я посмотрю "На последнем дыхании".

[9:39:50 PM] Маргулис:
Аризонскую месту
 Не на последнем дыхании, а Аризонскую мечту
 И главное: ничего мне о фильмах не пиши
 Совсем

--- Saturday, April 8, 2017 ---

[12:59:55 AM] Маргулис:
А насчёт планов—я действительно пойду в музей на выходных, и была на других выходных в музее, но я хожу туда одна и не хочу это менять. Я это уже довольно ясно выражала

[1:01:13 AM] Михаил:
Я тоже думаю, лучше по отдельности ходить.

[1:01:26 AM] Маргулис:
Кстати, убеди уж меня в своей интересности
 Я в последнее время только разочаровываюсь
 Тебе стоит понимать, я убеждённый анархист со всеми прилагающимися к этому моральными принципами
 И тоталитарность или насилие для высшей цели, как в пиле, я не приемлю
 Ни в каком виде

[1:05:06 AM] Михаил:
Насилие-то я тоже не приемлю.

[1:05:18 AM] Маргулис:
Если ты знаешь фильм Игра, то это квинтэссенция моего отвращения

[1:05:34 AM] Михаил:
Я тоталитарен в отношении самого себя.

[1:05:49 AM] Маргулис:
Нет, ты высказываешь тоталитарные мысли
 Дело не в том, насколько ты это понимаешь
 Дело в том, что этот механизм в тебя встроен
 Извини, я занята
 Приду позже, ты можешь писать
 Но я сегодня уже вряд ли отвечу
 Сейчас ты просто навязываешь мне разговоры об искусстве и просишь быть каким-то твоим проводником
 А я хочу говорить с готовым продуктом на важные темы
 Какой мне-то в этом интерес? Я теряю время в метро на совместные поездки, например. Это очень важное для меня время
 Я читаю на синей ветке
 Приходя, я увидела математиков, с которыми больше не о чем разговаривать, и пока что никто не изменил моего мнения. Убеди меня в обратном. Просто докажи мне это.
 Я люблю говорить с людьми, когда они разбираются в чем-то не хуже меня
 Не в чем-то, а в том, о чем мы говорим, я имею в виду
 Блин, опять потратила на тебя время тогда, когда оно мне нужно
edited 
[1:46:49 AM] Михаил:
Каждого человека разные люди воспринимают по-разному. Моя интересность - это то, что находится в твоем сознании.
 Однако я понимаю, что ты чувствуешь.
 Не нужно искать в людях своего отражения. Хотя я сам так раньше делал. Это путь одиночества. В людях надо искать этих самых людей.
 Мне давно было не о чем говорить с людьми. И дело во мне. Не в том, что люди глупые. Каждый человек - вселенная. С каждым можно разговаривать, только нужны искреннее желание и немного навыка.
 Насчет меня - я никогда не стану тобой. Ты на меня благотворно влияешь, но искоренить все мои несовершенства ты не сможешь. Твоей копии нигде нет и никогда не будет, как бы трагично это ни звучало. И моей тоже.

[3:53:14 AM] Маргулис:
Блин, ты умудряешься меня разочаровывать с каждым разом все сильнее
 Хватит писать очевидности
 Хватит пытаться меня принизить подобными разговорами
 Это касается многих твоих фраз
> Михаил Зыбин
> Это, конечно, печально. Мир несовершенен, человеческая глупость
Вот это тоже
> Михаил Зыбин
> Больно, что люди способны такими быть. Но человечество еще юно,
.
> Михаил Зыбин
> В мире, однако, есть добро, от этого тоже никуда не деться.
.
 Ты понимаешь, что это полная банальность?
 Ладно банальность, это клише
 Каждая фраза
edited 
Мне и не нужна копия меня, но и ты мне больше нравиться не начнёшь, какая бы там вселенная ни была, если продолжишь так себя вести
 Ты просто выводишь меня из себя
edited 
Я не нуждаюсь в рассуждениях уровня средней школы, это звучало бы мудро для меня классе в 5-6
 А твои рассуждения именно таковы

[4:18:46 AM] Маргулис:
Если ты думаешь, что ты психологически старше, опытнее и умнее меня, ты неправильно меня воспринимаешь. Я не ребёнок, мне не нужны те вещи, которые ты пишешь. Со своими проблемами я могу разобраться сама, я достаточно сильный человек для этого. Окружать меня заботой не нужно, это выглядит очень глупо. В конце концов, моя инфантильность—чисто разговорное явление, и как только я из разговора выхожу, она заканчивается. Я со всем могу разобраться и все выдержать. Из нас двоих только ты режешь себя в 18 лет, и именно это инфантильно. Я не дурочка, я просто иногда так начинаю себя вести, чтобы выглядеть мило. Я очень жёсткий человек, меня сложно разжалобить и выбить из колеи, и я с лёгкостью перестану общаться с человеком, если общение станет казаться мне неприятным. Общение с тобой начинает казаться таким. Я не милая в разговоре с моими настоящими близкими людьми, я серьёзный человек и могу вести серьезную беседу. Кстати, серьёзных бесед у нас с тобой не было. Говоря о мировоззрении—у тебя оно находится на гораздо более низком уровне развития, чем у меня, уж извини. Под этим я подразумеваю, что в твоей системе много проколов, она не учитывает сложность человеческого существа и вообще крайне подростковая. Сформированного раз и навсегда мировоззрения не бывает, но все-таки моё очень цельное в данный период времени. Я не должна выслушивать смешные поучения от тебя. И я должна это тебе написать, потому что самооценочка у вас немного зашкаливает, а для того, чтобы это было объективно, нужно делать ещё что-то кроме того, что делаешь ты. Раскачайся уже и выйди из своего круга мышления. Счастье для тебя—локальная штука, которая достигается чтением приятных книг, просмотром хороших фильмов, прогулками по природе и занятием любимым делом. А счастье не такое. В таком понимании счастье меня не интересует. Это примитивно и глупо. И достижение гармонии внутри себя как одно из представлений счастья—тоже. Оно гораздо сложнее в любом человеке. Не хорошее настроение каждый день. Меня твои слова просто раздражают. И я никогда не просила быть копией меня—думай своей головой, об этом я говорила выше. Я не хочу быть для тебя наставником, потому что хочу, чтобы ты сам ко всему пришёл, чтобы у тебя было своё мнение. Я не считаю, кстати, ничем хорошим то, что на кого-то сильно влияет книга. Тебе не хватает критической оценки, она не достаточно глубокая. С первой страницы нужно не верить, а анализировать и сопоставлять с тем, что ты знаешь. Тогда не получится так, что ты внезапно стал жить чужими, возможно, неверными, идеями.
 Я ни в ком не ищу своего отражения. Я хочу видеть перед собой другую сильную личность, которая может мне по-настоящему достойно возразить и поспорить со мной. И я пока знаю крайне мало таких людей, и все они не с матфака. Этого я хочу, этого я сейчас в тебе не нахожу. Как найдёшь это сам в себе—я тоже это замечу. Конечно, я не хочу не общаться с тобой, но я прошу свести на нет общение в телеграме на не насущные темы, потому что это отнимает мое время и никак тебе не помогает. Для такого есть личные разговоры. Все, это точка.
 Ну и, конечно, не всякий человек—вселенная, это тоже клише. Может, у тебя слишком хорошее окружение, и поэтому ты слишком мало видел в жизни, а потому мало понимаешь. Но это все очень по-детски. Некоторые люди действительно всего лишь ограниченные и глупые. Я не о тебе. Но если одни ограниченные и глупые настолько, что это очевидно, другие, конечно, тоже, только границе чуть шире. И мало кто действительно может эти границы постоянно расширять. И вот к этому расширению надо стремиться. И вот это уже ближе к счастью. Но расширение—это не просто чтение книжек, сами по себе они не дают ничего. Важна работа над собой, внутренняя сила, которую нужно научиться в себе открывать, которая даст понять, как тебе вообще существовать дальше. Это внутренняя свобода, которая должна сломать стереотипы твоего мышления, навязанные рамки, и только когда часть ограждений сломана, ты можешь расширять вселенную дальше.
 И пожалуйста, я правда очень тебя прошу—не отвечай ты на это и не пиши свои длинные письма, которые я читаю посреди фильма и не могу из-за этого смотреть его дальше. Они меня очень расстраивают. Я не хочу такое читать. Идеальный вариант—не отвечай мне на эти сообщения. Это единственный, кстати, вариант, я не хочу больше с тобой о таких вещах общаться и легко завтра тебя заблокирую при случае.
 Я недосмотрела один фильм пару дней назад, села смотреть его сейчас сначала, но ты меня сбил на том же месте и уже поздно
edited 
А смотреть третий раз с начала я не хочу. Трачу время на это. А фильм хороший, и ты испортил мне от него ощущение. Я пока что удаляюсь, а ты негодяй и не пиши мне ничего
 Я из-за тебя ещё и не усну теперь
 Не хочу общаться.
 Ещё один день испорчен
 Ни фильм не посмотрела, ни пораньше не встану. Класс. Спасибо.
 Ок, это не стоило того, чтобы будить человека, но я себя от твоих сообщений чувствую не лучше. Но все равно прости.

[4:55:26 AM] Михаил:
Пожалуйста, прости меня тоже, если тебе в самом деле не лучше, чем мне.

[5:19:20 AM] Маргулис:
Черт, я перегнула палку.
 Прости меня столько раз, сколько нужно, чтобы простить.

[5:37:38 AM] Маргулис:
Это больше никогда не повторится. Я начинаю что-то менять в жизни и это один из пунктов, которых коснутся перемены. Я буду себя контролировать и говорить такие вещи лично и таким образом, чтобы это не звучало уничижительно. Ты на самом деле умный, много читаешь и очень интересный человек, но я бы хотела запустить в тебе тот механизм, который, на мой взгляд, должен являться частью настоящей личности.
 Я бы не стала с тобой начинать общаться, если бы ты был неинтересным, это очень логично
 А прийти к обратному выводу за 2 недели нельзя
 Но я вижу новые грани тебя и они вызывают во мне яркий эмоциональный отклик
 И это хорошо, это значит, что меня твоя личность цепляет

[5:43:28 AM] Михаил:
Хорошо, я понял, все в порядке.
 Теперь уже не так больно.
 Красиво на улице. Интересно было смотреть, как ночь превращается в утро. И туман - я ведь так давно не видел туман.
 А запах этого утра - отличная причина, чтобы жить.

[11:31:26 AM] Михаил:
Противно находиться с собой в одной комнате.
 Видеть себя.
 Я не буду заниматься членовредительством, не волнуйся. Это не поможет.
 Не знал, что умею ненавидеть. Это горько и неприятно. Я ненавижу себя.
 Тебе по алгебре первую лекцию надо прочитать, ее выложили наверное.

[12:18:43 PM] Михаил:
Спасибо тебе за правду. Это не сарказм, я никогда не говорю сарказмов.

[12:44:53 PM] Михаил:
Я жил в клетке одиночества последние четыре года. Я не старался чувствовать книги, которые читал, потому что некому об этом рассказать. Зачем чувствовать, если это никому не нужно?
Моя мама не говорит ничего, кроме банальностей. За время, которое я себя помню, она прочитала книги три. Если ей сказать небанальность, она в лучшем случае не поймет. Если не повезет, начнет орать, что должна была не воспитывать меня, а делать карьеру, что я ее разочаровываю.
 Все, что она от меня хочет - чтобы я учился. Искусства для нее нет.
 У меня нет ни одного родственника, с которым я был бы близок.

[5:25:57 PM] Маргулис:
Слушай, Миш, тебе, наверное, стоит пересмотреть сове отношение к людям и то, насколько тебе важно их мнение.
 Тебе ведь, на самом деле, не так уж важно иметь родственника, с которым ты был бы очень близок
 В том, чтобы самостоятельно развиваться, есть много плюсов

--- Sunday, April 9, 2017 ---

[12:12:27 AM] Маргулис:
Как ты себя чувствуешь?

[12:48:38 AM] Михаил:
Хорошо, Маргулис. Ты правильно сделала. Подробно не буду писать. В понедельник поговорим.

[7:31:15 PM] Маргулис:
Извини, я не заходила сюда
 Хотя я не спала уже, когда ты написал
 Нет, это неконструктивно, нельзя жалеть о том, что жизнь так сложилась

[10:52:12 PM] Михаил:
Я не жалею, дело не в этом. Я рассказываю, чтобы ты лучше понимала, почему я такой, какой есть. Маргулис, приди завтра пораньше, пожалуйста, за пятнадцать минут или за полчаса, чтобы мы поговорили.

[11:14:27 PM] Маргулис:
До английского? Я постараюсь

[11:14:52 PM] Михаил:
Ты пойдешь на английский, да?

[11:15:02 PM] Маргулис:
Да
 Я, возможно, фильм ещё посмотрю, потому что Годар меня слишком сильно зацепил. Не знаю, насколько высплюсь
 И насколько смогу прийти пораньше

[11:22:48 PM] Михаил:
Есть перерыв между четвертой и пятой парой
 Можно поговорить тогда, я о нем забыл

[11:23:54 PM] Маргулис:
Да
 Хорошо

--- Monday, April 10, 2017 ---

[3:01:26 PM] Михаил:
Forwarded message: Sonya Ivanova [4/10/17] 
[Photo]
Если кому-то интересно, вот работа

[8:41:14 PM] Маргулис:
Странно
 Привыкла общаться с парнями больше, чем с девочками
 Потому что со мной общались в моей жизни почти только мальчики, которым я нравилась

[9:35:00 PM] Михаил:
Мне с парнями общаться тяжелее. Я это связываю с уходом отца из семьи.

[9:46:12 PM] Маргулис:
Ты хотя бы общался с девочками, которые тебе нравились, а я лично ни разу о чем-то интересном не говорила с теми людьми)
 Так что у тебя все хорошо)
 Что-то ты такой в метро грустный был, что мне самой взгрустнулось

[9:48:55 PM] Михаил:
Мне очень грустно от твоей истории.
 Она глубоко в меня проникла.
 Хотя я некоторые части не расслышал. Завтра поподробнее скажешь.

[9:52:08 PM] Маргулис:
Могу рассказать ещё вторую часть, про то, как в меня был влюблён один религиозный фанатик
 И Кутепов, кстати
 Из-за меня стал таким молчаливым
 По-моему

[9:54:21 PM] Михаил:
Да, я бы послушал. В сравнении с этим моя жизнь кажется несмешным анекдотом с последней страницы дешевой газеты.
 Я снова ничего не решаю.

[9:57:03 PM] Маргулис:
Зря

--- Tuesday, April 11, 2017 ---

[12:58:52 AM] Михаил:
[Photo]
 [Photo]
 [Photo]
 [Photo]
 [Photo]
 [Photo]
 [Photo]
 [Photo]
 [Photo]
 [Photo]

[2:35:28 AM] Маргулис:
Я все-таки завтра прочитаю
 Сейчас лягу уже

[2:36:04 AM] Михаил:
Ага

[2:36:14 AM] Маргулис:
Спокойной ночи
 Не спи как я
 Ты начинаешь мало спать

[2:37:07 AM] Михаил:
Мне нравится. Спокойной ночи, Маргулис.

[2:37:18 AM] Маргулис:
Это ужасно
 Я чувствую себя виноватой
 Ну ладно, спокойной ночи

[12:25:09 PM] Маргулис:
Я проспала
 Извини
 Я не знаю, как это произошло
 Я хотела прийти

[1:29:10 PM] Михаил:
Бывает)

--- Wednesday, April 12, 2017 ---

[2:05:35 AM] Маргулис:
Миш, а Шварцман не отложил дедлайн?

[2:06:33 AM] Михаил:
Он не говорил об этом, нет.

[2:06:55 AM] Маргулис:
Жаль
 Ты сдавал?

[2:07:57 AM] Михаил:
Три из восьмого, ничего из седьмого.
edited 
[2:08:34 AM] Маргулис:
А у восьмого тоже дедлайн 15-го?

[2:09:32 AM] Михаил:
Я не думаю, но точно не помню, когда он.

[2:09:49 AM] Маргулис:
Ничего не сказано
 Я опять плохо готова к философии
 Не все прочла

[2:10:28 AM] Михаил:
Я ничего не прочел.
 Надо брать себя в руки.

[2:11:26 AM] Маргулис:
Возьми

[2:12:26 AM] Михаил:
Этот воздух еще. Запах пьянящий у него.
 Невмоготу мириться с тем, что есть апрель, как говорится.

[2:16:16 AM] Маргулис:
Мне понравилась статья про круг понимания

[2:17:14 AM] Михаил:
Я рад. Ее можно прочитать минут за тридцать?

[2:23:30 AM] Маргулис:
Да

[2:51:49 AM] Михаил:
Скоро будешь ложиться спать?

[2:53:20 AM] Маргулис:
Да
 Сейчас буду

[2:54:54 AM] Михаил:
Спокойной ночи тогда тебе, Маргулис.

[2:55:31 AM] Маргулис:
И тебе, Миш

[5:50:50 PM] Маргулис:
Встретила двух своих учительниц, они не остановились поговорить

[7:50:59 PM] Михаил:
Обидно, наверное.

[9:18:15 PM] Михаил:
Лягу сейчас. Все равно ничего полезнее не сделаю.

[9:23:53 PM] Маргулис:
Ок
 Не грусти только

[9:29:19 PM] Михаил:
Завтра утренник в 9:15, я пойду.

[9:34:32 PM] Маргулис:
Я—нет
 С 9:15 до 18:20 не усижу там

--- Thursday, April 13, 2017 ---

[7:34:58 PM] Маргулис:
https://meduza.io/news/2017/04/13/ermitazhu-zapretili-rasprostranyat-informatsiyu-o-rasprodazhe-predmetov-iskusstva-pri-sovetskoy-vlasti
Снова хорошие новости. 
Кошмар. 
Неужели они искренне считают себя наследниками совка?

Meduza
Эрмитажу запретили распространять информацию о распродаже предметов искусства пр...
Государственный Эрмитаж обвинили в незаконном распространении информации о продаже предметов искусства, проводившейся советским правительством в 1920-...

[7:48:44 PM] Михаил:
Да, это возмутительно, конечно.

[9:26:43 PM] Михаил:
[Photo]
 [Photo]
 [Photo]
 Первая - это через месяц-полтора, точно не помню.

[10:51:48 PM] Маргулис:
Тебе идёт)

[11:08:36 PM] Маргулис:
Убрали Шварцмана
 За что?
 Я не люблю Тихомирова

[11:10:21 PM] Михаил:
Эхх, да. Ни у кого нет такого искреннего, душевного остроумия.

[11:10:33 PM] Маргулис:
Грустно
 Я так много его лекций пропустила
 Мне плохо.
 Ладно, я в норме

[11:12:22 PM] Михаил:
Да, я понимаю. Все так быстротечно.

[11:12:37 PM] Маргулис:
С 9 до 19 теперь на матфаке в четверг

[11:13:03 PM] Михаил:
Это весело

[11:14:41 PM] Маргулис:
Угу

[11:15:36 PM] Михаил:
Номер два из анализа мне не нравится.

[11:17:04 PM] Маргулис:
Не хочу думать о листочках
 Мне что-то страшно из-за них стало
 Тупая бесполезная система

[11:18:41 PM] Михаил:
Нет, это хорошая система, она заставляет студента постоянно работать.
 Я буду сейчас решать.

[11:19:22 PM] Маргулис:
Да, это идиотская система
 Которая при самом одностороннем в мире образовании ещё и не даёт времени развиваться
 Потому что ни в одном американском вузе не позволили бы давать студентам всего два нематематических курса за 4 года
 И слово "заставлять" выводит меня из себя.
 Алло

[11:43:31 PM] Маргулис:
Нет такого контекста, в котором слово заставлять имело бы положительный смысл
 Обычно экзамены больше, чем у нас, и это и является нормальным контролем знаний
 А в обычное время у человека есть возможность выбирать, чем ему заниматься
 Потому со дз есть
 Но его можно и не делать
 Можешь не отвечать, не хочу об этом читать
 В метро я хотела рассказать важную вещь, но забыла

--- Friday, April 14, 2017 ---

[12:09:09 AM] Маргулис:
Правда, что по дискре письменный экзамен?

[1:36:42 AM] Михаил:
Не знаю про экзамен.

[2:48:19 AM] Маргулис:
Жуткое ощущение, что я за этот год стала хуже думать
 http://arzamas.academy/micro/zagadka/6

Я не хочу смотреть ответ, но не уверена, что правильно разгадала

Arzamas
Загадка дня: Нанайская
Что загадывали друг другу представители разных народов
 Ответ природное явление или связан с человеком?
 Или тоже не хочешь смотреть?

[2:57:17 AM] Михаил:
Я ее 10 апреля видел, хотел подумать, но забыл. Я хочу думать, думаю.

[2:57:48 AM] Маргулис:
А я только сейчас увидела
 Самое обидное, что я случайно посмотрела и не подумала про китайскую, про стаю домашних гусей

[2:59:06 AM] Михаил:
Никогда не думал, что загадки - это интересно. Теперь я думаю иначе.

[2:59:53 AM] Маргулис:
Потому что видел только детские
 Ок, они все детские

[3:00:36 AM] Михаил:
Спать хочу. Спокойной ночи, Маша.

[3:00:41 AM] Маргулис:
И тебе

[9:36:41 AM] Маргулис:
[Photo]

[3:27:15 PM] Маргулис:
Я сейчас ухожу, наверное

--- Saturday, April 15, 2017 ---

[2:44:37 AM] Михаил:
Я посмотрел "На последнем дыхании".

[1:38:16 PM] Маргулис:
И как тебе?

[3:21:20 PM] Михаил:
У меня большое мнение, не буду его писать. Если коротко, главный герой - это наиболее интересный и сложный образ, который я видел в кино.

[3:34:26 PM] Маргулис:
Ещё надо посторонних посмотреть
edited 
Ещё называется иногда банда аутсайдеров. Это тот же фильм

--- Sunday, April 16, 2017 ---

[2:22:25 AM] Михаил:
Был вчера в Пушкинском музее, потом посмотрел посторонних.

[4:12:28 AM] Маргулис:
Я посмотрела безумного пьеро
 Он хуже
 Хотя есть очень красивые вещи
 Посмотри его и скажи, что тебе больше понравилось
 Похож на на последнем дыхании

[5:35:13 AM] Маргулис:
Кстати, было бы здорово рассказывать о том, где ты был и что смотрел, лично, потому что так я могу сразу положительно отреагировать и не буду ждать возможности обсудить это, и нам обоим будет приятнее
 А то как-то смешно, на отчёт похоже
 Я вот праздновала др мамы сегодня
 И смотрела безумного пьеро, как можно было догадаться
 О выставках, на которых я не была, не хочу говорить
 Ну это так, к слову
 Не рассказывай ничего
 И Годара тоже больше пока не хочу обсуждать, не заводи разговор об этом
 Не рассказывай ничего—в плане мнений тоже
 Я не люблю что-то знать о чужих мнениях
 О том, где меня не было

[4:18:28 PM] Михаил:
Мне нравится погода.

[8:00:34 PM] Маргулис:
Ты мне дашь геометрию?)
 Я ленивая)

[9:36:35 PM] Маргулис:
Надо подговориться к Ф

[9:38:52 PM] Михаил:
Не нравится мне, что я потакаю твоей лени.

[9:39:46 PM] Маргулис:
Вредный Миша?)

[9:42:24 PM] Михаил:
Ну ладно, ты убедительная.
 Что по философии?

[9:42:58 PM] Маргулис:
Смешно) ну вообще, я в четверг сдаю, если ты так хочешь, чтобы я занималась, у меня есть ещё пара дней
 Ну там кант
 Хотя я ещё во вторник хотела попробовать поймать адлера
 Там две книжки канта
 [Photo]
 Тут книжки и темы вперемешку
 И фраза про "как возможно" — это кусок названия главы

[9:45:01 PM] Михаил:
Спасибо

[10:10:16 PM] Маргулис:
Как сыграли в шляпу, кстати? Вы долго этим занимались?

--- Monday, April 17, 2017 ---

[1:09:44 AM] Михаил:
Первая игра мне не понравилась, потому что мой партнер не знал, что такое Откровение Иоанна Богослова. Вторая игра понравилась больше, потому что порядка половины слов придумал я, и было весело смотреть, как люди с трудом их объясняют. В целом, уровень игры достаточно низкий. Нужно регулярно играть, чтобы что-то уметь. До пяти часов я играл.
 Я и сам уровень теряю.

[1:24:39 AM] Маргулис:
Если бы посидела с на проективке, встретила бы тебя

[1:55:43 AM] Маргулис:
Прошел патруль, стуча мечами,
Дурной монах прокрался к милой.
Над островерхими домами
Неведомое опочило.

Но мы спокойны, мы поспорим
Со стражами Господня гнева,
И пахнет звездами и морем
Твой плащ широкий, Женевьева.

Ты помнишь ли, как перед нами
Встал храм, чернеющий во мраке,
Над сумрачными алтарями
Горели огненные знаки.

Торжественный, гранитнокрылый,
Он охранял наш город сонный,
В нем пели молоты и пилы,
В ночи работали масоны.

Слова их скупы и случайны,
Но взоры ясны и упрямы.
Им древние открыты тайны,
Как строить каменные храмы.

Поцеловав порог узорный, 
Свершив коленопреклоненье,
Мы попросили так покорно
Тебе и мне благословенья.

Великий Мастер с нивелиром
Стоял средь грохота и гула
И прошептал: «Идите с миром,
Мы побеждаем Вельзевула».

Пока живут они на свете,
Творят закон святого сева,
Мы смело можем быть как дети,
Любить друг друга, Женевьева.

[9:10:46 AM] Маргулис:
Ещё раз не хочу идти на Ракитина
 Мне слишком хорошо здесь
 Нет, надо все же пойти

[10:58:44 PM] Маргулис:
Голова болит

[11:52:28 PM] Маргулис:
Миша, научи меня читать только главное по философии

--- Tuesday, April 18, 2017 ---

[12:07:54 AM] Маргулис:
Скорей скорей скорей
 Эх ты
 У меня упала температура до 35,9

[12:58:43 AM] Михаил:
Тебе надо отдохнуть и выспаться.
 Я негодяй, снова сюда редко захожу, извини.
 Температура не очень низкая, это пройдет.

[1:07:11 AM] Маргулис:
Ага
 Негодяй
 Мне тяжело в таком количестве читать канта, я это не люблю, я с ним  не согласна и я и без него много об этом думала
 Но кирсберг принимает только ответы по канту
 Поговори завтра со мной об этом всем
 Я не могу без разговоров

[1:09:56 AM] Михаил:
Да. Я в чем-то с Кантом согласен.

[1:10:13 AM] Маргулис:
Есть ощущение, что никто не хочет со мной разговаривать
 А я ведь классно умею спорить
 Вы не хотите или не умеете?
 Меня избаловал сын двух выпускников философского факультета мгу?
 Хотя нет, я передумала, не хочу больше говорить с людьми, которые со мной не согласны
 Ты в скайпе есть?

[1:13:44 AM] Михаил:
Не вполне.
 Может, есть, но не знаю об этом.

[1:14:21 AM] Маргулис:
Ясно
 Я спать

[1:14:45 AM] Михаил:
Да, надо.

[1:14:51 AM] Маргулис:
Я ждала хоть кого-то три часа
 И никто не подошёл
 Я обиделась
edited 
[1:15:13 AM] Михаил:
В скайпе?

[1:15:35 AM] Маргулис:
Тут
 Ждала
 Тебя о ещё двух человек
 Всем на меня плевать
 Я ушла
 Все ок, не думай об этом
 (Не думал же три часа, и сейчас не думай)

[1:20:22 AM] Михаил:
Я исправлюсь, так больше не будет.

[11:09:21 AM] Маргулис:
Он умер, конечно. Даже смешно думать по-другому. И в Посторонних я права. Ты слишком лично к ним относишься, поэтому придумываешь ерунду
 Я даже не знаю, как в на последнем дыхании можно не понять, что он умер
 И, кстати, она это сделала не для того, чтобы он исправился
 Ты слишком наивный.
 Меня это раздражает.

[11:52:26 AM] Маргулис:
Где мы?

[8:16:53 PM] Маргулис:
Мишка
 Какая же я плохая
 Не обижайся на меня

[8:18:03 PM] Михаил:
За что?

[8:18:11 PM] Маргулис:
Ну вообще
 Про то, что я написала выше

[8:18:57 PM] Михаил:
Да, хорошо, конечно.

[8:19:07 PM] Маргулис:
Я такая сонная
 И боюсь философии
 Я хочу за семинар получить оценку

[8:19:49 PM] Михаил:
Я сейчас почти сплю.
 Надо хотеть получить удовольствие, а не бояться.

[8:20:42 PM] Маргулис:
Это как?
 Я не поняла
edited 
А, от философии
 Я однажды написала слово "филосовский"
 Было смешно

[8:21:46 PM] Михаил:
Львофский

[8:22:34 PM] Маргулис:
[Photo]
 Видел эту картину?

[8:23:01 PM] Михаил:
Да

[8:23:06 PM] Маргулис:
Хорошо
 На моей фотке видно, что губы и глаза были
 Но их он закрасил
 А в жизни не так видно

[9:34:05 PM] Маргулис:
Если нужно
$ http://anna-ganzha.narod.ru/kant_omn_fr.pdf
[kant_omn_fr.pdf] 226 KB$
 Это укорочённый кант

[9:35:27 PM] Михаил:
Спасибо.

[9:35:34 PM] Маргулис:
Но ты уже все прочитал
 Я знаю
 Кинь Мише, я не знаю, как его тут зовут
 Он есть в бмт 1?

[9:36:32 PM] Михаил:
Он есть у меня.
 Он обрадовался.

[9:37:09 PM] Маргулис:
Я уже отправила ему
 Блин
 Ну неважно

[10:23:40 PM] Маргулис:
Мне стало менее грустно жить за последние полторы недели
 А тебе?

[10:33:39 PM] Михаил:
Об этом лучше наяву поговорить.
 За тебя я рад. Мне еще нужно время, чтобы психика перестала приносить сюрпризы.

[10:35:23 PM] Маргулис:
Я только недавно поняла, что конец апреля через неделю.
 Мне страшно
 Надо успеть сдать 7 и 8 лист по геоме, 12 лист по матану, дискру
 И контра по дискре

[10:37:04 PM] Михаил:
Мы справимся, я уверен.

[10:37:17 PM] Маргулис:
Да, знаю
 Делаю матан

[10:37:52 PM] Михаил:
Я тоже делаю, за час ничего не решил.
 Ха-ха-ха, моя бабушка сейчас пытается третий раз вызвать скорую за сегодня.

[10:39:22 PM] Маргулис:
Поговори с ней?

[10:41:00 PM] Михаил:
Нет, ее невозможно в чем-то переубедить, тем более она ужасно плохо слышит.

--- Wednesday, April 19, 2017 ---

[1:51:42 PM] Маргулис:
Насчёт твоего примера: как желание убивать может быть независимо от эмпирики?

[3:55:52 PM] Михаил:
Оно зависит.

[6:09:20 PM] Маргулис:
Ну тогда твой аргумент ни чему не противоречит
 Дай мне геомку

[6:13:11 PM] Михаил:
[Photo]
 [Photo]
 [Photo]
 [Photo]
 [Photo]
 [Photo]
 [Photo]
 [Photo]
 [Photo]
edited 
Я имел ввиду, что в правиле "хочу ли я, чтобы максима моего действия стала всеобщим законом" есть субъективность. Разные люди могут хотеть разных всеобщих законов. Да и всеобщий закон невозможен, это фантазия.

[6:21:03 PM] Маргулис:
Разумеется
 Ну у тебя фотки перевёрнуты)
 Ладно, сама поверну
 Нет, подожди, смотря что ты имеешь в виду
 Не уходи
 А, ты тут
 А что ты подразумеваешь под всеобщим? Что все люди однажды к нему придут? Не придут, разумеется
 Это не означает, что не существует того идеального закона, к которому может прийти один человек и к которому стоит прийти остальным
 Проблема этого всего в том, что люди более всего ожидают наличия там правила вроде библейских заповедей
 Я имею в виду конкретно их
 Про не убий и не укради, почитай отца и мать и тд
 Но именно они совершенно ущербны
 Ты решил молчать?
 Да ну тебя
 Неужели нельзя ответить?

[6:32:13 PM] Михаил:
Я думаю

[6:32:27 PM] Маргулис:
Ладно
 Извини
 Кстати, я не уверена, что он хоть как-то показывает, что набор моральных законов—не пустое множество

[6:34:20 PM] Михаил:
Он говорит только один принцип.

[6:34:55 PM] Маргулис:
> Михаил Зыбин
> Я имел ввиду, что в правиле "хочу ли я, чтобы максима моего дейс
Если есть субъективность, то это твоя проблема, он говорит, что ты должен стремиться максимально от этого очиститься
> Михаил Зыбин
> Он говорит только один принцип.
Не поняла, к чему ты это сказал. Это принцип выбора, он не показывает, что можно выбрать что-то
 Спасибо за геому
 Ну лично я не могу тебе предьявить даже в сыром виде такую максиму

[6:36:53 PM] Михаил:
Нет, я ничего не понимаю
 У меня такое мировоззрение, которое предполагает, что я никому никогда не буду о нем рассказывать. Надо его перестраивать.

[6:39:23 PM] Маргулис:
Ок, на самом деле я не верю в её существование
> Маргулис
> Ну лично я не могу тебе предьявить даже в сыром виде такую макси
Я об этом выше сказала

[6:41:44 PM] Михаил:
На это всё надо с биологической точки зрения как-то смотреть.

[6:43:00 PM] Маргулис:
Нет, не надо
 Ок, давай так 
Твоё побуждение здесь—стать душевно возвышенней, вопреки внешним страданиям, и именно это понятие о возвышенности совсем не совпадает у людей
 Кант жил в довольно однородном обществе
 Культурно
 И ему представлялось, что это действительно примерно одно и то же
edited 
Ну вот Татьяна же поступала сообразно с той максимой, которую считала всеобщей
 Я про Ларину
 В Онегине
 В конце

[6:47:32 PM] Михаил:
Мне сейчас ужинать надо, извини.

[6:47:38 PM] Маргулис:
Ок
 Не обижайся

[8:24:18 PM] Маргулис:
Там 4-9 задания?

[8:28:29 PM] Михаил:
1-3 тоже

[8:29:30 PM] Маргулис:
Во-первых, я научу тебя писать цифру 7
 Она у тебя ужасна

[8:29:56 PM] Михаил:
А мне нравится.

[8:30:08 PM] Маргулис:
[Photo]
 Знаешь, знаковая система должна быть однозначной
 А твоя семь—это то ли 1, то ли л

[8:31:47 PM] Михаил:
[Photo]
 Я семерку не перечеркиваю, потому что она в печатных текстах не перечеркнута.

[8:33:25 PM] Маргулис:
Тогда пиши л с верхней перемычкой
edited 
Тогда на пиши черту под единицей
 Это непонятная цифра

[8:34:32 PM] Михаил:
Под единицей черта часто есть.

[8:34:37 PM] Маргулис:
Не
 Нет
 Слушай, рукописный шрифт придумали не просто так
 Это правда непонятно
 Ты не можешь быстро и чётко её писать

[8:36:41 PM] Михаил:
7 или 1?

[8:36:47 PM] Маргулис:
7
 Когда я писала 1 как ты, все думали, что это 2
 Но это неважно
 Правда, твоя 7 раздражает

[8:38:12 PM] Михаил:
Я проработаю этот вопрос.

[8:40:13 PM] Маргулис:
Слушай
 Я не кажусь тебе толстой?

[8:40:46 PM] Михаил:
Нет, конечно.
 Совершенной - да.

[8:41:37 PM] Маргулис:
Спасибо

[10:00:51 PM] Маргулис:
Миша, я ложусь уже сейчас
 Не могу больше

[10:01:52 PM] Михаил:
Хорошо, спи сладко, если можно так выразиться.

[10:02:17 PM] Маргулис:
Спасибо
 Ты тоже
 Я не написала геометрию, на семинаре этим займусь, пока у тебя будет шварцман

[10:03:34 PM] Михаил:
Где-то нужны будут мои комментарии.

[10:03:45 PM] Маргулис:
Потом на перемене
 И перед приёмом задач

--- Thursday, April 20, 2017 ---
edited 
[10:12:15 AM] Маргулис:
Я немножко опоздаю
 Если ты будешь ближе к той части аудитории, где я захожу, займи место, если нет—не занимай

[12:59:19 PM] Маргулис:
стикеры

[9:11:13 PM] Маргулис:
Forwarded message: 
Каган Александр [4/20/17] 
Так я же просто так ходил, я вообще на дискотеку записан)

[9:11:23 PM] Маргулис:
Каган вот на дискотеку записан
 Чего ты молчишь?
 Пойдёшь на питру?

[9:29:13 PM] Михаил:
Не пойду на Питру.

[9:31:31 PM] Маргулис:
Почему?
 Ты не сможешь отписаться
 Иди на питру
 Мне скучно там одной

[9:33:33 PM] Михаил:
У меня и учебника нет. И оттого, что я не сделаю задание, неловко будет.

[9:34:43 PM] Маргулис:
Ну так сделай
 Учебник я не ношу
 Так что это нормально
 Хотя я и сама не хочу делать задание

[9:35:51 PM] Михаил:
Сравнить что угодно, связанное с путешествием?

[9:37:16 PM] Маргулис:
Да
 Это смешно, конечно
 Хотя, конечно, я не заставляю тебя ходить
 Ну так же лучше
 Меня все ещё удивляет твой рейтинг. Что у тебя было по истории?

[9:38:49 PM] Михаил:
9

[9:39:06 PM] Маргулис:
А у меня 10)
 А за реферат что у тебя было?

[9:39:34 PM] Михаил:
10

[9:39:39 PM] Маргулис:
Странно
 Я думала, только у меня за него 10
edited 
У меня же лучший реферат, по её словам
> Маргулис
> Я думала, только у меня за него 10
Это шутка. И про то, что странно, тоже шутка
> Маргулис
> У меня же лучший реферат, по её словам
А это не шутка
 Это я хвастаюсь
 Надо было выбирать хорошую тему, как у меня
 А, у тебя же и так 10

[9:41:50 PM] Маргулис:
Смешная я

[9:41:51 PM] Михаил:
Моя - реформа в Англии 1851 года.

[9:42:01 PM] Маргулис:
1968, Париж
 Секунду, найду кое-что
 "Представь себе: война, а на неё никто не пошёл!"
 Кстати, такое было в Толстого

Wikipedia

Майские события 1968 года, или «Красный май» или Май 1968 (фр. Mai 68) — социальный кризис во Франции, начавшийся с леворадикальных студенческих вы...
 Я не знаю, почему у нас эта тема не популярна
 Ну, скорее всего, просто плохой год, в союзе его не любят
 Прага, танки, все дела
 Если ты вдруг не слышал об этом

[9:59:18 PM] Михаил:
Впервые об этом узнал на выставке в Пушкинском.

[9:59:25 PM] Маргулис:
А, точно
 (Ну как так можно, кошмар, разбойник Миша)

[10:01:47 PM] Михаил:
Глупый. Мне вовсе кажется, что меня не существует.

[10:02:03 PM] Маргулис:
Ну прекрати
 Сам же знаешь, что это не так

[10:03:40 PM] Михаил:
Я не знаю, у меня проблемы с самовосприятием давно.
 И давно периодически кажется, что меня не существует.

[10:04:18 PM] Маргулис:
Ты решил писать текст по англу?

[10:04:28 PM] Маргулис:
Если напишешь, мне тоже придётся
 А я спать хочу
 Так бы я ей по почте отправила попозже
 Но если ты да, то и я тоже

[10:05:15 PM] Михаил:
Я не решил еще.

[10:05:23 PM] Маргулис:
Какой объем нужен?

[10:05:50 PM] Михаил:
Я не знаю, я же не был.

[10:08:08 PM] Маргулис:
И я не знаю

[10:59:13 PM] Маргулис:
Кстати
 Вы с Каганом обсуждали, что не знаете, как ту девочку зовут
 Когда мы шли к метро
 Мы остались вдвоём и она спросила, как зовут вас

[11:00:29 PM] Михаил:
Смешно

[11:21:28 PM] Маргулис:
Ну ему же прострелили тело, он истёк кровью, что тут медицински не так
 ?

[11:26:01 PM] Михаил:
Пока он бежит, пятно на рубашке небольшое и не увеличивается; когда он лежит, вокруг него не растекается кровь.

[11:27:14 PM] Маргулис:
Ну это так сняли
 Умер он
 Я пойду в ванную и спать

[11:28:02 PM] Михаил:
Я согласен уже, что он умер.

[11:28:19 PM] Маргулис:
Ладно
 Спокойной ночи

[11:28:32 PM] Михаил:
Спокойной ночи, Маргулис.

--- Friday, April 21, 2017 ---

[12:29:50 AM] Маргулис:
И тебе

[9:39:46 AM] Михаил:
Я еду. Опоздаю немного.

[10:38:46 AM] Маргулис:
А я не еду
 Смешно

[10:40:21 AM] Михаил:
Ладно, я буду готовиться к контрольной с Мишей.

[12:04:27 PM] Михаил:
Я почему-то у Хорошкина.

[12:15:06 PM] Маргулис:
Почему?

[1:15:25 PM] Михаил:
Миша меня туда привел, а я не мог сопротивляться, потому что было ужасное настроение.

[1:22:23 PM] Маргулис:
Где ты?

[2:11:48 PM] Маргулис:
Миша, напиши мне после контрольной, я не пойду на проективку, можем немножко погулять

[3:22:09 PM] Маргулис:
Мишка, не бросай меня
 Миша
 Я в комп классе

[10:33:45 PM] Михаил:
Как твое настроение сейчас?

[11:02:02 PM] Михаил:
Спишь уже, наверное.

--- Saturday, April 22, 2017 ---

[1:13:37 AM] Маргулис:
Извини
 Не спала

edited 
[1:13:44 AM] Маргулис:
Настроение было хорошее
 Захлебнулась искусством
 Не могу удостовериться, действительно ли христиане настолько круты
 Пока что убеждаюсь
 Но я не могу в это верить всецело

[2:53:28 AM] Маргулис:
Миш, не заставляй меня наступать на родные грабли, мы не встречаемся и я не хочу. Не обнимай меня так часто, тем более на матфаке.

[1:28:14 PM] Михаил:
Я не требую от тебя ничего, Маша, кроме твоего существования. Встречаться - я ведь ни с кем не встречался, поэтому не очень понимаю, что это и зачем. Мы и так видимся на матфаке, и ты там все время рядом.
 Не хочу, чтобы тебе из-за меня было как-то неприятно. Буду обнимать тебя, только если ты захочешь.

[2:04:16 PM] Михаил:
Ты наиболее духовно богатый человек из всех, что я знаю. Я очень ценю тебя за это и не хочу, чтобы моя любовь мешала нашему общению.

[2:31:05 PM] Маргулис:
Хорошо
 Ты извини, что я это написала
 Просто я очень за тебя переживаю

[2:31:46 PM] Маргулис:
Forwarded message: Arzamas [4/22/17] 
Поматросил и бросил

Лермонтов забавы ради увел невесту у друга, бросил ее, написал сам на себя анонимную кляузу, а затем изобразил все это в романе

http://arzamas.academy/materials/119?src=tbot

Arzamas
Поматросил и бросил
Лермонтов поступил низко и изобразил все это в романе

[2:31:54 PM] Маргулис:
Немного о Лермонтове
 Извини меня. Ты очень хороший, мне приятно с тобой общаться
 Правда

[3:54:16 PM] Михаил:
Я совершенно из-за тебя не страдаю. Я принимаю и люблю все черты твоего характера.

[9:21:58 PM] Маргулис:
[Photo]

[10:27:00 PM] Михаил:
А я ведь ничего об атеизме не читал. Что ты читала? Эту книгу рекомендуешь, или она слишком пропагандистская?

[10:33:25 PM] Маргулис:
Это я не читала

[10:33:29 PM] Маргулис:
Она советская
 Нашла у своих
 Бабушки с дедушкой
 Она познавательная, но в общем пропагандистском духе

[11:01:31 PM] Маргулис:
У меня есть ощущение, что я постоянно пытаюсь попросить о помощи, но никто меня не слышит

[11:03:04 PM] Михаил:
Какая помощь тебе нужна?

[11:03:36 PM] Маргулис:
Психологическая, разумеется, о другой бы я так пафосно не написала бы

[11:04:24 PM] Михаил:
Как ты сейчас себя чувствуешь?

[11:04:33 PM] Маргулис:
Я чувствую себя чужой
 Я чувствую, что это едва ли возможно исправить

[11:06:43 PM] Михаил:
Честно, я сам не знаю, что с этим делать.

[11:07:04 PM] Маргулис:
Ты не понимаешь меня?

[11:07:28 PM] Михаил:
Понимаю, у меня ведь почти то же самое.

[11:07:39 PM] Маргулис:
Где у тебя почти то же самое?
 Мне безумно тяжело общаться с людьми

[11:10:18 PM] Михаил:
До марта я почти не мог общаться.

[11:10:28 PM] Маргулис:
Неправда

[11:11:04 PM] Михаил:
А ты хочешь, чтобы было легче?

[11:11:14 PM] Маргулис:
Я не понимаю, как ты можешь заговорить с малознакомым человеком на матфаке
 Хочу, чтобы было легче

[11:12:04 PM] Михаил:
С малознакомыми мне легче общаться, чем со знакомыми.

[11:12:15 PM] Маргулис:
Почему?
 Я тебе рассказывала о своих отношениях с одноклассниками?

[11:16:42 PM] Михаил:
Потому что они обо мне ничего не знают, и больше есть тем для разговора. Разговаривать с человеком, которого давно знаешь - это интересная и сложная для меня задача. Да, ты говорила, тебя не любили, ты не общалась.

[11:17:29 PM] Маргулис:
Значит, я говорила не подробно?
 Тем для разговора с незнакомым человеком не больше
 Ты же, подходя к человеку, не начитаешь с ним разговор о том, насколько золотой золотой век и насколько серебряный—серебряный
 Не могу понять, как тут нормально поставить знаки препинания

[11:21:02 PM] Михаил:
Я не умею рассуждать об общении.

[11:21:41 PM] Маргулис:
Ты же не можешь с незнакомым человеком начать разговор о миссионерской деятельности и влияние на мировую культуру ордена иезуитов
 Я вот так и не поняла, насколько для Японии важна христианская община 
Для Кореи вроде важнее 
И не поняла, насколько один иезуит повлиял на изо в Китае. Наверное, не так уж и повлиял

[11:23:40 PM] Михаил:
Я нашел курс про исторические фальсификации.

[11:23:53 PM] Маргулис:
Где?

[11:24:05 PM] Михаил:
На арзамасе.
 Номер два.

[11:24:39 PM] Маргулис:
А, старый
 Был такой, да
 Я не смотрела
 Только читала что-то
 Мне часто неудобно смотреть лекции

[11:25:56 PM] Маргулис:
Читать удобнее

[11:26:11 PM] Михаил:
Посмотрел первую часть нимфоманки и Безумного Пьеро.
 Читать действительно удобнее.

[11:28:20 PM] Маргулис:
Как тебе?

[11:28:21 PM] Михаил:
Ты затронула серьезную тему. О ней сложно здесь общаться.
 Я про одиночество.

[11:28:43 PM] Маргулис:
Я поняла
 Вы все друг друга знаете, вы все чувствуете себя уверенно

[11:29:18 PM] Михаил:
Кто?

[11:29:26 PM] Маргулис:
Люди на матфаке

[11:31:40 PM] Михаил:
Никто меня не знает, в том числе я.
 Я не являюсь частью какого-либо "вы".

[11:33:47 PM] Маргулис:
Ок, они
 Являться частью какого-либо вы—ты для этого выступаешь как объект, я решаю, являешься ли ты. Ты решаешь только, являешься ли частью какого-либо твоего "мы"

[11:36:39 PM] Михаил:
Да, действительно.

[11:38:00 PM] Маргулис:
Тебе понравились фильмы?

[11:38:11 PM] Михаил:
Да.

[11:39:43 PM] Маргулис:
Оба?
 Ещё надо посмотреть Бартона Финка, я люблю его
 Это фильм, не режиссёр

[11:40:18 PM] Михаил:
Нимфоманку завтра, видимо, досмотрю. Оба.

[11:40:31 PM] Маргулис:
Но безумный пьеро же так себе

[11:42:50 PM] Михаил:
Мне тяжело писать об этом в телеграме.
 Об одиночестве тоже. Но я тебя понимаю.

[11:44:30 PM] Маргулис:
Я думаю, не очень
 У тебя обо мне не особо хорошее мнение.

[11:45:44 PM] Михаил:
Forwarded message: Михаил Зыбин [4/22/17] 
Ты наиболее духовно богатый человек из всех, что я знаю. Я очень ценю тебя за это и не хочу, чтобы моя любовь мешала нашему общению.

[11:46:07 PM] Михаил:
Наиболее духовно богатый.

[11:47:09 PM] Маргулис:
И что это должно значить?
 Может, ты бы это сказал матери Терезе
 Растяжимое понятие
 Не растяжимое даже, просто его можно понять как угодно
 Дама из Чистого понедельника духовно богата?

[11:50:38 PM] Михаил:
Мне тяжело писать об этом в телеграме. У меня не получится сейчас убедить тебя, что я тебя понимаю.
 Я хочу пойти спать.

[11:53:23 PM] Маргулис:
Хорошо
 Спокойной ночи
 Не грусти только
 Я справлюсь

[11:54:11 PM] Михаил:
Я тебе помогу.

[11:54:57 PM] Михаил:
Правда, я хорошо знаком с одиночеством.
 Спокойной ночи.

--- Sunday, April 23, 2017 ---

[4:22:56 PM] Маргулис:
Красивое небо
 У тебя стоит
 Мне нравится, как оно смотрится рядом с моей картиной
 Я вот задумалась, что люди находят в комиксах? Ты хоть раз читал их?

[5:13:07 PM] Михаил:
Не читал, но супергеройское кино по их мотивам смотрел. Это только развлечение, чтобы отдохнуть и убить время; вся эмоциональная реакция в общем сводится к фразе "Ух ты, круто". Кому-то другого и не нужно.

[6:39:39 PM] Михаил:
[ Flёur – Расскажи мне о своей катастрофе  ] 14.3 MB
 
[ Flёur – Для того, кто умел верить  ] 5.2 MB
 
[ Flёur – ...и солнце встает над руинами  ] 12.1 MB
 
[ Flёur – Голос  ] 5.7 MB
 
 Я очень много и часто раньше слушал Флер, в том числе эти песни. Они объединены общим мотивом: сейчас очень плохо, но когда-то потом станет хорошо.

Больше двух лет я ждал и искал какого-то человека. Ждал рая за углом, хирурга, который сотрет шрамы; любви, которая осушит озеро слез, островов утешения, и так далее. Теперь я дождался. Ты для меня - вот это вот всё. Такое у меня о тебе мнение.
 Извини, пожалуйста, если это звучит глупо, и если песни тебе покажутся плохими.
 Очень я хочу помочь тебе, как ты помогла мне.

[9:44:55 PM] Маргулис:
Мне не хватает вдохновения жить
 Меня в классе не просто не любили, это была вполне оформившаяся травля, особенно в 11 классе.

[9:48:58 PM] Михаил:
Жаль тебя.

[9:49:19 PM] Маргулис:
Однажды в 10 классе мне на химии втыкали в волосы спички, коробок на это ушёл. Я заметила штук 7, думала, что вытащила все.
 Потом наша классная увидела, она с ними разговаривала, но одна из этих девочек была её крестной дочерью
 И дочерью её лучшей подруги


[9:50:32 PM] Михаил:
Ужасно.

[9:50:38 PM] Маргулис:
Второй она на выпускном говорила, что не представляет свою школьную жизнь без неё
 Эта девочка сейчас учится на платном в вышке, на рекламе, вроде
edited 
Третья девочка весила килограмм 150, и, наверное, она — самый тупой человек из моих знакомых

[9:53:17 PM] Михаил:
Я надеюсь, что такого больше не будет в твоей жизни.
 Я тебе сочувствую.

[9:53:50 PM] Маргулис:
На самом деле хорошо, что я тогда удержалась. Я была готова поджечь спичку и отправить её назад.

[9:54:23 PM] Михаил:
Очень несправедливо с тобой обходились.
 Слепые люди.

[9:56:58 PM] Маргулис:
Да нет, не слепые, у них особо не было возможности что-то рассмотреть. Эта троица просто была в другой касте по интеллекту. Я бы и сама не дала им с собой общаться.
 Не только из-за интеллекта. Ту, что в вышке, я знаю 12 лет, она всегда была тварью
 Да, если ты там считаешь меня глупой, ты не знаешь глупых людей.

[9:59:13 PM] Михаил:
Господи, Маша, я не считаю тебя глупой.
 Ты их не простила, верно?

[10:00:27 PM] Маргулис:
> Михаил Зыбин
> Господи, Маша, я не считаю тебя глупой.
Нет, считаешь.
 Не простила.
 Думай хоть, что ты говоришь. Я чуть не разрыдалась, когда ты это сказал в 214.

[10:01:34 PM] Михаил:
Я этого не сказал.

[10:03:19 PM] Маргулис:
Я их не смогу простить. По крайней мере пока помню это чувство.
 Я достаточно неплохо с этим справлялась. Но бывали моменты, когда меня это цепляло за живое.
 
[10:04:59 PM] Михаил:
Не знаю, насколько уместно предложить тебе простить всем всё, что тебе делали.

[10:05:09 PM] Маргулис:
Забавно было смотреть, как люди, которые неплохо со мной общались раньше, постепенно перестали даже здороваться.
edited 
Максимально смешно было видеть, что часть людей, например, новенькая девочка, и еще несколько, кто поздно пришёл в школу и раньше меня не знал, видели, что я умная, и что со мной интересно, но не начинали общаться, потому что не хотели терять основную массу коллектива.

[10:08:12 PM] Михаил:
У меня нет слов.

[10:10:18 PM] Маргулис:
Люди, которые хоть немного меня ценили, постепенно разошлись из школы. А тот человек, который мне нравился и который свалил уже после 8, наверное, единственный из них, кто мог бы за меня заступиться. Но это уже неважно. Хотя, конечно, если бы я рассказала, он считал бы это моей слабостью.
 Если бы рассказала сейчас
 Моего бывшего они тоже не любили, кстати. Несколько раз что-то рисовали фломастером на его кофте. Весёлая у меня была жизнь.

[10:13:43 PM] Михаил:
Мне очень жаль тебя. Такого больше не будет в твоей жизни, я надеюсь.

[10:15:07 PM] Маргулис:
Я тоже надеюсь.
 Не жалей, все нормально уже.
 Ну, почти.
 Иногда ощущаю себя как раньше.

[10:19:07 PM] Михаил:
Я бы написал, что всегда буду рядом и поддержу тебя, но это немного похоже на клише.

[10:19:47 PM] Маргулис:
Я бы могла что-то сделать, но у моих нервов есть дурацкая особенность, я очень легко начинаю плакать без повода, и поэтому не люблю разговаривать с людьми.
 Я вот не могу разговаривать с родителями из-за этого

[10:20:42 PM] Михаил:
Я уверен, с этим можно справиться.
 Нужно время.

[10:21:36 PM] Маргулис:
Мне все время кажется, что кто-то стал хуже ко мне относиться

[10:22:13 PM] Михаил:
Кто, и как давно кажется?

[10:23:10 PM] Маргулис:
Кто угодно из наших девочек. Как только начинаем меньше общаться, мне кажется, что что-то не так. Как только у неё просто настроение так себе — для меня проблема во мне

[10:26:26 PM] Маргулис:
Все хорошо. Не волнуйся за меня.

[10:27:37 PM] Михаил:
Я люблю тебя все-таки, и твои проблемы мне важны.

[10:27:44 PM] Маргулис:
Кстати, мне с сентября кажется, что я сильно не нравлюсь Насте

[10:28:54 PM] Михаил:
Мне так не казалось, не знаю.
 Я у тебя спрашивал, читала ли ты письма Ван Гога к брату?

[10:32:56 PM] Маргулис:
По-моему, все я точно не читала, а часть читала на одной выставке в гараже

[10:33:47 PM] Михаил:
Они могут помочь тебе.

[10:34:56 PM] Маргулис:
Вряд ли

[10:35:43 PM] Михаил:
Могут.

[10:35:47 PM] Маргулис:
Ну ок
 Попробую
 Всё хорошо.

[10:38:09 PM] Михаил:
А ночь в полнолуние не прочитала?

[10:41:18 PM] Маргулис:
Я не поняла, что за странный порядок страниц, но в любом случае, нет. Не хочу читать сейчас ничего романтичного. Извини. Но я помню об этом.
edited 
Мне очень стыдно за то, что я не уверена, что сама смогла бы в такой ситуации что-то изменить, если бы дело было с кем-то другим.

[10:46:03 PM] Михаил:
Господин Никто, Облачный атлас, Кон-тики - эти фильмы дали мне вдохновение, чтобы жить. И этот рассказ тоже.

[10:46:19 PM] Маргулис:
Господин никто—гадость
 Остальные не смотрела
 Другой его фильм лучше, но концовку он запорол

[10:47:10 PM] Михаил:
Я не знаю, кто режиссер.

[10:48:00 PM] Маргулис:
Я тоже
 Ну, не помню
 Фамилию
 Другой фильм—Новейший завет
 Ты не грусти
 На самом деле нормальный фильм, кстати
 Извини

[10:55:11 PM] Михаил:
Ладно, неважно. Давай задачи решать какие-нибудь.

[10:55:59 PM] Маргулис:
Хорошо

--- Monday, April 24, 2017 ---

[12:41:24 AM] Маргулис:
Если я их прощу, это ведь значит, что я сама была в этом виновата. Я знаю, я сейчас очень по-детски говорю.

[1:02:27 AM] Михаил:
По-моему, прощают тех, кто виноват.
 Мне однажды рассказали такое рассуждение. Если тебя обидел кто-то, кто не хотел тебя обидеть, то не нужно на него обижаться, так как он и сам наверняка страдает, и от обиды всем будет только хуже. А если человек хотел тебя обидеть, то тоже не нужно обижаться, поскольку иначе ты ведешь себя так, как он хотел; то есть он добился своей цели - родить в тебе плохие чувства.
 Я думаю, если ты простишь их, то ты их победишь этим. Груз обиды, который они намеренно на тебя положили, стоит сбросить.
 Об этом есть притча. Она меня очень впечатлила когда-то. https://elims.org.ua/pritchi/pritcha-budda-i-zhiteli-derevni/

Мудрые и короткие притчи
Притча: Будда и жители деревни
Мудрая притча о оскорблениях и как на них реагировать: Однажды Будда с учениками шёл мимо деревни, в которой жили противники буддизма. Жители высыпали...

[1:32:57 AM] Маргулис:
Спасибо, что заботишься обо мне
 Я хотела поговорить со своей любимой учительницей по математике весь последний год. Я хотела поговорить наедине, понятно почему. (Кстати, по-моему, если в конце стоит вопросительное слово, запятую не ставят и я правильно всё написала.) Она задерживалась, чтобы что-то с каким-нибудь моим одноклассником обсудить. Я ждала. Она уходила. Она уходила, даже не поворачивая голову в мою сторону. А мне она была нужна. Самое странное, что у неё тоже сложилось какое-то странное представление о моих отношениях с Серёжей и Васей. Ну, это неважно. В общем, она так и не помогла мне. Она тратила время, чтобы найти общий язык с моими одноклассниками, общий язык, когда нужно было показать им, что здесь она учитель и она считает их поведение недопустимым, а не идти у них на поводу. Я не об отношениях со мной, просто они к 11 классу стали ленивыми свиньями. Потом один из них умер и мне совсем не жаль его. Я — почти единственный человек, не пришедший на его похороны. Мне искренне на них плевать. Неважно. Но её я больше не люблю. Она, конечно, была завучем, ей было тяжело и много дел — но на звонки моему бывшему её хватало.
 Наверное, она с моей стороны любви не чувствовала, наверное, не поняла, почему я перестала улыбаться и стала жёстче в общении. Возможно, ничего не поняла про отношения с классом.
 Ты ещё не спишь?

[1:56:39 AM] Михаил:
Засыпаю.

[1:58:15 AM] Маргулис:
Хорошо

[1:58:16 AM] Маргулис:
Спи
 Сладких снов
 Не грусти из-за меня

[2:00:54 AM] Михаил:
Да, спокойной ночи.

[2:01:04 AM] Маргулис:
Миш, всё хорошо
 На англ идёшь?

[2:01:31 AM] Михаил:
Иду, но я ничего не делал.

[2:01:44 AM] Маргулис:
А что нужно было?
 Я забыла

[2:01:54 AM] Михаил:
Не знаю.

[2:01:58 AM] Маргулис:
Что бываю дз по англу
 Ну ладно, до завтра

[2:02:25 AM] Михаил:
Да. Хотя оно уже.

[2:02:31 AM] Маргулис:
Знаю
 Даже стала так воспринимать сутки

[9:10:26 PM] Маргулис:
У тебя потрясающее чувство юмора, Миш. С тобой очень весело и хорошо

[10:05:01 PM] Михаил:
Спасибо, Маша.

[10:26:52 PM] Маргулис:
Я обидела человека

[11:08:17 PM] Михаил:
Можно немного подождать и попросить прощения. В телеграме обидела?
 Это твой друг, он важен тебе?

[11:09:47 PM] Михаил:
Ты думаешь, зря обидела?

[11:48:59 PM] Маргулис:
Извини
 Я не видела сообщений
 В телеграме
 Сильно
 Зря
 Просто этот человек периодически меня обижает
 Который Серёжа
 Он может говорить мне разные гадости, но я, наверное, неадекватно отвечаю

[11:51:50 PM] Михаил:
Почему он говорит тебе гадости?

[11:52:47 PM] Маргулис:
Потому что любит меня и не может принять то, что мы не вместе и что я постоянно отдаляюсь от него

[11:54:12 PM] Михаил:
Да, понятно. Будь с ним мягче.

[11:55:49 PM] Маргулис:
Ну, это тяжело, когда в какой-то момент он называет меня шлюхой
 Это неоправданно, если у тебя есть сомнения.

[11:56:49 PM] Михаил:
Кошмар какой.

[11:57:49 PM] Маргулис:
Не волнуйся, все хорошо

--- Tuesday, April 25, 2017 ---

[12:00:11 AM] Михаил:
Не знаю, не могу придумать, как ему можно помочь и что тут делать.

[12:40:53 AM] Маргулис:
Миша
 О чем шутил Саша, когда говорили о сексе втроём?
 Я не понимаю ваших шуток

[12:55:27 AM] Михаил:
Я уже не помню. Наверное, мы смеялись над самим фактом того, что обсуждаем это.

[12:57:17 AM] Маргулис:
Ок. Я ничего не понимаю. До матфака я не знала, что часто не въезжаю в шутки
 А, со мной же никто не шутил
 Я забыла

[12:59:43 AM] Михаил:
Шутящий сам может не въезжать. Не думай об этом сильно.

[1:00:15 AM] Маргулис:
Ок
 Добыча кузнечика?
 Ой, чёрт, только не кузнечик.
 Это ужасно.
 Я однажды оторвала ноги кузнечику, потому что думала, что они отрастут. Это было года в 4. А недавно мне снился сон про гигантского кузнечика
 Он гнался за мной

[1:12:25 AM] Михаил:
Это неприятно всё, я понимаю.

[1:14:58 AM] Маргулис:
Разве есть разница в 3а и в 3б?
 Если что, это цифра три
 А не буква З

[1:16:09 AM] Михаил:
Да. Боюсь, я неправильно ее почувствовал, но есть.
 З3

[1:16:24 AM] Маргулис:
А какой ответ в них?

[1:16:30 AM] Михаил:
3ЗЗ3З
 Забавно.

[1:17:10 AM] Михаил:
Я получил три пятых и три десятых.
 Судную ночь ты смотрела?

[1:18:54 AM] Маргулис:
Нет
 Не поняла, как ты это получил

[1:20:32 AM] Михаил:
Я немножко понимаю. Слегка.

[1:20:56 AM] Маргулис:
И это правильный ответ?

[1:21:19 AM] Михаил:
Не спрашивал у людей.

[1:21:30 AM] Маргулис:
Наверное, я ничего не понимаю

[1:22:11 AM] Михаил:
У тебя получится.
 И у меня получится.

[1:22:34 AM] Маргулис:
Ок
 И Гегеля прочитать получится
 Опять убила день на того человека. Идиотка. Больше не отвечу ему никогда.

[1:23:54 AM] Михаил:
Да, нехорошо получилось.

[1:25:12 AM] Маргулис:
Завтра есть лекция утром?

[1:29:12 AM] Михаил:
Нет.
 Я не вспомнил бы, спасибо.
 У Гегеля есть слово "специфизирование". Еще он часто пользуется словом "тожество".

[2:23:03 AM] Михаил:
Спокойной ночи, Маргулис.
 Я лег.

[2:55:35 AM] Маргулис:
И тебе
 Я легла

[12:11:25 PM] Маргулис:
Ты где, Миш?

[12:13:22 PM] Михаил:
Проспал, буду к 13:30.

[12:22:15 PM] Маргулис:
Ок
 Везёт тебе
 Я бы тоже хотела проспать

[1:02:38 PM] Маргулис:
Давай встретимся в комнате отдыха, потом пойдём по месту назначения

[4:29:40 PM] Маргулис:
Парень за моей спиной звал меня на свидание в начале года

[5:52:15 PM] Маргулис:
Я чувствовала себя совсем по-другому, когда была верующей

[8:57:45 PM] Михаил:
Я сейчас буду решать чего-нибудь. Потом отвечу, или завтра поговорим.

[8:58:08 PM] Маргулис:
Ок

[10:04:34 PM] Маргулис:
Я потеряла свои решения.
 Блеск.
 Миш, у тебя правильный ответ в 2а был?

[10:32:47 PM] Маргулис:
Ок, ты, видимо, вернулся к своей модели не отмечания мне
> Михаил Зыбин
> Я сейчас буду решать чего-нибудь. Потом отвечу, или завтра погов
Ну я же через полминуты ответила. Неважно.

[11:14:10 PM] Михаил:
Сегодня я ничего не сдал - не нашел принимающих, пошел на чаепитие с Владленом.
 Не знаю про свой 2а.

[11:20:46 PM] Маргулис:
А какой там знаменатель?

[11:21:34 PM] Маргулис:
У меня упало настроение
 Предварительно оно взлетело

[11:22:28 PM] Михаил:
[Photo]
 Надо взять себя под контроль.

[11:23:38 PM] Маргулис:
М, у меня не только он отличается
 Ещё 2, а не 4

[11:24:23 PM] Михаил:
Я не претендую на правильность.

[11:24:26 PM] Маргулис:
Я слишком чуткий человек для этого
 Для того, чтобы брать под контроль
 Я научилась за несколько лет улавливать слишком много вещей.

[11:25:22 PM] Михаил:
Это без причины случилось, или есть причина?

[11:25:29 PM] Маргулис:
Есть несколько
 Энергия не нашла выхода и израсходовалась
 Впустую
 Ещё расстроилась из-за волн, хотя это смешно, конечно, расстраиваться из-за таких вещей.
 И Гегеля мы хотели обсудить, но не обсудили

[11:28:39 PM] Михаил:
С Георгием надо завтра поговорить, разубедить его. Но расстраиваться не надо из-за него.
 Про Гегеля действительно зря забыли.

[11:29:00 PM] Маргулис:
Да дело даже не очень в этом
 Ты занят?

[11:29:20 PM] Михаил:
Пытаюсь

[11:29:36 PM] Маргулис:
Тогда я одно сообщение ещё напишу
 Ты пока можешь отойти, минут через 5 я закончу
 Дело в том, что, когда я говорю, что я атеистка, люди начинают смотреть более косо. Я вижу, что им кажется, что моя картина мира слишком упрощённая, я вижу, что люди стали хуже ко мне относиться, и меня это расстраивает. Каждый разговор об этом — какое-то оправдание с моей стороны. И я забыла, что ещё хотела написать

[11:34:06 PM] Михаил:
Когда люди начинали смотреть на тебя косо в этой ситуации?

[11:34:15 PM] Маргулис:
Георгий

[11:35:29 PM] Михаил:
Мне так не показалось.
 Я хуже отношусь к людям, если они не атеисты.

[11:36:23 PM] Маргулис:
Точно, вот вторая часть разговора
 Зачем? И почему ты при этом считаешь, что нет правильных и неправильных взглядов на мораль? Ты ведь ещё совсем недавно был верующим. Концепция Бога же не делает людей хуже

[11:39:41 PM] Михаил:
Это не из-за морали, а из-за способа мышления. Если человек признает существование бога, то он неправильно мыслит.

[11:40:05 PM] Маргулис:
Да брось, ты бы это даже не почувствовал
 Тяжело затронуть ту область мышления, на которую влияет эта концепция
 Ты и так мыслишь набором предрассудков, любой человек, это такое бытовое мышление
 Кроме ряда случаев, о котором ты ещё и не говоришь с людьми
 Лучше завтра обсудить

[11:42:30 PM] Михаил:
Я мыслю набором предрассудков? Это звучит обидно. Наверное, ты не это имела ввиду. Лучше завтра обсудить.

[11:42:54 PM] Маргулис:
> Маргулис
> Ты и так мыслишь набором предрассудков, любой человек, это такое
Любой человек мыслит так
 Я это имела в виду
 В бытовом случае

[11:46:49 PM] Маргулис:
Ты понимаешь, что одинаково мысля, нельзя прийти к разным выводам?

[11:47:44 PM] Михаил:
Да. Лучше завтра обсудить?
edited 
[11:47:55 PM] Маргулис:
Да, если тебе так удобнее
 И ты недавно признавал, что нет никакой правильности мышления
 Мне кажется, мне стоит уйти с матфака. Извини, я не хочу тебя расстроить.

--- Wednesday, April 26, 2017 ---
 Мы бы продолжили общаться, наверное.
edited 
[12:20:34 AM] Маргулис:
Эх.

[1:26:47 AM] Михаил:
У тебя действительно испортилось настроение.
 Мы не обсудили тему одиночества, надо обсудить.
 Спокойной ночи, Маша. Я пораньше решил лечь.

[1:36:29 AM] Маргулис:
Хорошо
 Спокойной тебе ночи
 Я сейчас дочитаю Гегеля
 Слушай, это смешно, но мне правда тяжело отвечать ему. Пойми это, пожалуйста, и не относись ко мне хуже.
 У меня начинается паническая атака

[1:38:02 AM] Михаил:
Понятно, ничего плохого в этом нет.

[1:38:12 AM] Маргулис:
И к доске мне совсем тяжело выходить из-за этого же.

[1:38:29 AM] Михаил:
Хотя тебе, конечно, плохо.

[1:38:41 AM] Маргулис:
Я ничего не соображаю в таких ситуациях
 То есть я однажды ошиблась в чем-то вроде 6+3 от нервов

[1:39:36 AM] Маргулис:
Ладно, до завтра

[1:40:20 AM] Михаил:
Хотя оно уже)

[1:44:12 AM] Маргулис:
Ты хороший и добрый. Не хочу, чтобы ты грустил

[5:09:45 PM] Михаил:
Я называю это "приступ", "воспаление сознания" или "помутнение". Мне стыдно, что вы оказались рядом, когда это случилось.

[5:10:01 PM] Маргулис:
Ничего страшного, мы тебя любим

[5:11:39 PM] Михаил:
А ты?
 Об этом лучше говорить, видя друг друга, конечно.

[6:25:32 PM] Маргулис:
Прости
 Я написала и ушла

[7:43:20 PM] Маргулис:
А брить длинную бороду неудобно?

[7:52:03 PM] Михаил:
Ей можно придавать форму устройством по имени триммер, которого у меня нет. Бритвой можно только сбривать совсем.

[7:54:34 PM] Маргулис:
Нет, я имею в виду не это
edited 
Я имею в виду, когда ты совсем сбривал бороду, было хуже, чем когда ты брил щетину?
 Или ты сначала её постриг?

[7:55:55 PM] Михаил:
Постриг, да. Потом сбрил, что осталось.

[8:15:04 PM] Маргулис:
А ещё Варя каждую неделю пьёт

[9:40:36 PM] Михаил:
[Photo]

[9:46:57 PM] Маргулис:
Ты все по дискре сделал?
 Тебе лучше?

[9:47:23 PM] Михаил:
О, дискра.

[9:48:23 PM] Маргулис:
Ты не ответил
 И контра ещё завтра
 
> Михаил Зыбин
>  Photo
Откуда это?
 https://meduza.io/feature/2017/04/26/neyroseti-rasisty

Meduza
Нейросети-расисты: Как FaceApp, MSQRD и другие приложения становятся виновниками...
В январе 2017 года российские программисты выпустили приложение для редактирования селфи FaceApp. Оно быстро стало популярным не только в России, но и...

[10:44:47 PM] Михаил:
Я не могу ничего с собой поделать, но мне кажется, что в пятой задаче ответы 6, 6 и дальше все 1/2.

[10:45:27 PM] Маргулис:
> Маргулис
> Откуда это?
Откуда это?

[10:45:53 PM] Михаил:
Из беседе вконтакте.

[10:46:08 PM] Маргулис:
Ладно
 А картинка у тебя с луной откуда?

[10:48:58 PM] Михаил:
Не скажу.

[10:51:17 PM] Маргулис:
Почему?
 Миш?

[10:52:21 PM] Михаил:
Не волнуйся.

[10:52:28 PM] Маргулис:
Ты во мне разочаровался?

[10:52:43 PM] Михаил:
Нет

[10:54:35 PM] Маргулис:
Я вбила её в поиск по картинкам

[10:55:21 PM] Михаил:
Не знаю, что у тебя получилось.
 Я не пробовал.

[10:55:42 PM] Маргулис:
Много всего
 Разные сайты, с новостями о суперлунии 2012

[10:56:40 PM] Маргулис:
Почему не скажешь?
 Тебе плохо сейчас?
 Тебе ответить на твой вопрос сейчас или завтра?
> Михаил Зыбин
> А ты?
Этот вопрос

[11:24:47 PM] Маргулис:
Я завтра утром буду делать и сдавать дискру, за три часа должна успеть сделать и за три другие часа—сдать

[11:43:43 PM] Маргулис:
Спокойной ночи

 Ответь мне

[11:44:36 PM] Михаил:
Спокойной ночи.

[11:44:41 PM] Маргулис:
Миш
 В чем дело?

[11:45:38 PM] Михаил:
Не знаю
 Я такой загадочный...

[11:47:06 PM] Маргулис:
МИША
 Миш
 Я не лягу
 Пока ты не объяснишь

[11:47:45 PM] Михаил:
Просто приступ депрессии, ничего особенного.

[11:48:23 PM] Маргулис:
Не надо грустить
 Ты достоин того, чтобы быть счастливым

[11:49:12 PM] Маргулис:
Прошу тебя
 С этим можно бороться

[11:49:41 PM] Михаил:
Ну ладно

[11:50:00 PM] Маргулис:
Миша
 Миш
 Ну прости меня
 Ты меня простишь?

[11:50:58 PM] Михаил:
Да, конечно.

[11:51:20 PM] Маргулис:
А за что?

[11:52:02 PM] Михаил:
За то, что ты меня никогда не полюбишь.

[11:53:03 PM] Маргулис:
Миш, боже мой. Прошу тебя, мне тоже очень тяжело. Я переживаю за тебя. Я хотела на матфаке ни с кем не общаться, чтобы такого не было.

[11:53:39 PM] Михаил:
Пойду спать.

[11:54:11 PM] Маргулис:
Прости меня.
 Спокойных снов.

--- Thursday, April 27, 2017 ---

[12:41:52 AM] Михаил:
Поговорил с Трибуналом (это мой внутренний судья). Мне намного легче, честно-честно.

[7:46:45 AM] Маргулис:
Займи мне место
 Если ты не против, на 2 ряду
 Желательно в левой части
 Сам сядь справа от меня

[7:23:02 PM] Маргулис:
На Арзамасе неразгадываемые загадки

[10:13:07 PM] Маргулис:
Как ты?
 Чего делаешь?

[10:13:50 PM] Михаил:
Арзамас читаю.

[10:14:00 PM] Маргулис:
Что именно?

[10:14:31 PM] Михаил:
Журнал. Статьи какие-то.

[11:00:21 PM] Маргулис:
Кому завтра можно сдать геометрию?
 Моих нет
 Сдам после майских в общем

[11:02:12 PM] Михаил:
Не знаю.

[11:02:31 PM] Маргулис:
Я написала соломие, её нет

[11:29:03 PM] Маргулис:
Мишаня
 [Photo]

--- Friday, April 28, 2017 ---

[12:30:35 AM] Михаил:
Великолепная картинка.

[12:31:12 AM] Маргулис:
Я не знаю, посмотреть фильм или лечь
 Подскажи
 До 2 точно дойдёт, если я буду его смотреть

[12:34:00 AM] Михаил:
Лучше лечь. Целые каникулы впереди, на них посмотришь.

[12:34:06 AM] Маргулис:
Хорошо
 Я пока прочитаю рассказ Бредбери, который ты прислал
 Давно прислал

[12:36:52 AM] Маргулис:
Спокойной ночи
 Мишка
 Сладких снов, будь весёлым

[12:38:01 AM] Михаил:
Спасибо, Маша, тебе тоже спокойной ночи.

[10:39:38 AM] Маргулис:
Мишка
 Ты не на матфаке?
 Я приехала, но к Питре не пошла
 [Photo]
 [Photo]

[11:13:10 AM] Михаил:
Еду, к двенадцати буду.

[7:26:15 PM] Маргулис:
Кстати, я поняла, что никогда четвёртого раза не было

[9:02:58 PM] Михаил:
Поясни, пожалуйста.

[9:04:41 PM] Маргулис:
Я говорила, что 4 раза влюблялась
 Я не считаю 4 раз

[10:01:01 PM] Маргулис:
Я не люблю читать о космосе

[11:21:51 PM] Маргулис:
Ты как?

[11:22:27 PM] Михаил:
Четвертый - это Вася? И ты поняла, что не влюблялась в него, верно?

[11:22:39 PM] Маргулис:
Нет
 А, я изначально говорила, что трижды влюблялась?
 Четвёртый—это тот, о котором я не говорила, видимо. А с Васей все всегда было понятно

[11:24:28 PM] Михаил:
Я и помню вроде трех людей. Четвертый - он после Васи, да?

[11:24:56 PM] Маргулис:
Нет, забей.
 Не было ничего. Могло быть, но я подавила это
 Вася мне никогда не мешал, он тут не точка отсчета

[11:26:12 PM] Михаил:
Понятно

[11:26:39 PM] Маргулис:
Я понимаю, это плохо звучит
 Дело не в том, что я бы ему изменяла, но нельзя духовно изменять тому, с кем ничего не было
 Я имею в виду, влюбленность в данном случае—не измена
 Тот человек, о котором я сейчас исправилась, просто нравился мне

[11:29:52 PM] Михаил:
Хорошо, я понимаю.

[11:29:58 PM] Маргулис:
> Маргулис
> Дело не в том, что я бы ему изменяла, но нельзя духовно изменять
Чёрт, кошмарная фраза
 Я не хочу сказать, что у нас с Васей были не духовного рода отношения
 Я имею в виду только то, что я не любила его
 Я могу получить медаль за косноязычие
 Все, мне стыдно за то, что я разучилась писать

[11:33:28 PM] Михаил:
Ничего страшного, я немножко понял.

[11:33:38 PM] Маргулис:
Смешно
 Откуда картинка-то?

[11:41:06 PM] Михаил:
Нет, не скажу.

[11:41:34 PM] Маргулис:
Почему?

[11:44:03 PM] Михаил:
Хороший вопрос.
 Я почему-то фильмы смотрю дольше, чем они идут. Сейчас закончил Парфюмера, а начал в 20:00, притом что он длится два с половиной часа.

[11:57:54 PM] Маргулис:
Я так же
 Я не час дольше иногда смотрю

[11:58:14 PM] Михаил:
Мне не понравился фильм, но я давно хотел его посмотреть.
 У него тот же режиссер, что у Облачного атласа, оказывается. Том Тыквер.

[11:44:03 PM] Михаил:
Хороший вопрос.
 Я почему-то фильмы смотрю дольше, чем они идут. Сейчас закончил Парфюмера, а начал в 20:00, притом что он длится два с половиной часа.

[11:57:54 PM] Маргулис:
Я так же
 Я не час дольше иногда смотрю

[11:58:14 PM] Михаил:
Мне не понравился фильм, но я давно хотел его посмотреть.
 У него тот же режиссер, что у Облачного атласа, оказывается. Том Тыквер.

--- Saturday, April 29, 2017 ---

[12:00:43 AM] Маргулис:
Я таки говорила)

[12:00:55 AM] Михаил:
Смотрела "Он умер с фалафелем в руке"?

[12:01:24 AM] Маргулис:
Нет
 Не говори ничего про 
То

[12:02:06 AM] Михаил:
Не буду говорить про То.

[12:02:14 AM] Маргулис:
Ой
 Опечатка
 Это
 Хотя нет, говори, ничего о нем не слышала

[12:03:56 AM] Михаил:
Я его совсем не понял, но не знаю, из-за того это, что я был глупый, или из-за того, что фильм дурацкий.
 Это в январе было, по-моему.

[12:11:11 AM] Маргулис:
А Саша Леонтьева тебе нравится?

[12:12:35 AM] Михаил:
Маша, мне нравишься только ты. Ты самая красивая на нашем курсе и вообще.

Аризонская мечта абсурдностью происходящего напомнила мне этот фильм. И неявный драматизм там, наверное, тоже есть.

[12:13:21 AM] Маргулис:
Да, только я, Крутовский и Гаицгори)
 Я шучу

[12:13:50 AM] Михаил:
Мне интересно, что ты там увидишь. Я бы рискнул его тебе посоветовать посмотреть.

[12:14:39 AM] Маргулис:
Потом

[12:16:33 AM] Михаил:
В Парфюмере мне показалось глупым, что герой случайно за полминуты задушил девушку. По-моему, это просто невозможно, и вообще очень нелепо.

[12:16:59 AM] Маргулис:
Блин, Миш, ты издеваешься?
 Они должны показывать это 5 минут?
 Неважно
 Я ушла смотреть кино
 Настроение испортил
 Как всегда

[12:17:53 AM] Михаил:
Да, извини.

[12:18:11 AM] Маргулис:
Ну нельзя прибираться к таким вещам
 Ты же не удивляешься, что в кино могут показывать события, между которыми прошло много лет

[12:20:03 AM] Михаил:
Там в сцене явно время идет линейно и нормально. Явно полминуты. Но я тебя понял.

[12:20:29 AM] Маргулис:
Иванова меня позвала на английский
 Было ужасно
 Я плохо написала её тесты
 Зря согласилась пойти
 И сама Иванова ужасно неприятный человек
 Зачем я пошла? Я же знала, что не готова к тесту
 Как всегда, я написала, а ты ушёл

[1:09:30 AM] Михаил:
Не расстраивайся, бывает. Но я не воспринимаю Соню как неприятного человека. Я плохо умею не любить людей.
 Спокойной тебе ночи. Я лег.

[4:18:27 AM] Маргулис:
Иногда мне хочется отвечать тебе последними словами из на дне
 Она самый скучный человек на матфаке, тут нечего воспринимать. Я так её воспринимаю, потому что чувствую её отношение ко мне
 Она несколько раз вела себя очень неприятно
 Посмотри рассекая волны
edited 
[5:02:13 AM] Маргулис:
Миш, какого чёрта ты вообще решил сказать, что не считаешь её неприятным человеком? Где тут вопрос, что ты о ней думаешь? Если я так говорю, то у меня на это есть причины, тебе, мать твою, ясно?
 Если что, я правда тебя заблокировала, можно не писать.
 Хочется отвечать последними словами из на дне потому, что ты "такой фильм испортил"
 Объяснение для тех, кто наверняка не помнит.

[2:02:23 PM] Маргулис:
Мне стыдно.
 Я себя ненавижу.

[4:01:48 PM] Михаил:
Ладно, ничего, бывает. Я тебя простил. Ты меня тоже?

[4:04:21 PM] Маргулис:
Тоже.
 Да не за что тебя прощать. Ты ничего не сделал плохого.
 Мишенька

[7:01:02 PM] Михаил:
Ты часто смотришь больше одного фильма в день?

[7:01:21 PM] Маргулис:
Нет, почти никогда
 Давай не будем о фильмах, а ещё очень расстроена
 Я опять расстроилась.
 И ты опять не прочитал, хотя я сразу ответила
 Я удалила телеграм на айпэде, а телефон редко проверяю
 Извини

[11:34:27 PM] Маргулис:
Рассекая волны не посмотрел?

[11:50:02 PM] Михаил:
Нет, Новейший завет сегодня. У меня очередь, Рассекая волны дня через два посмотрю.

--- Sunday, April 30, 2017 ---

[12:11:47 AM] Маргулис:
Я таки зарегалась на Кинопоиске

[12:13:40 AM] Михаил:
Я сходил в лес, там так хорошо, что грустить неудобно и нелепо.
 Я многим фильмам оценки понизил после завета.

[12:20:19 AM] Маргулис:
Боже мой, неужели тебе понравилось это? Он же ужасно слащавый

[12:41:35 AM] Михаил:
Ну так, красивая сказка.

[12:41:54 AM] Маргулис:
Концовка плохая, но в целом приятно
 Я слишком злая, на самом деле он довольно ничего
 А что ещё в списке?
 До рассекая волны

[12:42:59 AM] Михаил:
Лучше не бывает и Мечтатели.

[12:44:50 AM] Маргулис:
Да, смотри

[12:45:47 AM] Михаил:
Я пойду спать. Спокойной тебе ночи.

[12:52:22 AM] Маргулис:
Я не помню, что я смотрела

[1:21:52 AM] Маргулис:
Посмотри бойцовую рыбку. Это важно

[4:15:37 PM] Маргулис:
А ещё посмотри выживут только любовники, это самый красивый в мире фильм

[4:49:21 PM] Маргулис:
Ты молчишь?
 Вообще, я могу отметить почти все фильмы из топ-250, но я не хочу
 Мне неудобно ставить некоторым фильмам низкие оценки
edited 
Например, я не люблю Форреста гампа

[4:56:56 PM] Михаил:
Почему неудобно? Ставь, что хочешь. Там, кстати, можно отметить фильм как просмотренный, а оценку не ставить.

[5:28:21 PM] Маргулис:
Забыла все фильмы ужасов

[6:23:06 PM] Михаил:
На мои посмотри.

[6:33:36 PM] Маргулис:
Посмотрела
 Ты тоже можешь посмотреть на то, чему я ставила 8-9
 Опять ни одной 10

[11:26:15 PM] Михаил:
[Photo]
 [Photo]
 [Photo]
 [Photo]
 [Photo]
 [Photo]
 Случайно пришел к заброшенным постройкам. Как ты относишься такому?

[11:30:35 PM] Маргулис:
Я бы не пошла с тобой внутрь
 Но красиво
 
> Михаил Зыбин
>  Photo
Как ты оказался наверху?

[11:35:08 PM] Михаил:
[Photo]
 Залез по этому.

[11:35:50 PM] Маргулис:
На этот гараж?
 Ты сумасшедший

[11:36:41 PM] Михаил:
Вполне нормально, по-моему.

\end{document}